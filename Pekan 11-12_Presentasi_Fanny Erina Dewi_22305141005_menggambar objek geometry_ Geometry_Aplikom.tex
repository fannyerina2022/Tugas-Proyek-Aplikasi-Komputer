\documentclass[a4paper,10pt]{article}
\usepackage{eumat}

\begin{document}
\begin{eulernotebook}
\eulersubheading{}
\begin{eulercomment}
Nama : Fanny Erina Dewi\\
NIM  : 22305141005\\
Kelas: Matematika B 2022\\
\end{eulercomment}
\eulersubheading{}
\begin{eulercomment}
SUB BAB 2: Menggambar objek-objek geometri.

\begin{eulercomment}
\eulerheading{MENGGAMBAR OBJEK OBJEK GEOMETRI}
\begin{eulercomment}
Objek-objek geometri meliputi titik, garis, bidang, bentuk bentuk
geometri pada bidang, beserta sifat sifatnya. Tiga unsur geometri
yaitu titik, garis, dan bidang. Ketiga unsur tersebut juga disebut
sebagai tiga unsur yang tak didefinisikan.

\end{eulercomment}
\begin{eulerprompt}
>load geometry
\end{eulerprompt}
\begin{euleroutput}
  Numerical and symbolic geometry.
\end{euleroutput}
\eulersubheading{Titik}
\begin{eulercomment}
Objek atau unsur pangkal dalam Geometri yaitu titik. Suatu titik
dipikirkan sebagai suatu tempat/ posisi dalam ruang. Titik tidak
memiliki panjang maupun ketebalan. Bekas tusukan jaru dan sentuhan
pertama ujung pensil di atas kertas dapat dikatakan sebuah noktah
dengan diberi nama suatu huruf alphabet kapital. Noktah sendiri adalah
bintik atau titik kecil.


Latihan 1\\
\end{eulercomment}
\eulersubheading{}
\begin{eulercomment}
Gambarlah titik A di koordinat (1,5)!
\end{eulercomment}
\begin{eulerprompt}
>setPlotRange(7); // mendefinisikan bidang kordinat baru
\end{eulerprompt}
\begin{eulercomment}
Langkah pertama yaitu membuat bidang koordinat dengan jarak 7. Pada
setPlotRange menampilkan bidang dengan jarak yang sama dengan masing
masing sumbu.
\end{eulercomment}
\begin{eulerprompt}
>A=[1,5];
\end{eulerprompt}
\begin{eulercomment}
Langkah kedua yaitu memanggil titik A untuk menggambar titik A di
bidang koordinat.
\end{eulercomment}
\begin{eulerprompt}
>plotPoint(A,"A"):
\end{eulerprompt}
\eulerimg{27}{images/Pekan 11-12_Presentasi_Fanny Erina Dewi_22305141005_menggambar objek geometry_ Geometry_Aplikom-001.png}
\begin{eulercomment}
Langkah ketiga yaitu menggambar titik dengan fungsi plotPoint. Fungsi
ini untuk menggambar titik A dengan memberi nama "A". Titik A disini
merupakan titik koordinat (1,5). 1 sebagai sumbu x dan 5 sebagai sumbu
y.

\end{eulercomment}
\begin{eulerprompt}
>setPlotRange(-7,3,-7,3);
>A=[1,5]; plotPoint(A,"A"):
\end{eulerprompt}
\eulerimg{27}{images/Pekan 11-12_Presentasi_Fanny Erina Dewi_22305141005_menggambar objek geometry_ Geometry_Aplikom-002.png}
\begin{eulercomment}
Latihan 2\\
\end{eulercomment}
\eulersubheading{}
\begin{eulercomment}
Gambarlah titik B di koordinat (-3,-5)!
\end{eulercomment}
\begin{eulerprompt}
>setPlotRange(-7,3,-7,3);
\end{eulerprompt}
\begin{eulercomment}
Langkah pertama yaitu membuat bidang koordinat dengan x1=-7, x2=3,
y1=-7, y2=3. bidang koordinat ini menentukan x dan y dengan x1
menunjukkan batas terkecil dan x2 menunjukkan batas terbesar sumbu x
sedangkan y1 menunjukkan batas terkecil dan y2 menunjukkan batas
terbesar sumbu y. 
\end{eulercomment}
\begin{eulerprompt}
>B=[-3,-5]; plotPoint(B,"B"):
\end{eulerprompt}
\eulerimg{27}{images/Pekan 11-12_Presentasi_Fanny Erina Dewi_22305141005_menggambar objek geometry_ Geometry_Aplikom-003.png}
\begin{eulercomment}
Langkah kedua yaitu memulis titik B dengan memanggil plotPoint dengan
menggambar titik B dengan memberi nama "B". Titik B ini (-3,-5) dengan
x =-3 dan y=-5.

Latihan 3\\
\end{eulercomment}
\eulersubheading{}
\begin{eulercomment}
Gambarlah 4 di titik C (2,7), titik D (-3,-5), titik E (-4,6), titik F
(2,-4) di bidang koordinat!
\end{eulercomment}
\begin{eulerprompt}
>setPlotRange(7);
\end{eulerprompt}
\begin{eulercomment}
Langkah pertama adalah menggambar bidang koordinatnya, disini saya
mengambil batas 7 di setiap sumbunya
\end{eulercomment}
\begin{eulerprompt}
>C=[2,7]; D=[-3,-5]; E=[-4,6]; F=[2,-4]; // mendefinisikan 3 titik
\end{eulerprompt}
\begin{eulercomment}
Langkah kedua yaitu menulis ketiga titik, titik yang diambil bisa di
jadikan satu perintah dengan menulis titik C dilanjut titik dua dan
spasi kemudian tulis titik berikutnya
\end{eulercomment}
\begin{eulerprompt}
> plotPoint(C,"C"); plotPoint(D,"D"); plotPoint(E,"E"); plotPoint(F,"F"):
\end{eulerprompt}
\eulerimg{27}{images/Pekan 11-12_Presentasi_Fanny Erina Dewi_22305141005_menggambar objek geometry_ Geometry_Aplikom-004.png}
\begin{eulercomment}
Langkah terakhir yaitu memanggil fungsi plotPoint. Menggambar titik
ini bisa dijadikan satu perintah juga.

Latihan 4\\
\end{eulercomment}
\eulersubheading{}
\begin{eulercomment}
Tentukan jarak antara titik P dan titik Q!\\
Titik P = (0,-5)\\
Titik Q = (-5,0)
\end{eulercomment}
\begin{eulerprompt}
> setPlotRange(-6,1,-6,1);
\end{eulerprompt}
\begin{eulercomment}
Langkah pertama adalah menggambar bidang koordinatnya terlebih dahulu
\end{eulercomment}
\begin{eulerprompt}
> P=[0,-5]; Q=[-5,0]; // mendefinisikan titik
\end{eulerprompt}
\begin{eulercomment}
Langkah kedua yaitu mendefinisikan titik dengan menulis titiknya.
\end{eulercomment}
\begin{eulerprompt}
> plotPoint(P,"P"); plotPoint(Q,"Q"): //menggambarkan titik
\end{eulerprompt}
\eulerimg{27}{images/Pekan 11-12_Presentasi_Fanny Erina Dewi_22305141005_menggambar objek geometry_ Geometry_Aplikom-005.png}
\begin{eulercomment}
Langkah ketiga adalah menggambar titik, seperti pada latihan di atas
cara menggambar titik adalah dengan plotPoint
\end{eulercomment}
\begin{eulerprompt}
>distance(P,Q) // menentukan jarak P dan Q
\end{eulerprompt}
\begin{euleroutput}
  7.07106781187
\end{euleroutput}
\begin{eulercomment}
Langkah terakhir adalah dengan fungsi distance. Fungsi distance
digunakan untuk menentukan jarak antara titik 1 dengan titik lainnya

Cara manual untuk menentukan jarak P dan Q adalah dengan rumus
phytagoras

\end{eulercomment}
\begin{eulerformula}
\[
A = -5, B =-5, C= ?
\]
\end{eulerformula}
\begin{eulerformula}
\[
\sqrt{50}
\]
\end{eulerformula}
\begin{eulerformula}
\[
5\sqrt{2} atau sekitar 7,07107
\]
\end{eulerformula}
\begin{eulercomment}
Latihan 5\\
\end{eulercomment}
\eulersubheading{}
\begin{eulercomment}
Tentukan titik tengah RS!\\
Titik R = (5,1)\\
Titik S = (3,6)
\end{eulercomment}
\begin{eulerprompt}
>R=[5,1]; S=[3,6]; //mendefinisikan titik R dan S
>middlePerpendicular(R,S)
\end{eulerprompt}
\begin{euleroutput}
  [2,  -5,  -9.5]
\end{euleroutput}
\begin{eulercomment}
Latihan 6\\
\end{eulercomment}
\eulersubheading{}
\begin{eulercomment}
Tentukan kuadrat jarak titik R dan titik T!
\end{eulercomment}
\begin{eulerprompt}
>R=[5,1]; S=[3,6];
>distanceSquared(R,S)
\end{eulerprompt}
\begin{euleroutput}
  29
\end{euleroutput}
\begin{eulercomment}
Latihan 7\\
\end{eulercomment}
\eulersubheading{}
\begin{eulercomment}
Tentukan kuadrat jarak titik R dan titik S!
\end{eulercomment}
\begin{eulerprompt}
>R=[5,1]; S=[3,6]; 
>quadrance(R,S)
\end{eulerprompt}
\begin{euleroutput}
  29
\end{euleroutput}
\eulersubheading{Contoh 1}
\begin{eulerprompt}
>setPlotRange(0,9,0,9);
>S=[3,4]; plotPoint(A,"A");
>V=[0.5,5]; plotPoint(B,"B");
>T=[1,1]; plotPoint(C,"C"):
\end{eulerprompt}
\eulerimg{27}{images/Pekan 11-12_Presentasi_Fanny Erina Dewi_22305141005_menggambar objek geometry_ Geometry_Aplikom-006.png}
\begin{eulerprompt}
>color(3); plotSegment(A,B,"AB"):
\end{eulerprompt}
\eulerimg{27}{images/Pekan 11-12_Presentasi_Fanny Erina Dewi_22305141005_menggambar objek geometry_ Geometry_Aplikom-007.png}
\begin{eulerprompt}
>color(8); plotSegment(B,C,"BC"):
\end{eulerprompt}
\eulerimg{27}{images/Pekan 11-12_Presentasi_Fanny Erina Dewi_22305141005_menggambar objek geometry_ Geometry_Aplikom-008.png}
\eulersubheading{Menggambar Ruas Garis}
\begin{eulercomment}
Ruas Garis adalah sebagian dari garis yang dibatasi oleh dua titik
ujung yang berbeda, dan memuat semua titik pada garis di antara
ujung-ujungnya. Ruas garia memiliki titik awal dan titik akhir.

Dalam latihan 8 kita menggambar ruas garis

Apabila kita ingin menambahkan warna pada ruas garis dengan rumu pada
fungsi yaitu
\end{eulercomment}
\begin{eulerprompt}
>color(2); plotSegment(A,C,"AC"):
\end{eulerprompt}
\eulerimg{27}{images/Pekan 11-12_Presentasi_Fanny Erina Dewi_22305141005_menggambar objek geometry_ Geometry_Aplikom-009.png}
\begin{eulercomment}
Latihan 1\\
\end{eulercomment}
\eulersubheading{}
\begin{eulercomment}
Gambarlah ruas garis dengan titik A (0,5) dan titik B (1,5)!
\end{eulercomment}
\begin{eulerprompt}
>setPlotRange(5);
\end{eulerprompt}
\begin{eulercomment}
Membuat koordinat dengan setPlotRange (5)
\end{eulercomment}
\begin{eulerprompt}
>A=[0,5]; plotPoint(A,"A"):
\end{eulerprompt}
\eulerimg{27}{images/Pekan 11-12_Presentasi_Fanny Erina Dewi_22305141005_menggambar objek geometry_ Geometry_Aplikom-010.png}
\begin{eulercomment}
Menentukan titik A dan menggambar titik A
\end{eulercomment}
\begin{eulerprompt}
>B=[1,5]; plotPoint(B,"B"):
\end{eulerprompt}
\eulerimg{27}{images/Pekan 11-12_Presentasi_Fanny Erina Dewi_22305141005_menggambar objek geometry_ Geometry_Aplikom-011.png}
\begin{eulercomment}
Menentukan titik B dan menggambar titik B menggunakan fungsi plotPoint
\end{eulercomment}
\begin{eulerprompt}
>plotSegment(A,B,"AB",3):
\end{eulerprompt}
\eulerimg{27}{images/Pekan 11-12_Presentasi_Fanny Erina Dewi_22305141005_menggambar objek geometry_ Geometry_Aplikom-012.png}
\begin{eulercomment}
Menggambar ruas garis dengan batas titik A (0,5) dan titik B (1,5)!

Latihan 2\\
\end{eulercomment}
\eulersubheading{}
\begin{eulercomment}
Gambarlah ruas garis dengan titik C (6,8) dan titik D (8,6)!
\end{eulercomment}
\begin{eulerprompt}
>setPlotRange(10);
\end{eulerprompt}
\begin{eulercomment}
Membuat garis kartesius/ bidang koordinat
\end{eulercomment}
\begin{eulerprompt}
> C=[6,8]; D=[8,6]; plotPoint(C,"C"); plotPoint(D,"D"):
\end{eulerprompt}
\eulerimg{27}{images/Pekan 11-12_Presentasi_Fanny Erina Dewi_22305141005_menggambar objek geometry_ Geometry_Aplikom-013.png}
\begin{eulercomment}
Menentukan titik dan menggambar titik C dan D
\end{eulercomment}
\begin{eulerprompt}
>plotSegment(C,D,"CD",20):
\end{eulerprompt}
\eulerimg{27}{images/Pekan 11-12_Presentasi_Fanny Erina Dewi_22305141005_menggambar objek geometry_ Geometry_Aplikom-014.png}
\begin{eulercomment}
Menggambar ruas garis CD dengan titik C(6,8) dan titik D(8,6) dengan
jarak label dari ruas garis adalah 20

Latihan 3\\
\end{eulercomment}
\eulersubheading{}
\begin{eulercomment}
Tentukan titik tengah di ruas AB, jika diketahui titik A(0,2), titik
B(6,2)!
\end{eulercomment}
\begin{eulerprompt}
>setPlotRange(10);
\end{eulerprompt}
\begin{eulercomment}
Langkah pertama adalah dengan membuat bidang koordinatnya
\end{eulercomment}
\begin{eulerprompt}
>A=[0,2]; plotPoint(A,"A"):
\end{eulerprompt}
\eulerimg{27}{images/Pekan 11-12_Presentasi_Fanny Erina Dewi_22305141005_menggambar objek geometry_ Geometry_Aplikom-015.png}
\begin{eulercomment}
Langkah kedua dan ketiga adalah mendefisikan dan menggambar titiknya
\end{eulercomment}
\begin{eulerprompt}
>B=[6,2]; plotPoint(B,"B"):
\end{eulerprompt}
\eulerimg{27}{images/Pekan 11-12_Presentasi_Fanny Erina Dewi_22305141005_menggambar objek geometry_ Geometry_Aplikom-016.png}
\begin{eulerprompt}
>plotSegment(A,B,"c"):
\end{eulerprompt}
\eulerimg{27}{images/Pekan 11-12_Presentasi_Fanny Erina Dewi_22305141005_menggambar objek geometry_ Geometry_Aplikom-017.png}
\begin{eulercomment}
Kemudian menggambar ruas garis yang melalui 2 titik yaitu A dan B 
\end{eulercomment}
\begin{eulerprompt}
>h= middlePerpendicular(A,B):
\end{eulerprompt}
\eulerimg{27}{images/Pekan 11-12_Presentasi_Fanny Erina Dewi_22305141005_menggambar objek geometry_ Geometry_Aplikom-018.png}
\begin{eulercomment}
Selanjutnya buat garis h yang tegak lurus ruas garis AB dan memotong
tepat di tengah ruas garis AB
\end{eulercomment}
\begin{eulerprompt}
>plotLine(h):
\end{eulerprompt}
\eulerimg{27}{images/Pekan 11-12_Presentasi_Fanny Erina Dewi_22305141005_menggambar objek geometry_ Geometry_Aplikom-019.png}
\begin{eulercomment}
Gambar garis h
\end{eulercomment}
\begin{eulerprompt}
>D=lineIntersection(h,lineThrough(A,B)):
\end{eulerprompt}
\eulerimg{27}{images/Pekan 11-12_Presentasi_Fanny Erina Dewi_22305141005_menggambar objek geometry_ Geometry_Aplikom-020.png}
\begin{eulercomment}
Menentukan titik potong garis AB 
\end{eulercomment}
\begin{eulerprompt}
>plotPoint(D,value=1):
\end{eulerprompt}
\eulerimg{27}{images/Pekan 11-12_Presentasi_Fanny Erina Dewi_22305141005_menggambar objek geometry_ Geometry_Aplikom-021.png}
\begin{eulercomment}
Menggambar titik potongnya

\end{eulercomment}
\eulersubheading{Menggambar Garis}
\begin{eulercomment}
Sebuah garis dipikirkan sebagai suatu himpunan titik berderet yang
panjang tak terbatas , tetapi tidak memiliki lebar. Sebuah garis
direpresentasikan dengan sebuah gambar sinar dengan mata di kedua
ujungnya yang menunjukkan bahwa garis tersebut tak berakhir.

Sebuah garis itu lurus sempurna, tidak memiliki ketebalan. Garis bisa
dimodelkan sebagai garis lurus yang tidak ada awalan dan akhirannya.

Latihan 1\\
\end{eulercomment}
\eulersubheading{}
\begin{eulercomment}
Gambarlah garis dengan persamaan 5x+2y=7!\\
Persamaan 5x+2y=7 bisa dituliskan sebagai [5,2,7]
\end{eulercomment}
\begin{eulerprompt}
>setPlotRange(0,10,0,10);
\end{eulerprompt}
\begin{eulercomment}
Lankah pertama adalah menentukan batas koordinat yaitu x1=0
,x2=10,y1=0 dan y2=10
\end{eulercomment}
\begin{eulerprompt}
>plotLine([5,2,7],"g",10):
\end{eulerprompt}
\eulerimg{27}{images/Pekan 11-12_Presentasi_Fanny Erina Dewi_22305141005_menggambar objek geometry_ Geometry_Aplikom-022.png}
\begin{eulercomment}
Menentukan garis dengan fungsi plotline(persamaan garis,"label",jarak
label)

Latihan 2\\
\end{eulercomment}
\eulersubheading{}
\begin{eulercomment}
Gambarlah garis yang melalui titik C (2,3) dan titik D(1,4)!
\end{eulercomment}
\begin{eulerprompt}
>setPlotRange(0,7,0,7);
\end{eulerprompt}
\begin{eulercomment}
Langkah pertama adalah menggambar bidang kartesius terlebih dahulu
\end{eulercomment}
\begin{eulerprompt}
>C=[2,3]; D=[1,4];
\end{eulerprompt}
\begin{eulercomment}
Menentukan titik C dan titik D
\end{eulercomment}
\begin{eulerprompt}
>g=$lineThrough(C,D)
\end{eulerprompt}
\begin{eulerformula}
\[
\left[ -\left(D-C\right)_{2} , \left(D-C\right)_{1} , \left[ -
 \left(D-C\right)_{2} , \left(D-C\right)_{1} \right] \cdot C \right] 
\]
\end{eulerformula}
\begin{eulercomment}
Latihan 3\\
\end{eulercomment}
\eulersubheading{}
\begin{eulercomment}
Tentukan vektor arah (gradien) garis g!
\end{eulercomment}
\begin{eulerprompt}
>$getLineDirection(g)
\end{eulerprompt}
\begin{eulerformula}
\[
\left[ g_{2} , -g_{1} \right] 
\]
\end{eulerformula}
\begin{eulercomment}
Latihan 4\\
\end{eulercomment}
\eulersubheading{}
\begin{eulercomment}
Menentukkan garis yang melalui titik A (3,4) dan tegak lurus terhadap
garis (g) 
\end{eulercomment}
\begin{eulerprompt}
>$h=perpendicular(A,g)
\end{eulerprompt}
\begin{eulerformula}
\[
h=\left[ g_{2} , -g_{1} , \left[ g_{2} , -g_{1} \right] \cdot A
  \right] 
\]
\end{eulerformula}
\begin{eulercomment}
Latihan 5\\
\end{eulercomment}
\eulersubheading{}
\begin{eulercomment}
Menentukan garis yang melalui titik dan sejajar terhadap garis g
\end{eulercomment}
\begin{eulerprompt}
>$m=parallel(A,g)
\end{eulerprompt}
\begin{eulerformula}
\[
m=\left[ g_{1} , g_{2} , \left[ g_{1} , g_{2} \right] \cdot A
  \right] 
\]
\end{eulerformula}
\begin{eulercomment}
Latihan 6\\
\end{eulercomment}
\eulersubheading{}
\begin{eulercomment}
Menentukan titik potong dua garis, yaitu garis g dan garis h
\end{eulercomment}
\begin{eulerprompt}
>$lineIntersection(g,h)
\end{eulerprompt}
\begin{eulerformula}
\[
\left[ \frac{g_{2}\,h_{3}-h_{2}\,g_{3}}{h_{1}\,g_{2}-g_{1}\,h_{2}}
  , \frac{h_{1}\,g_{3}-g_{1}\,h_{3}}{h_{1}\,g_{2}-g_{1}\,h_{2}}
  \right] 
\]
\end{eulerformula}
\begin{eulercomment}
Latihan 7\\
\end{eulercomment}
\eulersubheading{}
\begin{eulercomment}
Menentukan proyeksi titik A pada garis g
\end{eulercomment}
\begin{eulerprompt}
>$projectToLine(A,g)
\end{eulerprompt}
\begin{eulerformula}
\[
\left[ \frac{g_{2}\,\left[ g_{2} , -g_{1} \right] \cdot A-g_{1}\,
 \left[ -g_{1} , -g_{2} \right] \cdot A}{g_{2}^2+g_{1}^2} , \frac{-g
 _{1}\,\left[ g_{2} , -g_{1} \right] \cdot A-g_{2}\,\left[ -g_{1} , -
 g_{2} \right] \cdot A}{g_{2}^2+g_{1}^2} \right] 
\]
\end{eulerformula}
\begin{eulercomment}
Latihan 8\\
\end{eulercomment}
\eulersubheading{}
\begin{eulercomment}
Menentukkan garis dengan titik F(7,2), dan A(3,4) dan jarak label 10!
\end{eulercomment}
\begin{eulerprompt}
>setPlotRange(0,10,0,10)
\end{eulerprompt}
\begin{euleroutput}
  [0,  10,  0,  10]
\end{euleroutput}
\begin{eulercomment}
Menentukan bidang kartesius dengan batas terkecil x maupun y yaitu 0
dan batas terbesar x maupun y yaitu 10
\end{eulercomment}
\begin{eulerprompt}
>F=[7,2]; A:=[3,4];
\end{eulerprompt}
\begin{eulercomment}
Menentukkan titik F dan titik A
\end{eulercomment}
\begin{eulerprompt}
>m=middlePerpendicular(F,A);
\end{eulerprompt}
\begin{eulercomment}
Membuat garis FA denngan fungsi middlePerpendicular
\end{eulercomment}
\begin{eulerprompt}
>plotLine(m,"m",10):
\end{eulerprompt}
\eulerimg{27}{images/Pekan 11-12_Presentasi_Fanny Erina Dewi_22305141005_menggambar objek geometry_ Geometry_Aplikom-029.png}
\begin{eulercomment}
Menggambar garis dan jarak label menggunakan fungsi plotLine

Latihan 8\\
\end{eulercomment}
\eulersubheading{}
\begin{eulercomment}
Gambar garis sumbu ruas AB dengan titik A(2,2) dan B(-1,-2)!
\end{eulercomment}
\begin{eulerprompt}
>A=[2,2]; B=[-1,-2]; // menentukan titik titiknya
>c1=circleWithCenter(A,distance(A,B)); //membuat lingkaran dengnan pusat A dan jari jari r
>c2=circleWithCenter(B,distance(A,B));
>\{P1,P2,f\}=circleCircleIntersections(c1,c2);
>l=lineThrough(P1,P2);
>setPlotRange(5); plotCircle(c1); plotCircle(c2):
\end{eulerprompt}
\eulerimg{27}{images/Pekan 11-12_Presentasi_Fanny Erina Dewi_22305141005_menggambar objek geometry_ Geometry_Aplikom-030.png}
\begin{eulerprompt}
>plotPoint(A); plotPoint(B); plotSegment(A,B); plotLine(l):
\end{eulerprompt}
\eulerimg{27}{images/Pekan 11-12_Presentasi_Fanny Erina Dewi_22305141005_menggambar objek geometry_ Geometry_Aplikom-031.png}
\eulersubheading{Contoh 1}
\begin{eulercomment}
Apabila ingin menggambar garis kita dapat menggunakan rumus sebagai
berikut:

\end{eulercomment}
\begin{eulerprompt}
>n=angleBisector(A,B,C); color(7); plotLine(n):
\end{eulerprompt}
\eulerimg{27}{images/Pekan 11-12_Presentasi_Fanny Erina Dewi_22305141005_menggambar objek geometry_ Geometry_Aplikom-032.png}
\eulersubheading{Menggambar Bidang/ segibanyak}
\begin{eulercomment}
Bidang adalah permukaan rata dan tentu batasnya. Bidang didefinisikan
sebagai permukaan datar yang didefinisikan oleh serangkaian garis yang
tidak berpotongan

Dengan contoh 1 kita dapat menggambar segibanyak

\end{eulercomment}
\eulersubheading{Segilima}
\begin{eulercomment}
Langkah pertama adalah menggambar ketiga titik
\end{eulercomment}
\begin{eulerprompt}
>setPlotRange(-5,5,-5,5); O=[0,0]; plotPoint(O,"O"); A=[0,4]; plotPoint(A,"A"); B=turn(A,-72°); plotPoint(B,"B"):
\end{eulerprompt}
\eulerimg{27}{images/Pekan 11-12_Presentasi_Fanny Erina Dewi_22305141005_menggambar objek geometry_ Geometry_Aplikom-033.png}
\eulersubheading{Segienam}
\begin{eulerprompt}
>setPlotRange(-8,8,-8,8); O=[0,0]; plotPoint(O,"O"); A=[-2.3,4]; plotPoint(A,"A"); B=turn(A,-60°); plotPoint(B,"B"):
\end{eulerprompt}
\eulerimg{27}{images/Pekan 11-12_Presentasi_Fanny Erina Dewi_22305141005_menggambar objek geometry_ Geometry_Aplikom-034.png}
\eulersubheading{Segitujuh}
\begin{eulerprompt}
>setPlotRange(-5,8,-5,5); O=[0,0]; plotPoint(O,"O"); A=[0,4]; plotPoint(A,"A"); B=turn(A,-360°/7); plotPoint(B,"B"):
\end{eulerprompt}
\eulerimg{27}{images/Pekan 11-12_Presentasi_Fanny Erina Dewi_22305141005_menggambar objek geometry_ Geometry_Aplikom-035.png}
\begin{eulerprompt}
>setPlotRange(-5,5,-5,5);
>Q=[-2,4]; plotPoint(Q,"Q");
>R=[1,4]; plotPoint(R, "R");
>S=[2,2]; plotPoint(S, "S");
>T=[-3,2]; plotPoint(T, "T");
>M=[-2,0]; plotPoint(M, "M");
>N=[1,0]; plotPoint(N, "N"):
\end{eulerprompt}
\eulerimg{27}{images/Pekan 11-12_Presentasi_Fanny Erina Dewi_22305141005_menggambar objek geometry_ Geometry_Aplikom-036.png}
\begin{eulerprompt}
>color(2); plotSegment(Q,R," "); plotSegment(S,R," ");  plotSegment(S,T," "); plotSegment(Q,T," "):
\end{eulerprompt}
\eulerimg{27}{images/Pekan 11-12_Presentasi_Fanny Erina Dewi_22305141005_menggambar objek geometry_ Geometry_Aplikom-037.png}
\begin{eulercomment}
Langkah kedua adalah menggambar garis AC, BC, AC
\end{eulercomment}
\begin{eulerprompt}
>plotSegment(Q,R," ");
>plotSegment(R,S," ");
>plotSegment(S,T," ");
>plotSegment(T,Q," "):
\end{eulerprompt}
\eulerimg{27}{images/Pekan 11-12_Presentasi_Fanny Erina Dewi_22305141005_menggambar objek geometry_ Geometry_Aplikom-038.png}
\eulerheading{Trapesium Sama Kaki}
\begin{eulercomment}
\end{eulercomment}
\begin{eulerprompt}
>setPlotRange(-5,5,-5,5)
\end{eulerprompt}
\begin{euleroutput}
  [-5,  5,  -5,  5]
\end{euleroutput}
\begin{eulerprompt}
>Q=[-2,4]; plotPoint(Q,"Q")
>R=[1,4]; plotPoint(R, "R")
>S=[2,2]; plotPoint(S, "S");
>T=[-3,2]; plotPoint(T, "T"); 
>M=[-2,0]; plotPoint(M, "M");
>N=[1,0]; plotPoint(N, "N"):
\end{eulerprompt}
\eulerimg{27}{images/Pekan 11-12_Presentasi_Fanny Erina Dewi_22305141005_menggambar objek geometry_ Geometry_Aplikom-039.png}
\begin{eulerprompt}
>color(2); plotSegment(Q,R," "); plotSegment(S,R," "); plotSegment(S,T," "); plotSegment(Q,T," "):
\end{eulerprompt}
\eulerimg{27}{images/Pekan 11-12_Presentasi_Fanny Erina Dewi_22305141005_menggambar objek geometry_ Geometry_Aplikom-040.png}
\begin{eulerprompt}
>color(5); plotSegment(Q,R," "); plotSegment(R,S," ");  plotSegment(S,N," "); plotSegment(N,M," "); plotSegment(M,T," "); plotSegment(T,Q," "):
\end{eulerprompt}
\eulerimg{27}{images/Pekan 11-12_Presentasi_Fanny Erina Dewi_22305141005_menggambar objek geometry_ Geometry_Aplikom-041.png}
\begin{eulercomment}
.....
\end{eulercomment}
\begin{eulerprompt}
>setPlotRange(-5,17,-5,17); //mendefinisikan bidang koordinat
>A=[-3,5]; B=[-3,15]; C=[5,5]; D=[5,15]; //mendefinisikan dan menggambar 4 titik
>plotPoint(A,"A"); plotPoint(B,"B"); plotPoint(C,"C"); plotPoint(D,"D");
\end{eulerprompt}
\begin{eulercomment}
Menggambar 4 titik
\end{eulercomment}
\begin{eulerprompt}
>plotSegment(A,B,"a"); plotSegment(B,D,"b"); plotSegment(C,D,"c"); plotSegment(A,C,"d"):
\end{eulerprompt}
\eulerimg{27}{images/Pekan 11-12_Presentasi_Fanny Erina Dewi_22305141005_menggambar objek geometry_ Geometry_Aplikom-042.png}
\begin{eulercomment}
Membuat ulang garis A,B dan C

Latihan 2\\
\end{eulercomment}
\eulersubheading{}
\begin{eulercomment}
Menggambarkan segitiga siku siku dengan titik D(2,4), E(2,8), F(6,4)!
\end{eulercomment}
\begin{eulerprompt}
>setPlotRange(-5,10,-5,10);
\end{eulerprompt}
\begin{eulercomment}
Menentukan gambar bidang kartesius
\end{eulercomment}
\begin{eulerprompt}
>D=[2,4]; E=[2,8]; F=[6,4];
\end{eulerprompt}
\begin{eulercomment}
Menentukan 3 titik D,E dan F
\end{eulercomment}
\begin{eulerprompt}
>plotPoint(D,"D"); plotPoint(E,"E"); plotPoint(F,"F"):
\end{eulerprompt}
\eulerimg{27}{images/Pekan 11-12_Presentasi_Fanny Erina Dewi_22305141005_menggambar objek geometry_ Geometry_Aplikom-043.png}
\begin{eulercomment}
Menggambar 3 titik
\end{eulercomment}
\begin{eulerprompt}
>plotSegment(D,E,"DE",10); plotSegment(E,F,"EF",10); plotSegment(F,A,"FA",10):
\end{eulerprompt}
\eulerimg{27}{images/Pekan 11-12_Presentasi_Fanny Erina Dewi_22305141005_menggambar objek geometry_ Geometry_Aplikom-044.png}
\begin{eulercomment}
membuat 3 garis yaitu DE,EF,FA dengan jarak label 10
\end{eulercomment}
\begin{eulercomment}
Latihan 3\\
\end{eulercomment}
\eulersubheading{}
\begin{eulercomment}
Gambar dan tebaklah suatu bidang dengan titik A(-2,-5),B(2,-5),C(2,-1)
dan D(-2,-1)
\end{eulercomment}
\begin{eulerprompt}
>setPlotRange(-5,10,-5,10); // membuat bidang kartesius
>A=[-2,-5]; B=[2,-5]; C=[2,-1]; D=[-2,-1]; // menentukan titik
>plotPoint(A,"A"); plotPoint(B,"B"); plotPoint(C,"C"); plotPoint(D,"D"): // menggambar titik
\end{eulerprompt}
\eulerimg{27}{images/Pekan 11-12_Presentasi_Fanny Erina Dewi_22305141005_menggambar objek geometry_ Geometry_Aplikom-045.png}
\begin{eulerprompt}
>plotSegment(A,B,"a",10); plotSegment(B,C,"b",10); plotSegment(C,D,"c"); plotSegment(D,A,"d"): // menggambar garis
\end{eulerprompt}
\eulerimg{27}{images/Pekan 11-12_Presentasi_Fanny Erina Dewi_22305141005_menggambar objek geometry_ Geometry_Aplikom-046.png}
\begin{eulercomment}
Latihan 4\\
\end{eulercomment}
\eulersubheading{}
\begin{eulercomment}
Gambarlah objek geometri segitiga dengan 3 titik yaitu A (1,0), B(0,1)
dan C (2,2)!

\end{eulercomment}
\begin{eulerprompt}
>setPlotRange(-0.5,2.5,-0.5,2.5); // mendefinisikan bidang koordinat baru
>A=[1,0]; plotPoint(A,"A"); // definisi dan gambar tiga titik
>B=[0,1]; plotPoint(B,"B");
>C=[2,2]; plotPoint(C,"C");
>plotSegment(A,B,"c"); // c=AB
>plotSegment(B,C,"a"); // a=BC
>plotSegment(A,C,"b"); // b=AC
>lineThrough(B,C) // garis yang melalui B dan C
\end{eulerprompt}
\begin{euleroutput}
  [-1,  2,  2]
\end{euleroutput}
\begin{eulerprompt}
>h=perpendicular(A,lineThrough(B,C)); // garis h tegak lurus BC melalui A
>D=lineIntersection(h,lineThrough(B,C)); // D adalah titik potong h dan BC
>plotPoint(D,value=1); // koordinat D ditampilkan
>aspect(1); plotSegment(A,D): // tampilkan semua gambar hasil plot...()
\end{eulerprompt}
\eulerimg{27}{images/Pekan 11-12_Presentasi_Fanny Erina Dewi_22305141005_menggambar objek geometry_ Geometry_Aplikom-047.png}
\begin{eulerprompt}
>l=angleBisector(A,C,B): // garis bagi <ACB
\end{eulerprompt}
\eulerimg{27}{images/Pekan 11-12_Presentasi_Fanny Erina Dewi_22305141005_menggambar objek geometry_ Geometry_Aplikom-048.png}
\begin{eulerprompt}
>g=angleBisector(C,A,B): // garis bagi <CAB
\end{eulerprompt}
\eulerimg{27}{images/Pekan 11-12_Presentasi_Fanny Erina Dewi_22305141005_menggambar objek geometry_ Geometry_Aplikom-049.png}
\begin{eulerprompt}
>P=lineIntersection(l,g) // titik potong kedua garis bagi sudut
\end{eulerprompt}
\begin{euleroutput}
  [0.86038,  0.86038]
\end{euleroutput}
\eulersubheading{Segitiga Siku-siku}
\begin{eulercomment}
Langkah pertama dalam menggambarkan objek objek geometri yaitu membuat
perintah untuk geometry dengan "load geometry".
\end{eulercomment}
\begin{eulerprompt}
>load geometry
\end{eulerprompt}
\begin{euleroutput}
  Numerical and symbolic geometry.
\end{euleroutput}
\begin{eulercomment}
Langkah kedua kita akan menentukan rentang sumbu terlebih dahulu
\end{eulercomment}
\begin{eulerprompt}
>setPlotRange(-5,8,-5,8):
\end{eulerprompt}
\eulerimg{27}{images/Pekan 11-12_Presentasi_Fanny Erina Dewi_22305141005_menggambar objek geometry_ Geometry_Aplikom-050.png}
\begin{eulercomment}
Langkah ketiga adalah menentukan ketiga titik bidang koordinat
\end{eulercomment}
\begin{eulerprompt}
>A=[6,0]; plotPoint(A,"A"):
\end{eulerprompt}
\eulerimg{27}{images/Pekan 11-12_Presentasi_Fanny Erina Dewi_22305141005_menggambar objek geometry_ Geometry_Aplikom-051.png}
\begin{eulerprompt}
>B=[2,0]; plotPoint(B,"B"):
\end{eulerprompt}
\eulerimg{27}{images/Pekan 11-12_Presentasi_Fanny Erina Dewi_22305141005_menggambar objek geometry_ Geometry_Aplikom-052.png}
\begin{eulerprompt}
>C=[2,6]; plotPoint(C,"C"):
\end{eulerprompt}
\eulerimg{27}{images/Pekan 11-12_Presentasi_Fanny Erina Dewi_22305141005_menggambar objek geometry_ Geometry_Aplikom-053.png}
\begin{eulercomment}
Langkah keempat menggambar ruas garis di ketiga titik
\end{eulercomment}
\begin{eulerprompt}
>plotSegment(A,B,"c"):
\end{eulerprompt}
\eulerimg{27}{images/Pekan 11-12_Presentasi_Fanny Erina Dewi_22305141005_menggambar objek geometry_ Geometry_Aplikom-054.png}
\begin{eulerprompt}
>plotSegment(B,C,"a"):
\end{eulerprompt}
\eulerimg{27}{images/Pekan 11-12_Presentasi_Fanny Erina Dewi_22305141005_menggambar objek geometry_ Geometry_Aplikom-055.png}
\begin{eulerprompt}
>plotSegment(A,C,"b"):
\end{eulerprompt}
\eulerimg{27}{images/Pekan 11-12_Presentasi_Fanny Erina Dewi_22305141005_menggambar objek geometry_ Geometry_Aplikom-056.png}
\begin{eulercomment}
Langkah kelima menghitung luas segitiga siku siku
\end{eulercomment}
\begin{eulerprompt}
>areaTriangle(A,B,C)
\end{eulerprompt}
\begin{euleroutput}
  12
\end{euleroutput}
\begin{eulercomment}
membuktikan menggunakan perintah manual:
\end{eulercomment}
\begin{eulerprompt}
>norm(A-B) // menghitung panjaang alas dari panjang garis AB 
\end{eulerprompt}
\begin{euleroutput}
  4
\end{euleroutput}
\begin{eulerprompt}
>norm(B-C) // menghitung panjang tinggi dari panjang garis BC
\end{eulerprompt}
\begin{euleroutput}
  6
\end{euleroutput}
\begin{eulerprompt}
>norm(A-B)*norm(B-C)/2
\end{eulerprompt}
\begin{euleroutput}
  12
\end{euleroutput}
\begin{eulercomment}
Hasilnya sesuai dengan perintah menggunakan fungi geometri yang
bernilai 12 satuan
\end{eulercomment}
\eulersubheading{Menggambar Lingkaran}
\begin{eulercomment}
Lingkaran adalah tempat kedudukan/himpunan titik-titik yang berjarak
sama terhadap suatu titik tertentu. Jarak yang sama disebut panjang
jari-jari linngkaran dan titik tertentu disebut pusat lingkaran.

Contoh 1:\\
Melanjutkan dari luas segitiga siku siku Gambar suatu lingkaran yang
mengelilingi segitiga siku siku ABC.
\end{eulercomment}
\begin{eulerprompt}
>m=circleThrough(A,B,C);
>R=getCircleRadius(m):
\end{eulerprompt}
\eulerimg{27}{images/Pekan 11-12_Presentasi_Fanny Erina Dewi_22305141005_menggambar objek geometry_ Geometry_Aplikom-057.png}
\begin{eulerprompt}
>O=getCircleCenter(m):
\end{eulerprompt}
\eulerimg{27}{images/Pekan 11-12_Presentasi_Fanny Erina Dewi_22305141005_menggambar objek geometry_ Geometry_Aplikom-058.png}
\begin{eulerprompt}
>plotPoint(O,"O"):
\end{eulerprompt}
\eulerimg{27}{images/Pekan 11-12_Presentasi_Fanny Erina Dewi_22305141005_menggambar objek geometry_ Geometry_Aplikom-059.png}
\begin{eulerprompt}
> plotCircle(1,"Lingkaran Luar m"):
\end{eulerprompt}
\begin{euleroutput}
  Index 3 out of bounds!
  Try "trace errors" to inspect local variables after errors.
  plotCircle:
      t=linspace(0,2pi,100*floor(max(1,c[3])));
\end{euleroutput}
\begin{eulercomment}
Langkah kedua yaitu menentukan jari jari lingkaran
\end{eulercomment}
\begin{eulerprompt}
>R
\end{eulerprompt}
\begin{euleroutput}
  3.60555127546
\end{euleroutput}
\begin{eulercomment}
Langkah ketiga yaitumenghitung luas lingkaran luar m
\end{eulercomment}
\begin{eulerprompt}
>pi*(3.605551275462)^2
\end{eulerprompt}
\begin{euleroutput}
  40.8407044966
\end{euleroutput}
\begin{eulerformula}
\[
L_{m}= \pi \times r^2
\]
\end{eulerformula}
\begin{eulerformula}
\[
L_{m}= \pi \times 3.60555127546^2
\]
\end{eulerformula}
\begin{eulerformula}
\[
L_{m}= 40.8407044966
\]
\end{eulerformula}
\begin{eulercomment}
Jadi luas lingkaran luar m adalah  40.8407044966 satuan

Latihan 1\\
\end{eulercomment}
\eulersubheading{}
\begin{eulerprompt}
>setPlotRange(20);
>P=[5,5]; r=[14];
>plotCircle(circleWithCenter(P,r),"Lingkaran"); // gambar lingkaran
\end{eulerprompt}
\begin{eulercomment}
Cari titik pusat lingkaran dan jari jari lingkaran
\end{eulercomment}
\begin{eulerprompt}
>setPlotRange(20);
>P=[1,3]; plotPoint(P,"P");
>L=[-1,-3]; plotPoint(L,"L"):
\end{eulerprompt}
\eulerimg{27}{images/Pekan 11-12_Presentasi_Fanny Erina Dewi_22305141005_menggambar objek geometry_ Geometry_Aplikom-060.png}
\begin{eulerprompt}
>c1=circleWithCenter(P,distance(P,L)); plotCircle(c1);
>plotSegment(P,L," "):
\end{eulerprompt}
\eulerimg{27}{images/Pekan 11-12_Presentasi_Fanny Erina Dewi_22305141005_menggambar objek geometry_ Geometry_Aplikom-061.png}
\begin{eulercomment}
Latihan 2\\
\end{eulercomment}
\eulersubheading{}
\begin{eulercomment}
Buatlah lingkaran yang melalui tiga titik yaitu titik A (0,2), B(4,2)
dan C(2,4)!
\end{eulercomment}
\begin{eulerprompt}
>setPlotRange(-1,6,-1,6); // menentukan rentang x dan y pada bidang koordinat
>A=[0,2]; plotPoint(A,"A"); 
>B=[4,2]; plotPoint(B,"B");
>C=[2,4]; plotPoint(C,"C");
>c=circleThrough(A,B,C); // luar lingkaran ABC
>o=getCircleCenter(c); // menentukan titik pusat lingkaran
>plotCircle(c, "luar lingkaran ABC"):
\end{eulerprompt}
\eulerimg{27}{images/Pekan 11-12_Presentasi_Fanny Erina Dewi_22305141005_menggambar objek geometry_ Geometry_Aplikom-062.png}
\begin{eulercomment}
Latihan 3\\
\end{eulercomment}
\eulersubheading{}
\begin{eulercomment}
Gambarlah lingkaran yang melalui titik A (0,2), B (-2,-1), C (0,3) dan
tentukan titik pusatnya!
\end{eulercomment}
\begin{eulerprompt}
>setPlotRange(-10,8,-10,8):
\end{eulerprompt}
\eulerimg{27}{images/Pekan 11-12_Presentasi_Fanny Erina Dewi_22305141005_menggambar objek geometry_ Geometry_Aplikom-063.png}
\begin{eulercomment}
Langkah pertama yaitu menentukan batas tiap sumbu bidang koordinatnya
\end{eulercomment}
\begin{eulerprompt}
>A = [0,2]; B = [-2,-1]; C = [0,3];
\end{eulerprompt}
\begin{eulercomment}
Langkah kedua menentukan titiknya
\end{eulercomment}
\begin{eulerprompt}
>plotPoint(A,"A"); plotPoint(B,"B"); plotPoint(C,"C")
\end{eulerprompt}
\begin{eulercomment}
Langkah ketiga yaitu menggambar ketiga titiknya
\end{eulercomment}
\begin{eulerprompt}
>c = circleThrough(A,B,C);
\end{eulerprompt}
\begin{eulercomment}
Langkah keempat merupakan fungsi lingkaran pada titik A,B dan C
\end{eulercomment}
\begin{eulerprompt}
>P = getCircleCenter(c);
\end{eulerprompt}
\begin{eulercomment}
Langkah kelima adalah menentukan titik pusat lingkarannya, caranya
dengan fungsi getCircleCenter(A,B,C)
\end{eulercomment}
\begin{eulerprompt}
>plotPoint(P,"P"); 
\end{eulerprompt}
\begin{eulercomment}
Langkah keenam yaitu menggambar titik pusatnya
\end{eulercomment}
\begin{eulerprompt}
>plotCircle(c,"Lingkaran c"):
\end{eulerprompt}
\eulerimg{27}{images/Pekan 11-12_Presentasi_Fanny Erina Dewi_22305141005_menggambar objek geometry_ Geometry_Aplikom-064.png}
\begin{eulercomment}
Langkah terakhir yaitu memanggil lingkaran dan memberi label
"lingkaran c"

Latihan 4\\
\end{eulercomment}
\eulersubheading{}
\begin{eulerprompt}
>setPlotRange(-5,5,-5,5); 
\end{eulerprompt}
\begin{eulercomment}
Membuat bidang kartesiusnya
\end{eulercomment}
\begin{eulerprompt}
>A=[0,2]; plotPoint(A,"A"): //mendefinisikan titik dan gambarnya
\end{eulerprompt}
\eulerimg{27}{images/Pekan 11-12_Presentasi_Fanny Erina Dewi_22305141005_menggambar objek geometry_ Geometry_Aplikom-065.png}
\begin{eulerprompt}
>B=[4,2]; plotPoint(B,"B"): 
\end{eulerprompt}
\eulerimg{27}{images/Pekan 11-12_Presentasi_Fanny Erina Dewi_22305141005_menggambar objek geometry_ Geometry_Aplikom-066.png}
\begin{eulerprompt}
>C=[2,4]; plotPoint(C,"C"):
\end{eulerprompt}
\eulerimg{27}{images/Pekan 11-12_Presentasi_Fanny Erina Dewi_22305141005_menggambar objek geometry_ Geometry_Aplikom-067.png}
\begin{eulerprompt}
>c=circleThrough(A,B,C): //lingkaran melalui 3 titik
\end{eulerprompt}
\eulerimg{27}{images/Pekan 11-12_Presentasi_Fanny Erina Dewi_22305141005_menggambar objek geometry_ Geometry_Aplikom-068.png}
\begin{eulerprompt}
>o=getCircleCenter(c): // menentukan pusat lingkaran c
\end{eulerprompt}
\eulerimg{27}{images/Pekan 11-12_Presentasi_Fanny Erina Dewi_22305141005_menggambar objek geometry_ Geometry_Aplikom-069.png}
\begin{eulerprompt}
>plotCircle(c): //Menggambar lingkaran c dengan label nama "c"
\end{eulerprompt}
\eulerimg{27}{images/Pekan 11-12_Presentasi_Fanny Erina Dewi_22305141005_menggambar objek geometry_ Geometry_Aplikom-070.png}
\begin{eulerprompt}
>o
\end{eulerprompt}
\begin{euleroutput}
  [2,  2]
\end{euleroutput}
\begin{eulercomment}
Latihan 5\\
\end{eulercomment}
\eulersubheading{}
\begin{eulercomment}
Gambarlah lingkaran jika diketahui 3 titik yang membentuk segitiga
\end{eulercomment}
\begin{eulerprompt}
>setPlotRange(5)
\end{eulerprompt}
\begin{euleroutput}
  [-5,  5,  -5,  5]
\end{euleroutput}
\begin{eulerprompt}
>A::=[-1,-1]; B::=[2,0]; C::=[1,2];
>plotPoint(A,"A"); plotPoint(B,"B"); plotPoint(C,"C");
>plotSegment(A,B,""); plotSegment(B,C,""); plotSegment(C,A,""):
\end{eulerprompt}
\eulerimg{27}{images/Pekan 11-12_Presentasi_Fanny Erina Dewi_22305141005_menggambar objek geometry_ Geometry_Aplikom-071.png}
\begin{eulerprompt}
>LL &= circleThrough(A,B,C); $getCircleEquation(LL,x,y)
\end{eulerprompt}
\begin{eulerformula}
\[
\left(y-\frac{5}{14}\right)^2+\left(x-\frac{3}{14}\right)^2=\frac{
 325}{98}
\]
\end{eulerformula}
\begin{eulerprompt}
>O &= getCircleCenter(LL); $O
\end{eulerprompt}
\begin{eulerformula}
\[
\left[ \frac{3}{14} , \frac{5}{14} \right] 
\]
\end{eulerformula}
\begin{eulerprompt}
>plotCircle(LL()); plotPoint(O(),"O"):
\end{eulerprompt}
\eulerimg{27}{images/Pekan 11-12_Presentasi_Fanny Erina Dewi_22305141005_menggambar objek geometry_ Geometry_Aplikom-074.png}
\begin{eulercomment}
Latihan 6\\
\end{eulercomment}
\eulersubheading{}
\begin{eulercomment}
Menggambar lingkaran menggunakan plot 2D dengan menggunakan rumus umum
lingkaran\\
\end{eulercomment}
\begin{eulerformula}
\[
x^2+y^2=r^2
\]
\end{eulerformula}
\begin{eulerprompt}
>aspect(1);
>plot2d("x^2+y^2-4",r=3,level=1):
\end{eulerprompt}
\eulerimg{27}{images/Pekan 11-12_Presentasi_Fanny Erina Dewi_22305141005_menggambar objek geometry_ Geometry_Aplikom-075.png}
\begin{eulerprompt}
>aspect(2);
>plot2d("((x^2)/4)+((y^2)/4)-1",r=10,level=4,contourcolor=red):
\end{eulerprompt}
\eulerimg{13}{images/Pekan 11-12_Presentasi_Fanny Erina Dewi_22305141005_menggambar objek geometry_ Geometry_Aplikom-076.png}
\eulersubheading{Menggambar Parabola}
\begin{eulercomment}
Parabola adalah garis lengkung datar yang berbentuk jika suatu bidang
memotong kerucut sejajar dengan garis titik sudut puncak dengan salah
satu titik pada keliling alas. Contohnya seperti antena tv berbentuk
bundar seperti piring cekung yang dapat menangkap siaran jarak jauh.

Contoh Parabola\\
\end{eulercomment}
\begin{eulerformula}
\[
x^2-2x-2y+5
\]
\end{eulerformula}
\begin{eulercomment}
Gambar Parabolanya
\end{eulercomment}
\begin{eulerprompt}
>aspect(1);
>plot2d("x^2-2x-2y+5"): //parabola
\end{eulerprompt}
\eulerimg{27}{images/Pekan 11-12_Presentasi_Fanny Erina Dewi_22305141005_menggambar objek geometry_ Geometry_Aplikom-077.png}
\begin{eulercomment}
Latihan 1\\
\end{eulercomment}
\eulersubheading{}
\begin{eulercomment}
Misalkan persamaan parabolanya\\
\end{eulercomment}
\begin{eulerformula}
\[
y=ax^2+bx+c
\]
\end{eulerformula}
\begin{eulercomment}
Gambarlah bentuk parabolanya!
\end{eulercomment}
\begin{eulerprompt}
>function f(x) &=a*x^2+b*x+c;
>setPlotRange(-5,6,-11,2);
>A=[-2,0]; plotPoint(A,"A");
>B=[5,0]; plotPoint(B,"B");
>C=[2,0]; plotPoint(C,"C");
>&powerdisp:true;
>&f(-2)=0
\end{eulerprompt}
\begin{euleroutput}
  
                            4 a - 2 b + c = 0
  
\end{euleroutput}
\begin{eulerprompt}
>aspect(1)
>plot2d("4*x^2-2*x+1"): //parabola
\end{eulerprompt}
\eulerimg{27}{images/Pekan 11-12_Presentasi_Fanny Erina Dewi_22305141005_menggambar objek geometry_ Geometry_Aplikom-078.png}
\begin{eulercomment}
Latihan 2\\
\end{eulercomment}
\eulersubheading{}
\begin{eulerprompt}
>setPlotRange(3)
\end{eulerprompt}
\begin{euleroutput}
  [-3,  3,  -3,  3]
\end{euleroutput}
\begin{eulerprompt}
>p &= getHesseForm(lineThrough(A,B),x,y,C)-distance([x,y],C); $p='0
\end{eulerprompt}
\begin{eulerformula}
\[
-\sqrt{\left(1-x\right)^2+\left(2-y\right)^2}+\frac{2-x+3\,y}{
 \sqrt{10}}=0
\]
\end{eulerformula}
\begin{eulerprompt}
>plot2d(p,level=0,add=1,contourcolor=2):
\end{eulerprompt}
\eulerimg{27}{images/Pekan 11-12_Presentasi_Fanny Erina Dewi_22305141005_menggambar objek geometry_ Geometry_Aplikom-080.png}
\begin{eulercomment}
Latihan 3\\
\end{eulercomment}
\eulersubheading{}
\begin{eulerformula}
\[
f(x)=x^2+5x+6
\]
\end{eulerformula}
\begin{eulerprompt}
>function f(x) := x^2+5x+6
>aspect(2), plot2d("f(x)",-10,5):
\end{eulerprompt}
\eulerimg{13}{images/Pekan 11-12_Presentasi_Fanny Erina Dewi_22305141005_menggambar objek geometry_ Geometry_Aplikom-081.png}
\begin{eulercomment}
Latihan 4\\
\end{eulercomment}
\eulersubheading{}
\begin{eulerprompt}
>aspect(5);
>plot2d("4x^2-9y^2"):
\end{eulerprompt}
\eulerimg{4}{images/Pekan 11-12_Presentasi_Fanny Erina Dewi_22305141005_menggambar objek geometry_ Geometry_Aplikom-082.png}
\eulerheading{Hiperbola}
\begin{eulerprompt}
>aspect(1);
>plot3d("4x^2-9y^2-36"): //hiperbola
\end{eulerprompt}
\eulerimg{27}{images/Pekan 11-12_Presentasi_Fanny Erina Dewi_22305141005_menggambar objek geometry_ Geometry_Aplikom-083.png}
\begin{eulerprompt}
>aspect(3);
>plot3d("4*x^2-2*x+1"):
\end{eulerprompt}
\eulerimg{8}{images/Pekan 11-12_Presentasi_Fanny Erina Dewi_22305141005_menggambar objek geometry_ Geometry_Aplikom-084.png}
\begin{eulercomment}
\end{eulercomment}
\eulersubheading{Latihan}
\begin{eulercomment}
1. Gambarkan lingkaran luar dan dalam segitiga ABC dengan A(-7,-2),
B(7,-4),C(11,8).
\end{eulercomment}
\begin{eulerprompt}
>setPlotRange(15);
>A=[-7,-2]; plotPoint (A,"A"); // definisi dan menggambar ketiga titik
>B=[7,-6]; plotPoint (B,"B"); 
>C=[11,8]; plotPoint (C,"C");
>plotSegment (A,B,"c"); // c=AB
>plotSegment (B,C,"a"); // a=BC
>plotSegment (C,A,"b"); // b=CA
>m=circleThrough(A,B,C); // lingkaran luar segitiga
>P=getCircleCenter(m); // titik pusat lingkaran m
>R=getCircleRadius(m); // jari-jari lingkaran m
>plotPoint(P,"P"); // gambar titik "P"
>plotCircle(m,"Lingkaran luar segitiga ABC"):
\end{eulerprompt}
\eulerimg{8}{images/Pekan 11-12_Presentasi_Fanny Erina Dewi_22305141005_menggambar objek geometry_ Geometry_Aplikom-085.png}
\begin{eulerprompt}
>l=angleBisector(A,C,B); // garis bagi <ACB
>g=angleBisector(C,A,B); // garis bagi <CAB
>P=lineIntersection(l,g) // titik potong kedua garis bagi sudut
\end{eulerprompt}
\begin{euleroutput}
  [4.07107,  -0.727922]
\end{euleroutput}
\begin{eulerprompt}
>color(5); plotLine(l); plotLine(g); color(1); // gambar kedua garis bagi sudut
>plotPoint(P,"P"); // gambar titik potongnya
>r=norm(P-projectToLine(P,lineThrough(A,B))) // jari-jari lingkaran dalam
\end{eulerprompt}
\begin{euleroutput}
  4.26458963757
\end{euleroutput}
\begin{eulerprompt}
>plotCircle(circleWithCenter(P,r),"Lingkaran dalam segitiga ABC"): // gambar lingkaran dalam
\end{eulerprompt}
\eulerimg{8}{images/Pekan 11-12_Presentasi_Fanny Erina Dewi_22305141005_menggambar objek geometry_ Geometry_Aplikom-086.png}
\begin{eulercomment}
2. Diberikan ruas garis s melalui dua titik A dan titik B. Dengan
koordinat titik A dan titik B berturut-turut (0,1) dan (7,1). Tentukan
koordinat titik tengah ruas garis s!
\end{eulercomment}
\begin{eulerprompt}
>setPlotRange(10); //menggambar bidang kartesius
>A=[0,1]; plotPoint(A,"A"): // membuat dan menggambar titik A dan B
\end{eulerprompt}
\eulerimg{8}{images/Pekan 11-12_Presentasi_Fanny Erina Dewi_22305141005_menggambar objek geometry_ Geometry_Aplikom-087.png}
\begin{eulerprompt}
>B=[7,1]; plotPoint(B,"B"):
\end{eulerprompt}
\eulerimg{8}{images/Pekan 11-12_Presentasi_Fanny Erina Dewi_22305141005_menggambar objek geometry_ Geometry_Aplikom-088.png}
\begin{eulerprompt}
>plotSegment(A,B,"s"): // menggambar garis dengan label s
\end{eulerprompt}
\eulerimg{8}{images/Pekan 11-12_Presentasi_Fanny Erina Dewi_22305141005_menggambar objek geometry_ Geometry_Aplikom-089.png}
\begin{eulerprompt}
>h= middlePerpendicular(A,B): //titik tengah AB
\end{eulerprompt}
\eulerimg{8}{images/Pekan 11-12_Presentasi_Fanny Erina Dewi_22305141005_menggambar objek geometry_ Geometry_Aplikom-090.png}
\begin{eulerprompt}
>plotLine(h): //garis melalui titik tengah 
\end{eulerprompt}
\eulerimg{8}{images/Pekan 11-12_Presentasi_Fanny Erina Dewi_22305141005_menggambar objek geometry_ Geometry_Aplikom-091.png}
\begin{eulerprompt}
>D=lineIntersection(h,lineThrough(A,B)): // titik potong garis h dan garis yang melalui AB
\end{eulerprompt}
\eulerimg{8}{images/Pekan 11-12_Presentasi_Fanny Erina Dewi_22305141005_menggambar objek geometry_ Geometry_Aplikom-092.png}
\begin{eulerprompt}
>plotPoint(D,value=1):
\end{eulerprompt}
\eulerimg{8}{images/Pekan 11-12_Presentasi_Fanny Erina Dewi_22305141005_menggambar objek geometry_ Geometry_Aplikom-093.png}
\begin{eulercomment}
3. Diberikan dua garis yang melewati titik koordinat sebagai berikut:\\
garis AB: titik A(-4,4) dan titik B(4,0)\\
garis CD: titik C(0,0) dan titik D(0,2)\\
tentukan koordinat titik potong kedua garis tersebut!
\end{eulercomment}
\begin{eulerprompt}
>setPlotRange(5); // menggambar bidang kartesius
>A=[-4,4]; plotPoint(A,"A"): // membuat dan menggambar 3 titik
\end{eulerprompt}
\eulerimg{8}{images/Pekan 11-12_Presentasi_Fanny Erina Dewi_22305141005_menggambar objek geometry_ Geometry_Aplikom-094.png}
\begin{eulerprompt}
>B=[4,0]; plotPoint(B,"B"):
\end{eulerprompt}
\eulerimg{8}{images/Pekan 11-12_Presentasi_Fanny Erina Dewi_22305141005_menggambar objek geometry_ Geometry_Aplikom-095.png}
\begin{eulerprompt}
>C=[0,0]; plotPoint(C,"C"):
\end{eulerprompt}
\eulerimg{8}{images/Pekan 11-12_Presentasi_Fanny Erina Dewi_22305141005_menggambar objek geometry_ Geometry_Aplikom-096.png}
\begin{eulerprompt}
>D=[0,2]; plotPoint(D,"D"):
\end{eulerprompt}
\eulerimg{8}{images/Pekan 11-12_Presentasi_Fanny Erina Dewi_22305141005_menggambar objek geometry_ Geometry_Aplikom-097.png}
\begin{eulerprompt}
>plotSegment(A,B,"c"): // c=AB menggambar garis c dan b
\end{eulerprompt}
\eulerimg{8}{images/Pekan 11-12_Presentasi_Fanny Erina Dewi_22305141005_menggambar objek geometry_ Geometry_Aplikom-098.png}
\begin{eulerprompt}
>plotSegment(C,D,"b"): // b=CD
\end{eulerprompt}
\eulerimg{8}{images/Pekan 11-12_Presentasi_Fanny Erina Dewi_22305141005_menggambar objek geometry_ Geometry_Aplikom-099.png}
\begin{eulerprompt}
>lineThrough(A,B): // garis yang melalui titik A dan titik B
\end{eulerprompt}
\eulerimg{8}{images/Pekan 11-12_Presentasi_Fanny Erina Dewi_22305141005_menggambar objek geometry_ Geometry_Aplikom-100.png}
\begin{eulerprompt}
>lineThrough(C,D): // garis yang melalui titik C dan titik D
\end{eulerprompt}
\eulerimg{8}{images/Pekan 11-12_Presentasi_Fanny Erina Dewi_22305141005_menggambar objek geometry_ Geometry_Aplikom-101.png}
\begin{eulerprompt}
>E=lineIntersection(lineThrough(A,B),lineThrough(C,D));
>plotPoint(E,value=1):
\end{eulerprompt}
\eulerimg{8}{images/Pekan 11-12_Presentasi_Fanny Erina Dewi_22305141005_menggambar objek geometry_ Geometry_Aplikom-102.png}
\begin{eulercomment}
4. Gambarlah suatu parabola yang melalui 3 titik yang diketahui

Petunjuk:\\
- Misalkan persamaan bolanya y=ax\textasciicircum{}2+bx+c\\
- Substitusikan koordinat titik titik yang diketahui ke persamaan
berikut\\
- Selesaikan SPL yang terbentuk untuk mendapatkan nilai nilai a,b,c
\end{eulercomment}
\begin{eulerprompt}
>function f(x) &= a*x^2+b*x+c;
>setPlotRange(-5,6,-11,2);
>A=[-2,0]; plotPoint(A,"A");
>B=[5,0]; plotPoint(B,"B");
>C=[0,-10]; plotPoint(C,"C");
>&powerdisp:true;
>&f(-2)=0
\end{eulerprompt}
\begin{euleroutput}
  
                            4 a - 2 b + c = 0
  
\end{euleroutput}
\begin{eulerprompt}
>&f(5)=0
\end{eulerprompt}
\begin{euleroutput}
  
                            25 a + 5 b + c = 0
  
\end{euleroutput}
\begin{eulerprompt}
>&f(0)=-10
\end{eulerprompt}
\begin{euleroutput}
  
                                 c = - 10
  
\end{euleroutput}
\begin{eulerprompt}
>&solve([4*a-2*b-10=0,25*a+5*b-10=0],[a,b])
\end{eulerprompt}
\begin{euleroutput}
  
                            [[a = 1, b = - 3]]
  
\end{euleroutput}
\begin{eulerprompt}
>plot2d("x^2-3*x-10",r=15,xmax=6,xmin=-3); plot2d(-2,0,>add,>points); plot2d(5,0,>add,>points); plot2d(1.5,-12):
\end{eulerprompt}
\eulerimg{8}{images/Pekan 11-12_Presentasi_Fanny Erina Dewi_22305141005_menggambar objek geometry_ Geometry_Aplikom-103.png}
\begin{eulercomment}
5. Gambarlah suatu ellips jika diketahui kedua titik fokusnya,
misalkan P dan Q. Ingat ellips fokus P dan Q adalah tempat kedudukan
titik-titik yang jumlah jarak ke P dan ke Q selalu sama (konstan).\\
elesaikan SPL yang terbentuk untuk mendapatkan nilai nilai a,b,c
\end{eulercomment}
\begin{eulerprompt}
>P= [-1,0];
>Q= [1,0];
>distance(P,Q)
\end{eulerprompt}
\begin{euleroutput}
  2
\end{euleroutput}
\begin{eulerprompt}
>t=linspace(0,2pi,1000); plot2d(1*distance(P,Q)*cos(t),sin(t),r=3);
>plot2d(-1,0,>add,>points); plot2d(1,0,>add,>points):
\end{eulerprompt}
\eulerimg{8}{images/Pekan 11-12_Presentasi_Fanny Erina Dewi_22305141005_menggambar objek geometry_ Geometry_Aplikom-104.png}
\begin{eulercomment}
6. Gambarlah suatu hiperbola jika diketahui kedua titik fokusnya,
misalnya P dan Q. Ingat ellips dengan\\
fokus P dan Q adalah tempat kedudukan titik-titik yang selisih jarak
ke P dan ke Q selalu sama (konstan). 
\end{eulercomment}
\begin{eulerprompt}
>P=[-2,0];
>Q=[2,0];
>t=linspace(-5,5,1000); plot2d(cosh(t),sinh(t),r=3); plot2d(-cosh(t),-sinh(t),>add):
\end{eulerprompt}
\eulerimg{8}{images/Pekan 11-12_Presentasi_Fanny Erina Dewi_22305141005_menggambar objek geometry_ Geometry_Aplikom-105.png}
\begin{eulerprompt}
> 
\end{eulerprompt}
\end{eulernotebook}
\end{document}

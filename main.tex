\documentclass[12pt,arial,letterpaper]{book}
\linespread{1.5}
\usepackage{eumat}
\usepackage{graphicx}
\graphicspath{{images/}}
\usepackage{titlesec}
\titleformat{\chapter}[display]
    {\huge\bfseries\centering}
    {\chaptertitlename\ \thechapter}{20pt}{\huge}
\renewcommand{\chaptername}{BAB}
\pagestyle{plain}
\renewcommand{\contentsname}{Daftar Isi}

\begin{document}
\clearpage
\thispagestyle{empty}
\frontmatter
\begin{center}
    \huge{\textbf{Tugas Project \\ Aplikasi Komputer}}
\end{center}
\begin{center}
    \large{Untuk Memenuhi Tugas Mata Kuliah Aplikasi Komputer \\ Dosen Pengampu: Drs. Sahid M.Sc.}
\end{center}
\vspace{2cm}
\begin{minipage}{17cm}
    \begin{center}
        \includegraphics[width=6cm]{images/Universitas Negeri Yogyakarta.jpg}
    \end{center}
\end{minipage}

\vspace{3cm}

\begin{center}
    \large{Disusun Oleh: \\ Fanny Erina Dewi (22305141005/ Matematika B 2022)}
\end{center}

\vspace{4cm}
\begin{center}
    \large{\textbf{PROGRAM STUDI MATEMATIKA}} \\
    \large{\textbf{DEPARTEMEN PENDIDIKAN MATEMATIKA}} \\
    \large{\textbf{FAKULTAS MATEMATIKA DAN ILMU PENGETAHUAN ALAM}} \\
    \large{\textbf{UNIVERSITAS NEGERI YOGYAKARTA}} \\
    \large{\textbf{2023}}
\end{center}

\clearpage
\thispagestyle{empty}
\tableofcontents
\mainmatter

\chapter{Software EMT}
\begin{eulernootebook}   
\eulerheading{Pendahuluan dan Pengenalan Cara Kerja EMT}
\begin{eulercomment}
Selamat datang! Ini adalah pengantar pertama ke Euler Math Toolbox
(disingkat EMT atau Euler). EMT adalah sistem terintegrasi yang
merupakan perpaduan kernel numerik Euler dan program komputer aljabar
Maxima.

- Bagian numerik, GUI, dan komunikasi dengan Maxima telah dikembangkan
oleh R. Grothmann, seorang profesor matematika di Universitas
Eichstätt, Jerman. Banyak algoritma numerik dan pustaka software open
source yang digunakan di dalamnya.

- Maxima adalah program open source yang matang dan sangat kaya untuk
perhitungan simbolik dan aritmatika tak terbatas. Software ini
dikelola oleh sekelompok pengembang di internet.

- Beberapa program lain (LaTeX, Povray, Tiny C Compiler, Python) dapat
digunakan di Euler untuk memungkinkan perhitungan yang lebih cepat
maupun tampilan atau grafik yang lebih baik.

Yang sedang Anda baca (jika dibaca di EMT) ini adalah berkas notebook
di EMT. Notebook aslinya bawaan EMT (dalam bahasa Inggris) dapat
dibuka melalui menu File, kemudian pilih "Open Tutorias and Example",
lalu pilih file "00 First Steps.en". Perhatikan, file notebook EMT
memiliki ekstensi ".en". Melalui notebook ini Anda akan belajar
menggunakan software Euler untuk menyelesaikan berbagai masalah
matematika.
\end{eulercomment}
\begin{eulercomment}
Panduan ini ditulis dengan Euler dalam bentuk notebook Euler, yang berisi teks
(deskriptif), baris-baris perintah, tampilan hasil perintah (numerik, ekspresi
matematika, atau gambar/plot), dan gambar yang disisipkan dari file gambar.

Untuk menambah jendela EMT, Anda dapat menekan [F11]. EMT akan menampilkan
jendela grafik di layar desktop Anda. Tekan [F11] lagi untuk kembali ke tata
letak favorit Anda. Tata letak disimpan untuk sesi berikutnya.

Anda juga dapat menggunakan [Ctrl]+[G] untuk menyembunyikan jendela grafik.
Selanjutnya Anda dapat beralih antara grafik dan teks dengan tombol [TAB].

Seperti yang Anda baca, notebook ini berisi tulisan (teks) berwarna hijau, yang
dapat Anda edit dengan mengklik kanan teks atau tekan menu Edit -\textgreater{} Edit Comment
atau tekan [F5], dan juga baris perintah EMT yang ditandai dengan "\textgreater{}" dan
berwarna merah. Anda dapat menyisipkan baris perintah baru dengan cara menekan
tiga tombol bersamaan: [Shift]+[Ctrl]+[Enter].

\end{eulercomment}
\eulersubheading{Komentar (Teks Uraian)}
\begin{eulercomment}
Komentar atau teks penjelasan dapat berisi beberapa "markup" dengan sintaks
sebagai berikut.

\end{eulercomment}
\begin{eulerttcomment}
   - * Judul
   - ** Sub-Judul
   - latex: F (x) = \(\backslash\)int_a^x f (t) \(\backslash\), dt
   - mathjax: \(\backslash\)frac\{x^2-1\}\{x-1\} = x + 1
   - maxima: 'integrate(x^3,x) = integrate(x^3,x) + C
   - http://www.euler-math-toolbox.de
   - See: http://www.google.de | Google
   - image: hati.png
   - ---
\end{eulerttcomment}
\begin{eulercomment}

Hasil sintaks-sintaks di atas (tanpa diawali tanda strip) adalah sebagai berikut.

\begin{eulercomment}
\eulerheading{Judul}
\begin{eulercomment}
\end{eulercomment}
\eulersubheading{Sub-Judul}
\begin{eulercomment}
\end{eulercomment}
\begin{eulerformula}
\[
F(x) = \int_a^x f(t) \, dt
\]
\end{eulerformula}
\begin{eulerformula}
\[
\frac{x^2-1}{x-1} = x + 1
\]
\end{eulerformula}
\begin{eulercomment}
maxima: 'integrate(x\textasciicircum{}3,x) = integrate(x\textasciicircum{}3,x) + C\\
http://www.euler-math-toolbox.de\\
See: http://www.google.de \textbar{} Google\\
image: hati.png\\
\end{eulercomment}
\eulersubheading{}
\begin{eulercomment}
Gambar diambil dari folder images di tempat file notebook berada dan tidak dapat
dibaca dari Web. Untuk "See:", tautan (URL)web lokal dapat digunakan.

Paragraf terdiri atas satu baris panjang di editor. Pergantian baris akan memulai
baris baru. Paragraf harus dipisahkan dengan baris kosong.
\end{eulercomment}
\begin{eulerprompt}
>// baris perintah diawali dengan >, komentar (keterangan) diawali dengan //
\end{eulerprompt}
\eulerheading{Baris Perintah}
\begin{eulercomment}
Mari kita tunjukkan cara menggunakan EMT sebagai kalkulator yang sangat
canggih.

EMT berorientasi pada baris perintah. Anda dapat menuliskan satu atau lebih
perintah dalam satu baris perintah. Setiap perintah harus diakhiri dengan koma
atau titik koma.

- Titik koma menyembunyikan output (hasil) dari perintah.\\
- Sebuah koma mencetak hasilnya.\\
- Setelah perintah terakhir, koma diasumsikan secara otomatis (boleh tidak
ditulis).

Dalam contoh berikut, kita mendefinisikan variabel r yang diberi nilai 1,25.
Output dari definisi ini adalah nilai variabel. Tetapi karena tanda titik koma,
nilai ini tidak ditampilkan. Pada kedua perintah di belakangnya, hasil kedua
perhitungan tersebut ditampilkan.
\end{eulercomment}
\begin{eulerprompt}
>r=1.25; pi*r^2, 2*pi*r
\end{eulerprompt}
\begin{euleroutput}
  4.90873852123
  7.85398163397
\end{euleroutput}
\eulersubheading{Latihan untuk Anda}
\begin{eulercomment}
- Sisipkan beberapa baris perintah baru\\
- Tulis perintah-perintah baru untuk melakukan suatu perhitungan yang
Anda inginkan, boleh menggunakan variabel, boleh tanpa variabel.\\
\end{eulercomment}
\eulersubheading{}
\begin{eulerprompt}
>12.5+102
\end{eulerprompt}
\begin{euleroutput}
  114.5
\end{euleroutput}
\begin{eulerprompt}
>s=7;
>luas=s^2
\end{eulerprompt}
\begin{euleroutput}
  49
\end{euleroutput}
\begin{eulerprompt}
>r=14;
>A=pi*r^2 // luas lingkaran
\end{eulerprompt}
\begin{euleroutput}
  615.752160104
\end{euleroutput}
\begin{eulerprompt}
>r=7;pi*r^2,2*pi*r
\end{eulerprompt}
\begin{euleroutput}
  Space between commands expected!
  Found: pi*r^2,2*pi*r (character 112)
  You can disable this in the Options menu.
  Error in:
  r=7;pi*r^2,2*pi*r ...
      ^
\end{euleroutput}
\begin{eulerprompt}
>7*4//5
\end{eulerprompt}
\begin{euleroutput}
  28
\end{euleroutput}
\begin{eulerprompt}
>12/5
\end{eulerprompt}
\begin{euleroutput}
  2.4
\end{euleroutput}
\begin{eulerprompt}
>100-34
\end{eulerprompt}
\begin{euleroutput}
  66
\end{euleroutput}
\begin{eulerprompt}
>(1+sqrt(5))/2
\end{eulerprompt}
\begin{euleroutput}
  1.61803398875
\end{euleroutput}
\begin{eulerprompt}
>z1 = 6-8I
\end{eulerprompt}
\begin{euleroutput}
  6-8i
\end{euleroutput}
\begin{eulerprompt}
>z2 = 4+I
\end{eulerprompt}
\begin{euleroutput}
  4+1i
\end{euleroutput}
\begin{eulerprompt}
>z1+z2
\end{eulerprompt}
\begin{euleroutput}
  10-7i
\end{euleroutput}
\begin{eulerprompt}
>z1-z2
\end{eulerprompt}
\begin{euleroutput}
  2-9i
\end{euleroutput}
\begin{eulerprompt}
>z1/z2
\end{eulerprompt}
\begin{euleroutput}
  0.941176-2.23529i
\end{euleroutput}
\begin{eulerprompt}
>z2*z2^2
\end{eulerprompt}
\begin{euleroutput}
  52+47i
\end{euleroutput}
\begin{eulerprompt}
>r := 1.25 // Komentar: Menggunakan  := sebagai ganti =
\end{eulerprompt}
\begin{euleroutput}
  1.25
\end{euleroutput}
\begin{eulerprompt}
>p=1.00*5, l=2*3; t=3 //balok
\end{eulerprompt}
\begin{euleroutput}
  5
  3
\end{euleroutput}
\begin{eulerprompt}
>p*l*t //volume
\end{eulerprompt}
\begin{euleroutput}
  90
\end{euleroutput}
\begin{eulerprompt}
>(2*p*l)+(2*l*t)+(2*p*t) //luas permukaan
\end{eulerprompt}
\begin{euleroutput}
  126
\end{euleroutput}
\eulersubheading{}
\begin{eulercomment}
Beberapa catatan yang harus Anda perhatikan tentang penulisan sintaks
perintah EMT.

- Pastikan untuk menggunakan titik desimal, bukan koma desimal untuk
bilangan!\\
- Gunakan * untuk perkalian dan \textasciicircum{} untuk eksponen (pangkat).\\
- Seperti biasa, * dan / bersifat lebih kuat daripada + atau -.\\
- \textasciicircum{} mengikat lebih kuat dari *, sehingga pi * r \textasciicircum{} 2 merupakan rumus
luas lingkaran.\\
- Jika perlu, Anda harus menambahkan tanda kurung, seperti pada 2 \textasciicircum{} (2
\textasciicircum{} 3).

Perintah r = 1.25 adalah menyimpan nilai ke variabel di EMT. Anda juga
dapat menulis r: = 1.25 jika mau. Anda dapat menggunakan spasi sesuka
Anda.

Anda juga dapat mengakhiri baris perintah dengan komentar yang diawali
dengan dua garis miring (//).

Argumen atau input untuk fungsi ditulis di dalam tanda kurung.
\end{eulercomment}
\begin{eulerprompt}
>sin(45°), cos(pi), log(sqrt(E))
\end{eulerprompt}
\begin{euleroutput}
  0.707106781187
  -1
  0.5
\end{euleroutput}
\begin{eulercomment}
Seperti yang Anda lihat, fungsi trigonometri bekerja dengan radian, dan derajat
dapat diubah dengan °. Jika keyboard Anda tidak memiliki karakter derajat tekan
[F7], atau gunakan fungsi deg() untuk mengonversi.

EMT menyediakan banyak sekali fungsi dan operator matematika.Hampir semua fungsi
matematika sudah tersedia di EMT. Anda dapat melihat daftar lengkap fungsi-fungsi
matematika di EMT pada berkas Referensi (klik menu Help -\textgreater{} Reference)

Untuk membuat rangkaian komputasi lebih mudah, Anda dapat merujuk ke hasil
sebelumnya dengan "\%". Cara ini sebaiknya hanya digunakan untuk merujuk hasil
perhitungan dalam baris perintah yang sama.
\end{eulercomment}
\begin{eulerprompt}
>(sqrt(5)+1)/2, %^2-%+1 // Memeriksa solusi x^2-x+1=0
\end{eulerprompt}
\begin{euleroutput}
  1.61803398875
  2
\end{euleroutput}
\eulersubheading{Latihan untuk Anda}
\begin{eulercomment}
- Buka berkas Reference dan baca fungsi-fungsi matematika yang
tersedia di EMT.\\
- Sisipkan beberapa baris perintah baru.\\
- Lakukan contoh-contoh perhitungan menggunakan fungsi-fungsi
matematika di EMT.\\
\end{eulercomment}
\eulersubheading{}
\begin{eulerprompt}
>p=7, l=4; 2*(p+l), p*l
\end{eulerprompt}
\begin{euleroutput}
  7
  22
  28
\end{euleroutput}
\begin{eulerprompt}
>r:=12.5
\end{eulerprompt}
\begin{euleroutput}
  12.5
\end{euleroutput}
\begin{eulerprompt}
>cos(0)
\end{eulerprompt}
\begin{euleroutput}
  1
\end{euleroutput}
\begin{eulerprompt}
>cos(0), sin(pi), log(sqrt(E))
\end{eulerprompt}
\begin{euleroutput}
  1
  0
  0.5
\end{euleroutput}
\begin{eulerprompt}
>(2+2)/2, (sqrt(4)+1)/2
\end{eulerprompt}
\begin{euleroutput}
  2
  1.5
\end{euleroutput}
\begin{eulerprompt}
>2km -> miles, 2 inch -> "mm"
\end{eulerprompt}
\begin{euleroutput}
  1.24274238447
  Commands must be separated by semicolon or comma!
  Found: inch -> "mm" (character 105)
  You can disable this in the Options menu.
  Error in:
  2km -> miles, 2 inch -> "mm" ...
                  ^
\end{euleroutput}
\begin{eulerprompt}
>50miles // 1 mil = 1609,344 m
\end{eulerprompt}
\begin{euleroutput}
  80467.2
\end{euleroutput}
\begin{eulerprompt}
>pi
\end{eulerprompt}
\begin{euleroutput}
  3.14159265359
\end{euleroutput}
\begin{eulerprompt}
>longest pi
\end{eulerprompt}
\begin{euleroutput}
        3.141592653589793 
\end{euleroutput}
\begin{eulerprompt}
>7*longest pi
\end{eulerprompt}
\begin{euleroutput}
        3.141592653589793 
  Wrong argument.
  Cannot use a string here.
  
  Error in:
  7*longest pi ...
              ^
\end{euleroutput}
\begin{eulerprompt}
>short pi
\end{eulerprompt}
\begin{euleroutput}
  3.1416
\end{euleroutput}
\begin{eulerprompt}
>plot2d("sin(x)",0,2pi,-1.2,1.2,grid=3,xl="x",yl="sin(x)"):
\end{eulerprompt}
\eulerimg{27}{images/Pekan 1-2_Fanny Erina Dewi_22305141005_EMT00-FisrtSteps_Aplikom-001.png}
\begin{eulerprompt}
>plot3d("x^2+y^2",>spectral,>contour,n=100):
\end{eulerprompt}
\eulerimg{27}{images/Pekan 1-2_Fanny Erina Dewi_22305141005_EMT00-FisrtSteps_Aplikom-002.png}
\begin{eulerprompt}
>1miles  // 1 mil = 1609,344 m
\end{eulerprompt}
\begin{euleroutput}
  1609.344
\end{euleroutput}
\begin{eulerprompt}
>w=normal(1000); // 0-1-normal distribution
>\{x,y\}=histo(w,10,v=[-6,-4,-2,-1,0,1,2,4,6]); // interval bounds v
>plot2d(x,y,>bar):
\end{eulerprompt}
\eulerimg{27}{images/Pekan 1-2_Fanny Erina Dewi_22305141005_EMT00-FisrtSteps_Aplikom-003.png}
\begin{eulerprompt}
>x=-2:0.05:2; y=x'; plot3d(x,y,x*y,scale=[3,3,1]; grid=30):
\end{eulerprompt}
\eulerimg{27}{images/Pekan 1-2_Fanny Erina Dewi_22305141005_EMT00-FisrtSteps_Aplikom-004.png}
\begin{eulerprompt}
>aspect(2);
>plot2d(["sin(x)","cos(x)"],0,2pi):
\end{eulerprompt}
\eulerimg{13}{images/Pekan 1-2_Fanny Erina Dewi_22305141005_EMT00-FisrtSteps_Aplikom-005.png}
\eulersubheading{}
\eulerheading{Satuan}
\begin{eulercomment}
EMT dapat mengubah unit satuan menjadi sistem standar internasional
(SI). Tambahkan satuan di belakang angka untuk konversi sederhana.

Beberapa satuan yang sudah dikenal di dalam EMT adalah sebagai
berikut. Semua unit diakhiri dengan tanda dolar (\textdollar{}), namun boleh tidak
perlu ditulis dengan mengaktifkan easyunits.

kilometer\textdollar{}:=1000;\\
km\textdollar{}:=kilometer\textdollar{};\\
cm\textdollar{}:=0.01;\\
mm\textdollar{}:=0.001;\\
minute\textdollar{}:=60;\\
min\textdollar{}:=minute\textdollar{};\\
minutes\textdollar{}:=minute\textdollar{};\\
hour\textdollar{}:=60*minute\textdollar{};\\
h\textdollar{}:=hour\textdollar{};\\
hours\textdollar{}:=hour\textdollar{};\\
day\textdollar{}:=24*hour\textdollar{};\\
days\textdollar{}:=day\textdollar{};\\
d\textdollar{}:=day\textdollar{};\\
year\textdollar{}:=365.2425*day\textdollar{};\\
years\textdollar{}:=year\textdollar{};\\
y\textdollar{}:=year\textdollar{};\\
inch\textdollar{}:=0.0254;\\
in\textdollar{}:=inch\textdollar{};\\
feet\textdollar{}:=12*inch\textdollar{};\\
foot\textdollar{}:=feet\textdollar{};\\
ft\textdollar{}:=feet\textdollar{};\\
yard\textdollar{}:=3*feet\textdollar{};\\
yards\textdollar{}:=yard\textdollar{};\\
yd\textdollar{}:=yard\textdollar{};\\
mile\textdollar{}:=1760*yard\textdollar{};\\
miles\textdollar{}:=mile\textdollar{};\\
kg\textdollar{}:=1;\\
sec\textdollar{}:=1;\\
ha\textdollar{}:=10000;\\
Ar\textdollar{}:=100;\\
Tagwerk\textdollar{}:=3408;\\
Acre\textdollar{}:=4046.8564224;\\
pt\textdollar{}:=0.376mm;

Untuk konversi ke dan antar unit, EMT menggunakan operator khusus,
yakni -\textgreater{}.
\end{eulercomment}
\begin{eulerprompt}
>4km -> miles, 4inch -> " mm"
\end{eulerprompt}
\begin{euleroutput}
  2.48548476895
  101.6 mm
\end{euleroutput}
\eulerheading{Format Tampilan Nilai}
\begin{eulercomment}
Akurasi internal untuk nilai bilangan di EMT adalah standar IEEE,
sekitar 16 digit desimal. Aslinya, EMT tidak mencetak semua digit
suatu bilangan. Ini untuk menghemat tempat dan agar terlihat lebih
baik. Untuk mengatrtamilan satu bilangan, operator berikut dapat
digunakan.
\end{eulercomment}
\begin{eulerprompt}
>pi
\end{eulerprompt}
\begin{euleroutput}
  3.14159265359
\end{euleroutput}
\begin{eulerprompt}
>longest pi
\end{eulerprompt}
\begin{euleroutput}
        3.141592653589793 
\end{euleroutput}
\begin{eulerprompt}
>long pi
\end{eulerprompt}
\begin{euleroutput}
  3.14159265359
\end{euleroutput}
\begin{eulerprompt}
>short pi
\end{eulerprompt}
\begin{euleroutput}
  3.1416
\end{euleroutput}
\begin{eulerprompt}
>shortest pi
\end{eulerprompt}
\begin{euleroutput}
     3.1 
\end{euleroutput}
\begin{eulerprompt}
>fraction pi
\end{eulerprompt}
\begin{euleroutput}
  312689/99532
\end{euleroutput}
\begin{eulerprompt}
>short 1200*1.03^10, long E, longest pi
\end{eulerprompt}
\begin{euleroutput}
  1612.7
  2.71828182846
        3.141592653589793 
\end{euleroutput}
\begin{eulercomment}
Format aslinya untuk menampilkan nilai menggunakan sekitar 10 digit.
Format tampilan nilai dapat diatur secara global atau hanya untuk satu
nilai.

Anda dapat mengganti format tampilan bilangan untuk semua perintah
selanjutnya. Untuk mengembalikan ke format aslinya dapat digunakan
perintah "defformat" atau "reset".
\end{eulercomment}
\begin{eulerprompt}
>longestformat; pi, defformat; pi
\end{eulerprompt}
\begin{euleroutput}
  3.141592653589793
  3.14159265359
\end{euleroutput}
\begin{eulercomment}
Kernel numerik EMT bekerja dengan bilangan titik mengambang (floating point)
dalam presisi ganda IEEE (berbeda dengan bagian simbolik EMT). Hasil numerik
dapat ditampilkan dalam bentuk pecahan.
\end{eulercomment}
\begin{eulerprompt}
>1/7+1/4, fraction %
\end{eulerprompt}
\begin{euleroutput}
  0.392857142857
  11/28
\end{euleroutput}
\eulerheading{Perintah Multibaris}
\begin{eulercomment}
Perintah multi-baris membentang di beberapa baris yang terhubung
dengan "..." di setiap akhir baris, kecuali baris terakhir. Untuk
menghasilkan tanda pindah baris tersebut, gunakan tombol
[Ctrl]+[Enter]. Ini akan menyambung perintah ke baris berikutnya dan
menambahkan "..." di akhir baris sebelumnya. Untuk menggabungkan suatu
baris ke baris sebelumnya, gunakan [Ctrl]+[Backspace].

Contoh perintah multi-baris berikut dapat dijalankan setiap kali
kursor berada di salah satu barisnya. Ini juga menunjukkan bahwa ...
harus berada di akhir suatu baris meskipun baris tersebut memuat
komentar.
\end{eulercomment}
\begin{eulerprompt}
>a=4; b=15; c=2; // menyelesaikan a*x^2+b*x+c=0 secara manual ...
>D=sqrt(b^2/(a^2*4)-c/a); ...
>-b/(2*a) + D, ...
>-b/(2*a) - D
\end{eulerprompt}
\begin{euleroutput}
  -0.138444501319
  -3.61155549868
\end{euleroutput}
\eulerheading{Menampilkan Daftar Variabe}
\begin{eulercomment}
Untuk menampilkan semua variabel yang sudah pernah Anda definisikan
sebelumnya (dan dapat dilihat kembali nilainya), gunakan perintah
"listvar".
\end{eulercomment}
\begin{eulerprompt}
>listvar
\end{eulerprompt}
\begin{euleroutput}
  A                   615.752160103599
  r                   12.5
  s                   7
  z2                  4+1i
  X                   2
  p                   7
  l                   4
  luas                49
  z1                  6+-8i
  t                   3
  w                   Type: Real Matrix (1x1000)
  x                   Type: Real Matrix (1x81)
  y                   Type: Real Matrix (81x1)
  a                   4
  b                   15
  c                   2
  D                   1.73655549868123
\end{euleroutput}
\begin{eulercomment}
Perintah listvar hanya menampilkan variabel buatan pengguna.
Dimungkinkan untuk menampilkan variabel lain, dengan menambahkan
string  termuat di dalam nama variabel yang diinginkan.

Perlu Anda perhatikan, bahwa EMT membedakan huruf besar dan huruf
kecil. Jadi variabel "d" berbeda dengan variabel "D".

Contoh berikut ini menampilkan semua unit yang diakhiri dengan "m"
dengan mencari semua variabel yang berisi "m\textdollar{}".
\end{eulercomment}
\begin{eulerprompt}
>listvar m$
\end{eulerprompt}
\begin{euleroutput}
  km$                 1000
  cm$                 0.01
  mm$                 0.001
  nm$                 1853.24496
  gram$               0.001
  m$                  1
  hquantum$           6.62606957e-34
  atm$                101325
\end{euleroutput}
\begin{eulercomment}
Untuk menghapus variabel tanpa harus memulai ulang EMT gunakan
perintah "remvalue".
\end{eulercomment}
\begin{eulerprompt}
>remvalue a,b,c,D
>D
\end{eulerprompt}
\begin{euleroutput}
  Variable D not found!
  Error in:
  D ...
   ^
\end{euleroutput}
\eulerheading{Menampilkan Panduan}
\begin{eulercomment}
Untuk mendapatkan panduan tentang penggunaan perintah atau fungsi di EMT, buka
jendela panduan dengan menekan [F1] dan cari fungsinya. Anda juga dapat
mengklik dua kali pada fungsi yang tertulis di baris perintah atau di teks
untuk membuka jendela panduan.

Coba klik dua kali pada perintah "intrandom" berikut ini!
\end{eulercomment}
\begin{eulerprompt}
>intrandom(10,6)
\end{eulerprompt}
\begin{euleroutput}
  [5,  6,  5,  3,  6,  4,  6,  2,  1,  1]
\end{euleroutput}
\begin{eulercomment}
Di jendela panduan, Anda dapat mengklik kata apa saja untuk menemukan
referensi atau fungsi.

Misalnya, coba klik kata "random" di jendela panduan. Kata tersebut
boleh ada dalam teks atau di bagian "See:" pada panduan. Anda akan
menemukan penjelasan fungsi "random", untuk menghasilkan bilangan acak
berdistribusi uniform antara 0,0 dan 1,0. Dari panduan untuk "random"
Anda dapat menampilkan panduan untuk fungsi "normal", dll.
\end{eulercomment}
\begin{eulerprompt}
>random(10)
\end{eulerprompt}
\begin{euleroutput}
  [0.536575,  0.682446,  0.630154,  0.903174,  0.108693,  0.328172,
  0.692209,  0.127043,  0.321076,  0.395025]
\end{euleroutput}
\begin{eulerprompt}
>normal(10)
\end{eulerprompt}
\begin{euleroutput}
  [0.647754,  -1.48432,  -0.430514,  0.865282,  -0.91311,  -0.0955159,
  0.807261,  -1.01303,  -0.73066,  -0.607451]
\end{euleroutput}
\eulerheading{Matriks dan Vektor}
\begin{eulercomment}
EMT merupakan suatu aplikasi matematika yang mengerti "bahasa matriks". Artinya,
EMT menggunakan vektor dan matriks untuk perhitungan-perhitungan tingkat lanjut.
Suatu vektor atau matriks dapat didefinisikan dengan tanda kurung siku.
Elemen-elemennya dituliskan di dalam tanda kurung siku, antar elemen dalam satu
baris dipisahkan oleh koma(,), antar baris dipisahkan oleh titik koma (;).

Vektor dan matriks dapat diberi nama seperti variabel biasa.
\end{eulercomment}
\begin{eulerprompt}
>v=[4,5,6,3,2,1]
\end{eulerprompt}
\begin{euleroutput}
  [4,  5,  6,  3,  2,  1]
\end{euleroutput}
\begin{eulerprompt}
>A=[1,2,3;4,5,6;7,8,9]
\end{eulerprompt}
\begin{euleroutput}
              1             2             3 
              4             5             6 
              7             8             9 
\end{euleroutput}
\begin{eulercomment}
Karena EMT mengerti bahasa matriks, EMT memiliki kemampuan yang sangat canggih
untuk melakukan perhitungan matematis untuk masalah-masalah aljabar linier,
statistika, dan optimisasi.

Vektor juga dapat didefinisikan dengan menggunakan rentang nilai dengan interval
tertentu menggunakan tanda titik dua (:),seperti contoh berikut ini.
\end{eulercomment}
\begin{eulerprompt}
>c=1:5
\end{eulerprompt}
\begin{euleroutput}
  [1,  2,  3,  4,  5]
\end{euleroutput}
\begin{eulerprompt}
>w=0:0.1:1
\end{eulerprompt}
\begin{euleroutput}
  [0,  0.1,  0.2,  0.3,  0.4,  0.5,  0.6,  0.7,  0.8,  0.9,  1]
\end{euleroutput}
\begin{eulerprompt}
>mean(w^2)
\end{eulerprompt}
\begin{euleroutput}
  0.35
\end{euleroutput}
\eulerheading{Bilangan Kompleks}
\begin{eulercomment}
EMT juga dapat menggunakan bilangan kompleks. Tersedia banyak fungsi untuk
bilangan kompleks di EMT. Bilangan imaginer

\end{eulercomment}
\begin{eulerformula}
\[
i = \sqrt{-1}
\]
\end{eulerformula}
\begin{eulercomment}
dituliskan dengan huruf I (huruf besar I), namun akan ditampilkan dengan huruf i
(i kecil).

\end{eulercomment}
\begin{eulerttcomment}
  re(x) : bagian riil pada bilangan kompleks x.
  im(x) : bagian imaginer pada bilangan kompleks x.
  complex(x) : mengubah bilangan riil x menjadi bilangan kompleks.
  conj(x) : Konjugat untuk bilangan bilangan komplkes x.
  arg(x) : argumen (sudut dalam radian) bilangan kompleks x.
  real(x) : mengubah x menjadi bilangan riil.
\end{eulerttcomment}
\begin{eulercomment}

Apabila bagian imaginer x terlalu besar, hasilnya akan menampilkan pesan
kesalahan.

\end{eulercomment}
\begin{eulerttcomment}
  >sqrt(-1) // Error!
  >sqrt(complex(-1))
\end{eulerttcomment}
\begin{eulerprompt}
>z=2+3*I, re(z), im(z), conj(z), arg(z), deg(arg(z)), deg(arctan(3/2))
\end{eulerprompt}
\begin{euleroutput}
  2+3i
  2
  3
  2-3i
  0.982793723247
  56.309932474
  56.309932474
\end{euleroutput}
\begin{eulerprompt}
>deg(arg(I)) // 90°
\end{eulerprompt}
\begin{euleroutput}
  90
\end{euleroutput}
\begin{eulerprompt}
>sqrt(-1)
\end{eulerprompt}
\begin{euleroutput}
  Floating point error!
  Error in sqrt
  Error in:
  sqrt(-1) ...
          ^
\end{euleroutput}
\begin{eulerprompt}
>sqrt(complex(-1))
\end{eulerprompt}
\begin{euleroutput}
  0+1i
\end{euleroutput}
\begin{eulercomment}
EMT selalu menganggap semua hasil perhitungan berupa bilangan riil dan tidak
akan secara otomatis mengubah ke bilangan kompleks.

Jadi akar kuadrat -1 akan menghasilkan kesalahan, tetapi akar kuadrat kompleks
didefinisikan untuk bidang koordinat dengan cara seperti biasa. Untuk mengubah
bilangan riil menjadi kompleks, Anda dapat menambahkan 0i atau menggunakan
fungsi "complex".
\end{eulercomment}
\begin{eulerprompt}
>complex(-1), sqrt(%)
\end{eulerprompt}
\begin{euleroutput}
  -1+0i 
  0+1i
\end{euleroutput}
\eulerheading{Matematika Simbolik}
\begin{eulercomment}
EMT dapat melakukan perhitungan matematika simbolis (eksak) dengan bantuan
software Maxima. Software Maxima otomatis sudah terpasang di komputer Anda ketika
Anda memasang EMT. Meskipun demikian, Anda dapat juga memasang software Maxima
tersendiri (yang terpisah dengan instalasi Maxima di EMT).

Pengguna Maxima yang sudah mahir harus memperhatikan bahwa terdapat sedikit
perbedaan dalam sintaks antara sintaks asli Maxima dan sintaks ekspresi simbolik
di EMT.

Untuk melakukan perhitungan matematika simbolis di EMT, awali perintah Maxima
dengan tanda "\&". Setiap ekspresi yang dimulai dengan "\&" adalah ekspresi
simbolis dan dikerjakan oleh Maxima.
\end{eulercomment}
\begin{eulerprompt}
>&(a+b)^2
\end{eulerprompt}
\begin{euleroutput}
  
                                        2
                                 (b + a)
  
\end{euleroutput}
\begin{eulerprompt}
>&expand((a+b)^2), &factor(x^2+5*x+6)
\end{eulerprompt}
\begin{euleroutput}
  
                              2            2
                             b  + 2 a b + a
  
  
                             (x + 2) (x + 3)
  
\end{euleroutput}
\begin{eulerprompt}
>&solve(a*x^2+b*x+c,x) // rumus abc
\end{eulerprompt}
\begin{euleroutput}
  
                       2                         2
               - sqrt(b  - 4 a c) - b      sqrt(b  - 4 a c) - b
          [x = ----------------------, x = --------------------]
                        2 a                        2 a
  
\end{euleroutput}
\begin{eulerprompt}
>&(a^2-b^2)/(a+b), &ratsimp(%) // ratsimp menyederhanakan bentuk pecahan
\end{eulerprompt}
\begin{euleroutput}
  
                                  2    2
                                 a  - b
                                 -------
                                  b + a
  
  
                                  a - b
  
\end{euleroutput}
\begin{eulerprompt}
>10! // nilai faktorial (modus EMT)
\end{eulerprompt}
\begin{euleroutput}
  3628800
\end{euleroutput}
\begin{eulerprompt}
>&10! //nilai faktorial (simbolik dengan Maxima)
\end{eulerprompt}
\begin{euleroutput}
  
                                 3628800
  
\end{euleroutput}
\begin{eulercomment}
Untuk menggunakan perintah Maxima secara langsung (seperti perintah pada layar
Maxima) awali perintahnya dengan tanda "::" pada baris perintah EMT. Sintaks
Maxima disesuaikan dengan sintaks EMT (disebut "modus kompatibilitas").
\end{eulercomment}
\begin{eulerprompt}
>factor(1000) // mencari semua faktor 1000 (EMT)
\end{eulerprompt}
\begin{euleroutput}
  [2,  2,  2,  5,  5,  5]
\end{euleroutput}
\begin{eulerprompt}
>:: factor(1000) // faktorisasi prima 1000 (dengan Maxima) 
\end{eulerprompt}
\begin{euleroutput}
  
                                   3  3
                                  2  5
  
\end{euleroutput}
\begin{eulerprompt}
>:: factor(20!)
\end{eulerprompt}
\begin{euleroutput}
  
                          18  8  4  2
                         2   3  5  7  11 13 17 19
  
\end{euleroutput}
\begin{eulercomment}
Jika Anda sudah mahir menggunakan Maxima, Anda dapat menggunakan sintaks asli
perintah Maxima dengan menggunakan tanda ":::" untuk mengawali setiap perintah
Maxima di EMT. Perhatikan, harus ada spasi antara ":::" dan perintahnya.
\end{eulercomment}
\begin{eulerprompt}
>::: binomial(5,2); // nilai C(5,2)
\end{eulerprompt}
\begin{euleroutput}
  
                                    10
  
\end{euleroutput}
\begin{eulerprompt}
>::: binomial(m,4); // C(m,4)=m!/(4!(m-4)!)
\end{eulerprompt}
\begin{euleroutput}
  
                        (m - 3) (m - 2) (m - 1) m
                        -------------------------
                                   24
  
\end{euleroutput}
\begin{eulerprompt}
>::: trigexpand(cos(x+y)); // rumus cos(x+y)=cos(x) cos(y)-sin(x)sin(y) 
\end{eulerprompt}
\begin{euleroutput}
  
                      cos(x) cos(y) - sin(x) sin(y)
  
\end{euleroutput}
\begin{eulerprompt}
>::: trigexpand(sin(x+y));
\end{eulerprompt}
\begin{euleroutput}
  
                      cos(x) sin(y) + sin(x) cos(y)
  
\end{euleroutput}
\begin{eulerprompt}
>::: trigsimp(((1-sin(x)^2)*cos(x))/cos(x)^2+tan(x)*sec(x)^2) //menyederhanakan fungsi trigonometri
\end{eulerprompt}
\begin{euleroutput}
  
                                         4
                             sin(x) + cos (x)
                             ----------------
                                    3
                                 cos (x)
  
\end{euleroutput}
\begin{eulercomment}
Untuk menyimpan ekspresi simbolik ke dalam suatu variabel digunakan tanda "\&=".
\end{eulercomment}
\begin{eulerprompt}
>p1 &= (x^3+1)/(x+1)
\end{eulerprompt}
\begin{euleroutput}
  
                                   3
                                  x  + 1
                                  ------
                                  x + 1
  
\end{euleroutput}
\begin{eulerprompt}
>&ratsimp(p1)
\end{eulerprompt}
\begin{euleroutput}
  
                                 2
                                x  - x + 1
  
\end{euleroutput}
\begin{eulercomment}
Untuk mensubstitusikan suatu nilai ke dalam variabel dapat digunakan perintah
"with".
\end{eulercomment}
\begin{eulerprompt}
>&p1 with x=3 // (3^3+1)/(3+1)
\end{eulerprompt}
\begin{euleroutput}
  
                                    7
  
\end{euleroutput}
\begin{eulerprompt}
>&p1 with x=a+b, &ratsimp(%) //substitusi dengan variabel baru
\end{eulerprompt}
\begin{euleroutput}
  
                                      3
                               (b + a)  + 1
                               ------------
                                b + a + 1
  
  
                       2                  2
                      b  + (2 a - 1) b + a  - a + 1
  
\end{euleroutput}
\begin{eulerprompt}
>&diff(p1,x) //turunan p1 terhadap x
\end{eulerprompt}
\begin{euleroutput}
  
                                2      3
                             3 x      x  + 1
                             ----- - --------
                             x + 1          2
                                     (x + 1)
  
\end{euleroutput}
\begin{eulerprompt}
>&integrate(p1,x) // integral p1 terhadap x
\end{eulerprompt}
\begin{euleroutput}
  
                               3      2
                            2 x  - 3 x  + 6 x
                            -----------------
                                    6
  
\end{euleroutput}
\eulerheading{Tampilan Matematika Simbolik dengan LaTeX}
\begin{eulercomment}
Anda dapat menampilkan hasil perhitunagn simbolik secara lebih bagus
menggunakan LaTeX. Untuk melakukan hal ini, tambahkan tanda dolar (\textdollar{}) di depan
tanda \& pada setiap perintah Maxima.\\
Perhatikan, hal ini hanya dapat menghasilkan tampilan yang diinginkan apabila
komputer Anda sudah terpasang software LaTeX.
\end{eulercomment}
\begin{eulerprompt}
>$&(a+b)^2
\end{eulerprompt}
\begin{eulerformula}
\[
\left(b+a\right)^2
\]
\end{eulerformula}
\begin{eulerprompt}
>$&expand((a+b)^2), $&factor(x^2+5*x+6)
\end{eulerprompt}
\begin{eulerformula}
\[
b^2+2\,a\,b+a^2
\]
\end{eulerformula}
\begin{eulerformula}
\[
\left(x+2\right)\,\left(x+3\right)
\]
\end{eulerformula}
\begin{eulerprompt}
>$&solve(a*x^2+b*x+c,x) // rumus abc
\end{eulerprompt}
\begin{eulerformula}
\[
\left[ x=\frac{-\sqrt{b^2-4\,a\,c}-b}{2\,a} , x=\frac{\sqrt{b^2-4\,
 a\,c}-b}{2\,a} \right] 
\]
\end{eulerformula}
\begin{eulerprompt}
>$&(a^2-b^2)/(a+b), $&ratsimp(%)
\end{eulerprompt}
\begin{eulerformula}
\[
\frac{a^2-b^2}{b+a}
\]
\end{eulerformula}
\begin{eulerformula}
\[
a-b
\]
\end{eulerformula}
\eulerheading{Selamat Belajar dan Berlatih!}
\begin{eulercomment}
Baik, itulah sekilas pengantar penggunaan software EMT. Masih banyak
kemampuan EMT yang akan Anda pelajari dan praktikkan.

Sebagai latihan untuk memperlancar penggunaan perintah-perintah EMT
yang sudah dijelaskan di atas, silakan Anda lakukan hal-hal sebagai
berikut.

- Carilah soal-soal matematika dari buku-buku Matematika.\\
- Tambahkan beberapa baris perintah EMT pada notebook ini.\\
- Selesaikan soal-soal matematika tersebut dengan menggunakan EMT.\\
Pilih soal-soal yang sesuai dengan perintah-perintah yang sudah
dijelaskan dan dicontohkan di atas.

soal 1:\\
Sebuah desa yang jumlah penduduknya 1000 orang terkena covid-19.
Jumlah masyarakat yang terpapar setelah t hari semenjak menyebarnya
wabah penyakit dapat dimodelkan:

P(t) = 1000/(1+999*exp(-0.603*10))
\end{eulercomment}
\begin{eulerprompt}
>P = 1000/(1+999*exp(-0.603*10))
\end{eulerprompt}
\begin{euleroutput}
  293.850719578
\end{euleroutput}
\begin{eulerprompt}
>round(P)
\end{eulerprompt}
\begin{euleroutput}
  294
\end{euleroutput}
\begin{eulerprompt}
>&44!
\end{eulerprompt}
\begin{euleroutput}
  
          2658271574788448768043625811014615890319638528000000000
  
\end{euleroutput}
\begin{eulerprompt}
>$&solve(8*x+2,x), $&x with %[1]
\end{eulerprompt}
\begin{eulerformula}
\[
\left[ x=-\frac{1}{4} \right] 
\]
\end{eulerformula}
\begin{eulerformula}
\[
-\frac{1}{4}
\]
\end{eulerformula}
\begin{eulerprompt}
>$&factor(x^6+x^2+1) | modulus:=13 // factor in Z modulo 13
\end{eulerprompt}
\begin{eulerformula}
\[
\left(x^2+6\right)\,\left(x^4-6\,x^2-2\right)
\]
\end{eulerformula}
\begin{eulerprompt}
>$&solve(x^3-3*x^2+6*x+1,x), $&x with %[1]
\end{eulerprompt}
\begin{eulerformula}
\[
\left[ x=\frac{\frac{1}{2}-\frac{\sqrt{3}\,i}{2}}{\left(\frac{
 \sqrt{29}}{2}-\frac{5}{2}\right)^{\frac{1}{3}}}+\left(\frac{\sqrt{29
 }}{2}-\frac{5}{2}\right)^{\frac{1}{3}}\,\left(-\frac{\sqrt{3}\,i}{2}
 -\frac{1}{2}\right)+1 , x=\left(\frac{\sqrt{29}}{2}-\frac{5}{2}
 \right)^{\frac{1}{3}}\,\left(\frac{\sqrt{3}\,i}{2}-\frac{1}{2}
 \right)-\frac{-\frac{\sqrt{3}\,i}{2}-\frac{1}{2}}{\left(\frac{\sqrt{
 29}}{2}-\frac{5}{2}\right)^{\frac{1}{3}}}+1 , x=\left(\frac{\sqrt{29
 }}{2}-\frac{5}{2}\right)^{\frac{1}{3}}-\frac{1}{\left(\frac{\sqrt{29
 }}{2}-\frac{5}{2}\right)^{\frac{1}{3}}}+1 \right] 
\]
\end{eulerformula}
\begin{eulerformula}
\[
\frac{\frac{1}{2}-\frac{\sqrt{3}\,i}{2}}{\left(\frac{\sqrt{29}}{2}-
 \frac{5}{2}\right)^{\frac{1}{3}}}+\left(\frac{\sqrt{29}}{2}-\frac{5
 }{2}\right)^{\frac{1}{3}}\,\left(-\frac{\sqrt{3}\,i}{2}-\frac{1}{2}
 \right)+1
\]
\end{eulerformula}
\begin{eulerprompt}
>$&expand((a+b)^4), $&factor(x^4+5*x^2+6)
\end{eulerprompt}
\begin{eulerformula}
\[
b^4+4\,a\,b^3+6\,a^2\,b^2+4\,a^3\,b+a^4
\]
\end{eulerformula}
\begin{eulerformula}
\[
\left(x^2+2\right)\,\left(x^2+3\right)
\]
\end{eulerformula}
\begin{eulerprompt}
>$&solve(x^3-2*x+1,x), $&x with %[1]
\end{eulerprompt}
\begin{eulerformula}
\[
\left[ x=\frac{-\sqrt{5}-1}{2} , x=\frac{\sqrt{5}-1}{2} , x=1
  \right] 
\]
\end{eulerformula}
\begin{eulerformula}
\[
\frac{-\sqrt{5}-1}{2}
\]
\end{eulerformula}
\end{eulernotebook}

\chapter{EMT Perhitungan Aljabar}
\begin{eulernootebook}
\eulerheading{EMT untuk Perhitungan Aljabar}
\begin{eulercomment}
Pada notebook ini Anda belajar menggunakan EMT untuk melakukan
berbagai perhitungan terkait dengan materi atau topik dalam Aljabar.
Kegiatan yang harus Anda lakukan adalah sebagai berikut:

- Membaca secara cermat dan teliti notebook ini;\\
- Menerjemahkan teks bahasa Inggris ke bahasa Indonesia;\\
- Mencoba contoh-contoh perhitungan (perintah EMT) dengan cara
meng-ENTER setiap perintah EMT yang ada (pindahkan kursor ke baris
perintah)\\
- Jika perlu Anda dapat memodifikasi perintah yang ada dan memberikan
keterangan/penjelasan tambahan terkait hasilnya.\\
- Menyisipkan baris-baris perintah baru untuk mengerjakan soal-soal
Aljabar dari file PDF yang saya berikan;\\
- Memberi catatan hasilnya.\\
- Jika perlu tuliskan soalnya pada teks notebook (menggunakan format
LaTeX).\\
- Gunakan tampilan hasil semua perhitungan yang eksak atau simbolik
dengan format LaTeX. (Seperti contoh-contoh pada notebook ini.)

\end{eulercomment}
\eulersubheading{Contoh pertama}
\begin{eulercomment}
Menyederhanakan bentuk aljabar:

\end{eulercomment}
\begin{eulerformula}
\[
6x^{-3}y^5\times -7x^2y^{-9}
\]
\end{eulerformula}
\begin{eulercomment}
\end{eulercomment}
\begin{eulerprompt}
>$&6*x^(-3)*y^5*-7*x^2*y^(-9)
\end{eulerprompt}
\begin{eulercomment}
CONTOH CONTOH SOAL

MENYEDERHANAKAN
\end{eulercomment}
\begin{eulerprompt}
>$&(8*y^5)*(9*y)
>$& (3*a^2)*(-7*a^4)
>$&(6*x*y^3)*(9*x^4*y^2)
\end{eulerprompt}
\begin{eulercomment}
Menjabarkan:

\end{eulercomment}
\begin{eulerformula}
\[
(6x^{-3}+y^5)(-7x^2-y^{-9})
\]
\end{eulerformula}
\begin{eulerprompt}
>$&showev('expand((6*x^(-3)+y^5)*(-7*x^2-y^(-9))))
>$&expand ((x+3)^2)
>$&expand ((n+6)*(n-6))
>$expand((y-5)^2)
>$&expand ((5*x-3)^2)
\end{eulerprompt}
\eulersubheading{Baris Perintah}
\begin{eulercomment}
Baris perintah Euler terdiri dari satu atau beberapa perintah Euler
diikuti dengan titik koma ";" atau koma ",". Titik koma mencegah
pencetakan hasil. Koma setelah perintah terakhir dapat dihilangkan.

Baris perintah berikut hanya akan mencetak hasil ekspresi, bukan tugas
atau perintah format.
\end{eulercomment}
\begin{eulerprompt}
>r:=2; h:=4; pi*r^2*h/3
\end{eulerprompt}
\begin{euleroutput}
  16.7551608191
\end{euleroutput}
\begin{eulercomment}
Perintah harus dipisahkan dengan tanda kosong. Baris perintah berikut
mencetak dua hasilnya.
\end{eulercomment}
\begin{eulerprompt}
>pi*2*r*h, %+2*pi*r*h // Ingat tanda % menyatakan hasil perhitungan terakhir sebelumnya
\end{eulerprompt}
\begin{euleroutput}
  50.2654824574
  100.530964915
\end{euleroutput}
\begin{eulercomment}
Baris perintah dieksekusi dalam urutan yang ditekan pengguna kembali.
Jadi, Anda mendapatkan nilai baru setiap kali menjalankan baris kedua.
\end{eulercomment}
\begin{eulerprompt}
>x := 1;
>x := cos(x) // nilai cosinus (x dalam radian)
\end{eulerprompt}
\begin{euleroutput}
  0.540302305868
\end{euleroutput}
\begin{eulerprompt}
>x := cos(x)
\end{eulerprompt}
\begin{euleroutput}
  0.857553215846
\end{euleroutput}
\begin{eulercomment}
Jika dua jalur dihubungkan dengan "..." kedua jalur akan selalu
dijalankan secara bersamaan.
\end{eulercomment}
\begin{eulerprompt}
>x := 1.5; ...
>x := (x+2/x)/2, x := (x+2/x)/2, x := (x+2/x)/2, 
\end{eulerprompt}
\begin{euleroutput}
  1.41666666667
  1.41421568627
  1.41421356237
\end{euleroutput}
\begin{eulercomment}
Ini juga cara yang baik untuk menyebarkan perintah panjang ke dua
baris atau lebih. Anda dapat menekan Ctrl + Return untuk membagi garis
menjadi dua pada posisi kursor saat ini, atau Ctlr + Back untuk
menggabungkan garis.

Untuk melipat semua multi-garis tekan Ctrl + L. Kemudian garis
berikutnya hanya akan terlihat, jika salah satunya memiliki fokus.
Untuk melipat satu garis banyak, mulailah baris pertama dengan "\% +".
\end{eulercomment}
\begin{eulerprompt}
>%+ x=4+5; ...
\end{eulerprompt}
\begin{eulercomment}
Garis yang dimulai dengan \%\% akan benar-benar tidak terlihat.
\end{eulercomment}
\begin{euleroutput}
  81
\end{euleroutput}
\begin{eulercomment}
Euler mendukung loop dalam baris perintah, asalkan sesuai dengan satu
baris atau beberapa baris. Dalam program, batasan ini tidak berlaku,
tentu saja. Untuk informasi lebih lanjut, lihat pengantar berikut.
\end{eulercomment}
\begin{eulerprompt}
>x=1; for i=1 to 5; x := (x+2/x)/2, end; // menghitung akar 2
\end{eulerprompt}
\begin{euleroutput}
  1.5
  1.41666666667
  1.41421568627
  1.41421356237
  1.41421356237
\end{euleroutput}
\begin{eulerprompt}
>x := 1.5; // comments go here before the ...
>repeat xnew:=(x+2/x)/2; until xnew~=x; ...
>   x := xnew; ...
>end; ...
>x,
\end{eulerprompt}
\begin{euleroutput}
  1.41421356237
\end{euleroutput}
\begin{eulercomment}
Tidak apa-apa menggunakan multi-garis. Pastikan baris diakhiri dengan
"...".
\end{eulercomment}
\begin{eulerprompt}
>x := 1.5; // komentar diletakkan di sini sebelum ...
>ulang xnew:=(x+2/x)/2; until xnew~=x; ...
>   x := xnew; ...
>end; ...
>x,
\end{eulerprompt}
\begin{euleroutput}
  Variable ulang not found!
  Error in:
  x := 1.5; ulang xnew:=(x+2/x)/2; until xnew~=x;    x := xnew;  ...
                  ^
\end{euleroutput}
\begin{eulercomment}
Struktur bersyarat juga berfungsi.
\end{eulercomment}
\begin{eulerprompt}
>if E^pi>pi^E; then "Thought so!", endif;
\end{eulerprompt}
\begin{euleroutput}
  Thought so!
\end{euleroutput}
\begin{eulerprompt}
>if E^pi>pi^E; then "Berpikir begitu!", berakhir jika;
\end{eulerprompt}
\begin{euleroutput}
  Berpikir begitu!
  Variable berakhir not found!
  Error in:
  if E^pi>pi^E; then "Berpikir begitu!", berakhir jika; ...
                                                  ^
\end{euleroutput}
\begin{eulercomment}
Saat Anda menjalankan perintah, kursor dapat berada di posisi mana pun
di baris perintah. Anda dapat kembali ke perintah sebelumnya atau
melompat ke perintah berikutnya dengan tombol panah. Atau Anda dapat
mengklik bagian komentar di atas perintah untuk membuka perintah.

Saat Anda menggerakkan kursor di sepanjang garis, pasangan buka dan
tutup tanda kurung atau tanda kurung akan disorot. Juga, perhatikan
baris statusnya. Setelah kurung buka dari fungsi sqrt (), baris status
akan menampilkan teks bantuan untuk fungsi tersebut. Jalankan perintah
dengan tombol kembali.
\end{eulercomment}
\begin{eulerprompt}
>sqrt(sin(10°)/cos(20°))
\end{eulerprompt}
\begin{euleroutput}
  0.429875017772
\end{euleroutput}
\begin{eulercomment}
Untuk melihat bantuan untuk perintah terbaru, buka jendela bantuan
dengan F1. Di sana, Anda dapat memasukkan teks untuk dicari. Pada
baris kosong, bantuan untuk jendela bantuan akan ditampilkan. Anda
dapat menekan escape untuk menghapus garis, atau untuk menutup jendela
bantuan.

Anda dapat mengklik dua kali pada perintah apa pun untuk membuka
bantuan untuk perintah ini. Coba klik dua kali perintah exp di bawah
ini di baris perintah.
\end{eulercomment}
\begin{eulerprompt}
>exp(log(2.5))
\end{eulerprompt}
\begin{euleroutput}
  2.5
\end{euleroutput}
\begin{eulercomment}
Anda juga dapat menyalin dan menempel di Euler. Gunakan Ctrl-C dan
Ctrl-V untuk ini. Untuk menandai teks, seret mouse atau gunakan shift
bersama dengan tombol kursor apa pun. Selain itu, Anda dapat menyalin
tanda kurung yang disorot.
\end{eulercomment}
\eulersubheading{Sintaks Dasar}
\begin{eulercomment}
Euler mengetahui fungsi matematika biasa. Seperti yang Anda lihat di
atas, fungsi trigonometri bekerja dalam radian atau derajat. Untuk
mengonversi menjadi derajat, tambahkan simbol derajat (dengan tombol
F7) ke nilainya, atau gunakan fungsi rad (x). Fungsi akar kuadrat
disebut akar kuadrat di Euler. Tentu saja, x \textasciicircum{} (1/2) juga
dimungkinkan.

Untuk menyetel variabel, gunakan "=" atau ": =". Demi kejelasan,
pendahuluan ini menggunakan bentuk yang terakhir. Spasi tidak penting.
Tapi ruang antar perintah diharapkan.

Beberapa perintah dalam satu baris dipisahkan dengan "," atau ";".
Titik koma menekan keluaran dari perintah. Di akhir baris perintah,
"," diasumsikan, jika ";" hilang.
\end{eulercomment}
\begin{eulerprompt}
>g:=9.81; t:=2.5; 1/2*g*t^2
\end{eulerprompt}
\begin{euleroutput}
  30.65625
\end{euleroutput}
\begin{eulercomment}
EMT menggunakan sintaks pemrograman untuk ekspresi. Memasuki

\end{eulercomment}
\begin{eulerformula}
\[
e ^ 2(\frac{1}{3+4\log(0.6)}+\frac{1}{7})
\]
\end{eulerformula}
\begin{eulercomment}
Anda harus mengatur tanda kurung yang benar dan menggunakan / untuk
pecahan. Perhatikan tanda kurung yang disorot untuk bantuan.
Perhatikan bahwa konstanta Euler e dinamai E dalam EMT.
\end{eulercomment}
\begin{eulerprompt}
>E^2*(1/(3+4*log(0.6))+1/7)
\end{eulerprompt}
\begin{euleroutput}
  8.77908249441
\end{euleroutput}
\begin{eulercomment}
Untuk menghitung ekspresi yang rumit seperti

\end{eulercomment}
\begin{eulerformula}
\[
(\frac{\frac 17+\frac 18 + 2}{\frac 13+\frac 12})^ 2\ pi
\]
\end{eulerformula}
\begin{eulercomment}
Anda harus memasukkannya dalam bentuk baris.
\end{eulercomment}
\begin{eulerprompt}
>((1/7 + 1/8 + 2) / (1/3 + 1/2))^2 * pi
\end{eulerprompt}
\begin{euleroutput}
  23.2671801626
\end{euleroutput}
\begin{eulercomment}
Tempatkan tanda kurung dengan hati-hati di sekitar sub-ekspresi yang
perlu dihitung terlebih dahulu. EMT membantu Anda dengan menyoroti
ekspresi bahwa kurung tutup selesai. Anda juga harus memasukkan nama
"pi" untuk huruf Yunani pi.

Hasil dari perhitungan ini adalah bilangan floating point. Ini secara
default dicetak dengan akurasi sekitar 12 digit. Di baris perintah
berikut, kita juga belajar bagaimana kita bisa merujuk ke hasil
sebelumnya dalam baris yang sama.
\end{eulercomment}
\begin{eulerprompt}
>1/3+1/7, fraction %
\end{eulerprompt}
\begin{euleroutput}
  0.47619047619
  10/21
\end{euleroutput}
\begin{eulercomment}
Perintah Euler bisa berupa ekspresi atau perintah primitif. Ekspresi
dibuat dari operator dan fungsi. Jika perlu, itu harus berisi tanda
kurung untuk memaksa urutan eksekusi yang benar. Jika ragu, menetapkan
braket adalah ide yang bagus. Perhatikan bahwa EMT menampilkan tanda
kurung buka dan tutup saat mengedit baris perintah.
\end{eulercomment}
\begin{eulerprompt}
>(cos(pi/4)+1)^3*(sin(pi/4)+1)^2
\end{eulerprompt}
\begin{euleroutput}
  14.4978445072
\end{euleroutput}
\begin{eulercomment}
Operator numerik Euler termasuk

\end{eulercomment}
\begin{eulerttcomment}
 + unary atau operator plus
 - unary atau operator minus
 *, /
 . produk matriks
 a ^ b pangkat untuk positif a atau bilangan bulat b (a ** b bekerja
\end{eulerttcomment}
\begin{eulercomment}
juga)\\
\end{eulercomment}
\begin{eulerttcomment}
 n! operator faktorial
\end{eulerttcomment}
\begin{eulercomment}

dan masih banyak lagi.

Berikut beberapa fungsi yang mungkin Anda perlukan. Masih banyak lagi.

\end{eulercomment}
\begin{eulerttcomment}
 sin, cos, tan, atan, asin, acos, rad, deg
 log, exp, log10, sqrt, logbase
 bin, logbin, logfac, mod, floor, ceil, round, abs, sign
 konj, re, im, arg, konj, nyata, kompleks
 beta, betai, gamma, complexgamma, ellrf, ellf, ellrd, elle
 bitand, bitor, bitxor, bitnot
\end{eulerttcomment}
\begin{eulercomment}

Beberapa perintah memiliki alias, mis. ln untuk log.
\end{eulercomment}
\begin{eulerprompt}
>ln(E^2), arctan(tan(0.5))
\end{eulerprompt}
\begin{euleroutput}
  2
  0.5
\end{euleroutput}
\begin{eulerprompt}
>sin(30°)
\end{eulerprompt}
\begin{euleroutput}
  0.5
\end{euleroutput}
\begin{eulercomment}
Pastikan untuk menggunakan tanda kurung (tanda kurung bulat), setiap
kali ada keraguan tentang urutan eksekusi! Berikut ini tidak sama
dengan (2 \textasciicircum{} 3) \textasciicircum{} 4, yang merupakan default untuk 2 \textasciicircum{} 3 \textasciicircum{} 4 di EMT
(beberapa sistem numerik melakukannya dengan cara lain).
\end{eulercomment}
\begin{eulerprompt}
>2^3^4, (2^3)^4, 2^(3^4)
\end{eulerprompt}
\begin{euleroutput}
  2.41785163923e+24
  4096
  2.41785163923e+24
\end{euleroutput}
\eulersubheading{Bilangan Nyata}
\begin{eulercomment}
Tipe data utama di Euler adalah bilangan real. Real direpresentasikan
dalam format IEEE dengan akurasi sekitar 16 digit desimal.
\end{eulercomment}
\begin{eulerprompt}
>longest 1/3
\end{eulerprompt}
\begin{euleroutput}
       0.3333333333333333 
\end{euleroutput}
\begin{eulercomment}
Representasi ganda internal membutuhkan 8 byte.
\end{eulercomment}
\begin{eulerprompt}
>printdual(1/3)
\end{eulerprompt}
\begin{euleroutput}
  1.0101010101010101010101010101010101010101010101010101*2^-2
\end{euleroutput}
\begin{eulerprompt}
>printhex(1/3)
\end{eulerprompt}
\begin{euleroutput}
  5.5555555555554*16^-1
\end{euleroutput}
\eulersubheading{String}
\begin{eulercomment}
Sebuah string di Euler didefinisikan dengan "...".
\end{eulercomment}
\begin{eulerprompt}
>"Sebuah string bisa berisi apa saja."
\end{eulerprompt}
\begin{euleroutput}
  Sebuah string bisa berisi apa saja.
\end{euleroutput}
\begin{eulercomment}
String bisa digabungkan dengan \textbar{} atau dengan +. Ini juga berfungsi
dengan angka, yang diubah menjadi string dalam kasus itu.
\end{eulercomment}
\begin{eulerprompt}
>"The area of the circle with radius " + 2 + " cm is " + pi*4 + " cm^2."
\end{eulerprompt}
\begin{euleroutput}
  The area of the circle with radius 2 cm is 12.5663706144 cm^2.
\end{euleroutput}
\begin{eulercomment}
Fungsi cetak juga mengubah angka menjadi string. Ini bisa mengambil
sejumlah digit dan sejumlah tempat (0 untuk output padat), dan secara
optimal satu unit.
\end{eulercomment}
\begin{eulerprompt}
>"Golden Ratio : " + print((1+sqrt(5))/2,5,0)
\end{eulerprompt}
\begin{euleroutput}
  Golden Ratio : 1.61803
\end{euleroutput}
\begin{eulercomment}
Tidak ada string khusus, yang tidak dicetak. Itu dikembalikan oleh
beberapa fungsi, ketika hasilnya tidak penting. (Ini dikembalikan
secara otomatis, jika fungsi tidak memiliki pernyataan pengembalian.)
\end{eulercomment}
\begin{eulerprompt}
>none
\end{eulerprompt}
\begin{eulercomment}
Untuk mengonversi string menjadi angka, cukup evaluasi. Ini berfungsi
untuk ekspresi juga (lihat di bawah).
\end{eulercomment}
\begin{eulerprompt}
>"1234.5"()
\end{eulerprompt}
\begin{euleroutput}
  1234.5
\end{euleroutput}
\begin{eulercomment}
Untuk mendefinisikan vektor string, gunakan notasi vektor [...].
\end{eulercomment}
\begin{eulerprompt}
>v:=["affe","charlie","bravo"]
\end{eulerprompt}
\begin{euleroutput}
  affe
  charlie
  bravo
\end{euleroutput}
\begin{eulercomment}
Vektor string kosong dilambangkan dengan [tidak ada]. Vektor string
dapat digabungkan.
\end{eulercomment}
\begin{eulerprompt}
>w:=[none]; w|v|v
\end{eulerprompt}
\begin{euleroutput}
  affe
  charlie
  bravo
  affe
  charlie
  bravo
\end{euleroutput}
\begin{eulercomment}
String dapat berisi karakter Unicode. Secara internal, string ini
berisi kode UTF-8. Untuk menghasilkan string seperti itu, gunakan u
"..." dan salah satu entitas HTML.

String unicode dapat digabungkan seperti string lainnya.
\end{eulercomment}
\begin{eulerprompt}
>u"&alpha; = " + 45 + u"&deg;" // pdfLaTeX mungkin gagal menampilkan secara benar
\end{eulerprompt}
\begin{euleroutput}
  α = 45°
\end{euleroutput}
\begin{eulercomment}
I
\end{eulercomment}
\begin{eulercomment}
Di komentar, entitas yang sama seperti \& alpha ;, \& beta; dll. dapat
digunakan. Ini mungkin alternatif cepat untuk Latex. (Lebih detail
tentang komentar di bawah).
\end{eulercomment}
\begin{eulercomment}
Ada beberapa fungsi untuk membuat atau menganalisis string unicode.
Fungsi strtochar () akan mengenali string Unicode, dan
menerjemahkannya dengan benar.
\end{eulercomment}
\begin{eulerprompt}
>v=strtochar(u"&Auml; is a German letter")
\end{eulerprompt}
\begin{euleroutput}
  [196,  32,  105,  115,  32,  97,  32,  71,  101,  114,  109,  97,  110,
  32,  108,  101,  116,  116,  101,  114]
\end{euleroutput}
\begin{eulercomment}
Hasilnya adalah vektor bilangan Unicode. Fungsi kebalikannya adalah
chartoutf ().
\end{eulercomment}
\begin{eulerprompt}
>v[1]=strtochar(u"&Uuml;")[1]; chartoutf(v)
\end{eulerprompt}
\begin{euleroutput}
  Ü is a German letter
\end{euleroutput}
\begin{eulercomment}
Fungsi utf () dapat menerjemahkan string dengan entitas dalam variabel
menjadi string Unicode.
\end{eulercomment}
\begin{eulerprompt}
>s="We have &alpha;=&beta;."; utf(s) // pdfLaTeX mungkin gagal menampilkan secara benar
\end{eulerprompt}
\begin{euleroutput}
  We have α=β.
\end{euleroutput}
\begin{eulercomment}
Dimungkinkan juga untuk menggunakan entitas numerik.
\end{eulercomment}
\begin{eulerprompt}
>u"&#196;hnliches"
\end{eulerprompt}
\begin{euleroutput}
  Ähnliches
\end{euleroutput}
\eulersubheading{Nilai Boolean}
\begin{eulercomment}
Nilai Boolean diwakili dengan 1 = true atau 0 = false di Euler. String
dapat dibandingkan, seperti halnya angka.
\end{eulercomment}
\begin{eulerprompt}
>2<1, "apel"<"banana"
\end{eulerprompt}
\begin{euleroutput}
  0
  1
\end{euleroutput}
\begin{eulercomment}
"dan" adalah operator "\&\&" dan "atau" adalah operator "\textbar{}\textbar{}", seperti
dalam bahasa C. (Kata "dan" dan "atau" hanya dapat digunakan dalam
kondisi untuk "jika".)
\end{eulercomment}
\begin{eulerprompt}
>2<E && E<3
\end{eulerprompt}
\begin{euleroutput}
  1
\end{euleroutput}
\begin{eulercomment}
Operator Boolean mematuhi aturan bahasa matriks.
\end{eulercomment}
\begin{eulerprompt}
>(1:10)>5, nonzeros(%)
\end{eulerprompt}
\begin{euleroutput}
  [0,  0,  0,  0,  0,  1,  1,  1,  1,  1]
  [6,  7,  8,  9,  10]
\end{euleroutput}
\begin{eulercomment}
Anda dapat menggunakan fungsi nonzeros () untuk mengekstrak elemen
tertentu dari vektor. Dalam contoh, kami menggunakan isprime bersyarat
(n).
\end{eulercomment}
\begin{eulerprompt}
>N=2|3:2:99 // N berisi elemen 2 dan bilangan2 ganjil dari 3 s.d. 99
\end{eulerprompt}
\begin{euleroutput}
  [2,  3,  5,  7,  9,  11,  13,  15,  17,  19,  21,  23,  25,  27,  29,
  31,  33,  35,  37,  39,  41,  43,  45,  47,  49,  51,  53,  55,  57,
  59,  61,  63,  65,  67,  69,  71,  73,  75,  77,  79,  81,  83,  85,
  87,  89,  91,  93,  95,  97,  99]
\end{euleroutput}
\begin{eulerprompt}
>N[nonzeros(isprime(N))] //pilih anggota2 N yang prima
\end{eulerprompt}
\begin{euleroutput}
  [2,  3,  5,  7,  11,  13,  17,  19,  23,  29,  31,  37,  41,  43,  47,
  53,  59,  61,  67,  71,  73,  79,  83,  89,  97]
\end{euleroutput}
\eulersubheading{Format Keluaran}
\begin{eulercomment}
Format keluaran default EMT mencetak 12 digit. Untuk memastikan bahwa
kami melihat default, kami mengatur ulang format.
\end{eulercomment}
\begin{eulerprompt}
>defformat; pi
\end{eulerprompt}
\begin{euleroutput}
  3.14159265359
\end{euleroutput}
\begin{eulercomment}
Secara internal, EMT menggunakan standar IEEE untuk bilangan ganda
dengan sekitar 16 digit desimal. Untuk melihat jumlah digit secara
lengkap, gunakan perintah "longestformat", atau gunakan operator
"longest" untuk menampilkan hasil dalam format terpanjang.
\end{eulercomment}
\begin{eulerprompt}
>longest pi
\end{eulerprompt}
\begin{euleroutput}
        3.141592653589793 
\end{euleroutput}
\begin{eulercomment}
Berikut adalah representasi heksadesimal internal dari bilangan ganda.
\end{eulercomment}
\begin{eulerprompt}
>printhex(pi)
\end{eulerprompt}
\begin{euleroutput}
  3.243F6A8885A30*16^0
\end{euleroutput}
\begin{eulercomment}
Format keluaran dapat diubah secara permanen dengan perintah format.
\end{eulercomment}
\begin{eulerprompt}
>format(12,5); 1/3, pi, sin(1)
\end{eulerprompt}
\begin{euleroutput}
      0.33333 
      3.14159 
      0.84147 
\end{euleroutput}
\begin{eulercomment}
Standarnya adalah format (12).
\end{eulercomment}
\begin{eulerprompt}
>format(12); 1/3
\end{eulerprompt}
\begin{euleroutput}
  0.333333333333
\end{euleroutput}
\begin{eulercomment}
Fungsi seperti "shortestformat", "shortformat", "longformat" bekerja
untuk vektor dengan cara berikut.
\end{eulercomment}
\begin{eulerprompt}
>shortestformat; random(3,8)
\end{eulerprompt}
\begin{euleroutput}
    0.66    0.2   0.89   0.28   0.53   0.31   0.44    0.3 
    0.28   0.88   0.27    0.7   0.22   0.45   0.31   0.91 
    0.19   0.46  0.095    0.6   0.43   0.73   0.47   0.32 
\end{euleroutput}
\begin{eulercomment}
Format default untuk skalar adalah format (12). Tapi ini bisa diubah.
\end{eulercomment}
\begin{eulerprompt}
>setscalarformat(5); pi
\end{eulerprompt}
\begin{euleroutput}
  3.1416
\end{euleroutput}
\begin{eulercomment}
Fungsi "format terpanjang" mengatur format skalar juga.
\end{eulercomment}
\begin{eulerprompt}
>longestformat; pi
\end{eulerprompt}
\begin{euleroutput}
  3.141592653589793
\end{euleroutput}
\begin{eulercomment}
Sebagai referensi, berikut adalah daftar format keluaran terpenting.

\end{eulercomment}
\begin{eulerttcomment}
 format terpendek format pendek format panjang, format terpanjang
 format (panjang, digit) format yang baik (panjang)
 fracformat (panjang)
 defformat
\end{eulerttcomment}
\begin{eulercomment}

Akurasi internal EMT adalah sekitar 16 tempat desimal, yang merupakan
standar IEEE. Angka disimpan dalam format internal ini.

Tetapi format keluaran EMT dapat diatur dengan cara yang fleksibel.
\end{eulercomment}
\begin{eulerprompt}
>longestformat; pi,
\end{eulerprompt}
\begin{euleroutput}
  3.141592653589793
\end{euleroutput}
\begin{eulerprompt}
>format(10,5); pi
\end{eulerprompt}
\begin{euleroutput}
    3.14159 
\end{euleroutput}
\begin{eulercomment}
Standarnya adalah defformat ().
\end{eulercomment}
\begin{eulerprompt}
>defformat; // default
\end{eulerprompt}
\begin{eulercomment}
Ada operator pendek yang hanya mencetak satu nilai. Operator
"terpanjang" akan mencetak semua digit nomor yang valid.
\end{eulercomment}
\begin{eulerprompt}
>longest pi^2/2
\end{eulerprompt}
\begin{euleroutput}
        4.934802200544679 
\end{euleroutput}
\begin{eulercomment}
Ada juga operator singkat untuk mencetak hasil dalam format pecahan.
Kami telah menggunakannya di atas.
\end{eulercomment}
\begin{eulerprompt}
>fraction 1+1/2+1/3+1/4
\end{eulerprompt}
\begin{euleroutput}
  25/12
\end{euleroutput}
\begin{eulercomment}
Karena format internal menggunakan cara biner untuk menyimpan angka,
nilai 0.1 tidak akan direpresentasikan dengan tepat. Kesalahan
bertambah sedikit, seperti yang Anda lihat dalam perhitungan berikut.
\end{eulercomment}
\begin{eulerprompt}
>longest 0.1+0.1+0.1+0.1+0.1+0.1+0.1+0.1+0.1+0.1-1
\end{eulerprompt}
\begin{euleroutput}
   -1.110223024625157e-16 
\end{euleroutput}
\begin{eulercomment}
Tetapi dengan "longformat" default Anda tidak akan melihat ini. Untuk
kenyamanan, keluaran angka yang sangat kecil adalah 0.
\end{eulercomment}
\begin{eulerprompt}
>0.1+0.1+0.1+0.1+0.1+0.1+0.1+0.1+0.1+0.1-1
\end{eulerprompt}
\begin{euleroutput}
  0
\end{euleroutput}
\eulerheading{Ekspresi}
\begin{eulercomment}
String atau nama dapat digunakan untuk menyimpan ekspresi matematika,
yang dapat dievaluasi oleh EMT. Untuk ini, gunakan tanda kurung
setelah ekspresi. Jika Anda bermaksud menggunakan string sebagai
ekspresi, gunakan konvensi untuk menamainya "fx" atau "fxy" dll.
Ekspresi lebih diutamakan daripada fungsi.

Variabel global dapat digunakan dalam evaluasi.
\end{eulercomment}
\begin{eulerprompt}
>r:=2; fx:="pi*r^2"; longest fx()
\end{eulerprompt}
\begin{euleroutput}
        12.56637061435917 
\end{euleroutput}
\begin{eulercomment}
Parameter ditetapkan ke x, y, dan z dalam urutan itu. Parameter
tambahan dapat ditambahkan menggunakan parameter yang ditetapkan.
\end{eulercomment}
\begin{eulerprompt}
>fx:="a*sin(x)^2"; fx(5,a=-1)
\end{eulerprompt}
\begin{euleroutput}
  -0.919535764538
\end{euleroutput}
\begin{eulercomment}
Perhatikan bahwa ekspresi akan selalu menggunakan variabel global,
meskipun ada variabel dalam fungsi dengan nama yang sama. (Jika tidak,
evaluasi ekspresi dalam fungsi dapat memiliki hasil yang sangat
membingungkan bagi pengguna yang memanggil fungsi tersebut.)
\end{eulercomment}
\begin{eulerprompt}
>at:=4; function f(expr,x,at) := expr(x); ...
>f("at*x^2",3,5) // computes 4*3^2 not 5*3^2
\end{eulerprompt}
\begin{euleroutput}
  36
\end{euleroutput}
\begin{eulercomment}
Jika Anda ingin menggunakan nilai lain untuk "at" daripada nilai
global, Anda perlu menambahkan "at = value".
\end{eulercomment}
\begin{eulerprompt}
>at:=4; function f(expr,x,a) := expr(x,at=a); ...
>f("at*x^2",3,5)
\end{eulerprompt}
\begin{euleroutput}
  45
\end{euleroutput}
\begin{eulercomment}
Sebagai referensi, kami berkomentar bahwa koleksi panggilan (dibahas
di tempat lain) dapat berisi ekspresi. Jadi contoh diatas bisa kita
buat sebagai berikut.
\end{eulercomment}
\begin{eulerprompt}
>at:=4; function f(expr,x) := expr(x); ...
>f(\{\{"at*x^2",at=5\}\},3)
\end{eulerprompt}
\begin{euleroutput}
  45
\end{euleroutput}
\begin{eulercomment}
Ekspresi dalam x sering digunakan seperti halnya fungsi.\\
Perhatikan bahwa mendefinisikan fungsi dengan nama yang sama seperti
ekspresi simbolik global akan menghapus variabel ini untuk menghindari
kebingungan antara ekspresi simbolik dan fungsi.
\end{eulercomment}
\begin{eulerprompt}
>f &= 5*x;
>function f(x) := 6*x;
>f(2)
\end{eulerprompt}
\begin{euleroutput}
  12
\end{euleroutput}
\begin{eulercomment}
Dengan cara konvensi, ekspresi simbolik atau numerik harus diberi nama
fx, fxy dll. Skema penamaan ini tidak boleh digunakan untuk fungsi.
\end{eulercomment}
\begin{eulerprompt}
>fx &= diff(x^x,x); $&fx
\end{eulerprompt}
\begin{eulercomment}
Bentuk ekspresi khusus memungkinkan variabel apa pun sebagai parameter
tanpa nama untuk mengevaluasi ekspresi, tidak hanya "x", "y", dll.
Untuk ini, mulailah ekspresi dengan "@ (variabel) ...".
\end{eulercomment}
\begin{eulerprompt}
>"@(a,b) a^2+b^2", %(4,5)
\end{eulerprompt}
\begin{euleroutput}
  @(a,b) a^2+b^2
  41
\end{euleroutput}
\begin{eulercomment}
Hal ini memungkinkan untuk memanipulasi ekspresi dalam variabel lain
untuk fungsi EMT yang membutuhkan ekspresi dalam "x".

Cara paling dasar untuk mendefinisikan fungsi sederhana adalah dengan
menyimpan rumusnya dalam ekspresi simbolik atau numerik. Jika variabel
utamanya adalah x, ekspresi tersebut dapat dievaluasi seperti fungsi.

Seperti yang Anda lihat pada contoh berikut, variabel global terlihat
selama evaluasi.
\end{eulercomment}
\begin{eulerprompt}
>fx &= x^3-a*x;  ...
>a=1.2; fx(0.5)
\end{eulerprompt}
\begin{euleroutput}
  -0.475
\end{euleroutput}
\begin{eulercomment}
Semua variabel lain dalam ekspresi dapat ditentukan dalam evaluasi
menggunakan parameter yang ditetapkan.
\end{eulercomment}
\begin{eulerprompt}
>fx(0.5,a=1.1)
\end{eulerprompt}
\begin{euleroutput}
  -0.425
\end{euleroutput}
\begin{eulercomment}
Ekspresi tidak perlu simbolis. Ini diperlukan, jika ekspresi berisi
fungsi, yang hanya dikenal di kernel numerik, bukan di Maxima.

\begin{eulercomment}
\eulerheading{Matematika Simbolis}
\begin{eulercomment}
EMT melakukan matematika simbolis dengan bantuan Maxima. Untuk
detailnya, mulailah dengan tutorial berikut, atau telusuri referensi
untuk Maxima. Para ahli di Maxima harus memperhatikan bahwa ada
perbedaan dalam sintaks antara sintaks asli dari Maxima dan sintaks
default dari ekspresi simbolik di EMT.

Matematika simbolik terintegrasi mulus ke dalam Euler dengan \&.
Ekspresi apa pun yang dimulai dengan \& adalah ekspresi simbolis. Itu
dievaluasi dan dicetak oleh Maxima.

Pertama-tama, Maxima memiliki aritmatika "tak terbatas" yang dapat
menangani angka yang sangat besar.
\end{eulercomment}
\begin{eulerprompt}
>$&44!
\end{eulerprompt}
\begin{eulercomment}
Dengan cara ini, Anda dapat menghitung hasil yang besar dengan tepat.
Mari kita hitung

\end{eulercomment}
\begin{eulerformula}
\[
C (44,10) = \frac{44!}{34!\cdot10!}
\]
\end{eulerformula}
\begin{eulerprompt}
>$& 44!/(34!*10!) // nilai C(44,10)
\end{eulerprompt}
\begin{eulercomment}
Tentu saja, Maxima memiliki fungsi yang lebih efisien untuk ini
(seperti halnya bagian numerik EMT).
\end{eulercomment}
\begin{eulerprompt}
>$binomial(44,10) //menghitung C(44,10) menggunakan fungsi binomial()
\end{eulerprompt}
\begin{eulercomment}
Untuk mempelajari lebih lanjut tentang fungsi tertentu, klik dua kali
di atasnya. Misalnya, coba klik dua kali pada "\& binomial" di baris
perintah sebelumnya. Ini membuka dokumentasi Maxima yang disediakan
oleh penulis program itu.

Anda akan belajar bahwa yang berikut ini juga berfungsi.

\end{eulercomment}
\begin{eulerformula}
\[
C (x, 3) = \frac {x!} {(x-3)! 3!} = \frac {(x-2) (x-1) x} {6}
\]
\end{eulerformula}
\begin{eulerprompt}
>$binomial(x,3) // C(x,3)
\end{eulerprompt}
\begin{eulercomment}
Jika Anda ingin mengganti x dengan nilai tertentu, gunakan "dengan".
\end{eulercomment}
\begin{eulerprompt}
>$&binomial(x,3) with x=10 // substitusi x=10 ke C(x,3)
\end{eulerprompt}
\begin{eulercomment}
Dengan begitu, Anda bisa menggunakan solusi persamaan di persamaan
lain.

Ekspresi simbolik dicetak oleh Maxima dalam bentuk 2D. Alasannya
adalah adanya bendera simbolis khusus dalam string tersebut.

Seperti yang akan Anda lihat pada contoh sebelumnya dan berikut, jika
Anda menginstal LaTeX, Anda dapat mencetak ekspresi simbolik dengan
Latex. Jika tidak, perintah berikut akan mengeluarkan pesan kesalahan.

Untuk mencetak ekspresi simbolik dengan LaTeX, gunakan \textdollar{} infront dari
\& (atau Anda dapat menghilangkan \&) sebelum perintah. Jangan
menjalankan perintah Maxima dengan \textdollar{}, jika Anda belum menginstal
LaTeX.
\end{eulercomment}
\begin{eulerprompt}
>$(3+x)/(x^2+1)
\end{eulerprompt}
\begin{eulercomment}
Ekspresi simbolik diurai oleh Euler. Jika Anda membutuhkan sintaks
kompleks dalam satu ekspresi, Anda dapat mengapit ekspresi dalam
"...". Menggunakan lebih dari sekedar ekspresi sederhana dimungkinkan,
tetapi sangat tidak disarankan.
\end{eulercomment}
\begin{eulerprompt}
>&"v := 5; v^2"
\end{eulerprompt}
\begin{euleroutput}
  
                                    25
  
\end{euleroutput}
\begin{eulercomment}
Untuk kelengkapan, kami menyatakan bahwa ekspresi simbolik dapat
digunakan dalam program, tetapi perlu diapit tanda kutip. Selain itu,
jauh lebih efektif untuk memanggil Maxima pada waktu kompilasi jika
memungkinkan.
\end{eulercomment}
\begin{eulerprompt}
>$&expand((1+x)^4), $&factor(diff(%,x)) // diff: turunan, factor: faktor
\end{eulerprompt}
\begin{eulercomment}
Sekali lagi,\% mengacu pada hasil sebelumnya.

Untuk mempermudah, kami menyimpan solusi ke variabel simbolik.
Variabel simbolik didefinisikan dengan "\& =".
\end{eulercomment}
\begin{eulerprompt}
>fx &= (x+1)/(x^4+1); $&fx
\end{eulerprompt}
\begin{eulercomment}
Ekspresi simbolik dapat digunakan dalam ekspresi simbolik lainnya.
\end{eulercomment}
\begin{eulerprompt}
>$&factor(diff(fx,x))
\end{eulerprompt}
\begin{eulercomment}
Masukan langsung dari perintah Maxima juga tersedia. Mulai baris
perintah dengan "::". Sintaks Maxima disesuaikan dengan sintaks EMT
(disebut "mode kompatibilitas").
\end{eulercomment}
\begin{eulerprompt}
>&factor(20!)
\end{eulerprompt}
\begin{euleroutput}
  
                           2432902008176640000
  
\end{euleroutput}
\begin{eulerprompt}
>::: factor(10!)
\end{eulerprompt}
\begin{euleroutput}
  
                                 8  4  2
                                2  3  5  7
  
\end{euleroutput}
\begin{eulerprompt}
>:: factor(20!)
\end{eulerprompt}
\begin{euleroutput}
  
                          18  8  4  2
                         2   3  5  7  11 13 17 19
  
\end{euleroutput}
\begin{eulercomment}
Jika Anda ahli dalam Maxima, Anda mungkin ingin menggunakan sintaks
asli Maxima. Anda dapat melakukan ini dengan ":::".
\end{eulercomment}
\begin{eulerprompt}
>::: av:g$ av^2;
\end{eulerprompt}
\begin{euleroutput}
  
                                     2
                                    g
  
\end{euleroutput}
\begin{eulerprompt}
>fx &= x^3*exp(x), $fx
\end{eulerprompt}
\begin{euleroutput}
  
                                   3  x
                                  x  E
  
\end{euleroutput}
\begin{eulercomment}
Variabel semacam itu dapat digunakan dalam ekspresi simbolik lainnya.
Perhatikan, bahwa dalam perintah berikut, sisi kanan \& = dievaluasi
sebelum penugasan ke Fx.
\end{eulercomment}
\begin{eulerprompt}
>&(fx with x=5), $%, &float(%)
\end{eulerprompt}
\begin{euleroutput}
  
                                       5
                                  125 E
  
  
                            18551.64488782208
  
\end{euleroutput}
\begin{eulerprompt}
>fx(5)
\end{eulerprompt}
\begin{euleroutput}
  18551.6448878
\end{euleroutput}
\begin{eulercomment}
Untuk evaluasi ekspresi dengan nilai variabel tertentu, Anda dapat
menggunakan operator "dengan".

Baris perintah berikut juga menunjukkan bahwa Maxima bisa mengevaluasi
ekspresi secara numerik dengan float ().
\end{eulercomment}
\begin{eulerprompt}
>&(fx with x=10)-(fx with x=5), &float(%)
\end{eulerprompt}
\begin{euleroutput}
  
                                  10        5
                            1000 E   - 125 E
  
  
                           2.20079141499189e+7
  
\end{euleroutput}
\begin{eulerprompt}
>$factor(diff(fx,x,2))
\end{eulerprompt}
\begin{eulercomment}
Untuk mendapatkan kode Latex untuk ekspresi, Anda dapat menggunakan
perintah tex.
\end{eulercomment}
\begin{eulerprompt}
>tex(fx)
\end{eulerprompt}
\begin{euleroutput}
  x^3\(\backslash\),e^\{x\}
\end{euleroutput}
\begin{eulercomment}
Ekspresi simbolik dapat dievaluasi seperti ekspresi numerik.
\end{eulercomment}
\begin{eulerprompt}
>fx(0.5)
\end{eulerprompt}
\begin{euleroutput}
  0.206090158838
\end{euleroutput}
\begin{eulercomment}
Dalam ekspresi simbolik, ini tidak berfungsi, karena Maxima tidak
mendukungnya. Sebagai gantinya, gunakan sintaks "with" (bentuk yang
lebih bagus dari perintah at (...) Maxima).
\end{eulercomment}
\begin{eulerprompt}
>$&fx with x=1/2
\end{eulerprompt}
\begin{eulercomment}
Penugasan juga bisa bersifat simbolis.
\end{eulercomment}
\begin{eulerprompt}
>$&fx with x=1+t
\end{eulerprompt}
\begin{eulercomment}
Perintah memecahkan memecahkan ekspresi simbolik untuk variabel di
Maxima. Hasilnya adalah vektor solusi.
\end{eulercomment}
\begin{eulerprompt}
>$&solve(x^2+x=4,x)
\end{eulerprompt}
\begin{eulercomment}
Bandingkan dengan perintah "selesaikan" numerik di Euler, yang
membutuhkan nilai awal, dan secara opsional nilai target.
\end{eulercomment}
\begin{eulerprompt}
>solve("x^2+x",1,y=4)
\end{eulerprompt}
\begin{euleroutput}
  1.56155281281
\end{euleroutput}
\begin{eulercomment}
Nilai numerik dari solusi simbolik dapat dihitung dengan evaluasi
hasil simbolik. Euler akan membaca tugas x = dll. Jika Anda tidak
memerlukan hasil numerik untuk perhitungan lebih lanjut, Anda juga
dapat membiarkan Maxima menemukan nilai numerik.
\end{eulercomment}
\begin{eulerprompt}
>sol &= solve(x^2+2*x=4,x); $&sol, sol(), $&float(sol)
\end{eulerprompt}
\begin{euleroutput}
  [-3.23607,  1.23607]
\end{euleroutput}
\begin{eulercomment}
Untuk mendapatkan solusi simbolik tertentu, seseorang dapat
menggunakan "dengan" dan indeks.
\end{eulercomment}
\begin{eulerprompt}
>$&solve(x^2+x=1,x), x2 &= x with %[2]; $&x2
\end{eulerprompt}
\begin{eulercomment}
Untuk menyelesaikan sistem persamaan, gunakan vektor persamaan.
Hasilnya adalah vektor solusi.
\end{eulercomment}
\begin{eulerprompt}
>sol &= solve([x+y=3,x^2+y^2=5],[x,y]); $&sol, $&x*y with sol[1]
\end{eulerprompt}
\begin{eulercomment}
Ekspresi simbolis dapat memiliki bendera, yang menunjukkan perlakuan
khusus dalam Maxima. Beberapa flag juga dapat digunakan sebagai
perintah, yang lainnya tidak. Bendera ditambahkan dengan "\textbar{}" (bentuk
yang lebih bagus dari "ev (..., flags)")
\end{eulercomment}
\begin{eulerprompt}
>$& diff((x^3-1)/(x+1),x) //turunan bentuk pecahan
>$& diff((x^3-1)/(x+1),x) | ratsimp //menyederhanakan pecahan
>$&factor(%)
\end{eulerprompt}
\eulerheading{Fungsi}
\begin{eulercomment}
Dalam EMT, fungsi adalah program yang ditentukan dengan perintah
"fungsi". Ini bisa menjadi fungsi satu baris atau fungsi multiline.\\
Fungsi satu baris dapat berupa numerik atau simbolik. Fungsi satu
baris numerik ditentukan oleh ": =".
\end{eulercomment}
\begin{eulerprompt}
>function f(x) := x*sqrt(x^2+1)
\end{eulerprompt}
\begin{eulercomment}
Untuk gambaran umum, kami menunjukkan semua kemungkinan definisi untuk
fungsi satu baris. Sebuah fungsi dapat dievaluasi sama seperti fungsi
Euler bawaan lainnya.
\end{eulercomment}
\begin{eulerprompt}
>f(2)
\end{eulerprompt}
\begin{euleroutput}
  4.472135955
\end{euleroutput}
\begin{eulercomment}
Fungsi ini akan bekerja untuk vektor juga, mengikuti bahasa matriks
Euler, karena ekspresi yang digunakan dalam fungsi tersebut adalah
vektorisasi.
\end{eulercomment}
\begin{eulerprompt}
>f(0:0.1:1)
\end{eulerprompt}
\begin{euleroutput}
  [0,  0.100499,  0.203961,  0.313209,  0.430813,  0.559017,  0.699714,
  0.854459,  1.0245,  1.21083,  1.41421]
\end{euleroutput}
\begin{eulercomment}
Fungsi bisa diplot. Sebagai ganti ekspresi, kita hanya perlu
memberikan nama fungsi.

Berbeda dengan ekspresi simbolik atau numerik, nama fungsi harus
diberikan dalam sebuah string.
\end{eulercomment}
\begin{eulerprompt}
>solve("f",1,y=1)
\end{eulerprompt}
\begin{euleroutput}
  0.786151377757
\end{euleroutput}
\begin{eulercomment}
Secara default, jika Anda perlu menimpa fungsi built-in, Anda harus
menambahkan kata kunci "overwrite". Menimpa fungsi built-in berbahaya
dan dapat menyebabkan masalah pada fungsi lain tergantung pada
fungsinya.

Anda masih bisa memanggil fungsi built-in sebagai "\_...", jika itu
berfungsi di inti Euler.
\end{eulercomment}
\begin{eulerprompt}
>function overwrite sin (x) := _sin(x°) // redine sine in degrees
>sin(45)
\end{eulerprompt}
\begin{euleroutput}
  0.707106781187
\end{euleroutput}
\begin{eulercomment}
Lebih baik kita menghapus definisi ulang dosa ini.
\end{eulercomment}
\begin{eulerprompt}
>forget sin; sin(pi/4)
\end{eulerprompt}
\begin{euleroutput}
  0.707106781187
\end{euleroutput}
\eulersubheading{Parameter Default}
\begin{eulercomment}
Fungsi numerik dapat memiliki parameter default.
\end{eulercomment}
\begin{eulerprompt}
>function f(x,a=1) := a*x^2
\end{eulerprompt}
\begin{eulercomment}
Menghilangkan parameter ini menggunakan nilai default.
\end{eulercomment}
\begin{eulerprompt}
>f(4)
\end{eulerprompt}
\begin{euleroutput}
  16
\end{euleroutput}
\begin{eulercomment}
Menyetelnya menimpa nilai default.
\end{eulercomment}
\begin{eulerprompt}
>f(4,5)
\end{eulerprompt}
\begin{euleroutput}
  80
\end{euleroutput}
\begin{eulercomment}
Parameter yang ditetapkan juga menimpanya. Ini digunakan oleh banyak
fungsi Euler seperti plot2d, plot3d.
\end{eulercomment}
\begin{eulerprompt}
>f(4,a=1)
\end{eulerprompt}
\begin{euleroutput}
  16
\end{euleroutput}
\begin{eulercomment}
Jika variabel bukan parameter, itu harus global. Fungsi satu baris
dapat melihat variabel global.
\end{eulercomment}
\begin{eulerprompt}
>function f(x) := a*x^2
>a=6; f(2)
\end{eulerprompt}
\begin{euleroutput}
  24
\end{euleroutput}
\begin{eulercomment}
Tetapi parameter yang ditetapkan menggantikan nilai global.

Jika argumen tidak ada dalam daftar parameter yang ditentukan
sebelumnya, itu harus dideklarasikan dengan ": ="!
\end{eulercomment}
\begin{eulerprompt}
>f(2,a:=5)
\end{eulerprompt}
\begin{euleroutput}
  20
\end{euleroutput}
\begin{eulercomment}
Fungsi simbolik didefinisikan dengan "\& =". Mereka didefinisikan di
Euler dan Maxima, dan bekerja di kedua dunia. Ekspresi yang menentukan
dijalankan melalui Maxima sebelum definisi.
\end{eulercomment}
\begin{eulerprompt}
>function g(x) &= x^3-x*exp(-x); $&g(x)
\end{eulerprompt}
\begin{eulercomment}
Fungsi simbolik dapat digunakan dalam ekspresi simbolik.
\end{eulercomment}
\begin{eulerprompt}
>$&diff(g(x),x), $&% with x=4/3
\end{eulerprompt}
\begin{eulercomment}
Mereka juga dapat digunakan dalam ekspresi numerik. Tentu saja, ini
hanya akan berfungsi jika EMT dapat menafsirkan semua yang ada di
dalam fungsi tersebut.
\end{eulercomment}
\begin{eulerprompt}
>g(5+g(1))
\end{eulerprompt}
\begin{euleroutput}
  178.635099908
\end{euleroutput}
\begin{eulercomment}
Mereka dapat digunakan untuk mendefinisikan fungsi atau ekspresi
simbolik lainnya.
\end{eulercomment}
\begin{eulerprompt}
>function G(x) &= factor(integrate(g(x),x)); $&G(c) // integrate: mengintegralkan
>solve(&g(x),0.5)
\end{eulerprompt}
\begin{euleroutput}
  0.703467422498
\end{euleroutput}
\begin{eulercomment}
Cara berikut juga berlaku, karena Euler menggunakan ekspresi simbolik
dalam fungsi g, jika tidak menemukan variabel simbolis g, dan jika ada
fungsi simbolik g.
\end{eulercomment}
\begin{eulerprompt}
>solve(&g,0.5)
\end{eulerprompt}
\begin{euleroutput}
  0.703467422498
\end{euleroutput}
\begin{eulerprompt}
>function P(x,n) &= (2*x-1)^n; $&P(x,n)
>function Q(x,n) &= (x+2)^n; $&Q(x,n)
>$&P(x,4), $&expand(%)
>P(3,4)
\end{eulerprompt}
\begin{euleroutput}
  625
\end{euleroutput}
\begin{eulerprompt}
>$&P(x,4)+ Q(x,3), $&expand(%)
>$&P(x,4)-Q(x,3), $&expand(%), $&factor(%)
>$&P(x,4)*Q(x,3), $&expand(%), $&factor(%)
>$&P(x,4)/Q(x,1), $&expand(%), $&factor(%)
>function f(x) &= x^3-x; $&f(x)
\end{eulerprompt}
\begin{eulercomment}
Dengan \& = fungsinya adalah simbolik, dan dapat digunakan dalam
ekspresi simbolik lainnya.
\end{eulercomment}
\begin{eulerprompt}
>$&integrate(f(x),x)
\end{eulerprompt}
\begin{eulercomment}
Dengan: = fungsinya adalah numerik. Contoh yang baik adalah seperti
integral pasti

\end{eulercomment}
\begin{eulerformula}
\[
f (x) = \int_1^x t^t\, dt,
\]
\end{eulerformula}
\begin{eulercomment}
yang tidak dapat dievaluasi secara simbolis.

Jika kita mendefinisikan ulang fungsi dengan kata kunci "map", ini
dapat digunakan untuk vektor x. Secara internal, fungsi ini dipanggil
untuk semua nilai x satu kali, dan hasilnya disimpan dalam vektor.
\end{eulercomment}
\begin{eulerprompt}
>function map f(x) := integrate("x^x",1,x)
>f(0:0.5:2)
\end{eulerprompt}
\begin{euleroutput}
  [-0.783431,  -0.410816,  0,  0.676863,  2.05045]
\end{euleroutput}
\begin{eulercomment}
Fungsi dapat memiliki nilai default untuk parameter.
\end{eulercomment}
\begin{eulerprompt}
>function mylog (x,base=10) := ln(x)/ln(base);
\end{eulerprompt}
\begin{eulercomment}
Sekarang fungsi tersebut dapat dipanggil dengan atau tanpa parameter
"base".
\end{eulercomment}
\begin{eulerprompt}
>mylog(100), mylog(2^6.7,2)
\end{eulerprompt}
\begin{euleroutput}
  2
  6.7
\end{euleroutput}
\begin{eulercomment}
Selain itu, dimungkinkan untuk menggunakan parameter yang ditetapkan.
\end{eulercomment}
\begin{eulerprompt}
>mylog(E^2,base=E)
\end{eulerprompt}
\begin{euleroutput}
  2
\end{euleroutput}
\begin{eulercomment}
Seringkali, kami ingin menggunakan fungsi untuk vektor di satu tempat,
dan untuk elemen individu di tempat lain. Ini dimungkinkan dengan
parameter vektor.
\end{eulercomment}
\begin{eulerprompt}
>function f([a,b]) &= a^2+b^2-a*b+b; $&f(a,b), $&f(x,y)
\end{eulerprompt}
\begin{eulercomment}
Fungsi simbolik seperti itu dapat digunakan untuk variabel simbolik.

Tetapi fungsinya juga dapat digunakan untuk vektor numerik.
\end{eulercomment}
\begin{eulerprompt}
>v=[3,4]; f(v)
\end{eulerprompt}
\begin{euleroutput}
  17
\end{euleroutput}
\begin{eulercomment}
Ada juga fungsi simbolik murni, yang tidak dapat digunakan secara
numerik.
\end{eulercomment}
\begin{eulerprompt}
>function lapl(expr,x,y) &&= diff(expr,x,2)+diff(expr,y,2)//turunan parsial kedua
\end{eulerprompt}
\begin{euleroutput}
  
                   diff(expr, y, 2) + diff(expr, x, 2)
  
\end{euleroutput}
\begin{eulerprompt}
>$&realpart((x+I*y)^4), $&lapl(%,x,y)
\end{eulerprompt}
\begin{eulercomment}
Tetapi tentu saja, mereka dapat digunakan dalam ekspresi simbolik atau
dalam definisi fungsi simbolik.
\end{eulercomment}
\begin{eulerprompt}
>function f(x,y) &= factor(lapl((x+y^2)^5,x,y)); $&f(x,y)
\end{eulerprompt}
\begin{eulercomment}
Untuk meringkas

- \&= mendefinisikan fungsi simbolik,\\
- := mendefinisikan fungsi numerik,\\
- \&\&= mendefinisikan fungsi simbolik murni.

\begin{eulercomment}
\eulerheading{Memecahkan Ekspresi}
\begin{eulercomment}
Ekspresi dapat diselesaikan secara numerik dan simbolik.

Untuk menyelesaikan ekspresi sederhana dari satu variabel, kita dapat
menggunakan fungsi Solving (). Diperlukan nilai awal untuk memulai
pencarian. Secara internal, Solving () menggunakan metode garis
potong.
\end{eulercomment}
\begin{eulerprompt}
>solve("x^2-2",1)
\end{eulerprompt}
\begin{euleroutput}
  1.41421356237
\end{euleroutput}
\begin{eulercomment}
Ini bekerja untuk ekspresi simbolik juga. Ambil fungsi berikut.
\end{eulercomment}
\begin{eulerprompt}
>$&solve(x^2=2,x)
>$&solve(x^2-2,x)
>$&solve(a*x^2+b*x+c=0,x)
>$&solve([a*x+b*y=c,d*x+e*y=f],[x,y])
>px &= 4*x^8+x^7-x^4-x; $&px
\end{eulerprompt}
\begin{eulercomment}
Sekarang kita mencari titik, di mana polinomialnya adalah 2. Dalam
Solving (), nilai target default y = 0 dapat diubah dengan variabel
yang ditetapkan.\\
Kami menggunakan y = 2 dan memeriksa dengan mengevaluasi polinomial
pada hasil sebelumnya.
\end{eulercomment}
\begin{eulerprompt}
>solve(px,1,y=2), px(%)
\end{eulerprompt}
\begin{euleroutput}
  0.966715594851
  2
\end{euleroutput}
\begin{eulercomment}
Memecahkan ekspresi simbolis dalam bentuk simbolik mengembalikan
daftar solusi. Kami menggunakan penyelesaian pemecah simbolik () yang
disediakan oleh Maxima.
\end{eulercomment}
\begin{eulerprompt}
>sol &= solve(x^2-x-1,x); $&sol
\end{eulerprompt}
\begin{eulercomment}
Cara termudah untuk mendapatkan nilai numerik adalah dengan
mengevaluasi solusi secara numerik seperti ekspresi.
\end{eulercomment}
\begin{eulerprompt}
>longest sol()
\end{eulerprompt}
\begin{euleroutput}
      -0.6180339887498949       1.618033988749895 
\end{euleroutput}
\begin{eulercomment}
Untuk menggunakan solusi secara simbolis dalam ekspresi lain, cara
termudah adalah "dengan".
\end{eulercomment}
\begin{eulerprompt}
>$&x^2 with sol[1], $&expand(x^2-x-1 with sol[2])
\end{eulerprompt}
\begin{eulercomment}
Sistem pemecahan persamaan secara simbolis dapat dilakukan dengan
vektor persamaan dan penyelesaian pemecah simbolik (). Jawabannya
adalah daftar persamaan.
\end{eulercomment}
\begin{eulerprompt}
>$&solve([x+y=2,x^3+2*y+x=4],[x,y])
\end{eulerprompt}
\begin{eulercomment}
Fungsi f () dapat melihat variabel global. Namun seringkali kita ingin
menggunakan parameter lokal.

\end{eulercomment}
\begin{eulerformula}
\[
a^ x-x^a = 0,1
\]
\end{eulerformula}
\begin{eulercomment}
dengan a = 3.
\end{eulercomment}
\begin{eulerprompt}
>function f(x,a) := x^a-a^x;
\end{eulerprompt}
\begin{eulercomment}
Salah satu cara untuk meneruskan parameter tambahan ke f () adalah
dengan menggunakan daftar dengan nama fungsi dan parameternya (cara
lainnya adalah parameter titik koma).
\end{eulercomment}
\begin{eulerprompt}
>solve(\{\{"f",3\}\},2,y=0.1)
\end{eulerprompt}
\begin{euleroutput}
  2.54116291558
\end{euleroutput}
\begin{eulercomment}
Ini juga bekerja dengan ekspresi. Tapi kemudian, elemen daftar bernama
harus digunakan. (Lebih lanjut tentang daftar di tutorial tentang
sintaks EMT).
\end{eulercomment}
\begin{eulerprompt}
>solve(\{\{"x^a-a^x",a=3\}\},2,y=0.1)
\end{eulerprompt}
\begin{euleroutput}
  2.54116291558
\end{euleroutput}
\eulerheading{Menyelesaikan Pertidaksamaan}
\begin{eulercomment}
Untuk menyelesaikan pertidaksamaan, EMT tidak akan dapat melakukannya,
melainkan dengan bantuan Maxima, artinya secara eksak (simbolik).
Perintah Maxima yang digunakan adalah fourier\_elim(), yang harus
dipanggil dengan perintah "load(fourier\_elim)" terlebih dahulu.
\end{eulercomment}
\begin{eulerprompt}
>&load(fourier_elim)
\end{eulerprompt}
\begin{euleroutput}
  
          C:/Program Files/Euler x64/maxima/share/maxima/5.35.1/share/f\(\backslash\)
  ourier_elim/fourier_elim.lisp
  
\end{euleroutput}
\begin{eulerprompt}
>$&fourier_elim([x^2 - 1>0],[x]) // x^2-1 > 0
>$&fourier_elim([x^2 - 1<0],[x]) // x^2-1 < 0
>$&fourier_elim([x^2 - 1 # 0],[x]) // x^-1 <> 0
>$&fourier_elim([x # 6],[x])
>$&fourier_elim([x < 1, x > 1],[x]) // tidak memiliki penyelesaian
>$&fourier_elim([minf < x, x < inf],[x]) // solusinya R
>$&fourier_elim([x^3 - 1 > 0],[x])
>$&fourier_elim([cos(x) < 1/2],[x]) // ??? gagal
>$&fourier_elim([y-x < 5, x - y < 7, 10 < y],[x,y]) // sistem pertidaksamaan
>$&fourier_elim([y-x < 5, x - y < 7, 10 < y],[y,x])
>$&fourier_elim((x + y < 5) and (x - y >8),[x,y])
>$&fourier_elim(((x + y < 5) and x < 1) or  (x - y >8),[x,y])
>&fourier_elim([max(x,y) > 6, x # 8, abs(y-1) > 12],[x,y])
\end{eulerprompt}
\begin{euleroutput}
  
          [6 < x, x < 8, y < - 11] or [8 < x, y < - 11]
   or [x < 8, 13 < y] or [x = y, 13 < y] or [8 < x, x < y, 13 < y]
   or [y < x, 13 < y]
  
\end{euleroutput}
\begin{eulerprompt}
>$&fourier_elim([(x+6)/(x-9) <= 6],[x])
\end{eulerprompt}
\eulerheading{Bahasa Matriks}
\begin{eulercomment}
Dokumentasi inti EMT berisi diskusi terperinci tentang bahasa matriks
Euler.

Vektor dan matriks dimasukkan dengan tanda kurung siku, elemen
dipisahkan dengan koma, baris dipisahkan dengan titik koma.
\end{eulercomment}
\begin{eulerprompt}
>A=[1,2;3,4]
\end{eulerprompt}
\begin{euleroutput}
              1             2 
              3             4 
\end{euleroutput}
\begin{eulercomment}
Produk matriks dilambangkan dengan titik.
\end{eulercomment}
\begin{eulerprompt}
>b=[3;4]
\end{eulerprompt}
\begin{euleroutput}
              3 
              4 
\end{euleroutput}
\begin{eulerprompt}
>b' // transpose b
\end{eulerprompt}
\begin{euleroutput}
  [3,  4]
\end{euleroutput}
\begin{eulerprompt}
>inv(A) //inverse A
\end{eulerprompt}
\begin{euleroutput}
             -2             1 
            1.5          -0.5 
\end{euleroutput}
\begin{eulerprompt}
>A.b //perkalian matriks
\end{eulerprompt}
\begin{euleroutput}
             11 
             25 
\end{euleroutput}
\begin{eulerprompt}
>A.inv(A)
\end{eulerprompt}
\begin{euleroutput}
              1             0 
              0             1 
\end{euleroutput}
\begin{eulercomment}
Poin utama dari bahasa matriks adalah bahwa semua fungsi dan operator
mengerjakan elemen untuk elemen.
\end{eulercomment}
\begin{eulerprompt}
>A.A
\end{eulerprompt}
\begin{euleroutput}
              7            10 
             15            22 
\end{euleroutput}
\begin{eulerprompt}
>A^2 //perpangkatan elemen2 A
\end{eulerprompt}
\begin{euleroutput}
              1             4 
              9            16 
\end{euleroutput}
\begin{eulerprompt}
>A.A.A
\end{eulerprompt}
\begin{euleroutput}
             37            54 
             81           118 
\end{euleroutput}
\begin{eulerprompt}
>power(A,3) //perpangkatan matriks
\end{eulerprompt}
\begin{euleroutput}
             37            54 
             81           118 
\end{euleroutput}
\begin{eulerprompt}
>A/A //pembagian elemen-elemen matriks yang seletak
\end{eulerprompt}
\begin{euleroutput}
              1             1 
              1             1 
\end{euleroutput}
\begin{eulerprompt}
>A/b //pembagian elemen2 A oleh elemen2 b kolom demi kolom (karena b vektor kolom)
\end{eulerprompt}
\begin{euleroutput}
       0.333333      0.666667 
           0.75             1 
\end{euleroutput}
\begin{eulerprompt}
>A\(\backslash\)b // hasilkali invers A dan b, A^(-1)b 
\end{eulerprompt}
\begin{euleroutput}
             -2 
            2.5 
\end{euleroutput}
\begin{eulerprompt}
>inv(A).b
\end{eulerprompt}
\begin{euleroutput}
             -2 
            2.5 
\end{euleroutput}
\begin{eulerprompt}
>A\(\backslash\)A   //A^(-1)A
\end{eulerprompt}
\begin{euleroutput}
              1             0 
              0             1 
\end{euleroutput}
\begin{eulerprompt}
>inv(A).A
\end{eulerprompt}
\begin{euleroutput}
              1             0 
              0             1 
\end{euleroutput}
\begin{eulerprompt}
>A*A //perkalin elemen-elemen matriks seletak
\end{eulerprompt}
\begin{euleroutput}
              1             4 
              9            16 
\end{euleroutput}
\begin{eulercomment}
Ini bukan hasil perkalian matriks, tetapi perkalian elemen dengan
elemen. Hal yang sama juga berlaku untuk vektor.
\end{eulercomment}
\begin{eulerprompt}
>b^2 // perpangkatan elemen-elemen matriks/vektor
\end{eulerprompt}
\begin{euleroutput}
              9 
             16 
\end{euleroutput}
\begin{eulercomment}
Jika salah satu operan adalah vektor atau skalar, itu diperluas dengan
cara alami.
\end{eulercomment}
\begin{eulerprompt}
>2*A
\end{eulerprompt}
\begin{euleroutput}
              2             4 
              6             8 
\end{euleroutput}
\begin{eulercomment}
Misalnya, jika operan adalah vektor kolom, elemennya diterapkan ke
semua baris A.
\end{eulercomment}
\begin{eulerprompt}
>[1,2]*A
\end{eulerprompt}
\begin{euleroutput}
              1             4 
              3             8 
\end{euleroutput}
\begin{eulercomment}
Jika itu adalah vektor baris, itu diterapkan ke semua kolom A.
\end{eulercomment}
\begin{eulerprompt}
>A*[2,3]
\end{eulerprompt}
\begin{euleroutput}
              2             6 
              6            12 
\end{euleroutput}
\begin{eulercomment}
Dapat dibayangkan perkalian ini seolah-olah vektor baris v telah
diduplikasi untuk membentuk matriks dengan ukuran yang sama dengan A.
\end{eulercomment}
\begin{eulerprompt}
>dup([1,2],2) // dup: menduplikasi/menggandakan vektor [1,2] sebanyak 2 kali (baris)
\end{eulerprompt}
\begin{euleroutput}
              1             2 
              1             2 
\end{euleroutput}
\begin{eulerprompt}
>A*dup([1,2],2) 
\end{eulerprompt}
\begin{euleroutput}
              1             4 
              3             8 
\end{euleroutput}
\begin{eulercomment}
Ini juga berlaku untuk dua vektor di mana satu adalah vektor baris dan
yang lainnya adalah vektor kolom. Kami menghitung i * j untuk i, j
dari 1 sampai 5. Triknya adalah mengalikan 1: 5 dengan transposenya.
Bahasa matriks Euler secara otomatis menghasilkan tabel nilai.
\end{eulercomment}
\begin{eulerprompt}
>(1:5)*(1:5)' // hasilkali elemen-elemen vektor baris dan vektor kolom
\end{eulerprompt}
\begin{euleroutput}
              1             2             3             4             5 
              2             4             6             8            10 
              3             6             9            12            15 
              4             8            12            16            20 
              5            10            15            20            25 
\end{euleroutput}
\begin{eulercomment}
Sekali lagi, ingatlah bahwa ini bukan hasil perkalian matriks!
\end{eulercomment}
\begin{eulerprompt}
>(1:5).(1:5)' // hasilkali vektor baris dan vektor kolom
\end{eulerprompt}
\begin{euleroutput}
  55
\end{euleroutput}
\begin{eulerprompt}
>sum((1:5)*(1:5)) // sama hasilnya
\end{eulerprompt}
\begin{euleroutput}
  55
\end{euleroutput}
\begin{eulercomment}
Bahkan operator seperti \textless{}atau == bekerja dengan cara yang sama.
\end{eulercomment}
\begin{eulerprompt}
>(1:10)<6 // menguji elemen-elemen yang kurang dari 6
\end{eulerprompt}
\begin{euleroutput}
  [1,  1,  1,  1,  1,  0,  0,  0,  0,  0]
\end{euleroutput}
\begin{eulercomment}
Misalnya, kita dapat menghitung jumlah elemen yang memenuhi kondisi
tertentu dengan fungsi sum ().
\end{eulercomment}
\begin{eulerprompt}
>sum((1:10)<6) // banyak elemen yang kurang dari 6
\end{eulerprompt}
\begin{euleroutput}
  5
\end{euleroutput}
\begin{eulercomment}
Euler memiliki operator perbandingan, seperti "==", yang memeriksa
kesetaraan.

Kami mendapatkan vektor 0 dan 1, di mana 1 berarti benar.
\end{eulercomment}
\begin{eulerprompt}
>t=(1:10)^2; t==25 //menguji elemen2 t yang sama dengan 25 (hanya ada 1)
\end{eulerprompt}
\begin{euleroutput}
  [0,  0,  0,  0,  1,  0,  0,  0,  0,  0]
\end{euleroutput}
\begin{eulercomment}
Dari vektor seperti itu, "nonzeros" memilih elemen bukan nol.

Dalam hal ini, kami mendapatkan indeks dari semua elemen yang lebih
besar dari 50.
\end{eulercomment}
\begin{eulerprompt}
>nonzeros(t>50) //indeks elemen2 t yang lebih besar daripada 50
\end{eulerprompt}
\begin{euleroutput}
  [8,  9,  10]
\end{euleroutput}
\begin{eulercomment}
Tentu saja, kita dapat menggunakan vektor indeks ini untuk mendapatkan
nilai t yang sesuai.
\end{eulercomment}
\begin{eulerprompt}
>t[nonzeros(t>50)] //elemen2 t yang lebih besar daripada 50
\end{eulerprompt}
\begin{euleroutput}
  [64,  81,  100]
\end{euleroutput}
\begin{eulercomment}
Sebagai contoh, mari kita cari semua kuadrat dari angka 1 sampai 1000,
yaitu 5 modulo 11 dan 3 modulo 13.
\end{eulercomment}
\begin{eulerprompt}
>t=1:1000; nonzeros(mod(t^2,11)==5 && mod(t^2,13)==3)
\end{eulerprompt}
\begin{euleroutput}
  [4,  48,  95,  139,  147,  191,  238,  282,  290,  334,  381,  425,
  433,  477,  524,  568,  576,  620,  667,  711,  719,  763,  810,  854,
  862,  906,  953,  997]
\end{euleroutput}
\begin{eulercomment}
EMT tidak sepenuhnya efektif untuk perhitungan integer. Ini
menggunakan titik mengambang presisi ganda secara internal. Namun,
seringkali ini sangat berguna.

Kita bisa memeriksa keutamaan. Mari kita cari tahu, berapa banyak
kuadrat ditambah 1 yang merupakan bilangan prima.
\end{eulercomment}
\begin{eulerprompt}
>t=1:1000; length(nonzeros(isprime(t^2+1)))
\end{eulerprompt}
\begin{euleroutput}
  112
\end{euleroutput}
\begin{eulercomment}
Fungsi nonzeros () hanya berfungsi untuk vektor. Untuk matriks, ada
mnonzeros ().
\end{eulercomment}
\begin{eulerprompt}
>seed(2); A=random(3,4)
\end{eulerprompt}
\begin{euleroutput}
       0.765761      0.401188      0.406347      0.267829 
        0.13673      0.390567      0.495975      0.952814 
       0.548138      0.006085      0.444255      0.539246 
\end{euleroutput}
\begin{eulercomment}
Ini mengembalikan indeks elemen, yang bukan nol.
\end{eulercomment}
\begin{eulerprompt}
>k=mnonzeros(A<0.4) //indeks elemen2 A yang kurang dari 0,4
\end{eulerprompt}
\begin{euleroutput}
              1             4 
              2             1 
              2             2 
              3             2 
\end{euleroutput}
\begin{eulercomment}
Indeks ini dapat digunakan untuk mengatur elemen ke nilai tertentu.
\end{eulercomment}
\begin{eulerprompt}
>mset(A,k,0) //mengganti elemen2 suatu matriks pada indeks tertentu
\end{eulerprompt}
\begin{euleroutput}
       0.765761      0.401188      0.406347             0 
              0             0      0.495975      0.952814 
       0.548138             0      0.444255      0.539246 
\end{euleroutput}
\begin{eulercomment}
Fungsi mset () juga dapat menyetel elemen pada indeks ke entri
beberapa matriks lainnya.
\end{eulercomment}
\begin{eulerprompt}
>mset(A,k,-random(size(A)))
\end{eulerprompt}
\begin{euleroutput}
       0.765761      0.401188      0.406347     -0.126917 
      -0.122404     -0.691673      0.495975      0.952814 
       0.548138     -0.483902      0.444255      0.539246 
\end{euleroutput}
\begin{eulercomment}
Dan dimungkinkan untuk mendapatkan elemen dalam vektor.
\end{eulercomment}
\begin{eulerprompt}
>mget(A,k)
\end{eulerprompt}
\begin{euleroutput}
  [0.267829,  0.13673,  0.390567,  0.006085]
\end{euleroutput}
\begin{eulercomment}
Fungsi berguna lainnya adalah extrema, yang mengembalikan nilai
minimal dan maksimal di setiap baris matriks dan posisinya.
\end{eulercomment}
\begin{eulerprompt}
>ex=extrema(A)
\end{eulerprompt}
\begin{euleroutput}
       0.267829             4      0.765761             1 
        0.13673             1      0.952814             4 
       0.006085             2      0.548138             1 
\end{euleroutput}
\begin{eulercomment}
Kita dapat menggunakan ini untuk mengekstrak nilai maksimal di setiap
baris.
\end{eulercomment}
\begin{eulerprompt}
>ex[,3]'
\end{eulerprompt}
\begin{euleroutput}
  [0.765761,  0.952814,  0.548138]
\end{euleroutput}
\begin{eulercomment}
Ini, tentu saja, sama dengan fungsi max ().
\end{eulercomment}
\begin{eulerprompt}
>max(A)'
\end{eulerprompt}
\begin{euleroutput}
  [0.765761,  0.952814,  0.548138]
\end{euleroutput}
\begin{eulercomment}
Tetapi dengan mget (), kita dapat mengekstrak indeks dan menggunakan
informasi ini untuk mengekstrak elemen pada posisi yang sama dari
matriks lain.
\end{eulercomment}
\begin{eulerprompt}
>j=(1:rows(A))'|ex[,4], mget(-A,j)
\end{eulerprompt}
\begin{euleroutput}
              1             1 
              2             4 
              3             1 
  [-0.765761,  -0.952814,  -0.548138]
\end{euleroutput}
\begin{eulercomment}
\begin{eulercomment}
\eulerheading{Fungsi Matriks Lainnya (Building Matrix)}
\begin{eulercomment}
Untuk membangun matriks, kita dapat menumpuk satu matriks di atas
matriks lainnya. Jika keduanya tidak memiliki jumlah kolom yang sama,
kolom yang lebih pendek diisi dengan 0.
\end{eulercomment}
\begin{eulerprompt}
>v=1:3; v_v
\end{eulerprompt}
\begin{euleroutput}
              1             2             3 
              1             2             3 
\end{euleroutput}
\begin{eulercomment}
Demikian juga, kita dapat melampirkan matriks ke sisi lain secara
berdampingan, jika keduanya memiliki jumlah baris yang sama.
\end{eulercomment}
\begin{eulerprompt}
>A=random(3,4); A|v'
\end{eulerprompt}
\begin{euleroutput}
       0.032444     0.0534171      0.595713      0.564454             1 
        0.83916      0.175552      0.396988       0.83514             2 
      0.0257573      0.658585      0.629832      0.770895             3 
\end{euleroutput}
\begin{eulercomment}
Jika mereka tidak memiliki jumlah baris yang sama, matriks yang lebih
pendek diisi dengan 0.

Ada pengecualian untuk aturan ini. Bilangan real yang melekat pada
matriks akan digunakan sebagai kolom yang diisi dengan bilangan real
tersebut.
\end{eulercomment}
\begin{eulerprompt}
>A|1
\end{eulerprompt}
\begin{euleroutput}
       0.032444     0.0534171      0.595713      0.564454             1 
        0.83916      0.175552      0.396988       0.83514             1 
      0.0257573      0.658585      0.629832      0.770895             1 
\end{euleroutput}
\begin{eulercomment}
Dimungkinkan untuk membuat matriks vektor baris dan kolom.
\end{eulercomment}
\begin{eulerprompt}
>[v;v]
\end{eulerprompt}
\begin{euleroutput}
              1             2             3 
              1             2             3 
\end{euleroutput}
\begin{eulerprompt}
>[v',v']
\end{eulerprompt}
\begin{euleroutput}
              1             1 
              2             2 
              3             3 
\end{euleroutput}
\begin{eulercomment}
Tujuan utamanya adalah untuk menafsirkan vektor ekspresi untuk vektor
kolom.
\end{eulercomment}
\begin{eulerprompt}
>"[x,x^2]"(v')
\end{eulerprompt}
\begin{euleroutput}
              1             1 
              2             4 
              3             9 
\end{euleroutput}
\begin{eulercomment}
Untuk mendapatkan ukuran A, kita dapat menggunakan fungsi-fungsi
berikut.
\end{eulercomment}
\begin{eulerprompt}
>C=zeros(2,4); rows(C), cols(C), size(C), length(C)
\end{eulerprompt}
\begin{euleroutput}
  2
  4
  [2,  4]
  4
\end{euleroutput}
\begin{eulercomment}
Untuk vektor, ada panjang ().
\end{eulercomment}
\begin{eulerprompt}
>length(2:10)
\end{eulerprompt}
\begin{euleroutput}
  9
\end{euleroutput}
\begin{eulercomment}
Ada banyak fungsi lain yang menghasilkan matriks.
\end{eulercomment}
\begin{eulerprompt}
>ones(2,2)
\end{eulerprompt}
\begin{euleroutput}
              1             1 
              1             1 
\end{euleroutput}
\begin{eulercomment}
Ini juga dapat digunakan dengan satu parameter. Untuk mendapatkan
vektor dengan angka selain 1, gunakan yang berikut ini.
\end{eulercomment}
\begin{eulerprompt}
>ones(5)*6
\end{eulerprompt}
\begin{euleroutput}
  [6,  6,  6,  6,  6]
\end{euleroutput}
\begin{eulercomment}
Juga matriks bilangan acak dapat dihasilkan dengan acak (distribusi
seragam) atau normal (distribusi Gauß).
\end{eulercomment}
\begin{eulerprompt}
>random(2,2)
\end{eulerprompt}
\begin{euleroutput}
        0.66566      0.831835 
          0.977      0.544258 
\end{euleroutput}
\begin{eulercomment}
Berikut adalah fungsi berguna lainnya, yang menyusun kembali
elemen-elemen matriks menjadi matriks lain.
\end{eulercomment}
\begin{eulerprompt}
>redim(1:9,3,3) // menyusun elemen2 1, 2, 3, ..., 9 ke bentuk matriks 3x3
\end{eulerprompt}
\begin{euleroutput}
              1             2             3 
              4             5             6 
              7             8             9 
\end{euleroutput}
\begin{eulercomment}
Dengan fungsi berikut, kita dapat menggunakan this dan fungsi dup
untuk menulis fungsi rep (), yang mengulang vektor sebanyak n kali.
\end{eulercomment}
\begin{eulerprompt}
>function rep(v,n) := redim(dup(v,n),1,n*cols(v))
\end{eulerprompt}
\begin{eulercomment}
Mari kita uji.
\end{eulercomment}
\begin{eulerprompt}
>rep(1:3,5)
\end{eulerprompt}
\begin{euleroutput}
  [1,  2,  3,  1,  2,  3,  1,  2,  3,  1,  2,  3,  1,  2,  3]
\end{euleroutput}
\begin{eulercomment}
Fungsi multdup () menduplikasi elemen vektor.
\end{eulercomment}
\begin{eulerprompt}
>multdup(1:3,5), multdup(1:3,[2,3,2])
\end{eulerprompt}
\begin{euleroutput}
  [1,  1,  1,  1,  1,  2,  2,  2,  2,  2,  3,  3,  3,  3,  3]
  [1,  1,  2,  2,  2,  3,  3]
\end{euleroutput}
\begin{eulercomment}
Fungsi flipx () dan flipy () mengembalikan urutan baris atau kolom
matriks. Yaitu, fungsi flipx () membalik secara horizontal.
\end{eulercomment}
\begin{eulerprompt}
>flipx(1:5) //membalik elemen2 vektor baris
\end{eulerprompt}
\begin{euleroutput}
  [5,  4,  3,  2,  1]
\end{euleroutput}
\begin{eulercomment}
Untuk rotasi, Euler memiliki rotleft () dan rotright ().
\end{eulercomment}
\begin{eulerprompt}
>rotleft(1:5) // memutar elemen2 vektor baris
\end{eulerprompt}
\begin{euleroutput}
  [2,  3,  4,  5,  1]
\end{euleroutput}
\begin{eulercomment}
Sebuah fungsi khusus adalah drop (v, i), yang menghilangkan elemen
dengan indeks di i dari vektor v.
\end{eulercomment}
\begin{eulerprompt}
>drop(10:20,3)
\end{eulerprompt}
\begin{euleroutput}
  [10,  11,  13,  14,  15,  16,  17,  18,  19,  20]
\end{euleroutput}
\begin{eulercomment}
Perhatikan bahwa vektor i dalam drop (v, i) mengacu pada indeks elemen
di v, bukan nilai elemen. Jika Anda ingin menghapus elemen, Anda harus
menemukan elemennya terlebih dahulu. Indeks fungsi (v, x) dapat
digunakan untuk mencari elemen x dalam vektor yang diurutkan v.
\end{eulercomment}
\begin{eulerprompt}
>v=primes(50), i=indexof(v,10:20), drop(v,i)
\end{eulerprompt}
\begin{euleroutput}
  [2,  3,  5,  7,  11,  13,  17,  19,  23,  29,  31,  37,  41,  43,  47]
  [0,  5,  0,  6,  0,  0,  0,  7,  0,  8,  0]
  [2,  3,  5,  7,  23,  29,  31,  37,  41,  43,  47]
\end{euleroutput}
\begin{eulercomment}
Seperti yang Anda lihat, tidak ada salahnya untuk menyertakan indeks
di luar rentang (seperti 0), indeks ganda, atau indeks yang tidak
disortir.
\end{eulercomment}
\begin{eulerprompt}
>drop(1:10,shuffle([0,0,5,5,7,12,12]))
\end{eulerprompt}
\begin{euleroutput}
  [1,  2,  3,  4,  6,  8,  9,  10]
\end{euleroutput}
\begin{eulercomment}
Ada beberapa fungsi khusus untuk mengatur diagonal atau untuk
menghasilkan matriks diagonal.

Kami mulai dengan matriks identitas.
\end{eulercomment}
\begin{eulerprompt}
>A=id(5) // matriks identitas 5x5
\end{eulerprompt}
\begin{euleroutput}
              1             0             0             0             0 
              0             1             0             0             0 
              0             0             1             0             0 
              0             0             0             1             0 
              0             0             0             0             1 
\end{euleroutput}
\begin{eulercomment}
Kemudian kami mengatur diagonal bawah (-1) menjadi 1: 4.
\end{eulercomment}
\begin{eulerprompt}
>setdiag(A,-1,1:4) //mengganti diagonal di bawah diagonal utama
\end{eulerprompt}
\begin{euleroutput}
              1             0             0             0             0 
              1             1             0             0             0 
              0             2             1             0             0 
              0             0             3             1             0 
              0             0             0             4             1 
\end{euleroutput}
\begin{eulercomment}
Perhatikan bahwa kami tidak mengubah matriks A. Kami mendapatkan
matriks baru sebagai hasil dari setdiag ().

Berikut adalah fungsi yang mengembalikan matriks tri-diagonal.
\end{eulercomment}
\begin{eulerprompt}
>function tridiag (n,a,b,c) := setdiag(setdiag(b*id(n),1,c),-1,a); ...
>tridiag(5,1,2,3)
\end{eulerprompt}
\begin{euleroutput}
              2             3             0             0             0 
              1             2             3             0             0 
              0             1             2             3             0 
              0             0             1             2             3 
              0             0             0             1             2 
\end{euleroutput}
\begin{eulercomment}
Diagonal matriks juga dapat diekstraksi dari matriks. Untuk
mendemonstrasikan ini, kami merestrukturisasi vektor 1: 9 menjadi
matriks 3x3.
\end{eulercomment}
\begin{eulerprompt}
>A=redim(1:9,3,3)
\end{eulerprompt}
\begin{euleroutput}
              1             2             3 
              4             5             6 
              7             8             9 
\end{euleroutput}
\begin{eulercomment}
Sekarang kita bisa mengekstrak diagonal.
\end{eulercomment}
\begin{eulerprompt}
>d=getdiag(A,0)
\end{eulerprompt}
\begin{euleroutput}
  [1,  5,  9]
\end{euleroutput}
\begin{eulercomment}
Misalnya. Kita dapat membagi matriks dengan diagonalnya. Bahasa
matriks memperhatikan bahwa vektor kolom d diterapkan ke matriks baris
demi baris.
\end{eulercomment}
\begin{eulerprompt}
>fraction A/d'
\end{eulerprompt}
\begin{euleroutput}
          1         2         3 
        4/5         1       6/5 
        7/9       8/9         1 
\end{euleroutput}
\eulerheading{Vektorisasi}
\begin{eulercomment}
Hampir semua fungsi di Euler bekerja untuk matriks dan input vektor
juga, jika memungkinkan.

Misalnya, fungsi sqrt () menghitung akar kuadrat dari semua elemen
vektor atau matriks.
\end{eulercomment}
\begin{eulerprompt}
>sqrt(1:3)
\end{eulerprompt}
\begin{euleroutput}
  [1,  1.41421,  1.73205]
\end{euleroutput}
\begin{eulercomment}
Jadi Anda dapat dengan mudah membuat tabel nilai. Ini adalah salah
satu cara untuk memplot fungsi (alternatifnya menggunakan ekspresi).
\end{eulercomment}
\begin{eulerprompt}
>x=1:0.01:5; y=log(x)/x^2; // terlalu panjang untuk ditampikan
\end{eulerprompt}
\begin{eulercomment}
Dengan ini dan operator titik dua a: delta: b, vektor nilai fungsi
dapat dibuat dengan mudah.

Dalam contoh berikut, kami menghasilkan vektor nilai t [i] dengan
jarak 0,1 dari -1 hingga 1. Kemudian kami menghasilkan vektor nilai
fungsi

lateks: s = t \textasciicircum{} 3-t
\end{eulercomment}
\begin{eulerprompt}
>t=-1:0.1:1; s=t^3-t
\end{eulerprompt}
\begin{euleroutput}
  [0,  0.171,  0.288,  0.357,  0.384,  0.375,  0.336,  0.273,  0.192,
  0.099,  0,  -0.099,  -0.192,  -0.273,  -0.336,  -0.375,  -0.384,
  -0.357,  -0.288,  -0.171,  0]
\end{euleroutput}
\begin{eulercomment}
EMT memperluas operator untuk skalar, vektor, dan matriks dengan cara
yang jelas.

Misalnya, vektor kolom dikali vektor baris mengembang menjadi matriks,
jika operator diterapkan. Berikut ini, v 'adalah vektor yang dialihkan
(vektor kolom).
\end{eulercomment}
\begin{eulerprompt}
>shortest (1:5)*(1:5)'
\end{eulerprompt}
\begin{euleroutput}
       1      2      3      4      5 
       2      4      6      8     10 
       3      6      9     12     15 
       4      8     12     16     20 
       5     10     15     20     25 
\end{euleroutput}
\begin{eulercomment}
Perhatikan, ini sangat berbeda dari hasil perkalian matriks. Produk
matriks dilambangkan dengan titik "." di EMT.
\end{eulercomment}
\begin{eulerprompt}
>(1:5).(1:5)'
\end{eulerprompt}
\begin{euleroutput}
  55
\end{euleroutput}
\begin{eulercomment}
Secara default, vektor baris dicetak dalam format kompak.
\end{eulercomment}
\begin{eulerprompt}
>[1,2,3,4]
\end{eulerprompt}
\begin{euleroutput}
  [1,  2,  3,  4]
\end{euleroutput}
\begin{eulercomment}
Untuk matriks, operator khusus. menunjukkan perkalian matriks, dan A
'menunjukkan transposing. Matriks 1x1 dapat digunakan seperti bilangan
real.
\end{eulercomment}
\begin{eulerprompt}
>v:=[1,2]; v.v', %^2
\end{eulerprompt}
\begin{euleroutput}
  5
  25
\end{euleroutput}
\begin{eulercomment}
Untuk mengubah urutan matriks, kami menggunakan apostrof.
\end{eulercomment}
\begin{eulerprompt}
>v=1:4; v'
\end{eulerprompt}
\begin{euleroutput}
              1 
              2 
              3 
              4 
\end{euleroutput}
\begin{eulercomment}
Sehingga kita dapat menghitung matriks A dikali vektor b.
\end{eulercomment}
\begin{eulerprompt}
>A=[1,2,3,4;5,6,7,8]; A.v'
\end{eulerprompt}
\begin{euleroutput}
             30 
             70 
\end{euleroutput}
\begin{eulercomment}
Perhatikan bahwa v masih merupakan vektor baris. Jadi v'.v berbeda
dari v.v '.
\end{eulercomment}
\begin{eulerprompt}
>v'.v
\end{eulerprompt}
\begin{euleroutput}
              1             2             3             4 
              2             4             6             8 
              3             6             9            12 
              4             8            12            16 
\end{euleroutput}
\begin{eulercomment}
v.v 'menghitung norma v kuadrat untuk vektor baris v. Hasilnya adalah
vektor 1x1, yang bekerja seperti bilangan real.
\end{eulercomment}
\begin{eulerprompt}
>v.v'
\end{eulerprompt}
\begin{euleroutput}
  30
\end{euleroutput}
\begin{eulercomment}
Ada juga norma fungsi (bersama dengan banyak fungsi lain dari Aljabar
Linear).
\end{eulercomment}
\begin{eulerprompt}
>norm(v)^2
\end{eulerprompt}
\begin{euleroutput}
  30
\end{euleroutput}
\begin{eulercomment}
Operator dan fungsi mematuhi bahasa matriks Euler.

Berikut adalah ringkasan aturannya.

- Fungsi yang diterapkan ke vektor atau matriks diterapkan ke setiap
elemen.

- Seorang operator yang beroperasi pada dua matriks dengan ukuran yang
sama diterapkan berpasangan ke elemen matriks.

- Jika kedua matriks memiliki dimensi yang berbeda, keduanya diperluas
dengan cara yang masuk akal, sehingga memiliki ukuran yang sama.

Misalnya, nilai skalar dikalikan vektor mengalikan nilai dengan setiap
elemen vektor. Atau matriks dikalikan dengan vektor (dengan *, bukan.)
Memperluas vektor ke ukuran matriks dengan menduplikasinya.

Berikut ini adalah kasus sederhana dengan operator \textasciicircum{}.
\end{eulercomment}
\begin{eulerprompt}
>[1,2,3]^2
\end{eulerprompt}
\begin{euleroutput}
  [1,  4,  9]
\end{euleroutput}
\begin{eulercomment}
Ini kasus yang lebih rumit. Vektor baris dikalikan kolom mengembang
kelelahan dengan menduplikasi.
\end{eulercomment}
\begin{eulerprompt}
>v:=[1,2,3]; v*v'
\end{eulerprompt}
\begin{euleroutput}
              1             2             3 
              2             4             6 
              3             6             9 
\end{euleroutput}
\begin{eulercomment}
Perhatikan bahwa produk skalar menggunakan produk matriks, bukan *!
\end{eulercomment}
\begin{eulerprompt}
>v.v'
\end{eulerprompt}
\begin{euleroutput}
  14
\end{euleroutput}
\begin{eulercomment}
Ada banyak fungsi untuk matriks. Kami memberikan daftar singkat. Anda
harus membaca dokumentasi untuk informasi lebih lanjut tentang
perintah ini.

\end{eulercomment}
\begin{eulerttcomment}
  sum, prod menghitung jumlah dan produk dari baris
  cumsum, cumprod melakukan hal yang sama secara kumulatif
  menghitung nilai ekstrem dari setiap baris
  extrema mengembalikan vektor dengan informasi ekstrem
  diag (A, i) mengembalikan diagonal ke-i
  setdiag (A, i, v) mengatur diagonal ke-i
  id (n) matriks identitas
  det (A) determinan
  charpoly (A) polinomial karakteristik
  eigenvalues ??(A) eigenvalues
\end{eulerttcomment}
\begin{eulerprompt}
>v*v, sum(v*v), cumsum(v*v)
\end{eulerprompt}
\begin{euleroutput}
  [1,  4,  9]
  14
  [1,  5,  14]
\end{euleroutput}
\begin{eulercomment}
Operator: menghasilkan vektor baris spasi yang sama, secara opsional
dengan ukuran langkah.
\end{eulercomment}
\begin{eulerprompt}
>1:4, 1:2:10
\end{eulerprompt}
\begin{euleroutput}
  [1,  2,  3,  4]
  [1,  3,  5,  7,  9]
\end{euleroutput}
\begin{eulercomment}
Untuk menggabungkan matriks dan vektor ada operator "\textbar{}" dan "\_".
\end{eulercomment}
\begin{eulerprompt}
>[1,2,3]|[4,5], [1,2,3]_1
\end{eulerprompt}
\begin{euleroutput}
  [1,  2,  3,  4,  5]
              1             2             3 
              1             1             1 
\end{euleroutput}
\begin{eulercomment}
Unsur-unsur matriks disebut dengan "A [i, j]".
\end{eulercomment}
\begin{eulerprompt}
>A:=[1,2,3;4,5,6;7,8,9]; A[2,3]
\end{eulerprompt}
\begin{euleroutput}
  6
\end{euleroutput}
\begin{eulercomment}
Untuk vektor baris atau kolom, v [i] adalah elemen ke-i dari vektor.
Untuk matriks, ini mengembalikan baris ke-i lengkap dari matriks
tersebut.
\end{eulercomment}
\begin{eulerprompt}
>v:=[2,4,6,8]; v[3], A[3]
\end{eulerprompt}
\begin{euleroutput}
  6
  [7,  8,  9]
\end{euleroutput}
\begin{eulercomment}
Indeks juga dapat berupa vektor baris indeks. : menunjukkan semua
indeks.
\end{eulercomment}
\begin{eulerprompt}
>v[1:2], A[:,2]
\end{eulerprompt}
\begin{euleroutput}
  [2,  4]
              2 
              5 
              8 
\end{euleroutput}
\begin{eulercomment}
Bentuk singkat dari: adalah menghilangkan indeks sepenuhnya.
\end{eulercomment}
\begin{eulerprompt}
>A[,2:3]
\end{eulerprompt}
\begin{euleroutput}
              2             3 
              5             6 
              8             9 
\end{euleroutput}
\begin{eulercomment}
Untuk tujuan vektorisasi, elemen-elemen matriks dapat diakses
seolah-olah mereka adalah vektor.
\end{eulercomment}
\begin{eulerprompt}
>A\{4\}
\end{eulerprompt}
\begin{euleroutput}
  4
\end{euleroutput}
\begin{eulercomment}
Matriks juga bisa diratakan, menggunakan fungsi redim (). Ini
diimplementasikan dalam fungsi flatten ().
\end{eulercomment}
\begin{eulerprompt}
>redim(A,1,prod(size(A))), flatten(A)
\end{eulerprompt}
\begin{euleroutput}
  [1,  2,  3,  4,  5,  6,  7,  8,  9]
  [1,  2,  3,  4,  5,  6,  7,  8,  9]
\end{euleroutput}
\begin{eulercomment}
Untuk menggunakan matriks untuk tabel, mari kita reset ke format
default, dan menghitung tabel nilai sinus dan cosinus. Perhatikan
bahwa sudut dalam radian secara default.
\end{eulercomment}
\begin{eulerprompt}
>defformat; w=0°:45°:360°; w=w'; deg(w)
\end{eulerprompt}
\begin{euleroutput}
              0 
             45 
             90 
            135 
            180 
            225 
            270 
            315 
            360 
\end{euleroutput}
\begin{eulercomment}
Sekarang kami menambahkan kolom ke matriks.
\end{eulercomment}
\begin{eulerprompt}
>M = deg(w)|w|cos(w)|sin(w)
\end{eulerprompt}
\begin{euleroutput}
              0             0             1             0 
             45      0.785398      0.707107      0.707107 
             90        1.5708             0             1 
            135       2.35619     -0.707107      0.707107 
            180       3.14159            -1             0 
            225       3.92699     -0.707107     -0.707107 
            270       4.71239             0            -1 
            315       5.49779      0.707107     -0.707107 
            360       6.28319             1             0 
\end{euleroutput}
\begin{eulercomment}
Dengan menggunakan bahasa matriks, kita dapat menghasilkan beberapa
tabel dari beberapa fungsi sekaligus.

Dalam contoh berikut, kami menghitung t [j] \textasciicircum{} i untuk i dari 1 ke n.
Kami mendapatkan matriks, di mana setiap baris adalah tabel t \textasciicircum{} i
untuk satu i. Yaitu, matriks memiliki elemen lateks: a\_ \{i, j\} = t\_j \textasciicircum{}
i, \textbackslash{} quad 1 \textbackslash{} le j \textbackslash{} le 101, \textbackslash{} quad 1 \textbackslash{} le i \textbackslash{} le n

Fungsi yang tidak bekerja untuk input vektor harus di-vectorisasi. Hal
ini dapat dicapai dengan kata kunci "peta" dalam definisi fungsi.
Kemudian fungsi tersebut akan dievaluasi untuk setiap elemen dari
parameter vektor.

Integrasi numerik integ () hanya berfungsi untuk batas interval
skalar. Jadi kita perlu melakukan vektorisasi.
\end{eulercomment}
\begin{eulerprompt}
>function map f(x) := integrate("x^x",1,x)
\end{eulerprompt}
\begin{eulercomment}
Kata kunci "map" memvektorisasi fungsi. Fungsi tersebut sekarang akan
bekerja\\
untuk vektor angka.
\end{eulercomment}
\begin{eulerprompt}
>f([1:5])
\end{eulerprompt}
\begin{euleroutput}
  [0,  2.05045,  13.7251,  113.336,  1241.03]
\end{euleroutput}
\eulerheading{Sub-Matriks dan Elemen-Matriks}
\begin{eulercomment}
Untuk mengakses elemen matriks, gunakan notasi braket.
\end{eulercomment}
\begin{eulerprompt}
>A=[1,2,3;4,5,6;7,8,9], A[2,2]
\end{eulerprompt}
\begin{euleroutput}
              1             2             3 
              4             5             6 
              7             8             9 
  5
\end{euleroutput}
\begin{eulercomment}
Kita dapat mengakses baris matriks yang lengkap.
\end{eulercomment}
\begin{eulerprompt}
>A[2]
\end{eulerprompt}
\begin{euleroutput}
  [4,  5,  6]
\end{euleroutput}
\begin{eulercomment}
Dalam kasus vektor baris atau kolom, ini mengembalikan elemen vektor.
\end{eulercomment}
\begin{eulerprompt}
>v=1:3; v[2]
\end{eulerprompt}
\begin{euleroutput}
  2
\end{euleroutput}
\begin{eulercomment}
Untuk memastikan, Anda mendapatkan baris pertama untuk matriks 1xn dan
mxn, tentukan semua kolom menggunakan indeks kedua yang kosong.
\end{eulercomment}
\begin{eulerprompt}
>A[2,]
\end{eulerprompt}
\begin{euleroutput}
  [4,  5,  6]
\end{euleroutput}
\begin{eulercomment}
Jika indeks adalah vektor indeks, Euler akan mengembalikan baris yang
sesuai dari matriks.

Di sini kami ingin baris pertama dan kedua dari A.
\end{eulercomment}
\begin{eulerprompt}
>A[[1,2]]
\end{eulerprompt}
\begin{euleroutput}
              1             2             3 
              4             5             6 
\end{euleroutput}
\begin{eulercomment}
Kami bahkan dapat menyusun ulang A menggunakan vektor indeks.
Tepatnya, kami tidak mengubah A di sini, tetapi menghitung versi A.
\end{eulercomment}
\begin{eulerprompt}
>A[[3,2,1]]
\end{eulerprompt}
\begin{euleroutput}
              7             8             9 
              4             5             6 
              1             2             3 
\end{euleroutput}
\begin{eulercomment}
Trik indeks juga bekerja dengan kolom.

Contoh ini memilih semua baris A dan kolom kedua dan ketiga.
\end{eulercomment}
\begin{eulerprompt}
>A[1:3,2:3]
\end{eulerprompt}
\begin{euleroutput}
              2             3 
              5             6 
              8             9 
\end{euleroutput}
\begin{eulercomment}
Untuk singkatan ":" menunjukkan semua indeks baris atau kolom.
\end{eulercomment}
\begin{eulerprompt}
>A[:,3]
\end{eulerprompt}
\begin{euleroutput}
              3 
              6 
              9 
\end{euleroutput}
\begin{eulercomment}
Cara lainnya, biarkan indeks pertama kosong.
\end{eulercomment}
\begin{eulerprompt}
>A[,2:3]
\end{eulerprompt}
\begin{euleroutput}
              2             3 
              5             6 
              8             9 
\end{euleroutput}
\begin{eulercomment}
Kita juga bisa mendapatkan baris terakhir A.
\end{eulercomment}
\begin{eulerprompt}
>A[-1]
\end{eulerprompt}
\begin{euleroutput}
  [7,  8,  9]
\end{euleroutput}
\begin{eulercomment}
Sekarang mari kita ubah elemen A dengan menetapkan submatrix dari A ke
beberapa nilai. Ini sebenarnya mengubah matriks A yang disimpan.
\end{eulercomment}
\begin{eulerprompt}
>A[1,1]=4
\end{eulerprompt}
\begin{euleroutput}
              4             2             3 
              4             5             6 
              7             8             9 
\end{euleroutput}
\begin{eulercomment}
Kami juga dapat menetapkan nilai ke baris A.
\end{eulercomment}
\begin{eulerprompt}
>A[1]=[-1,-1,-1]
\end{eulerprompt}
\begin{euleroutput}
             -1            -1            -1 
              4             5             6 
              7             8             9 
\end{euleroutput}
\begin{eulercomment}
Kami bahkan dapat menetapkan ke sub-matriks jika memiliki ukuran yang
sesuai.
\end{eulercomment}
\begin{eulerprompt}
>A[1:2,1:2]=[5,6;7,8]
\end{eulerprompt}
\begin{euleroutput}
              5             6            -1 
              7             8             6 
              7             8             9 
\end{euleroutput}
\begin{eulercomment}
Selain itu, beberapa pintasan diperbolehkan.
\end{eulercomment}
\begin{eulerprompt}
>A[1:2,1:2]=0
\end{eulerprompt}
\begin{euleroutput}
              0             0            -1 
              0             0             6 
              7             8             9 
\end{euleroutput}
\begin{eulercomment}
Peringatan: Indeks di luar batas menampilkan matriks kosong, atau
pesan kesalahan, bergantung pada pengaturan sistem. Standarnya adalah
pesan kesalahan. Ingat, bagaimanapun, bahwa indeks negatif dapat
digunakan untuk mengakses elemen matriks yang dihitung dari akhir.
\end{eulercomment}
\begin{eulerprompt}
>A[4]
\end{eulerprompt}
\begin{euleroutput}
  Row index 4 out of bounds!
  Error in:
  A[4] ...
      ^
\end{euleroutput}
\eulerheading{Menyortir dan Mengocok}
\begin{eulercomment}
Fungsi sort () mengurutkan vektor baris.
\end{eulercomment}
\begin{eulerprompt}
>sort([5,6,4,8,1,9])
\end{eulerprompt}
\begin{euleroutput}
  [1,  4,  5,  6,  8,  9]
\end{euleroutput}
\begin{eulercomment}
Seringkali perlu untuk mengetahui indeks dari vektor yang diurutkan
dalam vektor asli. Ini dapat digunakan untuk menyusun ulang vektor
lain dengan cara yang sama.

Mari kita mengacak vektor.
\end{eulercomment}
\begin{eulerprompt}
>v=shuffle(1:10)
\end{eulerprompt}
\begin{euleroutput}
  [4,  5,  10,  6,  8,  9,  1,  7,  2,  3]
\end{euleroutput}
\begin{eulercomment}
Indeks tersebut berisi urutan v.
\end{eulercomment}
\begin{eulerprompt}
>\{vs,ind\}=sort(v); v[ind]
\end{eulerprompt}
\begin{euleroutput}
  [1,  2,  3,  4,  5,  6,  7,  8,  9,  10]
\end{euleroutput}
\begin{eulercomment}
Ini bekerja untuk vektor string juga.
\end{eulercomment}
\begin{eulerprompt}
>s=["a","d","e","a","aa","e"]
\end{eulerprompt}
\begin{euleroutput}
  a
  d
  e
  a
  aa
  e
\end{euleroutput}
\begin{eulerprompt}
>\{ss,ind\}=sort(s); ss
\end{eulerprompt}
\begin{euleroutput}
  a
  a
  aa
  d
  e
  e
\end{euleroutput}
\begin{eulercomment}
Seperti yang Anda lihat, posisi entri ganda agak acak.
\end{eulercomment}
\begin{eulerprompt}
>ind
\end{eulerprompt}
\begin{euleroutput}
  [4,  1,  5,  2,  6,  3]
\end{euleroutput}
\begin{eulercomment}
Fungsi unik mengembalikan daftar elemen unik vektor yang diurutkan.
\end{eulercomment}
\begin{eulerprompt}
>intrandom(1,10,10), unique(%)
\end{eulerprompt}
\begin{euleroutput}
  [4,  4,  9,  2,  6,  5,  10,  6,  5,  1]
  [1,  2,  4,  5,  6,  9,  10]
\end{euleroutput}
\begin{eulercomment}
Ini bekerja untuk vektor string juga.
\end{eulercomment}
\begin{eulerprompt}
>unique(s)
\end{eulerprompt}
\begin{euleroutput}
  a
  aa
  d
  e
\end{euleroutput}
\eulerheading{Aljabar linier}
\begin{eulercomment}
EMT memiliki banyak fungsi untuk menyelesaikan masalah sistem linier,
sistem jarang, atau regresi.

Untuk sistem linier Ax = b, Anda dapat menggunakan algoritma Gauss,
matriks invers atau fit linier. Operator A \textbackslash{} b menggunakan versi
algoritma Gauss.
\end{eulercomment}
\begin{eulerprompt}
>A=[1,2;3,4]; b=[5;6]; A\(\backslash\)b
\end{eulerprompt}
\begin{euleroutput}
             -4 
            4.5 
\end{euleroutput}
\begin{eulercomment}
Untuk contoh lain, kami menghasilkan matriks 200x200 dan jumlah
barisnya. Kemudian kita menyelesaikan Ax = b menggunakan matriks
invers. Kami mengukur kesalahan sebagai deviasi maksimal semua elemen
dari 1, yang tentu saja merupakan solusi yang tepat.
\end{eulercomment}
\begin{eulerprompt}
>A=normal(200,200); b=sum(A); longest totalmax(abs(inv(A).b-1))
\end{eulerprompt}
\begin{euleroutput}
    8.790745908981989e-13 
\end{euleroutput}
\begin{eulercomment}
Jika sistem tidak memiliki solusi, kesesuaian linier meminimalkan
norma kesalahan Ax-b.
\end{eulercomment}
\begin{eulerprompt}
>A=[1,2,3;4,5,6;7,8,9]
\end{eulerprompt}
\begin{euleroutput}
              1             2             3 
              4             5             6 
              7             8             9 
\end{euleroutput}
\begin{eulercomment}
Determinan dari matriks ini adalah 0.
\end{eulercomment}
\begin{eulerprompt}
>det(A)
\end{eulerprompt}
\begin{euleroutput}
  0
\end{euleroutput}
\eulerheading{Matriks Simbolik}
\begin{eulercomment}
Maxima memiliki matriks simbolis. Tentu saja, Maxima dapat digunakan
untuk soal-soal aljabar linier sederhana. Kita dapat mendefinisikan
matriks untuk Euler dan Maxima dengan \&: =, dan kemudian
menggunakannya dalam ekspresi simbolik. Bentuk [...] biasa untuk
mendefinisikan matriks dapat digunakan di Euler untuk mendefinisikan
matriks simbolik.
\end{eulercomment}
\begin{eulerprompt}
>A &= [a,1,1;1,a,1;1,1,a]; $A
>$&det(A), $&factor(%)
>$&invert(A) with a=0
>A &= [1,a;b,2]; $A
\end{eulerprompt}
\begin{eulercomment}
Seperti semua variabel simbolik, matriks ini dapat digunakan dalam
ekspresi simbolik lainnya.
\end{eulercomment}
\begin{eulerprompt}
>$&det(A-x*ident(2)), $&solve(%,x)
\end{eulerprompt}
\begin{eulercomment}
Nilai eigen juga dapat dihitung secara otomatis. Hasilnya adalah
vektor dengan dua vektor nilai eigen dan kelipatannya.
\end{eulercomment}
\begin{eulerprompt}
>$&eigenvalues([a,1;1,a])
\end{eulerprompt}
\begin{eulercomment}
Untuk mengekstrak vektor eigen tertentu, perlu pengindeksan yang
cermat.
\end{eulercomment}
\begin{eulerprompt}
>$&eigenvectors([a,1;1,a]), &%[2][1][1]
\end{eulerprompt}
\begin{euleroutput}
  
                                 [1, - 1]
  
\end{euleroutput}
\begin{eulercomment}
Matriks simbolik dapat dievaluasi dalam Euler secara numerik seperti
ekspresi simbolik lainnya.
\end{eulercomment}
\begin{eulerprompt}
>A(a=4,b=5)
\end{eulerprompt}
\begin{euleroutput}
              1             4 
              5             2 
\end{euleroutput}
\begin{eulercomment}
Dalam ekspresi simbolik, gunakan dengan.
\end{eulercomment}
\begin{eulerprompt}
>$&A with [a=4,b=5]
\end{eulerprompt}
\begin{eulercomment}
Akses ke baris matriks simbolik berfungsi seperti halnya dengan
matriks numerik.
\end{eulercomment}
\begin{eulerprompt}
>$&A[1]
\end{eulerprompt}
\begin{eulercomment}
Ekspresi simbolis dapat berisi tugas. Dan itu mengubah matriks A.
\end{eulercomment}
\begin{eulerprompt}
>&A[1,1]:=t+1; $&A
\end{eulerprompt}
\begin{eulercomment}
Ada fungsi simbolik dalam Maxima untuk membuat vektor dan matriks.
Untuk ini, lihat dokumentasi Maxima atau tutorial tentang Maxima di
EMT.
\end{eulercomment}
\begin{eulerprompt}
>v &= makelist(1/(i+j),i,1,3); $v
\end{eulerprompt}
\begin{eulerttcomment}
 
\end{eulerttcomment}
\begin{eulerprompt}
>B &:= [1,2;3,4]; $B, $&invert(B)
\end{eulerprompt}
\begin{eulercomment}
Hasilnya dapat dievaluasi secara numerik di Euler. Untuk informasi
lebih lanjut tentang Maxima, lihat pengantar Maxima.
\end{eulercomment}
\begin{eulerprompt}
>$&invert(B)()
\end{eulerprompt}
\begin{euleroutput}
             -2             1 
            1.5          -0.5 
\end{euleroutput}
\begin{eulercomment}
Euler juga memiliki fungsi xinv () yang kuat, yang membuat upaya lebih
besar dan mendapatkan hasil yang lebih tepat.

Perhatikan, bahwa dengan \&: = matriks B telah didefinisikan sebagai
simbolik dalam ekspresi simbolik dan numerik dalam ekspresi numerik.
Jadi kita bisa menggunakannya di sini.
\end{eulercomment}
\begin{eulerprompt}
>longest B.xinv(B)
\end{eulerprompt}
\begin{euleroutput}
                        1                       0 
                        0                       1 
\end{euleroutput}
\begin{eulercomment}
Misalnya. nilai eigen dari A dapat dihitung secara numerik.
\end{eulercomment}
\begin{eulerprompt}
>A=[1,2,3;4,5,6;7,8,9]; real(eigenvalues(A))
\end{eulerprompt}
\begin{euleroutput}
  [16.1168,  -1.11684,  0]
\end{euleroutput}
\begin{eulercomment}
Atau secara simbolis. Lihat tutorial tentang Maxima untuk detailnya.
\end{eulercomment}
\begin{eulerprompt}
>$&eigenvalues(@A)
\end{eulerprompt}
\eulerheading{Nilai Numerik dalam Ekspresi simbolis}
\begin{eulercomment}
Ekspresi simbolik hanyalah string yang mengandung ekspresi. Jika kita
ingin mendefinisikan nilai untuk ekspresi simbolik dan ekspresi
numerik, kita harus menggunakan "\&: =".
\end{eulercomment}
\begin{eulerprompt}
>A &:= [1,pi;4,5]
\end{eulerprompt}
\begin{euleroutput}
              1       3.14159 
              4             5 
\end{euleroutput}
\begin{eulercomment}
Masih terdapat perbedaan antara bentuk numerik dan simbolik. Saat
mentransfer matriks ke bentuk simbolis, pendekatan pecahan untuk real
akan digunakan.
\end{eulercomment}
\begin{eulerprompt}
>$&A
\end{eulerprompt}
\begin{eulercomment}
Untuk menghindari hal ini, ada fungsi "mxmset (variabel)".
\end{eulercomment}
\begin{eulerprompt}
>mxmset(A); $&A
\end{eulerprompt}
\begin{eulercomment}
Maxima juga dapat menghitung dengan angka floating point, dan bahkan
dengan angka mengambang besar dengan 32 digit. Namun, evaluasinya jauh
lebih lambat.
\end{eulercomment}
\begin{eulerprompt}
>$&bfloat(sqrt(2)), $&float(sqrt(2))
\end{eulerprompt}
\begin{eulercomment}
Ketepatan angka floating point besar dapat diubah.
\end{eulercomment}
\begin{eulerprompt}
>&fpprec:=100; &bfloat(pi)
\end{eulerprompt}
\begin{euleroutput}
  
          3.14159265358979323846264338327950288419716939937510582097494\(\backslash\)
  4592307816406286208998628034825342117068b0
  
\end{euleroutput}
\begin{eulercomment}
Variabel numerik dapat digunakan dalam ekspresi simbolik apa pun yang
menggunakan "@var".

Perhatikan bahwa ini hanya diperlukan, jika variabel telah ditentukan
dengan ": =" atau "=" sebagai variabel numerik.
\end{eulercomment}
\begin{eulerprompt}
>B:=[1,pi;3,4]; $&det(@B)
\end{eulerprompt}
\eulerheading{Demo - Suku Bunga}
\begin{eulercomment}
Di bawah ini, kami menggunakan Euler Math Toolbox (EMT) untuk
menghitung suku bunga. Kami melakukannya secara numerik dan simbolis
untuk menunjukkan kepada Anda bagaimana Euler dapat digunakan untuk
memecahkan masalah kehidupan nyata.

Asumsikan Anda memiliki modal awal 5000 (katakanlah dalam dolar).
\end{eulercomment}
\begin{eulerprompt}
>K=5000
\end{eulerprompt}
\begin{euleroutput}
  5000
\end{euleroutput}
\begin{eulercomment}
Sekarang kami mengasumsikan tingkat bunga 3\% per tahun. Mari kita
tambahkan satu tingkat sederhana dan hitung hasilnya.
\end{eulercomment}
\begin{eulerprompt}
>K*1.03
\end{eulerprompt}
\begin{euleroutput}
  5150
\end{euleroutput}
\begin{eulercomment}
Euler akan memahami sintaks berikut juga.
\end{eulercomment}
\begin{eulerprompt}
>K+K*3%
\end{eulerprompt}
\begin{euleroutput}
  5150
\end{euleroutput}
\begin{eulercomment}
Tapi lebih mudah menggunakan faktornya
\end{eulercomment}
\begin{eulerprompt}
>q=1+3%, K*q
\end{eulerprompt}
\begin{euleroutput}
  1.03
  5150
\end{euleroutput}
\begin{eulercomment}
Selama 10 tahun, kita cukup mengalikan faktor dan mendapatkan nilai
akhir dengan suku bunga majemuk.
\end{eulercomment}
\begin{eulerprompt}
>K*q^10
\end{eulerprompt}
\begin{euleroutput}
  6719.58189672
\end{euleroutput}
\begin{eulercomment}
Untuk tujuan kami, kami dapat mengatur format menjadi 2 digit setelah
titik desimal.
\end{eulercomment}
\begin{eulerprompt}
>format(12,2); K*q^10
\end{eulerprompt}
\begin{euleroutput}
      6719.58 
\end{euleroutput}
\begin{eulercomment}
Mari kita cetak yang dibulatkan menjadi 2 digit itu dalam kalimat
lengkap.
\end{eulercomment}
\begin{eulerprompt}
>"Mulai dari " + K + "$ Anda mendapatkan " + round(K*q^10,2) + "$."
\end{eulerprompt}
\begin{euleroutput}
  Mulai dari 5000$ Anda mendapatkan 6719.58$.
\end{euleroutput}
\begin{eulercomment}
Bagaimana jika kita ingin mengetahui hasil antara tahun 1 sampai tahun
9? Untuk ini, bahasa matriks Euler sangat membantu. Anda tidak perlu
menulis loop, tetapi cukup masukkan
\end{eulercomment}
\begin{eulerprompt}
>K*q^(0:10)
\end{eulerprompt}
\begin{euleroutput}
  Real 1 x 11 matrix
  
      5000.00     5150.00     5304.50     5463.64     ...
\end{euleroutput}
\begin{eulercomment}
Bagaimana keajaiban ini bekerja? Pertama, ekspresi 0:10 mengembalikan
vektor bilangan bulat.
\end{eulercomment}
\begin{eulerprompt}
>short 0:10
\end{eulerprompt}
\begin{euleroutput}
  [0,  1,  2,  3,  4,  5,  6,  7,  8,  9,  10]
\end{euleroutput}
\begin{eulercomment}
Kemudian semua operator dan fungsi di Euler dapat diterapkan ke elemen
vektor untuk elemen. jadi
\end{eulercomment}
\begin{eulerprompt}
>short q^(0:10)
\end{eulerprompt}
\begin{euleroutput}
  [1,  1.03,  1.0609,  1.0927,  1.1255,  1.1593,  1.1941,  1.2299,
  1.2668,  1.3048,  1.3439]
\end{euleroutput}
\begin{eulercomment}
adalah vektor faktor q \textasciicircum{} 0 hingga q \textasciicircum{} 10. Ini dikalikan dengan K, dan
kita mendapatkan nilai vektor.
\end{eulercomment}
\begin{eulerprompt}
>VK=K*q^(0:10);
\end{eulerprompt}
\begin{eulercomment}
Tentu saja, cara realistis untuk menghitung suku bunga ini adalah
dengan membulatkan ke sen terdekat setiap tahun. Mari kita tambahkan
fungsi untuk ini.
\end{eulercomment}
\begin{eulerprompt}
>function oneyear (K) := round(K*q,2)
\end{eulerprompt}
\begin{eulercomment}
Mari kita bandingkan kedua hasil tersebut, dengan dan tanpa
pembulatan.
\end{eulercomment}
\begin{eulerprompt}
>longest oneyear(1234.57), longest 1234.57*q
\end{eulerprompt}
\begin{euleroutput}
                  1271.61 
                1271.6071 
\end{euleroutput}
\begin{eulercomment}
Sekarang tidak ada rumus sederhana untuk tahun ke-n, dan kita harus
mengulang selama bertahun-tahun. Euler memberikan banyak solusi untuk
ini.

Cara termudah adalah fungsi iterasi, yang mengulang fungsi tertentu
beberapa kali.
\end{eulercomment}
\begin{eulerprompt}
>VKr=iterate("oneyear",5000,10)
\end{eulerprompt}
\begin{euleroutput}
  Real 1 x 11 matrix
  
      5000.00     5150.00     5304.50     5463.64     ...
\end{euleroutput}
\begin{eulercomment}
Kami dapat mencetaknya dengan cara yang ramah, menggunakan format kami
dengan tempat desimal tetap.
\end{eulercomment}
\begin{eulerprompt}
>VKr'
\end{eulerprompt}
\begin{euleroutput}
      5000.00 
      5150.00 
      5304.50 
      5463.64 
      5627.55 
      5796.38 
      5970.27 
      6149.38 
      6333.86 
      6523.88 
      6719.60 
\end{euleroutput}
\begin{eulercomment}
Untuk mendapatkan elemen tertentu dari vektor, kami menggunakan indeks
dalam tanda kurung siku.
\end{eulercomment}
\begin{eulerprompt}
>VKr[2], VKr[1:3]
\end{eulerprompt}
\begin{euleroutput}
      5150.00 
      5000.00     5150.00     5304.50 
\end{euleroutput}
\begin{eulercomment}
Anehnya, kita juga bisa menggunakan indeks vektor. Ingat bahwa 1: 3
menghasilkan vektor [1,2,3].

Mari kita bandingkan elemen terakhir dari nilai yang dibulatkan dengan
nilai penuh.
\end{eulercomment}
\begin{eulerprompt}
>VKr[-1], VK[-1]
\end{eulerprompt}
\begin{euleroutput}
      6719.60 
      6719.58 
\end{euleroutput}
\begin{eulercomment}
Perbedaannya sangat kecil.

\begin{eulercomment}
\eulerheading{Memecahkan Persamaan}
\begin{eulercomment}
Sekarang kita ambil fungsi yang lebih maju, yang menambahkan jumlah
uang tertentu setiap tahun.
\end{eulercomment}
\begin{eulerprompt}
>function onepay (K) := K*q+R
\end{eulerprompt}
\begin{eulercomment}
Kami tidak harus menentukan q atau R untuk definisi fungsi. Hanya jika
kita menjalankan perintah, kita harus menentukan nilai-nilai ini. Kami
memilih R = 200.
\end{eulercomment}
\begin{eulerprompt}
>R=200; iterate("onepay",5000,10)
\end{eulerprompt}
\begin{euleroutput}
  Real 1 x 11 matrix
  
      5000.00     5350.00     5710.50     6081.82     ...
\end{euleroutput}
\begin{eulercomment}
Bagaimana jika kita menghapus jumlah yang sama setiap tahun?
\end{eulercomment}
\begin{eulerprompt}
>R=-200; iterate("onepay",5000,10)
\end{eulerprompt}
\begin{euleroutput}
  Real 1 x 11 matrix
  
      5000.00     4950.00     4898.50     4845.45     ...
\end{euleroutput}
\begin{eulercomment}
Kami melihat bahwa uang berkurang. Jelas, jika kita hanya mendapatkan
150 bunga di tahun pertama, tetapi menghapus 200, kita kehilangan uang
setiap tahun.

Bagaimana kita bisa menentukan berapa tahun uang itu akan bertahan?
Kami harus menulis loop untuk ini. Cara termudah adalah mengulanginya
cukup lama.
\end{eulercomment}
\begin{eulerprompt}
>VKR=iterate("onepay",5000,50)
\end{eulerprompt}
\begin{euleroutput}
  Real 1 x 51 matrix
  
      5000.00     4950.00     4898.50     4845.45     ...
\end{euleroutput}
\begin{eulercomment}
Dengan menggunakan bahasa matriks, kita dapat menentukan nilai negatif
pertama dengan cara berikut.
\end{eulercomment}
\begin{eulerprompt}
>min(nonzeros(VKR<0))
\end{eulerprompt}
\begin{euleroutput}
        48.00 
\end{euleroutput}
\begin{eulercomment}
Alasan untuk ini adalah bahwa nonzeros (VKR \textless{}0) mengembalikan vektor
indeks i, di mana VKR [i] \textless{}0, dan min menghitung indeks minimal.

Karena vektor selalu dimulai dengan indeks 1, jawabannya adalah 47
tahun.

Fungsi iterate () memiliki satu trik lagi. Ini bisa mengambil kondisi
akhir sebagai argumen. Maka itu akan mengembalikan nilai dan jumlah
iterasi.
\end{eulercomment}
\begin{eulerprompt}
>\{x,n\}=iterate("onepay",5000,till="x<0"); x, n,
\end{eulerprompt}
\begin{euleroutput}
       -19.83 
        47.00 
\end{euleroutput}
\begin{eulercomment}
Mari kita coba menjawab pertanyaan yang lebih ambigu. Asumsikan kita
tahu bahwa nilainya 0 setelah 50 tahun. Berapa tingkat bunganya?

Ini adalah pertanyaan yang hanya bisa dijawab secara numerik. Di bawah
ini, kami akan mendapatkan rumus yang diperlukan. Maka Anda akan
melihat bahwa tidak ada rumus mudah untuk suku bunga. Tetapi untuk
saat ini, kami bertujuan untuk solusi numerik.

Langkah pertama adalah menentukan fungsi yang melakukan iterasi
sebanyak n kali. Kami menambahkan semua parameter ke fungsi ini.
\end{eulercomment}
\begin{eulerprompt}
>function f(K,R,P,n) := iterate("x*(1+P/100)+R",K,n;P,R)[-1]
\end{eulerprompt}
\begin{eulercomment}
Iterasinya sama seperti di atas

lateks: x\_ \{n + 1\} = x\_n \textbackslash{} cdot \textbackslash{} kiri (1+ \textbackslash{} frac \{P\} \{100\} \textbackslash{} kanan) +
R

Tapi kami lebih lama menggunakan nilai global R dalam ekspresi kami.
Fungsi seperti iterate () memiliki trik khusus di Euler. Anda dapat
meneruskan nilai variabel dalam ekspresi sebagai parameter titik koma.
Dalam hal ini P dan R.

Apalagi kami hanya tertarik pada nilai terakhir. Jadi kami mengambil
indeks [-1].

Mari kita coba tes.
\end{eulercomment}
\begin{eulerprompt}
>f(5000,-200,3,47)
\end{eulerprompt}
\begin{euleroutput}
       -19.83 
\end{euleroutput}
\begin{eulercomment}
Sekarang kita bisa menyelesaikan masalah kita.
\end{eulercomment}
\begin{eulerprompt}
>solve("f(5000,-200,x,50)",3)
\end{eulerprompt}
\begin{euleroutput}
         3.15 
\end{euleroutput}
\begin{eulercomment}
Rutin menyelesaikan memecahkan ekspresi = 0 untuk variabel x.
Jawabannya adalah 3,15\% per tahun. Kami mengambil nilai awal 3\% untuk
algoritme. Fungsi Solving () selalu membutuhkan nilai awal.

Kita dapat menggunakan fungsi yang sama untuk menjawab pertanyaan
berikut: Berapa banyak yang dapat kita keluarkan per tahun sehingga
modal awal habis setelah 20 tahun dengan asumsi tingkat bunga 3\% per
tahun.
\end{eulercomment}
\begin{eulerprompt}
>solve("f(5000,x,3,20)",-200)
\end{eulerprompt}
\begin{euleroutput}
      -336.08 
\end{euleroutput}
\begin{eulercomment}
Perhatikan bahwa Anda tidak dapat menyelesaikan jumlah tahun, karena
fungsi kami mengasumsikan n sebagai nilai integer.

\end{eulercomment}
\eulersubheading{Solusi Simbolis untuk Masalah Suku Bunga}
\begin{eulercomment}
Kita dapat menggunakan bagian simbolik Euler untuk mempelajari
masalahnya. Pertama kita mendefinisikan fungsi onepay () secara
simbolis.
\end{eulercomment}
\begin{eulerprompt}
>function op(K) &= K*q+R; $&op(K)
\end{eulerprompt}
\begin{eulercomment}
Kami sekarang dapat mengulang ini.
\end{eulercomment}
\begin{eulerprompt}
>$&op(op(op(op(K)))), $&expand(%)
\end{eulerprompt}
\begin{eulercomment}
Kami melihat sebuah pola. Setelah n periode yang kita miliki

lateks: K\_n = q \textasciicircum{} n K + R (1 + q + \textbackslash{} ldots + q \textasciicircum{} \{n-1\}) = q \textasciicircum{} n K + \textbackslash{}
frac \{q \textasciicircum{} n-1\} \{q-1\} R

Rumusnya adalah rumus jumlah geometris, yang dikenal dengan Maxima.
\end{eulercomment}
\begin{eulerprompt}
>&sum(q^k,k,0,n-1); $& % = ev(%,simpsum)
\end{eulerprompt}
\begin{eulercomment}
Ini agak rumit. Jumlahnya dievaluasi dengan bendera "simpsum" untuk
menguranginya menjadi hasil bagi.

Mari kita buat fungsi untuk ini.
\end{eulercomment}
\begin{eulerprompt}
>function fs(K,R,P,n) &= (1+P/100)^n*K + ((1+P/100)^n-1)/(P/100)*R; $&fs(K,R,P,n)
\end{eulerprompt}
\begin{eulercomment}
Fungsinya sama dengan fungsi f kita sebelumnya. Tapi itu lebih
efektif.
\end{eulercomment}
\begin{eulerprompt}
>longest f(5000,-200,3,47), longest fs(5000,-200,3,47)
\end{eulerprompt}
\begin{euleroutput}
       -19.82504734650985 
       -19.82504734652684 
\end{euleroutput}
\begin{eulercomment}
Sekarang kita dapat menggunakannya untuk menanyakan waktu n. Kapan
modal kita habis? Tebakan awal kami adalah 30 tahun.
\end{eulercomment}
\begin{eulerprompt}
>solve("fs(5000,-330,3,x)",30)
\end{eulerprompt}
\begin{euleroutput}
        20.51 
\end{euleroutput}
\begin{eulercomment}
Jawaban ini mengatakan bahwa itu akan menjadi negatif setelah 21
tahun.

Kita juga dapat menggunakan sisi simbolik Euler untuk menghitung rumus
pembayaran.

Asumsikan kita mendapat pinjaman sebesar K, dan membayar n pembayaran
R (dimulai setelah tahun pertama) meninggalkan sisa utang Kn (pada
saat pembayaran terakhir). Rumusnya jelas
\end{eulercomment}
\begin{eulerprompt}
>equ &= fs(K,R,P,n)=Kn; $&equ
\end{eulerprompt}
\begin{eulercomment}
Biasanya rumus ini diberikan dalam bentuk

getah: i = \textbackslash{} frac \{P\} \{100\}
\end{eulercomment}
\begin{eulerprompt}
>equ &= (equ with P=100*i); $&equ
\end{eulerprompt}
\begin{eulercomment}
Kita bisa mencari nilai R secara simbolis.
\end{eulercomment}
\begin{eulerprompt}
>$&solve(equ,R)
\end{eulerprompt}
\begin{eulercomment}
Seperti yang Anda lihat dari rumusnya, fungsi ini mengembalikan
kesalahan titik mengambang untuk i = 0. Euler tetap merencanakannya.

Tentu saja, kami memiliki batasan berikut.
\end{eulercomment}
\begin{eulerprompt}
>$&limit(R(5000,0,x,10),x,0)
\end{eulerprompt}
\begin{eulercomment}
Jelas, tanpa bunga kita harus membayar kembali 10 bunga dari 500.

Persamaan ini juga bisa diselesaikan untuk n. Ini terlihat lebih
bagus, jika kita menerapkan beberapa penyederhanaan padanya.
\end{eulercomment}
\begin{eulerprompt}
>fn &= solve(equ,n) | ratsimp; $&fn
\end{eulerprompt}
\begin{eulercomment}
TUGAS APLIKOM (berdasarkan topik)\\
1.Melakukan operasi bentuk bentuk aljabar
\end{eulercomment}
\begin{eulerprompt}
>$&(8*y^5)*(9*y)
>$&(3*a^2)*(-7*a^4)
>$&expand((x+3)^2)
>$&expand((2*x^2+3*y)^2)
>$&expand ((5*-3)^2)
>$&factor(t^2+8*t+15)
>$&factor(2* x^2 + 11*x-21)
>$&expand((n+6)*(n-6))
\end{eulerprompt}
\begin{eulercomment}
2. menulis ekspresi padanan tanpa negatif eksponen
\end{eulercomment}
\begin{eulerprompt}
>$&3^(-7)
>$&(3*m^4)^3*(2*m^(-5))^4
>$&m^(-1)*n^(-12)/t^(-6)
\end{eulerprompt}
\begin{eulercomment}
3. Melakukan perhitungan dengan menggunakan bilangan kompleks
\end{eulercomment}
\begin{eulerprompt}
>(-5+3i)+(7+8i)
\end{eulerprompt}
\begin{euleroutput}
              2.00+11.00i 
\end{euleroutput}
\begin{eulerprompt}
>(-6-5i)+(9+2i)
\end{eulerprompt}
\begin{euleroutput}
               3.00-3.00i 
\end{euleroutput}
\begin{eulerprompt}
>(10+7i)-(5+3i)
\end{eulerprompt}
\begin{euleroutput}
               5.00+4.00i 
\end{euleroutput}
\begin{eulerprompt}
>(13+9i)-(8+2i)
\end{eulerprompt}
\begin{euleroutput}
               5.00+7.00i 
\end{euleroutput}
\begin{eulerprompt}
>(-6+7i)-(-5-21)
\end{eulerprompt}
\begin{euleroutput}
              20.00+7.00i 
\end{euleroutput}
\begin{eulercomment}
4. Melakukan perhitungan dengan perhitungan buatan sendiri
\end{eulercomment}
\begin{eulerprompt}
>function f(x):=3x+1
>function g(x):=x^2-2x-6
>function h(x):=x^3
>h(f(1))
\end{eulerprompt}
\begin{euleroutput}
        64.00 
\end{euleroutput}
\begin{eulerprompt}
>g(f(5))
\end{eulerprompt}
\begin{euleroutput}
       218.00 
\end{euleroutput}
\begin{eulerprompt}
>f(f(-4))
\end{eulerprompt}
\begin{euleroutput}
       -32.00 
\end{euleroutput}
\begin{eulerprompt}
>g(g(3))
\end{eulerprompt}
\begin{euleroutput}
         9.00 
\end{euleroutput}
\begin{eulerprompt}
>f(g(-1))
\end{eulerprompt}
\begin{euleroutput}
        -8.00 
\end{euleroutput}
\begin{eulercomment}
5. Menyelesaikan persamaan dan sistem persamaan\\
6. Menyelesaikan pertidaksamaan dan sistem pertidaksamaan\\
7. Melakukan manipulasi dan perhitungan matriks dan vektor
\end{eulercomment}
\begin{eulerprompt}
>$&solve(y^2+12*y+27)
>$&solve(x^2+100=20*x,x)
>$&solve(t^2+8*t+15)
>$&solve(((x^2+x-6)/(x^2+8*x+15))*((x^2-25)/(x^2-4*x+4)),x)
\end{eulerprompt}
\begin{eulercomment}
TUGAS APLIKOM MENGERJAKAN SOAL (berdasarkan soal)

R exercise 2\\
\end{eulercomment}
\eulersubheading{}
\begin{eulercomment}
No 49\\
Menyederhanakan:\\
\end{eulercomment}
\begin{eulerformula}
\[
\left(\frac{24a^{10}b^{-8}c^7}{12a^6b^{-3}c^5}\right)^{-5}
\]
\end{eulerformula}
\begin{eulerprompt}
>$&((24*a^(10)*b^(-8)*c^7)/(12*a^6*b^(-3)*c^5))^(-5)
\end{eulerprompt}
\begin{eulercomment}
No 50\\
Menyederhanakan:\\
\end{eulercomment}
\begin{eulerformula}
\[
\left(\frac{125p^{12}q^{-14}r^{22}}{25p^8q^6r^{-15}}\right)^{-4}
\]
\end{eulerformula}
\begin{eulerprompt}
>$&((125*p^(12)*q^(-14)*r^(22))/25*p^8*q^6*r^(-15))^(-4)
\end{eulerprompt}
\begin{eulercomment}
No 90\\
operasi bilangan matematika\\
\end{eulercomment}
\begin{eulerformula}
\[
2^6*2^{-3}/2^{10}/2^{-8}
\]
\end{eulerformula}
\begin{eulerprompt}
>2^6*2^-3/2^10/2^-8
\end{eulerprompt}
\begin{euleroutput}
         2.00 
\end{euleroutput}
\begin{eulercomment}
No 91\\
Operasi bilangan matematika\\
\end{eulercomment}
\begin{eulerformula}
\[
\left(\frac{4(8-6)^2-4*3+2*8}{3^1+9^0}\right)
\]
\end{eulerformula}
\begin{eulerprompt}
>(4*(8-6)^2 - 4*3 + 2*8)/(3^1+19^0)
\end{eulerprompt}
\begin{euleroutput}
         5.00 
\end{euleroutput}
\begin{eulercomment}
No 92\\
Calculate\\
\end{eulercomment}
\begin{eulerformula}
\[
\left(\frac{[4(8-6)^2-4](3+2*8)}{2^2(2^5+5)}\right)
\]
\end{eulerformula}
\begin{eulercomment}
\end{eulercomment}
\begin{eulerprompt}
>((4*(8-6)^2-4)*3+2*8)/(3^1+9^0)
\end{eulerprompt}
\begin{euleroutput}
        13.00 
\end{euleroutput}
\begin{eulercomment}
R exercise 3\\
\end{eulercomment}
\eulersubheading{}
\begin{eulercomment}
no 27\\
\end{eulercomment}
\begin{eulerformula}
\[
(x+3)^2
\]
\end{eulerformula}
\begin{eulerprompt}
> $&showev('expand((x+3)^2))
\end{eulerprompt}
\begin{eulercomment}
no 29\\
\end{eulercomment}
\begin{eulerformula}
\[
(y-5)^2
\]
\end{eulerformula}
\begin{eulerprompt}
>$&showev('expand((y-5)^2))
\end{eulerprompt}
\begin{eulercomment}
no 33\\
\end{eulercomment}
\begin{eulerformula}
\[
(2x+3y)^2
\]
\end{eulerformula}
\begin{eulerprompt}
>$&showev('expand((2*x+3*y)^2))
\end{eulerprompt}
\begin{eulercomment}
no 39\\
\end{eulercomment}
\begin{eulerformula}
\[
(3y+4)(3y-4)
\]
\end{eulerformula}
\begin{eulerprompt}
>$&showev('expand((3*y+4)*(3*y-4)))
\end{eulerprompt}
\begin{eulercomment}
no 42\\
\end{eulercomment}
\begin{eulerformula}
\[
(3x+5y)(3x-5y)
\]
\end{eulerformula}
\begin{eulerprompt}
>$&showev('expand ((3*x + 5*y)*(3*x - 5*y)))
\end{eulerprompt}
\begin{eulercomment}
R exercise 4\\
\end{eulercomment}
\eulersubheading{}
\begin{eulercomment}
no 24\\
\end{eulercomment}
\begin{eulerformula}
\[
y^2+12y+27
\]
\end{eulerformula}
\begin{eulerprompt}
>$&solve(y^2+12*y+27)
\end{eulerprompt}
\begin{eulercomment}
no 23\\
\end{eulercomment}
\begin{eulerformula}
\[
t^2+8t+15
\]
\end{eulerformula}
\begin{eulerprompt}
>$&solve(t^2+8*t+15)
\end{eulerprompt}
\begin{eulercomment}
Nomor 47\\
\end{eulercomment}
\begin{eulerformula}
\[
z^2-81
\]
\end{eulerformula}
\begin{eulerprompt}
>$&solve(z^2-81)
\end{eulerprompt}
\begin{eulercomment}
no 48\\
\end{eulercomment}
\begin{eulerformula}
\[
m^2-4
\]
\end{eulerformula}
\begin{eulerprompt}
>$&solve(m^2-4)
\end{eulerprompt}
\begin{eulercomment}
no 49\\
\end{eulercomment}
\begin{eulerformula}
\[
16x^2-9
\]
\end{eulerformula}
\begin{eulerprompt}
>$&solve(16*x^2-9)
\end{eulerprompt}
\begin{eulercomment}
R exercise 5\\
\end{eulercomment}
\eulersubheading{}
\begin{eulercomment}
soal no 36\\
tentukan nilai y\\
\end{eulercomment}
\begin{eulerformula}
\[
y^2-4y-45=0
\]
\end{eulerformula}
\begin{eulerprompt}
>$&solve(y^2-4*y-45,y)
\end{eulerprompt}
\begin{eulercomment}
soal no 38\\
tentukan nilai y\\
\end{eulercomment}
\begin{eulerformula}
\[
t^2+6t=0
\]
\end{eulerformula}
\begin{eulerprompt}
>$&solve(t^2+ 6*t,t)
\end{eulerprompt}
\begin{eulercomment}
soal no 41\\
tentukan nilai x\\
\end{eulercomment}
\begin{eulerformula}
\[
x^2 + 100 = 20x
\]
\end{eulerformula}
\begin{eulerprompt}
>$&solve(x^2+100=20*x,x)
\end{eulerprompt}
\begin{eulercomment}
soal no 42\\
tentukan nilai y\\
\end{eulercomment}
\begin{eulerformula}
\[
y^2+25=10y
\]
\end{eulerformula}
\begin{eulerprompt}
>$&solve(y^2+25=10*y,y)
\end{eulerprompt}
\begin{eulercomment}
soal no 45\\
tentukan nilai y\\
\end{eulercomment}
\begin{eulerformula}
\[
3y^2 + 8y + 4=0
\]
\end{eulerformula}
\begin{eulerprompt}
>$&solve(3*y^2 + 8*y + 4,y)
\end{eulerprompt}
\begin{eulercomment}
soal no 47\\
tentukan nilai z\\
\end{eulercomment}
\begin{eulerformula}
\[
12z^2+z=6
\]
\end{eulerformula}
\begin{eulerprompt}
>$&solve(12*z^2+z=6,z)
\end{eulerprompt}
\begin{eulercomment}
soal no 60\\
tentukan nilai x\\
\end{eulercomment}
\begin{eulerformula}
\[
5x^2-75=0
\]
\end{eulerformula}
\begin{eulerprompt}
>$&solve(5*x^2-75=0,x)
\end{eulerprompt}
\begin{eulercomment}
R exercise 6\\
\end{eulercomment}
\eulersubheading{}
\begin{eulercomment}
no 9\\
Menyederhanakan\\
\end{eulercomment}
\begin{eulerformula}
\[
\frac{x^{2}-4}{x^{2}-4x+4}
\]
\end{eulerformula}
\begin{eulerprompt}
>$&((x^2)/(x^2-4*x+4)), $&factor(%)
\end{eulerprompt}
\begin{eulercomment}
no 11\\
Menyederhanakan\\
\end{eulercomment}
\begin{eulerformula}
\[
\frac{x^{3}-6x^{2}+9x}{x^{3}-3x^{2}}
\]
\end{eulerformula}
\begin{eulerprompt}
>$&((x^3 - 6*x^2 + 9*x)/ (x^3 - 3*x^2)), $&factor(%)
\end{eulerprompt}
\begin{eulercomment}
no 14\\
Menyederhanakan\\
\end{eulercomment}
\begin{eulerformula}
\[
\frac{2x^{2}-20x+50}{10x^{2}-30x-100}
\]
\end{eulerformula}
\begin{eulerprompt}
>$&((2*x^2-20*x+50)/(10*x^2-30*x-100)), $&factor(%)
\end{eulerprompt}
\begin{eulercomment}
no 15\\
Menyederhanakan\\
\end{eulercomment}
\begin{eulerformula}
\[
\frac{4-x}{x^{2}+4x-32}
\]
\end{eulerformula}
\begin{eulerprompt}
>$&((4-x)/(x^2 +4*x-32)), $&factor(%)
\end{eulerprompt}
\begin{eulercomment}
no 16\\
Menyederhanakan\\
\end{eulercomment}
\begin{eulerformula}
\[
\frac{6-x}{x^{2}-36}
\]
\end{eulerformula}
\begin{eulerprompt}
>$&((6-x)/(x^2-36)), $&factor(%)
\end{eulerprompt}
\begin{eulercomment}
no 23
\end{eulercomment}
\begin{eulerprompt}
>$&(((m^2-n^2)/(r+s))/((m-n)/r+s)), $&factor(%)
\end{eulerprompt}
\begin{eulercomment}
no 25
\end{eulercomment}
\begin{eulerprompt}
>$&(((3*x+12)/(2*x-8))/(((x+4)^2)/(x-4)^2)), $&factor(%)
\end{eulerprompt}
\begin{eulercomment}
no 28
\end{eulercomment}
\begin{eulerprompt}
>$&(((c^3+8)/(c^2-4)) /((c^2 -2*c +4)/(c^2-4*c+4))), $&factor(%) 
\end{eulerprompt}
\begin{eulercomment}
Review\\
\end{eulercomment}
\eulersubheading{}
\begin{eulercomment}
no 73\\
\end{eulercomment}
\begin{eulerformula}
\[
(a^n-b^n)^x
\]
\end{eulerformula}
\begin{eulerprompt}
>function P(a,b,n,x) &= (a^n - b^n)^x; $&P(a,b,n,x)
>$&P(a,b,n,3), $&expand(%)
\end{eulerprompt}
\begin{eulercomment}
no 71
\end{eulercomment}
\begin{eulerprompt}
>function P(a,n) &= (t^a+t^(-a))^n; $&P(a,n)
>$&P(a,2), $&expand(%)
\end{eulerprompt}
\begin{eulercomment}
no 39
\end{eulercomment}
\begin{eulerprompt}
>$&solve(9*x^2-30*x+25,x)
\end{eulerprompt}
\begin{eulercomment}
no 41
\end{eulercomment}
\begin{eulerprompt}
>$&solve(18*x^2-3*x+6,x)
\end{eulerprompt}
\begin{eulercomment}
no 48
\end{eulercomment}
\begin{eulerprompt}
>$&solve((8-3*x)=(-7+2*x),x)
\end{eulerprompt}
\begin{eulercomment}
no 70
\end{eulercomment}
\begin{eulerprompt}
>$& '((x^n+10)*(x^n-4)) = expand(((x^n+10)*(x^n-4)))
\end{eulerprompt}
\begin{eulercomment}
no 71
\end{eulercomment}
\begin{eulerprompt}
>$& '((t^a+t^(-a))^2) = expand(((t^a+t^(-a))^2))
\end{eulerprompt}
\begin{eulercomment}
no 72
\end{eulercomment}
\begin{eulerprompt}
>$& '((y^b-z^c)*(y^b+z^c)) = expand((y^b-z^c)*(y^b+z^c))
\end{eulerprompt}
\begin{eulercomment}
no 73
\end{eulercomment}
\begin{eulerprompt}
>$& '((a^n-b^n)^3) = expand((a^n-b^n)^3)
\end{eulerprompt}
\begin{eulercomment}
chapter R test\\
\end{eulercomment}
\eulersubheading{}
\begin{eulerttcomment}
 no 32
\end{eulerttcomment}
\begin{eulerprompt}
>$& '(((x^2+x-6)/(x^2+8*x+15))*((x^2-25)/(x^2-4*x+4))) = simplify(((x^2+x-6)/(x^2+8*x+15))*((x^2-25)/(x^2-4*x+4)))
>$&solve(((x^2+x-6)/(x^2+8*x+15))*((x^2-25)/(x^2-4*x+4)),x)
\end{eulerprompt}
\begin{eulercomment}
no 33
\end{eulercomment}
\begin{eulerprompt}
>$& '(((x)/(x^2-1))-((3)/(x^2+4*x-5)))=simplify(((x)/(x^2-1))-((3)/(x^2+4*x-5)))
>$&solve(((x)/(x^2-1))-((3)/(x^2+4*x-5)))
\end{eulerprompt}
\begin{eulercomment}
exercise 2.3\\
\end{eulercomment}
\eulersubheading{}
\begin{eulercomment}
Cari

\end{eulercomment}
\begin{eulerformula}
\[
\left(f\circ g\right)\left(x\right) dan \left(g\circ f\right)\left(x\right)
\]
\end{eulerformula}
\begin{eulercomment}
dan domain nya !

\end{eulercomment}
\begin{eulerformula}
\[
1. f(x)=x+3\ ,\ g(x)=x-3
\]
\end{eulerformula}
\begin{eulercomment}
\end{eulercomment}
\begin{eulerformula}
\[
\left(f\circ g\right)\left(x\right)=
\]
\end{eulerformula}
\begin{eulerprompt}
>$&gx:=x-3; $&fx:=gx+3; $&fx
\end{eulerprompt}
\begin{eulercomment}
dengan domainnya

\end{eulercomment}
\begin{eulerformula}
\[
D_{f\circ g}=\left\{x\in\mathbb{R}\right\}
\]
\end{eulerformula}
\begin{eulercomment}
\end{eulercomment}
\begin{eulerformula}
\[
\left(g\circ f\right)\left(x\right)=
\]
\end{eulerformula}
\begin{eulerprompt}
>$&fx:=x+3; $&gx:=fx-3; $&gx
\end{eulerprompt}
\begin{eulercomment}
dengan domainnya\\
\end{eulercomment}
\begin{eulerformula}
\[
D_{g\circ f}=\left\{x\in\mathbb{R}\right\}
\]
\end{eulerformula}
\begin{eulercomment}
\end{eulercomment}
\begin{eulerformula}
\[
2. f(x)=4/(1-5x)\ \ ,\ g(x)=1/x
\]
\end{eulerformula}
\begin{eulerformula}
\[
\left(f\circ g\right)\left(x\right)=
\]
\end{eulerformula}
\begin{eulerprompt}
>$&gx:=1/x; $&fx:=4/(1-5*gx); $&fx
\end{eulerprompt}
\begin{eulercomment}
dengan domain\\
\end{eulercomment}
\begin{eulerformula}
\[
D_{f\circ g}=\left\{x\in\mathbb{R}|x\neq 0\cup x\neq 5\right\}
\]
\end{eulerformula}
\begin{eulercomment}
\end{eulercomment}
\begin{eulerformula}
\[
\left(g\circ f\right)\left(x\right)=
\]
\end{eulerformula}
\begin{eulerprompt}
>$&fx:=4/(1-5*x); $&gx:=1/fx; $&gx
\end{eulerprompt}
\begin{eulercomment}
dengan domainnya\\
\end{eulercomment}
\begin{eulerformula}
\[
D_{g\circ f}=\left\{x\in\mathbb{R}\right\}
\]
\end{eulerformula}
\begin{eulercomment}
Diberikan fungsi\\
\end{eulercomment}
\begin{eulerformula}
\[
f(x)=3x+1 , g(x)=x^2-2x-6 , h(x)=x^3
\]
\end{eulerformula}
\begin{eulercomment}
cari\\
\end{eulercomment}
\begin{eulerformula}
\[
3.\ \left(f\circ g\right)\left(1/3\right)
\]
\end{eulerformula}
\begin{eulerprompt}
>$x:=1/3; $&gx:=x^2-2*x-6; $&fx:=3*gx+1; $&fx
\end{eulerprompt}
\begin{eulerformula}
\[
4. \left(g\circ h\right)\left(1/2\right)
\]
\end{eulerformula}
\begin{eulerprompt}
>$x:=1/2; $&hx:=x^3; $&gx:=hx^2-2*hx-6; $&gx
\end{eulerprompt}
\begin{eulerformula}
\[
5. \left(g\circ g\right)\left(-2\right)
\]
\end{eulerformula}
\begin{eulerprompt}
>$x:=-2; $&gx:=x^2-2*x-6; $&gx:=gx^2-2*gx-6; $&gx
\end{eulerprompt}
\begin{eulercomment}
Exercise 3.1 \\
\end{eulercomment}
\eulersubheading{}
\begin{eulercomment}
no 37\\
\end{eulercomment}
\begin{eulerformula}
\[
x^2-2=15
\]
\end{eulerformula}
\begin{eulerprompt}
>$&solve(x^2-2= 15,x)
\end{eulerprompt}
\begin{euleroutput}
  Maxima said:
  solve: all variables must not be numbers.
   -- an error. To debug this try: debugmode(true);
  
  Error in:
   $&solve(x^2-2= 15,x) ...
                      ^
\end{euleroutput}
\begin{eulercomment}
no 39\\
\end{eulercomment}
\begin{eulerformula}
\[
5m^2+3m=2
\]
\end{eulerformula}
\begin{eulerprompt}
>$&solve(5*m^2+3*m=2,m)
\end{eulerprompt}
\begin{eulercomment}
no 40\\
\end{eulercomment}
\begin{eulerformula}
\[
2y^2-3y-2=0
\]
\end{eulerformula}
\begin{eulerprompt}
>$&solve(2*y^2-3*y-2=0,y)
\end{eulerprompt}
\begin{eulercomment}
no 83\\
\end{eulercomment}
\begin{eulerformula}
\[
y^4+4y^2-5=0
\]
\end{eulerformula}
\begin{eulerprompt}
>$&solve(y^4+4*y^2-5=0,y)
\end{eulerprompt}
\begin{eulercomment}
no 84\\
\end{eulercomment}
\begin{eulerformula}
\[
y^4-15y^2-16=0
\]
\end{eulerformula}
\begin{eulerprompt}
>$&solve(y^4-15*y^2-16=0,y)
\end{eulerprompt}
\begin{eulerttcomment}
 Exercise 3.4 
\end{eulerttcomment}
\eulersubheading{}
\begin{eulerttcomment}
 no 1
\end{eulerttcomment}
\begin{eulerprompt}
>$&solve(1/4+1/5=1/t,t)
\end{eulerprompt}
\begin{eulercomment}
no 2
\end{eulercomment}
\begin{eulerprompt}
>$&solve(1/4+1/5=1/t)$&solve(1/3-5/6=1/x,x):
\end{eulerprompt}
\begin{euleroutput}
  Maxima said:
  incorrect syntax: solve is not an infix operator
  &solve(
       ^
  
  Error in:
   $&solve(1/4+1/5=1/t)$&solve(1/3-5/6=1/x,x): ...
                                             ^
\end{euleroutput}
\begin{eulercomment}
no 6
\end{eulercomment}
\begin{eulerprompt}
>$&solve(1/t+1/2*t+1/3*t=5,t)
\end{eulerprompt}
\begin{eulercomment}
no 7
\end{eulercomment}
\begin{eulerprompt}
>$&solve(5/3*x+2=3/2*x,x)
\end{eulerprompt}
\begin{euleroutput}
  Maxima said:
  solve: all variables must not be numbers.
   -- an error. To debug this try: debugmode(true);
  
  Error in:
   $&solve(5/3*x+2=3/2*x,x) ...
                          ^
\end{euleroutput}
\begin{eulercomment}
no 10
\end{eulercomment}
\begin{eulerprompt}
>$&solve(x-12/x=1,x)
\end{eulerprompt}
\begin{euleroutput}
  Maxima said:
  solve: all variables must not be numbers.
   -- an error. To debug this try: debugmode(true);
  
  Error in:
   $&solve(x-12/x=1,x) ...
                     ^
\end{euleroutput}
\begin{eulercomment}
no 18
\end{eulercomment}
\begin{eulerprompt}
>$&solve(3*y+5/y^2+5*y +y+4/y+5=y+1/y,y)
\end{eulerprompt}
\begin{eulercomment}
Exercise 3.5 \\
\end{eulercomment}
\eulersubheading{}
\begin{eulerprompt}
>&load(fourier_elim)
\end{eulerprompt}
\begin{euleroutput}
  
          C:/Program Files/Euler x64/maxima/share/maxima/5.35.1/share/f\(\backslash\)
  ourier_elim/fourier_elim.lisp
  
\end{euleroutput}
\begin{eulercomment}
no 44
\end{eulercomment}
\begin{eulerprompt}
>$&fourier_elim([4*x]>20,[x]) // 4*x > 20
\end{eulerprompt}
\begin{euleroutput}
  Maxima said:
  Function "$elim" expects a symbol, instead found -2
   -- an error. To debug this try: debugmode(true);
  
  Error in:
   $&fourier_elim([4*x]>20,[x]) // 4*x > 20 ...
                               ^
\end{euleroutput}
\begin{eulercomment}
no 45
\end{eulercomment}
\begin{eulerprompt}
>$&fourier_elim([x+8]<9,[x])// x+8<9
\end{eulerprompt}
\begin{euleroutput}
  Maxima said:
  Function "$elim" expects a symbol, instead found -2
   -- an error. To debug this try: debugmode(true);
  
  Error in:
   $&fourier_elim([x+8]<9,[x])// x+8<9 ...
                             ^
\end{euleroutput}
\begin{eulercomment}
no 47
\end{eulercomment}
\begin{eulerprompt}
>$&fourier_elim([x+8]>= 9,[x])//x+8 >=9
\end{eulerprompt}
\begin{euleroutput}
  Maxima said:
  Function "$elim" expects a symbol, instead found -2
   -- an error. To debug this try: debugmode(true);
  
  Error in:
   $&fourier_elim([x+8]>= 9,[x])//x+8 >=9 ...
                               ^
\end{euleroutput}
\begin{eulercomment}
no 52
\end{eulercomment}
\begin{eulerprompt}
>$&fourier_elim([3*x+4]<13,[x])//3*x+4<13
\end{eulerprompt}
\begin{euleroutput}
  Maxima said:
  Function "$elim" expects a symbol, instead found -2
   -- an error. To debug this try: debugmode(true);
  
  Error in:
   $&fourier_elim([3*x+4]<13,[x])//3*x+4<13 ...
                                ^
\end{euleroutput}
\begin{eulercomment}
no 62
\end{eulercomment}
\begin{eulerprompt}
>$&fourier_elim([3*x+5]<0,[x])//3*x+5<0
\end{eulerprompt}
\begin{euleroutput}
  Maxima said:
  Function "$elim" expects a symbol, instead found -2
   -- an error. To debug this try: debugmode(true);
  
  Error in:
   $&fourier_elim([3*x+5]<0,[x])//3*x+5<0 ...
                               ^
\end{euleroutput}
\begin{eulercomment}
Chapter 3\\
\end{eulercomment}
\eulersubheading{}
\begin{eulercomment}
no 8
\end{eulercomment}
\begin{eulerprompt}
>$&solve(3/3*x+4 + 2/x-1 =2,x)
\end{eulerprompt}
\begin{euleroutput}
  Maxima said:
  solve: all variables must not be numbers.
   -- an error. To debug this try: debugmode(true);
  
  Error in:
   $&solve(3/3*x+4 + 2/x-1 =2,x) ...
                               ^
\end{euleroutput}
\begin{eulerprompt}
>$&load(fourier_elim)
\end{eulerprompt}
\begin{eulercomment}
no 11
\end{eulercomment}
\begin{eulerprompt}
>$&fourier_elim([x+4]=7,[x])//x+4=7
\end{eulerprompt}
\begin{euleroutput}
  Maxima said:
  Function "$elim" expects a symbol, instead found -2
   -- an error. To debug this try: debugmode(true);
  
  Error in:
   $&fourier_elim([x+4]=7,[x])//x+4=7 ...
                             ^
\end{euleroutput}
\begin{eulercomment}
no 12
\end{eulercomment}
\begin{eulerprompt}
>$&fourier_elim([4*y-3]=5,[x])//4*y-3=5
\end{eulerprompt}
\begin{euleroutput}
  Maxima said:
  Function "$elim" expects a symbol, instead found -2
   -- an error. To debug this try: debugmode(true);
  
  Error in:
   $&fourier_elim([4*y-3]=5,[x])//4*y-3=5 ...
                               ^
\end{euleroutput}
\begin{eulercomment}
no 13
\end{eulercomment}
\begin{eulerprompt}
>$&fourier_elim([x+3]<=4,[x])//x+3<=4
\end{eulerprompt}
\begin{euleroutput}
  Maxima said:
  Function "$elim" expects a symbol, instead found -2
   -- an error. To debug this try: debugmode(true);
  
  Error in:
   $&fourier_elim([x+3]<=4,[x])//x+3<=4 ...
                              ^
\end{euleroutput}
\begin{eulercomment}
no 15
\end{eulercomment}
\begin{eulerprompt}
>$&fourier_elim([x+5]>2,[x])//x+5>2
\end{eulerprompt}
\begin{euleroutput}
  Maxima said:
  Function "$elim" expects a symbol, instead found -2
   -- an error. To debug this try: debugmode(true);
  
  Error in:
   $&fourier_elim([x+5]>2,[x])//x+5>2 ...
                             ^
\end{euleroutput}
\begin{eulercomment}
no 19
\end{eulercomment}
\begin{eulerprompt}
>$&solve(x^2+4*x =1,x)
\end{eulerprompt}
\begin{euleroutput}
  Maxima said:
  solve: all variables must not be numbers.
   -- an error. To debug this try: debugmode(true);
  
  Error in:
   $&solve(x^2+4*x =1,x) ...
                       ^
\end{euleroutput}
\eulerheading{4.1 Exercise Set}
\begin{eulercomment}
Use the substitution to determine whether 2,3 and -1 are zeros of\\
Nomor 23
\end{eulercomment}
\begin{eulerprompt}
>function P(x) &= (x^3-9*x^2+14*x+24); $&P(x)
>P(4)
\end{eulerprompt}
\begin{euleroutput}
       -48.00 
\end{euleroutput}
\begin{eulerprompt}
>P(5)
\end{eulerprompt}
\begin{euleroutput}
       -48.00 
\end{euleroutput}
\begin{eulerprompt}
>P(-2)
\end{eulerprompt}
\begin{euleroutput}
       -48.00 
\end{euleroutput}
\begin{eulercomment}
Jadi hasil substitusi yang menghasilkan persamaan mempunyai nilai nol
adalah dengan mensubtsisusi angka 4

Nomor 24
\end{eulercomment}
\begin{eulerprompt}
>function P(x) &= (2*x^3-3*x^2+x+6);$&P(x)
>P(2)
\end{eulerprompt}
\begin{euleroutput}
       -24.00 
\end{euleroutput}
\begin{eulerprompt}
>P(3)
\end{eulerprompt}
\begin{euleroutput}
       -24.00 
\end{euleroutput}
\begin{eulerprompt}
>P(-1)
\end{eulerprompt}
\begin{euleroutput}
       -24.00 
\end{euleroutput}
\begin{eulercomment}
Jadi hasil substitusi yang menghasilkan persamaan mempunyai nilai nol
adalah dengan mensubtsisusi angka -1

Nomor 25
\end{eulercomment}
\begin{eulerprompt}
>function P(x) &= (x^4-6*x^3+8*x^2+6*x-9);$&P(x)
>P(2)
\end{eulerprompt}
\begin{euleroutput}
        75.00 
\end{euleroutput}
\begin{eulerprompt}
>P(3)
\end{eulerprompt}
\begin{euleroutput}
        75.00 
\end{euleroutput}
\begin{eulerprompt}
>P(-1)
\end{eulerprompt}
\begin{euleroutput}
        75.00 
\end{euleroutput}
\begin{eulercomment}
Jadi hasil substitusi yang menghasilkan persamaan mempunyai nilai nol
adalah dengan mensubtsisusi angka 3 dan -1

Nomor 37
\end{eulercomment}
\begin{eulerprompt}
>$&solve(x^4-4*x^2+3,x)
\end{eulerprompt}
\begin{euleroutput}
  Maxima said:
  solve: all variables must not be numbers.
   -- an error. To debug this try: debugmode(true);
  
  Error in:
   $&solve(x^4-4*x^2+3,x) ...
                        ^
\end{euleroutput}
\begin{eulercomment}
Nomor 39
\end{eulercomment}
\begin{eulerprompt}
>$&solve(x^3+3*x^2-x-3,x)
\end{eulerprompt}
\begin{euleroutput}
  Maxima said:
  solve: all variables must not be numbers.
   -- an error. To debug this try: debugmode(true);
  
  Error in:
   $&solve(x^3+3*x^2-x-3,x) ...
                          ^
\end{euleroutput}
\begin{eulercomment}
Exercise 4.3\\
\end{eulercomment}
\eulersubheading{}
\begin{eulercomment}
no\\
For the function\\
\end{eulercomment}
\begin{eulerformula}
\[
f(x)= x^4-6x^3+x^2+24x-20
\]
\end{eulerformula}
\begin{eulercomment}
use long division to determine whether each of the following is a
factor of f(x)

a) x+1\\
b) x-2\\
c) x + 5
\end{eulercomment}
\begin{eulerprompt}
>function f(x) &= (x^4-6*x^3+x^2+24*x-20);$&f(x)
>$&f(x+1), $&expand(%)
>$&f(x-2), $&expand(%)
>$&f(x+5), $&expand(%)
\end{eulerprompt}
\begin{eulercomment}
no 23\\
Use synthetic division to find the function values.\\
\end{eulercomment}
\begin{eulerformula}
\[
f(x) = x^3-6x^2+11x-6
\]
\end{eulerformula}
\begin{eulercomment}
find f(1), f(-2), dan f(3)
\end{eulercomment}
\begin{eulerprompt}
>function f(x) &= (x^3-6*x^2+11*x-6);$&f(x)
>f(1)
\end{eulerprompt}
\begin{euleroutput}
       -60.00 
\end{euleroutput}
\begin{eulerprompt}
>f(-2)
\end{eulerprompt}
\begin{euleroutput}
       -60.00 
\end{euleroutput}
\begin{eulerprompt}
>f(3)
\end{eulerprompt}
\begin{euleroutput}
       -60.00 
\end{euleroutput}
\begin{eulercomment}
no 24\\
\end{eulercomment}
\begin{eulerformula}
\[
f(x)=x63+7x^2-12x-3
\]
\end{eulerformula}
\begin{eulercomment}
find f(-3),f(-2), dan f(1)
\end{eulercomment}
\begin{eulerprompt}
>function f(x) &= (x^3+7*x^2-12*x-3);$&f(x)
>f(-3)
\end{eulerprompt}
\begin{euleroutput}
        41.00 
\end{euleroutput}
\begin{eulerprompt}
>f(-2)
\end{eulerprompt}
\begin{euleroutput}
        41.00 
\end{euleroutput}
\begin{eulerprompt}
>f(1)
\end{eulerprompt}
\begin{euleroutput}
        41.00 
\end{euleroutput}
\begin{eulercomment}
no 25\\
\end{eulercomment}
\begin{eulerformula}
\[
f(x) = x^4-3x^2+2x+8
\]
\end{eulerformula}
\begin{eulercomment}
find f(-1),f(4) dan f(-5)
\end{eulercomment}
\begin{eulerprompt}
>function f(x) &= (x^4-3*x^3+2*x+8);$&f(x)
>f(-1)
\end{eulerprompt}
\begin{euleroutput}
        44.00 
\end{euleroutput}
\begin{eulerprompt}
>f(4)
\end{eulerprompt}
\begin{euleroutput}
        44.00 
\end{euleroutput}
\begin{eulerprompt}
>f(-5)
\end{eulerprompt}
\begin{euleroutput}
        44.00 
\end{euleroutput}
\begin{eulercomment}
Factor the polynomial function f(x). Then Solve the equation f(x)=0\\
no 39\\
\end{eulercomment}
\begin{eulerformula}
\[
f(x)=x^3+4x^2+x-6
\]
\end{eulerformula}
\begin{eulerprompt}
>fx &= (x^3+4*x^2+x-6=0); $&fx
>$&factor((fx,x^3+4*x^2+x-6=0))
>$&solve(x^3+4*x^2+x-6=0,x)
\end{eulerprompt}
\begin{euleroutput}
  Maxima said:
  solve: all variables must not be numbers.
   -- an error. To debug this try: debugmode(true);
  
  Error in:
   $&solve(x^3+4*x^2+x-6=0,x) ...
                            ^
\end{euleroutput}
\begin{eulercomment}
no 40\\
\end{eulercomment}
\begin{eulerformula}
\[
f(x)=x^3+5x^2-2x-24
\]
\end{eulerformula}
\begin{eulerprompt}
>fx &= (x^3+5*x^2-2*x-24=0); $&fx
>$&factor((fx,x^3+5*x^2-2*x-24=0))
>$&solve(x^3+5*x^2-2*x-24=0,x)
\end{eulerprompt}
\begin{euleroutput}
  Maxima said:
  solve: all variables must not be numbers.
   -- an error. To debug this try: debugmode(true);
  
  Error in:
   $&solve(x^3+5*x^2-2*x-24=0,x) ...
                               ^
\end{euleroutput}
\begin{eulercomment}
Mid-Chapter Mixed Review\\
\end{eulercomment}
\eulersubheading{}
\begin{eulercomment}
Use synthetic division to find the function values\\
no 18\\
\end{eulercomment}
\begin{eulerformula}
\[
g(x) = x^3-9x^2+4x-10
\]
\end{eulerformula}
\begin{eulercomment}
find g(-5)
\end{eulercomment}
\begin{eulerprompt}
>function g(x) &= (x^3-9*x^2+4*x-10);$&g(x)
>g(-5)
\end{eulerprompt}
\begin{euleroutput}
       -62.00 
\end{euleroutput}
\begin{eulercomment}
no 19\\
\end{eulercomment}
\begin{eulerformula}
\[
f(x)=20x^2-40x
\]
\end{eulerformula}
\begin{eulercomment}
find f(1/2)
\end{eulercomment}
\begin{eulerprompt}
>function f(x) &= (20*x^2-40*x);$&f(x)
>f(1/2)
\end{eulerprompt}
\begin{euleroutput}
       160.00 
\end{euleroutput}
\begin{eulercomment}
no\\
-1,5;\\
\end{eulercomment}
\begin{eulerformula}
\[
f(x)=x^6-35x^4+259x^2-225
\]
\end{eulerformula}
\begin{eulerprompt}
>function f(x) &= (x^6-35*x^4+259*x^2-225);$&f(x)
>f(-1.5)
\end{eulerprompt}
\begin{euleroutput}
       315.00 
\end{euleroutput}
\begin{eulercomment}
Factor the polynomial function f(x). Then solve the equation f(x) =0.\\
no 23\\
\end{eulercomment}
\begin{eulerformula}
\[
h(x) = x^3-2x^2-55x+56
\]
\end{eulerformula}
\begin{eulerprompt}
>hx &= (x^3-2*x^2-55*x+56=0); $&hx
>$&factor((hx,x^3-2*x^2-55*x+56=0))
>$&solve(x^3-2*x^2-55*x+56=0,x)
\end{eulerprompt}
\begin{euleroutput}
  Maxima said:
  solve: all variables must not be numbers.
   -- an error. To debug this try: debugmode(true);
  
  Error in:
   $&solve(x^3-2*x^2-55*x+56=0,x) ...
                                ^
\end{euleroutput}
\begin{eulercomment}
no 24\\
\end{eulercomment}
\begin{eulerformula}
\[
g(x) = x^4-2x^3-13x^2+14x+24
\]
\end{eulerformula}
\begin{eulerprompt}
>gx &= (x^4-2*x^3-13*x^2+14*x+24=0); $&gx
>$&factor((gx,x^4-2*x^3-13*x^2+14*x+24=0))
>$&solve(x^4-2*x^3-13*x^2+14*x+24=0,x)
\end{eulerprompt}
\begin{euleroutput}
  Maxima said:
  solve: all variables must not be numbers.
   -- an error. To debug this try: debugmode(true);
  
  Error in:
   $&solve(x^4-2*x^3-13*x^2+14*x+24=0,x) ...
                                       ^
\end{euleroutput}
\begin{eulerprompt}
>function g(x,a=1) :=a*x^3+2*(a*x)^2+4*a*x-10
>g(-5)
\end{eulerprompt}
\begin{euleroutput}
      -105.00 
\end{euleroutput}
\begin{eulerprompt}
>function f(x,a=1) :=5*(a*x)^4+a*x^3-x
>f(-(2^(-1/2)))
\end{eulerprompt}
\begin{euleroutput}
         1.60 
\end{euleroutput}
\begin{eulerprompt}
>function f(x,a=1) :=10*a*x^2-40*a*x
>f(1/2)
\end{eulerprompt}
\begin{euleroutput}
       -17.50 
\end{euleroutput}
\begin{eulerprompt}
>$&solve((x^5-5)/(x+1))
>$&solve((3*x^4-x^3+2*x^2-6*x)/(x-2))
\end{eulercomment}
\end{eulernotebook}

\chapter{EMT plot 2D}
\begin{eulernotebook}
\eulerheading{Menggambar Grafik 2D dengan EMT}
\begin{eulercomment}
Notebook ini menjelaskan tentang cara menggambar berbagaikurva dan
grafik 2D dengan software EMT. EMT menyediakan fungsi plot2d() untuk
menggambar berbagai kurva dan grafik dua dimensi (2D).\\
\end{eulercomment}
\eulersubheading{Plot Dasar}
\begin{eulercomment}
Ada fungsi plot yang sangat mendasar. Ada koordinat layar, yang selalu
berkisar dari 0 hingga 1024 di setiap sumbu, tidak peduli apakah
layarnya persegi atau tidak. Terdapat koordinat plot, yang dapat
diatur dengan setplot(). Pemetaan antara koordinat tergantung pada
jendela plot saat ini. Sebagai contoh, default shrinkwindow()
menyisakan ruang untuk label sumbu dan judul plot.\\
Dalam contoh, kita hanya menggambar beberapa garis acak dalam berbagai
warna. Untuk detail mengenai fungsi-fungsi ini, pelajari fungsi inti
EMT.
\end{eulercomment}
\begin{eulerprompt}
>clg; // membersihkan layar
>window(0,0,1024,1024); // gunakan semua jendela
>setplot(0,1,0,1); // mengatur koordinat plot
>hold on; // mulai mode timpa
>n=100; X=random(n,2); Y=random(n,2);  // mendapatkan poin acak
>colors=rgb(random(n),random(n),random(n)); // mendapatkan warna acak
>loop 1 to n; color(colors[#]); plot(X[#],Y[#]); end; // plot
>hold off; // akhiri mode timpa
>insimg; // sisipkan ke buku catatan
\end{eulerprompt}
\eulerimg{27}{images/Pekan 5-6_Fanny Erina Dewi_22305141005_EMT00-Plot2D_Aplikom-001.png}
\begin{eulerprompt}
>reset;
\end{eulerprompt}
\begin{eulercomment}
Anda harus menahan grafik, karena perintah plot() akan menghapus
jendela plot. Untuk menghapus semua yang telah kita lakukan, kita
menggunakan reset().\\
Untuk menampilkan gambar hasil plot di layar notebook, perintah
plot2d() dapat diakhiri dengan titik dua (:). Cara lain adalah
perintah plot2d() diakhiri dengan titik koma (;), kemudian gunakan
perintah insimg() untuk menampilkan gambar hasil plot.\\
Sebagai contoh lain, kita menggambar plot sebagai inset dalam plot
lain. Hal ini dilakukan dengan mendefinisikan jendela plot yang lebih
kecil. Perhatikan bahwa jendela ini tidak menyediakan ruang untuk
label sumbu di luar jendela plot. Kita harus menambahkan beberapa
margin untuk hal ini sesuai kebutuhan. Perhatikan bahwa kita menyimpan
dan mengembalikan jendela penuh, dan menahan plot saat ini ketika kita
membuat inset.
\end{eulercomment}
\begin{eulerprompt}
>plot2d("x^3-x");
>xw=200; yw=100; ww=300; hw=300;
>ow=window();
>window(xw,yw,xw+ww,yw+hw);
>hold on;
>barclear(xw-50,yw-10,ww+60,ww+60);
>plot2d("x^4-x",grid=6):
\end{eulerprompt}
\eulerimg{27}{images/Pekan 5-6_Fanny Erina Dewi_22305141005_EMT00-Plot2D_Aplikom-002.png}
\begin{eulerprompt}
>hold off;
>window(ow);
>reset;
\end{eulerprompt}
\eulersubheading{Contoh plot dasar **}
\begin{eulerprompt}
>plot2d("x^2-x");
>xw=100; yw=200; ww=100; hw=50
\end{eulerprompt}
\begin{euleroutput}
  50
\end{euleroutput}
\begin{eulerprompt}
>ow=window();
>window(xw,yw,xw+ww,yw+hw);
>hold on;
>barclear(xw-10,yw-50,ww+0,ww+50);
>plot2d("x^8+x",grid=5):
\end{eulerprompt}
\eulerimg{27}{images/Pekan 5-6_Fanny Erina Dewi_22305141005_EMT00-Plot2D_Aplikom-003.png}
\begin{eulerprompt}
>hold off;
>window(ow);
>reset;
\end{eulerprompt}
\begin{eulercomment}
Plot dengan beberapa angka dicapai dengan cara yang sama. Ada fungsi
utility figure() untuk ini.

\end{eulercomment}
\eulersubheading{Aspek Plot}
\begin{eulercomment}
Plot default menggunakan jendela plot persegi. Anda dapat mengubahnya
dengan fungsi aspect(). Jangan lupa untuk mengatur ulang aspeknya
nanti. Anda juga dapat mengubah default ini di menu dengan "Set
Aspect" ke rasio aspek tertentu atau ke ukuran jendela grafis saat
ini.\\
Tetapi Anda juga dapat mengubahnya untuk satu plot. Untuk melakukan
ini, ukuran area plot saat ini diubah, dan jendela diatur sedemikian
rupa sehingga label memiliki ruang yang cukup.
\end{eulercomment}
\begin{eulerprompt}
>aspect(2); // rasio panjang dan lebar 2:1
>plot2d(["sin(x)","cos(x)"],0,2pi):
\end{eulerprompt}
\eulerimg{13}{images/Pekan 5-6_Fanny Erina Dewi_22305141005_EMT00-Plot2D_Aplikom-004.png}
\begin{eulerprompt}
>aspect();
>reset;
\end{eulerprompt}
\begin{eulercomment}
Fungsi reset() memulihkan default plot, termasuk rasio aspek.

\end{eulercomment}
\eulersubheading{Contoh Aspek Plot **}
\begin{eulerprompt}
>aspect(4); // rasio panjang dan lebar 3:4
>plot2d(["sin(x)","cos(x)"],0,4pi):
\end{eulerprompt}
\eulerimg{6}{images/Pekan 5-6_Fanny Erina Dewi_22305141005_EMT00-Plot2D_Aplikom-005.png}
\begin{eulerprompt}
>aspect();
>reset;
\end{eulerprompt}
\eulerheading{Plot 2D di Euler}
\begin{eulercomment}
EMT Math Toolbox memiliki plot dalam bentuk 2D, baik untuk data maupun
fungsi. EMT menggunakan fungsi plot2d. Fungsi ini dapat memplot fungsi
dan data.\\
Hal ini memungkinkan untuk memplot di Maxima menggunakan Gnuplot atau
di Python menggunakan Math Plot Lib. Euler dapat memplot plot 2D dari\\
-   ekspresi\\
-   fungsi, variabel, atau kurva yang diparameterkan,\\
-   vektor nilai x-y,\\
-   awan titik-titik di dalam pesawat,\\
-   kurva implisit dengan level atau wilayah level.\\
-   Fungsi yang kompleks\\
Gaya plot mencakup berbagai gaya untuk garis dan titik, plot batang
dan plot berbayang.


\begin{eulercomment}
\eulerheading{Plot Ekspresi atau Variabel}
\begin{eulercomment}
Ekspresi tunggal dalam "x" (misalnya "4*x\textasciicircum{}2") atau nama fungsi
(misalnya "f") menghasilkan grafik fungsi.

Berikut ini adalah contoh paling dasar, yang menggunakan rentang
default dan menetapkan rentang y yang tepat agar sesuai dengan plot
fungsi.

Catatan: Jika Anda mengakhiri baris perintah dengan tanda titik dua
":", plot akan disisipkan ke dalam jendela teks. Jika tidak, tekan TAB
untuk melihat plot jika jendela plot tertutup.
\end{eulercomment}
\begin{eulerprompt}
>plot2d("x^2"):
\end{eulerprompt}
\eulerimg{27}{images/Pekan 5-6_Fanny Erina Dewi_22305141005_EMT00-Plot2D_Aplikom-006.png}
\begin{eulerprompt}
>aspect(1.5); plot2d("x^3-x"):
\end{eulerprompt}
\eulerimg{17}{images/Pekan 5-6_Fanny Erina Dewi_22305141005_EMT00-Plot2D_Aplikom-007.png}
\begin{eulerprompt}
>a:=5.6; plot2d("exp(-a*x^2)/a"); insimg(30); // menampilkan gambar hasil plot setinggi 25 baris
\end{eulerprompt}
\eulerimg{17}{images/Pekan 5-6_Fanny Erina Dewi_22305141005_EMT00-Plot2D_Aplikom-008.png}
\begin{eulercomment}
Dari beberapa contoh sebelumnya Anda dapat melihat bahwa aslinya
gambar plot menggunakan sumbu X dengan rentang nilai dari -2 sampai
dengan 2. Untuk mengubah rentang nilai X dan Y, Anda dapat menambahkan
nilai-nilai batas X (dan Y) di belakang ekspresi yang digambar.

Rentang plot ditetapkan dengan parameter yang ditetapkan berikut ini

-   a,b: rentang-x (default -2,2)\\
-   c, d: rentang y (default: skala dengan nilai)\\
-   r: sebagai alternatif adalah radius di sekitar pusat plot\\
-   cx,cy: koordinat pusat plot (standar 0,0)
\end{eulercomment}
\begin{eulerprompt}
>plot2d("x^3-x",-1,2):
\end{eulerprompt}
\eulerimg{17}{images/Pekan 5-6_Fanny Erina Dewi_22305141005_EMT00-Plot2D_Aplikom-009.png}
\begin{eulerprompt}
>plot2d("sin(x)",-2*pi,2*pi): // plot sin(x) pada interval [-2pi, 2pi]
\end{eulerprompt}
\eulerimg{17}{images/Pekan 5-6_Fanny Erina Dewi_22305141005_EMT00-Plot2D_Aplikom-010.png}
\begin{eulerprompt}
>plot2d("cos(x)","sin(3*x)",xmin=0,xmax=2pi):
\end{eulerprompt}
\eulerimg{17}{images/Pekan 5-6_Fanny Erina Dewi_22305141005_EMT00-Plot2D_Aplikom-011.png}
\begin{eulercomment}
Alternatif untuk tanda titik dua adalah perintah insimg(lines), yang
menyisipkan plot yang menempati sejumlah baris teks tertentu.

Dalam opsi, plot dapat diatur untuk muncul\\
- dalam jendela terpisah yang dapat diubah ukurannya,\\
- di jendela buku catatan.

Lebih banyak gaya dapat dicapai dengan perintah plot tertentu.\\
Dalam hal apa pun, tekan tombol tabulator untuk melihat plot, jika
disembunyikan.

Untuk membagi jendela menjadi beberapa plot, gunakan perintah
figure(). Pada contoh, kita memplot x\textasciicircum{}1 hingga x\textasciicircum{}4 menjadi 4 bagian
jendela. figure(0) mengatur ulang jendela default.
\end{eulercomment}
\begin{eulerprompt}
>reset;
>figure(2,2); ...
>for n=1 to 4; figure(n); plot2d("x^"+n); end; ...
>figure(0):
\end{eulerprompt}
\eulerimg{27}{images/Pekan 5-6_Fanny Erina Dewi_22305141005_EMT00-Plot2D_Aplikom-012.png}
\begin{eulercomment}
Pada plot2d(), terdapat beberapa gaya alternatif yang tersedia dengan
grid=x. Sebagai gambaran umum, kami menampilkan berbagai gaya grid
pada satu gambar (lihat di bawah ini untuk perintah figure()). Gaya
grid=0 tidak disertakan. Gaya ini tidak menampilkan grid dan frame.
\end{eulercomment}
\begin{eulerprompt}
>figure(3,3); ...
>for k=1:9; figure(k); plot2d("x^3-x",-2,1,grid=k); end; ...
>figure(0):
\end{eulerprompt}
\eulerimg{27}{images/Pekan 5-6_Fanny Erina Dewi_22305141005_EMT00-Plot2D_Aplikom-013.png}
\begin{eulercomment}
Jika argumen untuk plot2d() adalah sebuah ekspresi yang diikuti oleh
empat angka, angka-angka ini adalah rentang x dan y untuk plot.

Atau, a, b, c, d dapat ditetapkan sebagai parameter yang ditetapkan
sebagai a=... dst.

Pada contoh berikut, kita mengubah gaya kisi, menambahkan label, dan
menggunakan label vertikal untuk sumbu y.
\end{eulercomment}
\begin{eulerprompt}
>aspect(1.5); plot2d("sin(x)",0,2pi,-1.2,1.2,grid=3,xl="x",yl="sin(x)"):
\end{eulerprompt}
\eulerimg{17}{images/Pekan 5-6_Fanny Erina Dewi_22305141005_EMT00-Plot2D_Aplikom-014.png}
\begin{eulerprompt}
>plot2d("sin(x)+cos(2*x)",0,4pi):
\end{eulerprompt}
\eulerimg{17}{images/Pekan 5-6_Fanny Erina Dewi_22305141005_EMT00-Plot2D_Aplikom-015.png}
\begin{eulercomment}
Gambar yang dihasilkan dengan menyisipkan plot ke dalam jendela teks
disimpan di direktori yang sama dengan buku catatan, secara default
dalam subdirektori bernama "images". Gambar-gambar tersebut juga
digunakan oleh ekspor HTML.

Anda cukup menandai gambar apa pun dan menyalinnya ke clipboard dengan
Ctrl-C. Tentu saja, Anda juga dapat mengekspor grafik saat ini dengan
fungsi-fungsi dalam menu File.

Fungsi atau ekspresi dalam plot2d dievaluasi secara adaptif. Untuk
kecepatan yang lebih tinggi, matikan plot adaptif dengan \textless{}adaptive dan
tentukan jumlah subinterval dengan n=... Hal ini hanya diperlukan pada
kasus- kasus yang jarang terjadi.
\end{eulercomment}
\begin{eulerprompt}
>plot2d("sign(x)*exp(-x^2)",-1,1,<adaptive,n=10000):
\end{eulerprompt}
\eulerimg{17}{images/Pekan 5-6_Fanny Erina Dewi_22305141005_EMT00-Plot2D_Aplikom-016.png}
\begin{eulerprompt}
>plot2d("x^x",r=1.2,cx=1,cy=1):
\end{eulerprompt}
\eulerimg{17}{images/Pekan 5-6_Fanny Erina Dewi_22305141005_EMT00-Plot2D_Aplikom-017.png}
\begin{eulercomment}
Perhatikan bahwa x\textasciicircum{}x tidak didefinisikan untuk x\textless{}=0. Fungsi plot2d
menangkap kesalahan ini, dan mulai memplot segera setelah fungsi
didefinisikan. Hal ini berlaku untuk semua fungsi yang mengembalikan
NAN di luar jangkauan definisinya.
\end{eulercomment}
\begin{eulerprompt}
>plot2d("log(x)",-0.1,2):
\end{eulerprompt}
\eulerimg{17}{images/Pekan 5-6_Fanny Erina Dewi_22305141005_EMT00-Plot2D_Aplikom-018.png}
\begin{eulercomment}
Parameter square=true (atau \textgreater{}square) memilih rentang y secara otomatis
sehingga hasilnya adalah jendela plot persegi. Perhatikan bahwa secara
default, Euler menggunakan ruang persegi di dalam jendela plot.
\end{eulercomment}
\begin{eulerprompt}
>plot2d("x^3-x",>square):
\end{eulerprompt}
\eulerimg{17}{images/Pekan 5-6_Fanny Erina Dewi_22305141005_EMT00-Plot2D_Aplikom-019.png}
\begin{eulerprompt}
>plot2d(''integrate("sin(x)*exp(-x^2)",0,x)'',0,2): // plot integral
\end{eulerprompt}
\eulerimg{17}{images/Pekan 5-6_Fanny Erina Dewi_22305141005_EMT00-Plot2D_Aplikom-020.png}
\begin{eulercomment}
Jika Anda membutuhkan lebih banyak ruang untuk label-y, panggil
shrinkwindow() dengan parameter lebih kecil, atau tetapkan nilai
positif untuk "lebih kecil" pada plot2d().
\end{eulercomment}
\begin{eulerprompt}
>plot2d("gamma(x)",1,10,yl="y-values",smaller=6,<vertical):
\end{eulerprompt}
\eulerimg{17}{images/Pekan 5-6_Fanny Erina Dewi_22305141005_EMT00-Plot2D_Aplikom-021.png}
\begin{eulercomment}
Ekspresi simbolik juga dapat digunakan, karena disimpan sebagai
ekspresi string sederhana.
\end{eulercomment}
\begin{eulerprompt}
>x=linspace(0,2pi,1000); plot2d(sin(5x),cos(7x)):
\end{eulerprompt}
\eulerimg{17}{images/Pekan 5-6_Fanny Erina Dewi_22305141005_EMT00-Plot2D_Aplikom-022.png}
\begin{eulerprompt}
>a:=5.6; expr &= exp(-a*x^2)/a; // mendefinisikan ekspresi
>plot2d(expr,-2,2): // plot dari -2 to 2
\end{eulerprompt}
\eulerimg{17}{images/Pekan 5-6_Fanny Erina Dewi_22305141005_EMT00-Plot2D_Aplikom-023.png}
\begin{eulerprompt}
>plot2d(expr,r=1,thickness=2): // plot dalam kotak di sekitar (0,0)
\end{eulerprompt}
\eulerimg{17}{images/Pekan 5-6_Fanny Erina Dewi_22305141005_EMT00-Plot2D_Aplikom-024.png}
\begin{eulerprompt}
>plot2d(&diff(expr,x),>add,style="--",color=red): // tambahkan plot lain
\end{eulerprompt}
\eulerimg{17}{images/Pekan 5-6_Fanny Erina Dewi_22305141005_EMT00-Plot2D_Aplikom-025.png}
\begin{eulerprompt}
>plot2d(&diff(expr,x,2),a=-2,b=2,c=-2,d=1): // plot dalam persegi panjang
\end{eulerprompt}
\eulerimg{17}{images/Pekan 5-6_Fanny Erina Dewi_22305141005_EMT00-Plot2D_Aplikom-026.png}
\begin{eulerprompt}
>plot2d(&diff(expr,x),a=-2,b=2,>square): // pertahankan plot tetap persegi
\end{eulerprompt}
\eulerimg{17}{images/Pekan 5-6_Fanny Erina Dewi_22305141005_EMT00-Plot2D_Aplikom-027.png}
\begin{eulerprompt}
>plot2d("x^2",0,1,steps=1,color=red,n=10):
\end{eulerprompt}
\eulerimg{17}{images/Pekan 5-6_Fanny Erina Dewi_22305141005_EMT00-Plot2D_Aplikom-028.png}
\begin{eulerprompt}
>plot2d("x^2",>add,steps=2,color=blue,n=10):
\end{eulerprompt}
\eulerimg{17}{images/Pekan 5-6_Fanny Erina Dewi_22305141005_EMT00-Plot2D_Aplikom-029.png}
\eulersubheading{Contoh Plot Ekspresi atau Variabel **}
\begin{eulerprompt}
>plot2d("x^2-3"):
\end{eulerprompt}
\eulerimg{17}{images/Pekan 5-6_Fanny Erina Dewi_22305141005_EMT00-Plot2D_Aplikom-030.png}
\begin{eulerprompt}
>aspect(1.5); plot2d("x^3"):
\end{eulerprompt}
\eulerimg{17}{images/Pekan 5-6_Fanny Erina Dewi_22305141005_EMT00-Plot2D_Aplikom-031.png}
\begin{eulerprompt}
>plot2d("x^3",-5,4):
\end{eulerprompt}
\eulerimg{17}{images/Pekan 5-6_Fanny Erina Dewi_22305141005_EMT00-Plot2D_Aplikom-032.png}
\begin{eulercomment}
\begin{eulercomment}
\eulerheading{Fungsi dalam satu Parameter}
\begin{eulercomment}
Fungsi plot yang paling penting untuk plot planar adalah plot2d().
Fungsi ini diimplementasikan dalam bahasa Euler dalam file "plot.e",
yang dimuat pada awal program.

Berikut adalah beberapa contoh penggunaan fungsi. Seperti biasa dalam
EMT, fungsi yang bekerja untuk fungsi atau eksekusi lain, Anda dapat
mengoper parameter tambahan (selain x) yang bukan variabel global ke
fungsi dengan parameter titik koma atau dengan koleksi panggilan.
\end{eulercomment}
\begin{eulerprompt}
>function f(x,a) := x^2/a+a*x^2-x; // mendefinisikan sebuah fungsi
>a=0.3; plot2d("f",0,1;a): // plot dengan a=0.3
\end{eulerprompt}
\eulerimg{17}{images/Pekan 5-6_Fanny Erina Dewi_22305141005_EMT00-Plot2D_Aplikom-033.png}
\begin{eulerprompt}
>plot2d("f",0,1;0.4): // plot dengan a=0.4
\end{eulerprompt}
\eulerimg{17}{images/Pekan 5-6_Fanny Erina Dewi_22305141005_EMT00-Plot2D_Aplikom-034.png}
\begin{eulerprompt}
>plot2d(\{\{"f",0.2\}\},0,1): // plot dengan a=0.2
\end{eulerprompt}
\eulerimg{17}{images/Pekan 5-6_Fanny Erina Dewi_22305141005_EMT00-Plot2D_Aplikom-035.png}
\begin{eulerprompt}
>plot2d(\{\{"f(x,b)",b=0.1\}\},0,1): // plot dengan 0.1
\end{eulerprompt}
\eulerimg{17}{images/Pekan 5-6_Fanny Erina Dewi_22305141005_EMT00-Plot2D_Aplikom-036.png}
\begin{eulerprompt}
>function f(x) := x^3-x; ...
>plot2d("f",r=1):
\end{eulerprompt}
\eulerimg{17}{images/Pekan 5-6_Fanny Erina Dewi_22305141005_EMT00-Plot2D_Aplikom-037.png}
\begin{eulercomment}
Berikut ini adalah ringkasan dari fungsi yang diterima

- ekspresi atau ekspresi simbolik dalam x\\
- fungsi atau fungsi simbolis dengan nama sebagai "f"\\
- fungsi simbolik hanya dengan nama f\\
Fungsi plot2d() juga menerima fungsi simbolik. Untuk fungsi simbolik,
nama saja sudah cukup.
\end{eulercomment}
\begin{eulerprompt}
>function f(x) &= diff(x^x,x)
\end{eulerprompt}
\begin{euleroutput}
  
                              x
                             x  (log(x) + 1)
  
\end{euleroutput}
\begin{eulerprompt}
>plot2d(f,0,2):
\end{eulerprompt}
\eulerimg{17}{images/Pekan 5-6_Fanny Erina Dewi_22305141005_EMT00-Plot2D_Aplikom-038.png}
\begin{eulercomment}
Tentu saja, untuk ekspresi atau ungkapan simbolik, nama variabel sudah
cukup untuk memplotnya.
\end{eulercomment}
\begin{eulerprompt}
>expr &= sin(x)*exp(-x)
\end{eulerprompt}
\begin{euleroutput}
  
                                - x
                               E    sin(x)
  
\end{euleroutput}
\begin{eulerprompt}
>plot2d(expr,0,3pi):
\end{eulerprompt}
\eulerimg{17}{images/Pekan 5-6_Fanny Erina Dewi_22305141005_EMT00-Plot2D_Aplikom-039.png}
\begin{eulerprompt}
>function f(x) &= x^x;
>plot2d(f,r=1,cx=1,cy=1,color=blue,thickness=2);
>plot2d(&diff(f(x),x),>add,color=red,style="-.-"):
\end{eulerprompt}
\eulerimg{17}{images/Pekan 5-6_Fanny Erina Dewi_22305141005_EMT00-Plot2D_Aplikom-040.png}
\begin{eulercomment}
Untuk gaya garis, terdapat berbagai opsi.\\
- style = "...". Pilih dari "-", "-", "-.", ".", ".-.", "-.-".\\
- warna: Lihat di bawah untuk warna.\\
- ketebalan: Standarnya adalah 1.\\
Warna dapat dipilih sebagai salah satu warna default, atau sebagai
warna RGB.\\
- 0..15: indeks warna default.\\
- konstanta warna: putih, hitam, merah, hijau, biru, cyan, zaitun,
abu-abu muda, abu-abu, abu-abu tua, oranye, hijau muda, biru
kehijauan, biru muda, oranye muda, kuning\\
- rgb (merah, hijau, biru): parameter dalam bentuk real dalam [0,1].
\end{eulercomment}
\begin{eulerprompt}
>plot2d("exp(-x^2)",r=2,color=red,thickness=3,style="--"):
\end{eulerprompt}
\eulerimg{17}{images/Pekan 5-6_Fanny Erina Dewi_22305141005_EMT00-Plot2D_Aplikom-041.png}
\begin{eulercomment}
Berikut ini adalah pemandangan warna EMT yang sudah ditetapkan
sebelumnya.
\end{eulercomment}
\begin{eulerprompt}
>aspect(2); columnsplot(ones(1,16),lab=0:15,grid=0,color=0:15):
\end{eulerprompt}
\eulerimg{13}{images/Pekan 5-6_Fanny Erina Dewi_22305141005_EMT00-Plot2D_Aplikom-042.png}
\begin{eulercomment}
Tetapi Anda bisa menggunakan warna apa pun.
\end{eulercomment}
\begin{eulerprompt}
>columnsplot(ones(1,16),grid=0,color=rgb(0,0,linspace(0,1,15))):
\end{eulerprompt}
\eulerimg{13}{images/Pekan 5-6_Fanny Erina Dewi_22305141005_EMT00-Plot2D_Aplikom-043.png}
\eulerheading{Menggambar Beberapa Kurva pada bidang koordinat yang sama}
\begin{eulercomment}
Memplot lebih dari satu fungsi (beberapa fungsi) ke dalam satu jendela
dapat dilakukan dengan berbagai cara. Salah satu caranya adalah dengan
menggunakan \textgreater{}add untuk beberapa pemanggilan ke plot2d secara
bersamaan, kecuali pemanggilan pertama. Kita telah menggunakan fitur
ini pada contoh di atas.
\end{eulercomment}
\begin{eulerprompt}
>aspect(); plot2d("cos(x)",r=2,grid=6); plot2d("x",style=".",>add):
\end{eulerprompt}
\eulerimg{27}{images/Pekan 5-6_Fanny Erina Dewi_22305141005_EMT00-Plot2D_Aplikom-044.png}
\begin{eulerprompt}
>aspect(1.5); plot2d("sin(x)",0,2pi); plot2d("cos(x)",color=blue,style="--",>add):
\end{eulerprompt}
\eulerimg{17}{images/Pekan 5-6_Fanny Erina Dewi_22305141005_EMT00-Plot2D_Aplikom-045.png}
\begin{eulercomment}
Salah satu kegunaan \textgreater{}add adalah untuk menambahkan titik pada kurva.
\end{eulercomment}
\begin{eulerprompt}
>plot2d("sin(x)",0,pi); plot2d(2,sin(2),>points,>add):
\end{eulerprompt}
\eulerimg{17}{images/Pekan 5-6_Fanny Erina Dewi_22305141005_EMT00-Plot2D_Aplikom-046.png}
\begin{eulercomment}
Kami menambahkan titik perpotongan dengan label (pada posisi "cl"
untuk kiri tengah), dan menyisipkan hasilnya ke dalam buku catatan.
Kami juga menambahkan judul ke plot.
\end{eulercomment}
\begin{eulerprompt}
>plot2d(["cos(x)","x"],r=1.1,cx=0.5,cy=0.5, ...
>  color=[black,blue],style=["-","."], ...
>  grid=1);
>x0=solve("cos(x)-x",1);  ...
>  plot2d(x0,x0,>points,>add,title="Intersection Demo");  ...
>  label("cos(x) = x",x0,x0,pos="cl",offset=20):
\end{eulerprompt}
\eulerimg{17}{images/Pekan 5-6_Fanny Erina Dewi_22305141005_EMT00-Plot2D_Aplikom-047.png}
\begin{eulercomment}
Dalam demo berikut ini, kami memplot fungsi sinc(x)=sin(x)/x dan
ekspansi Taylor ke-8 dan ke-16. Kami menghitung ekspansi ini
menggunakan Maxima melalui ekspresi simbolik.\\
Plot ini dilakukan dalam perintah multi-baris berikut ini dengan tiga
kali pemanggilan plot2d(). Pemanggilan kedua dan ketiga memiliki set
flag \textgreater{}add, yang membuat plot menggunakan rentang sebelumnya.

Kami menambahkan kotak label yang menjelaskan fungsinya.
\end{eulercomment}
\begin{eulerprompt}
>$taylor(sin(x)/x,x,0,4)
\end{eulerprompt}
\begin{eulerformula}
\[
\frac{x^4}{120}-\frac{x^2}{6}+1
\]
\end{eulerformula}
\begin{eulerprompt}
>plot2d("sinc(x)",0,4pi,color=green,thickness=2); ...
>  plot2d(&taylor(sin(x)/x,x,0,8),>add,color=blue,style="--"); ...
>  plot2d(&taylor(sin(x)/x,x,0,16),>add,color=red,style="-.-"); ...
>  labelbox(["sinc","T8","T16"],styles=["-","--","-.-"], ...
>    colors=[black,blue,red]):
\end{eulerprompt}
\eulerimg{17}{images/Pekan 5-6_Fanny Erina Dewi_22305141005_EMT00-Plot2D_Aplikom-049.png}
\begin{eulercomment}
Pada contoh berikut, kami menghasilkan Polinomial Bernstein.

\end{eulercomment}
\begin{eulerformula}
\[
B_i(x) = \binom{n}{i} x^i (1-x)^{n-i}
\]
\end{eulerformula}
\begin{eulerprompt}
>plot2d("(1-x)^10",0,1); // plot fungsi pertama
>for i=1 to 10; plot2d("bin(10,i)*x^i*(1-x)^(10-i)",>add); end;
>insimg;
\end{eulerprompt}
\eulerimg{17}{images/Pekan 5-6_Fanny Erina Dewi_22305141005_EMT00-Plot2D_Aplikom-050.png}
\begin{eulercomment}
Metode kedua adalah menggunakan sepasang matriks nilai x dan matriks
nilai y dengan ukuran yang sama. \\
Kita membuat sebuah matriks nilai dengan satu Bernstein-Polynomial di
setiap baris. Untuk ini, kita cukup menggunakan vektor kolom i.
Lihatlah pengantar tentang bahasa matriks untuk mempelajari lebih
lanjut.
\end{eulercomment}
\begin{eulerprompt}
>x=linspace(0,1,500);
>n=10; k=(0:n)'; // n adalah vektor baris, k adalah vektor kolom
>y=bin(n,k)*x^k*(1-x)^(n-k); // y adalah sebuah matriks maka
>plot2d(x,y):
\end{eulerprompt}
\eulerimg{17}{images/Pekan 5-6_Fanny Erina Dewi_22305141005_EMT00-Plot2D_Aplikom-051.png}
\begin{eulercomment}
Perhatikan bahwa parameter warna dapat berupa vektor. Kemudian setiap
warna digunakan untuk setiap baris matriks.
\end{eulercomment}
\begin{eulerprompt}
>x=linspace(0,1,200); y=x^(1:10)'; plot2d(x,y,color=1:10):
\end{eulerprompt}
\eulerimg{17}{images/Pekan 5-6_Fanny Erina Dewi_22305141005_EMT00-Plot2D_Aplikom-052.png}
\begin{eulercomment}
Metode lainnya adalah menggunakan vektor ekspresi (string). Anda
kemudian dapat menggunakan larik warna, larik gaya, dan larik
ketebalan dengan panjang yang sama.
\end{eulercomment}
\begin{eulerprompt}
>plot2d(["sin(x)","cos(x)"],0,2pi,color=4:5): 
\end{eulerprompt}
\eulerimg{17}{images/Pekan 5-6_Fanny Erina Dewi_22305141005_EMT00-Plot2D_Aplikom-053.png}
\begin{eulerprompt}
>plot2d(["sin(x)","cos(x)"],0,2pi): // plot vektor ekspresi
\end{eulerprompt}
\eulerimg{17}{images/Pekan 5-6_Fanny Erina Dewi_22305141005_EMT00-Plot2D_Aplikom-054.png}
\begin{eulercomment}
Kita bisa mendapatkan vektor seperti itu dari Maxima dengan
menggunakan makelist() dan mxm2str().
\end{eulercomment}
\begin{eulerprompt}
>v &= makelist(binomial(10,i)*x^i*(1-x)^(10-i),i,0,10) // membuat daftar
\end{eulerprompt}
\begin{euleroutput}
  
                 10            9              8  2             7  3
         [(1 - x)  , 10 (1 - x)  x, 45 (1 - x)  x , 120 (1 - x)  x , 
             6  4             5  5             4  6             3  7
  210 (1 - x)  x , 252 (1 - x)  x , 210 (1 - x)  x , 120 (1 - x)  x , 
            2  8              9   10
  45 (1 - x)  x , 10 (1 - x) x , x  ]
  
\end{euleroutput}
\begin{eulerprompt}
>mxm2str(v) // mendapatkan vektor string dari vektor simbolik
\end{eulerprompt}
\begin{euleroutput}
  (1-x)^10
  10*(1-x)^9*x
  45*(1-x)^8*x^2
  120*(1-x)^7*x^3
  210*(1-x)^6*x^4
  252*(1-x)^5*x^5
  210*(1-x)^4*x^6
  120*(1-x)^3*x^7
  45*(1-x)^2*x^8
  10*(1-x)*x^9
  x^10
\end{euleroutput}
\begin{eulerprompt}
>plot2d(mxm2str(v),0,1): // fungsi plot
\end{eulerprompt}
\eulerimg{17}{images/Pekan 5-6_Fanny Erina Dewi_22305141005_EMT00-Plot2D_Aplikom-055.png}
\begin{eulercomment}
Alternatif lain adalah dengan menggunakan bahasa matriks Euler.

Jika sebuah ekspresi menghasilkan matriks fungsi, dengan satu fungsi
di setiap baris, semua fungsi ini akan diplot ke dalam satu plot.

Untuk ini, gunakan vektor parameter dalam bentuk vektor kolom. Jika
sebuah larik warna ditambahkan, maka akan digunakan untuk setiap baris
plot.
\end{eulercomment}
\begin{eulerprompt}
>n=(1:10)'; plot2d("x^n",0,1,color=1:10):
\end{eulerprompt}
\eulerimg{17}{images/Pekan 5-6_Fanny Erina Dewi_22305141005_EMT00-Plot2D_Aplikom-056.png}
\begin{eulercomment}
Ekspresi dan fungsi satu baris dapat melihat variabel global.

Jika Anda tidak dapat menggunakan variabel global, Anda perlu
menggunakan fungsi dengan parameter tambahan, dan mengoper parameter
ini sebagai parameter titik koma.

Berhati-hatilah untuk meletakkan semua parameter yang ditetapkan di
akhir perintah plot2d. Pada contoh, kita memberikan a=5 ke fungsi f,
yang kita plot dari -10 ke 10.
\end{eulercomment}
\begin{eulerprompt}
>function f(x,a) := 1/a*exp(-x^2/a); ...
>plot2d("f",-10,10;5,thickness=2,title="a=5"):
\end{eulerprompt}
\eulerimg{17}{images/Pekan 5-6_Fanny Erina Dewi_22305141005_EMT00-Plot2D_Aplikom-057.png}
\begin{eulercomment}
Atau, gunakan koleksi dengan nama fungsi dan semua parameter tambahan.
Daftar khusus ini disebut koleksi panggilan, dan itu adalah cara yang
lebih disukai untuk meneruskan argumen ke fungsi yang dengan
sendirinya diteruskan sebagai argumen ke fungsi lain.

Pada contoh berikut ini, kita menggunakan loop untuk memplot beberapa
fungsi (lihat tutorial tentang pemrograman untuk loop).
\end{eulercomment}
\begin{eulerprompt}
>plot2d(\{\{"f",1\}\},-10,10); ...
>for a=2:10; plot2d(\{\{"f",a\}\},>add); end:
\end{eulerprompt}
\eulerimg{17}{images/Pekan 5-6_Fanny Erina Dewi_22305141005_EMT00-Plot2D_Aplikom-058.png}
\begin{eulercomment}
Kita dapat mencapai hasil yang sama dengan cara berikut menggunakan
bahasa matriks EMT. Setiap baris dari matriks f(x,a) adalah satu
fungsi. Selain itu, kita dapat mengatur warna untuk setiap baris
matriks. Klik dua kali pada fungsi getspectral() untuk penjelasannya.
\end{eulercomment}
\begin{eulerprompt}
>x=-10:0.01:10; a=(1:10)'; plot2d(x,f(x,a),color=getspectral(a/10)):
\end{eulerprompt}
\eulerimg{17}{images/Pekan 5-6_Fanny Erina Dewi_22305141005_EMT00-Plot2D_Aplikom-059.png}
\eulersubheading{Label Teks}
\begin{eulercomment}
Dekorasi sederhana dapat berupa\\
- judul dengan title = "..."\\
- Label x dan y dengan xl="...", yl="..."\\
- label teks lain dengan label("...",x,y)\\
Perintah label akan memplot ke dalam plot saat ini pada koordinat plot
(x,y). Perintah ini dapat menerima argumen posisi.

\end{eulercomment}
\begin{eulerprompt}
>plot2d("x^3-x",-1,2,title="y=x^3-x",yl="y",xl="x"):
\end{eulerprompt}
\eulerimg{17}{images/Pekan 5-6_Fanny Erina Dewi_22305141005_EMT00-Plot2D_Aplikom-060.png}
\begin{eulerprompt}
>expr := "log(x)/x"; ...
>  plot2d(expr,0.5,5,title="y="+expr,xl="x",yl="y"); ...
>  label("(1,0)",1,0); label("Max",E,expr(E),pos="lc"):
\end{eulerprompt}
\eulerimg{17}{images/Pekan 5-6_Fanny Erina Dewi_22305141005_EMT00-Plot2D_Aplikom-061.png}
\begin{eulercomment}
Ada juga fungsi labelbox(), yang dapat menampilkan fungsi dan teks.
Fungsi ini membutuhkan vektor string dan warna, satu item untuk setiap
fungsi.
\end{eulercomment}
\begin{eulerprompt}
>function f(x) &= x^2*exp(-x^2);  ...
>plot2d(&f(x),a=-3,b=3,c=-1,d=1);  ...
>plot2d(&diff(f(x),x),>add,color=blue,style="--"); ...
>labelbox(["function","derivative"],styles=["-","--"], ...
>   colors=[black,blue],w=0.4):
\end{eulerprompt}
\eulerimg{17}{images/Pekan 5-6_Fanny Erina Dewi_22305141005_EMT00-Plot2D_Aplikom-062.png}
\begin{eulercomment}
Kotak tersebut berlabuh di kanan atas secara default, tetapi \textgreater{}kiri
menambatkannya di kiri atas. Anda dapat memindahkannya ke tempat mana
pun yang Anda suka. Posisi jangkar adalah sudut kanan atas kotak, dan
angkanya adalah pecahan dari ukuran jendela grafik. Lebarnya adalah
otomatis.

Untuk plot titik, kotak label juga dapat digunakan. Tambahkan sebuah
parameter \textgreater{}titik, atau sebuah vektor bendera, satu untuk setiap label.

Pada contoh berikut ini, hanya ada satu fungsi. Jadi kita dapat
menggunakan string dan bukan vektor string. Kami mengatur warna teks
menjadi hitam untuk contoh ini.
\end{eulercomment}
\begin{eulerprompt}
>n=10; plot2d(0:n,bin(n,0:n),>addpoints); ...
>labelbox("Binomials",styles="[]",>points,x=0.1,y=0.1, ...
>tcolor=black,>left):
\end{eulerprompt}
\eulerimg{17}{images/Pekan 5-6_Fanny Erina Dewi_22305141005_EMT00-Plot2D_Aplikom-063.png}
\begin{eulercomment}
Gaya plot ini juga tersedia di statplot(). Seperti pada plot2d() warna
dapat diatur untuk setiap baris plot. Terdapat lebih banyak plot
khusus untuk keperluan statistik (lihat tutorial tentang statistik).
\end{eulercomment}
\begin{eulerprompt}
>statplot(1:10,random(2,10),color=[red,blue]):
\end{eulerprompt}
\eulerimg{17}{images/Pekan 5-6_Fanny Erina Dewi_22305141005_EMT00-Plot2D_Aplikom-064.png}
\begin{eulercomment}
Fitur yang serupa adalah fungsi textbox().

Lebarnya secara default adalah lebar maksimal baris teks. Tetapi, ini
juga dapat diatur oleh pengguna.
\end{eulercomment}
\begin{eulerprompt}
>function f(x) &= exp(-x)*sin(2*pi*x); ...
>plot2d("f(x)",0,2pi); ...
>textbox(latex("\(\backslash\)text\{Example of a damped oscillation\}\(\backslash\) f(x)=e^\{-x\}sin(2\(\backslash\)pi x)"),w=0.85):
\end{eulerprompt}
\eulerimg{17}{images/Pekan 5-6_Fanny Erina Dewi_22305141005_EMT00-Plot2D_Aplikom-065.png}
\begin{eulercomment}
Label teks, judul, kotak label, dan teks lainnya dapat berisi string
Unicode (lihat sintaks EMT untuk mengetahui lebih lanjut tentang
string Unicode).
\end{eulercomment}
\begin{eulerprompt}
>plot2d("x^3-x",title=u"x &rarr; x&sup3; - x"):
\end{eulerprompt}
\eulerimg{17}{images/Pekan 5-6_Fanny Erina Dewi_22305141005_EMT00-Plot2D_Aplikom-066.png}
\begin{eulercomment}
Label pada sumbu x dan y bisa vertikal, begitu juga dengan sumbu.
\end{eulercomment}
\begin{eulerprompt}
>plot2d("sinc(x)",0,2pi,xl="x",yl=u"x &rarr; sinc(x)",>vertical):
\end{eulerprompt}
\eulerimg{17}{images/Pekan 5-6_Fanny Erina Dewi_22305141005_EMT00-Plot2D_Aplikom-067.png}
\eulersubheading{LaTeX}
\begin{eulercomment}
Anda juga dapat memplot formula LaTeX jika Anda telah menginstal
sistem LaTeX. Saya merekomendasikan MiKTeX. Jalur ke binari "lateks"
dan "dvipng" harus berada di jalur sistem, atau Anda harus mengatur
LaTeX di menu opsi.

Perlu diperhatikan bahwa penguraian LaTeX berjalan lambat. Jika Anda
ingin menggunakan LaTeX dalam plot animasi, Anda harus memanggil
latex() sebelum perulangan sekali dan menggunakan hasilnya (gambar
dalam matriks RGB).

Pada plot berikut ini, kita menggunakan LaTeX untuk label x dan y,
sebuah label, sebuah kotak label, dan judul plot.
\end{eulercomment}
\begin{eulerprompt}
>plot2d("exp(-x)*sin(x)/x",a=0,b=2pi,c=0,d=1,grid=6,color=blue, ...
>  title=latex("\(\backslash\)text\{Function $\(\backslash\)Phi$\}"), ...
>  xl=latex("\(\backslash\)phi"),yl=latex("\(\backslash\)Phi(\(\backslash\)phi)")); ...
>textbox( ...
>  latex("\(\backslash\)Phi(\(\backslash\)phi) = e^\{-\(\backslash\)phi\} \(\backslash\)frac\{\(\backslash\)sin(\(\backslash\)phi)\}\{\(\backslash\)phi\}"),x=0.8,y=0.5); ...
>label(latex("\(\backslash\)Phi",color=blue),1,0.4):
\end{eulerprompt}
\eulerimg{17}{images/Pekan 5-6_Fanny Erina Dewi_22305141005_EMT00-Plot2D_Aplikom-068.png}
\begin{eulercomment}
Seringkali, kita menginginkan spasi dan label teks yang tidak sesuai
pada sumbu x. Kita dapat menggunakan xaxis() dan yaxis() seperti yang
akan kita tunjukkan nanti.

Cara termudah adalah dengan membuat plot kosong dengan sebuah frame
menggunakan grid=4, dan kemudian menambahkan grid dengan ygrid() dan
xgrid(). Pada contoh berikut, kita menggunakan tiga buah string LaTeX
untuk label pada sumbu x dengan xtick().\\
\end{eulercomment}
\begin{eulerttcomment}
 
\end{eulerttcomment}
\begin{eulerprompt}
>plot2d("sinc(x)",0,2pi,grid=4,<ticks); ...
>ygrid(-2:0.5:2,grid=6); ...
>xgrid([0:2]*pi,<ticks,grid=6);  ...
>xtick([0,pi,2pi],["0","\(\backslash\)pi","2\(\backslash\)pi"],>latex):
\end{eulerprompt}
\eulerimg{17}{images/Pekan 5-6_Fanny Erina Dewi_22305141005_EMT00-Plot2D_Aplikom-069.png}
\begin{eulercomment}
Tentu saja, fungsi juga dapat digunakan.
\end{eulercomment}
\begin{eulerprompt}
>function map f(x) ...
\end{eulerprompt}
\begin{eulerudf}
  if x>0 then return x^4
  else return x^2
  endif
  endfunction
\end{eulerudf}
\begin{eulercomment}
Parameter "map" membantu menggunakan fungsi untuk vektor. Untuk plot,
hal ini tidak diperlukan. Tetapi untuk menunjukkan bahwa vektorisasi
berguna, kami menambahkan beberapa titik kunci pada plot pada x = -1,
x = 0 dan x = 1.

Pada plot berikut, kita juga memasukkan beberapa kode LaTeX. Kami
menggunakannya untuk dua label dan sebuah kotak teks. Tentu saja, Anda
hanya dapat menggunakan LaTeX jika Anda telah menginstal LaTeX dengan
benar
\end{eulercomment}
\begin{eulerprompt}
>plot2d("f",-1,1,xl="x",yl="f(x)",grid=6);  ...
>plot2d([-1,0,1],f([-1,0,1]),>points,>add); ...
>label(latex("x^3"),0.72,f(0.72)); ...
>label(latex("x^2"),-0.52,f(-0.52),pos="ll"); ...
>textbox( ...
>  latex("f(x)=\(\backslash\)begin\{cases\} x^3 & x>0 \(\backslash\)\(\backslash\) x^2 & x \(\backslash\)le 0\(\backslash\)end\{cases\}"), ...
>  x=0.7,y=0.2):
\end{eulerprompt}
\eulerimg{17}{images/Pekan 5-6_Fanny Erina Dewi_22305141005_EMT00-Plot2D_Aplikom-070.png}
\begin{eulercomment}
\end{eulercomment}
\eulersubheading{Interaksi Pengguna}
\begin{eulercomment}
Ketika memplot fungsi atau ekspresi, parameter \textgreater{}user memungkinkan
pengguna untuk memperbesar dan menggeser plot dengan tombol kursor
atau mouse. \\
Pengguna dapat\\
- zoom dengan + atau -\\
- memindahkan plot dengan tombol kursor\\
- pilih jendela plot dengan mouse\\
- mengatur ulang tampilan dengan spasi\\
- keluar dengan kembali\\
Tombol spasi akan mengatur ulang plot ke jendela plot asli.

Saat memplot data, bendera \textgreater{}user hanya akan menunggu penekanan tombol.
\end{eulercomment}
\begin{eulerprompt}
>plot2d(\{\{"x^3-a*x",a=1\}\},>user,title="Press any key!"):
\end{eulerprompt}
\eulerimg{17}{images/Pekan 5-6_Fanny Erina Dewi_22305141005_EMT00-Plot2D_Aplikom-071.png}
\begin{eulerprompt}
>plot2d("exp(x)*sin(x)",user=true, ...
>  title="+/- or cursor keys (return to exit)"):
\end{eulerprompt}
\eulerimg{17}{images/Pekan 5-6_Fanny Erina Dewi_22305141005_EMT00-Plot2D_Aplikom-072.png}
\begin{eulercomment}
Berikut ini menunjukkan cara interaksi pengguna tingkat lanjut (lihat
tutorial mengenai pemrograman untuk detailnya).

Fungsi bawaan mousedrag() menunggu peristiwa mouse atau keyboard.
Fungsi ini melaporkan mouse ke bawah, mouse bergerak atau mouse ke
atas, dan penekanan tombol. Fungsi dragpoints() memanfaatkan hal ini,
dan mengizinkan pengguna untuk menyeret titik manapun di dalam plot.

Kita membutuhkan fungsi plot terlebih dahulu. Sebagai contoh, kita
melakukan interpolasi dalam 5 titik dengan polinomial. Fungsi ini
harus memplot ke dalam area plot yang tetap.
\end{eulercomment}
\begin{eulerprompt}
>function plotf(xp,yp,select) ...
\end{eulerprompt}
\begin{eulerudf}
    d=interp(xp,yp);
    plot2d("interpval(xp,d,x)";d,xp,r=2);
    plot2d(xp,yp,>points,>add);
    if select>0 then
      plot2d(xp[select],yp[select],color=red,>points,>add);
    endif;
    title("Drag one point, or press space or return!");
  endfunction
\end{eulerudf}
\begin{eulercomment}
Perhatikan parameter titik koma pada plot2d (d dan xp), yang
diteruskan ke evaluasi fungsi interp(). Tanpa ini, kita harus menulis
fungsi plotinterp() terlebih dahulu, untuk mengakses nilai secara
global.

Sekarang kita menghasilkan beberapa nilai acak, dan membiarkan
pengguna menyeret titik-titiknya
\end{eulercomment}
\begin{eulerprompt}
>t=-1:0.5:1; dragpoints("plotf",t,random(size(t))-0.5):
\end{eulerprompt}
\eulerimg{17}{images/Pekan 5-6_Fanny Erina Dewi_22305141005_EMT00-Plot2D_Aplikom-073.png}
\begin{eulercomment}
Ada juga fungsi yang memplot fungsi lain tergantung pada vektor
parameter, dan memungkinkan pengguna menyesuaikan parameter ini.

Pertama, kita memerlukan fungsi plot.
\end{eulercomment}
\begin{eulerprompt}
>function plotf([a,b]) := plot2d("exp(a*x)*cos(2pi*b*x)",0,2pi;a,b);
\end{eulerprompt}
\begin{eulercomment}
Kemudian kita membutuhkan nama untuk parameter, nilai awal dan matriks
rentang nx2, dan secara opsional, sebuah garis judul. Terdapat slider
interaktif, yang dapat mengatur nilai oleh pengguna. 

Fungsi dragvalues() menyediakan ini.
\end{eulercomment}
\begin{eulerprompt}
>dragvalues("plotf",["a","b"],[-1,2],[[-2,2];[1,10]], ...
>  heading="Drag these values:",hcolor=black):
\end{eulerprompt}
\eulerimg{17}{images/Pekan 5-6_Fanny Erina Dewi_22305141005_EMT00-Plot2D_Aplikom-074.png}
\begin{eulercomment}
Anda dapat membatasi nilai yang diseret menjadi bilangan bulat.
Sebagai contoh, kita menulis fungsi plot, yang memplot polinomial
Taylor dengan derajat n ke fungsi kosinus.
\end{eulercomment}
\begin{eulerprompt}
>function plotf(n) ...
\end{eulerprompt}
\begin{eulerudf}
  plot2d("cos(x)",0,2pi,>square,grid=6);
  plot2d(&"taylor(cos(x),x,0,@n)",color=blue,>add);
  textbox("Taylor polynomial of degree "+n,0.1,0.02,style="t",>left);
  endfunction
\end{eulerudf}
\begin{eulercomment}
Sekarang kita biarkan derajat n bervariasi dari 0 sampai 20 dalam 20
stop. Hasil dari dragvalues() digunakan untuk memplot sketsa dengan n
ini, dan untuk menyisipkan plot ke dalam buku catatan.
\end{eulercomment}
\begin{eulerprompt}
>nd=dragvalues("plotf","degree",2,[0,20],20,y=0.8, ...
>   heading="Drag the value:"); ...
>plotf(nd):
\end{eulerprompt}
\eulerimg{17}{images/Pekan 5-6_Fanny Erina Dewi_22305141005_EMT00-Plot2D_Aplikom-075.png}
\begin{eulercomment}
Berikut ini adalah peragaan sederhana dari fungsi ini. Pengguna dapat
menggambar di atas jendela plot, meninggalkan jejak titik.
\end{eulercomment}
\begin{eulerprompt}
>function dragtest ...
\end{eulerprompt}
\begin{eulerudf}
    plot2d(none,r=1,title="Drag with the mouse, or press any key!");
    start=0;
    repeat
      \{flag,m,time\}=mousedrag();
      if flag==0 then return; endif;
      if flag==2 then
        hold on; mark(m[1],m[2]); hold off;
      endif;
    end
  endfunction
\end{eulerudf}
\begin{eulerprompt}
>dragtest // lihat hasilnya dan cobalah lakukan!
\end{eulerprompt}
\eulersubheading{Gaya PLot 2D}
\begin{eulercomment}
Secara default, EMT menghitung tick sumbu otomatis dan menambahkan
label pada setiap tick. Hal ini dapat diubah dengan parameter
kisi-kisi. Gaya default sumbu dan label dapat dimodifikasi. Selain
itu, label dan judul dapat ditambahkan secara manual. Untuk mengatur
ulang ke gaya default, gunakan reset().
\end{eulercomment}
\begin{eulerprompt}
>aspect();
>figure(3,4); ...
> figure(1); plot2d("x^3-x",grid=0); ... // tidak ada grid, bingkai atau sumbu
> figure(2); plot2d("x^3-x",grid=1); ... // sumbu x-y
> figure(3); plot2d("x^3-x",grid=2); ... // kutu default
> figure(4); plot2d("x^3-x",grid=3); ... // sumbu x-y dengan label di dalamnya
> figure(5); plot2d("x^3-x",grid=4); ... // tidak ada tanda centang, hanya label
> figure(6); plot2d("x^3-x",grid=5); ... // default, tetapi tidak ada margin
> figure(7); plot2d("x^3-x",grid=6); ... // sumbu saja
> figure(8); plot2d("x^3-x",grid=7); ... // sumbu saja, centang pada sumbu
> figure(9); plot2d("x^3-x",grid=8); ... // sumbu saja, kutu yang lebih halus pada sumbu
> figure(10); plot2d("x^3-x",grid=9); ... // default, kutu kecil di dalamnya
> figure(11); plot2d("x^3-x",grid=10); ...// tidak ada kutu, hanya sumbu 
> figure(0):
\end{eulerprompt}
\eulerimg{27}{images/Pekan 5-6_Fanny Erina Dewi_22305141005_EMT00-Plot2D_Aplikom-076.png}
\begin{eulercomment}
Parameter \textless{}frame mematikan bingkai, dan framecolor=blue menetapkan
bingkai ke warna biru. 

Jika Anda menginginkan tanda centang Anda sendiri, Anda dapat
menggunakan style=0, dan menambahkan semuanya nanti.
\end{eulercomment}
\begin{eulerprompt}
>aspect(1.5); 
>plot2d("x^3-x",grid=0); // plot
>frame; xgrid([-1,0,1]); ygrid(0): // menambahkan frame dan grid
\end{eulerprompt}
\eulerimg{17}{images/Pekan 5-6_Fanny Erina Dewi_22305141005_EMT00-Plot2D_Aplikom-077.png}
\begin{eulercomment}
Untuk judul plot dan label sumbu, lihat contoh berikut.
\end{eulercomment}
\begin{eulerprompt}
>plot2d("exp(x)",-1,1);
>textcolor(black); // mengatur warna teks menjadi hitam
>title(latex("y=e^x")); // judul di atas plot
>xlabel(latex("x")); // "x" untuk sumbu x
>ylabel(latex("y"),>vertical); // vertikal "y" untuk sumbu y
>label(latex("(0,1)"),0,1,color=blue): // memberi label sebuah titik
\end{eulerprompt}
\eulerimg{17}{images/Pekan 5-6_Fanny Erina Dewi_22305141005_EMT00-Plot2D_Aplikom-078.png}
\begin{eulercomment}
Sumbu dapat digambar secara terpisah dengan sumbu x() dan sumbu y().
\end{eulercomment}
\begin{eulerprompt}
>plot2d("x^3-x",<grid,<frame);
>xaxis(0,xx=-2:1,style="->"); yaxis(0,yy=-5:5,style="->"):
\end{eulerprompt}
\eulerimg{17}{images/Pekan 5-6_Fanny Erina Dewi_22305141005_EMT00-Plot2D_Aplikom-079.png}
\begin{eulercomment}
Teks pada plot dapat diatur dengan label(). Pada contoh berikut ini,
"lc" berarti lower center. Ini mengatur posisi label relatif terhadap
koordinat plot.
\end{eulercomment}
\begin{eulerprompt}
>function f(x) &= x^3-x
\end{eulerprompt}
\begin{euleroutput}
  
                                   3
                                  x  - x
  
\end{euleroutput}
\begin{eulerprompt}
>plot2d(f,-1,1,>square);
>x0=fmin(f,0,1); // menghitung titik minimum
>label("Rel. Min.",x0,f(x0),pos="lc"): // tambahkan label di sana
\end{eulerprompt}
\eulerimg{17}{images/Pekan 5-6_Fanny Erina Dewi_22305141005_EMT00-Plot2D_Aplikom-080.png}
\begin{eulercomment}
Terdapat juga kotak teks.
\end{eulercomment}
\begin{eulerprompt}
>plot2d(&f(x),-1,1,-2,2); // fungsi
>plot2d(&diff(f(x),x),>add,style="--",color=red); // turunan
>labelbox(["f","f'"],["-","--"],[black,red]): // kotak label
\end{eulerprompt}
\eulerimg{17}{images/Pekan 5-6_Fanny Erina Dewi_22305141005_EMT00-Plot2D_Aplikom-081.png}
\begin{eulerprompt}
>plot2d(["exp(x)","1+x"],color=[black,blue],style=["-","-.-"]):
\end{eulerprompt}
\eulerimg{17}{images/Pekan 5-6_Fanny Erina Dewi_22305141005_EMT00-Plot2D_Aplikom-082.png}
\begin{eulerprompt}
>gridstyle("->",color=gray,textcolor=gray,framecolor=gray);  ...
> plot2d("x^3-x",grid=1);   ...
> settitle("y=x^3-x",color=black); ...
> label("x",2,0,pos="bc",color=gray);  ...
> label("y",0,6,pos="cl",color=gray); ...
> reset():
\end{eulerprompt}
\eulerimg{27}{images/Pekan 5-6_Fanny Erina Dewi_22305141005_EMT00-Plot2D_Aplikom-083.png}
\begin{eulercomment}
Untuk kontrol yang lebih banyak lagi, sumbu x dan sumbu y dapat
dilakukan secara manual.

Perintah fullwindow() memperluas jendela plot karena kita tidak lagi
membutuhkan tempat untuk label di luar jendela plot. Gunakan
shrinkwindow() atau reset() untuk mengatur ulang ke default.
\end{eulercomment}
\begin{eulerprompt}
>fullwindow; ...
> gridstyle(color=darkgray,textcolor=darkgray); ...
> plot2d(["2^x","1","2^(-x)"],a=-2,b=2,c=0,d=4,<grid,color=4:6,<frame); ...
> xaxis(0,-2:1,style="->"); xaxis(0,2,"x",<axis); ...
> yaxis(0,4,"y",style="->"); ...
> yaxis(-2,1:4,>left); ...
> yaxis(2,2^(-2:2),style=".",<left); ...
> labelbox(["2^x","1","2^-x"],colors=4:6,x=0.8,y=0.2); ...
> reset:
\end{eulerprompt}
\eulerimg{27}{images/Pekan 5-6_Fanny Erina Dewi_22305141005_EMT00-Plot2D_Aplikom-084.png}
\begin{eulercomment}
Berikut ini adalah contoh lain, di mana string Unicode digunakan dan
sumbu di luar area plot.
\end{eulercomment}
\begin{eulerprompt}
>aspect(1.5); 
>plot2d(["sin(x)","cos(x)"],0,2pi,color=[red,green],<grid,<frame); ...
> xaxis(-1.1,(0:2)*pi,xt=["0",u"&pi;",u"2&pi;"],style="-",>ticks,>zero);  ...
> xgrid((0:0.5:2)*pi,<ticks); ...
> yaxis(-0.1*pi,-1:0.2:1,style="-",>zero,>grid); ...
> labelbox(["sin","cos"],colors=[red,green],x=0.5,y=0.2,>left); ...
> xlabel(u"&phi;"); ylabel(u"f(&phi;)"):
\end{eulerprompt}
\eulerimg{17}{images/Pekan 5-6_Fanny Erina Dewi_22305141005_EMT00-Plot2D_Aplikom-085.png}
\eulerheading{Memplot Data 2D}
\begin{eulercomment}
Jika x dan y adalah vektor data, data ini akan digunakan sebagai
koordinat x dan y dari sebuah kurva. Dalam hal ini, a, b, c, dan d,
atau radius r dapat ditentukan, atau jendela plot akan menyesuaikan
secara otomatis dengan data. Atau, \textgreater{}square dapat diatur untuk
mempertahankan rasio aspek persegi.

Memplot ekspresi hanyalah singkatan dari plot data. Untuk plot data,
Anda memerlukan satu atau lebih baris nilai x, dan satu atau lebih
baris nilai y. Dari rentang dan nilai x, fungsi plot2d akan menghitung
data untuk diplot, secara default dengan evaluasi adaptif dari fungsi
tersebut. Untuk plot titik, gunakan "\textgreater{}points", untuk garis dan titik
campuran gunakan "\textgreater{}addpoints".

Tetapi Anda dapat memasukkan data secara langsung.

- Gunakan vektor baris untuk x dan y untuk satu fungsi.\\
- Matriks untuk x dan y diplot baris demi baris. 

Berikut adalah contoh dengan satu baris untuk x dan y.


\end{eulercomment}
\begin{eulerprompt}
>x=-10:0.1:10; y=exp(-x^2)*x; plot2d(x,y):
\end{eulerprompt}
\eulerimg{17}{images/Pekan 5-6_Fanny Erina Dewi_22305141005_EMT00-Plot2D_Aplikom-086.png}
\begin{eulercomment}
Data juga dapat diplot sebagai titik. Gunakan poin=true untuk ini.
Plot ini berfungsi seperti poligon, namun hanya menggambar
sudut-sudutnya saja.

- style = "...": Pilih dari "[]", "\textless{}\textgreater{}", "o", ".", "..", "+", "*",
"[]", "\textless{}\textgreater{}", "o", "..", "", "\textbar{}".

Untuk memplot kumpulan titik, gunakan \textgreater{}titik. Jika warna adalah vektor
warna, setiap titik mendapatkan warna yang berbeda. Untuk matriks
koordinat dan vektor kolom, warna berlaku untuk baris matriks.

Parameter \textgreater{}addpoints menambahkan titik ke segmen garis untuk plot
data.
\end{eulercomment}
\begin{eulerprompt}
>xdata=[1,1.5,2.5,3,4]; ydata=[3,3.1,2.8,2.9,2.7]; // data
>plot2d(xdata,ydata,a=0.5,b=4.5,c=2.5,d=3.5,style="."); // garis
>plot2d(xdata,ydata,>points,>add,style="o"): // tambahkan poin
\end{eulerprompt}
\eulerimg{17}{images/Pekan 5-6_Fanny Erina Dewi_22305141005_EMT00-Plot2D_Aplikom-087.png}
\begin{eulerprompt}
>p=polyfit(xdata,ydata,1); // mendapatkan garis regresi
>plot2d("polyval(p,x)",>add,color=red): // menambahkan plot garis
\end{eulerprompt}
\eulerimg{17}{images/Pekan 5-6_Fanny Erina Dewi_22305141005_EMT00-Plot2D_Aplikom-088.png}
\eulerheading{Menggambar Daerah Yang Dibatasi Kurva}
\begin{eulercomment}
Plot data benar-benar berupa poligon. Kita juga dapat memplot kurva
atau kurva terisi.

- filled = true mengisi plot.\\
- style = "...": Pilih dari "", "/", "\textbackslash{}", "\textbackslash{}/".\\
- fillcolor: Lihat di atas untuk warna yang tersedia.

Warna isian ditentukan oleh argumen "fillcolor", dan pada opsional
\textless{}outline mencegah menggambar batas untuk semua gaya kecuali yang
default.
\end{eulercomment}
\begin{eulerprompt}
>t=linspace(0,2pi,1000); // parameter untuk kurva
>x=sin(t)*exp(t/pi); y=cos(t)*exp(t/pi); // x(t) and y(t)
>figure(1,2); aspect(16/9)
>figure(1); plot2d(x,y,r=10); // plot kurva
>figure(2); plot2d(x,y,r=10,>filled,style="/",fillcolor=red); // mengisi kurva
>figure(0):
\end{eulerprompt}
\eulerimg{14}{images/Pekan 5-6_Fanny Erina Dewi_22305141005_EMT00-Plot2D_Aplikom-089.png}
\begin{eulercomment}
Pada contoh berikut ini, kami memplot elips yang terisi dan dua segi
enam yang terisi menggunakan kurva tertutup dengan 6 titik dengan gaya
isian yang berbeda.
\end{eulercomment}
\begin{eulerprompt}
>x=linspace(0,2pi,1000); plot2d(sin(x),cos(x)*0.5,r=1,>filled,style="/"):
\end{eulerprompt}
\eulerimg{14}{images/Pekan 5-6_Fanny Erina Dewi_22305141005_EMT00-Plot2D_Aplikom-090.png}
\begin{eulerprompt}
>t=linspace(0,2pi,6); ...
>plot2d(cos(t),sin(t),>filled,style="/",fillcolor=red,r=1.2):
\end{eulerprompt}
\eulerimg{14}{images/Pekan 5-6_Fanny Erina Dewi_22305141005_EMT00-Plot2D_Aplikom-091.png}
\begin{eulerprompt}
>t=linspace(0,2pi,6); plot2d(cos(t),sin(t),>filled,style="#"):
\end{eulerprompt}
\eulerimg{14}{images/Pekan 5-6_Fanny Erina Dewi_22305141005_EMT00-Plot2D_Aplikom-092.png}
\begin{eulercomment}
Contoh lainnya adalah septagon, yang kita buat dengan 7 titik pada
lingkaran satuan.
\end{eulercomment}
\begin{eulerprompt}
>t=linspace(0,2pi,7);  ...
> plot2d(cos(t),sin(t),r=1,>filled,style="/",fillcolor=red):
\end{eulerprompt}
\eulerimg{14}{images/Pekan 5-6_Fanny Erina Dewi_22305141005_EMT00-Plot2D_Aplikom-093.png}
\begin{eulercomment}
Berikut ini adalah himpunan nilai maksimal dari empat kondisi linier
yang kurang dari atau sama dengan 3. Ini adalah A[k].v\textless{}=3 untuk semua
barisan A. Untuk mendapatkan sudut-sudut yang bagus, kita menggunakan
n yang relatif besar.
\end{eulercomment}
\begin{eulerprompt}
>A=[2,1;1,2;-1,0;0,-1];
>function f(x,y) := max([x,y].A');
>plot2d("f",r=4,level=[0;3],color=green,n=111):
\end{eulerprompt}
\eulerimg{14}{images/Pekan 5-6_Fanny Erina Dewi_22305141005_EMT00-Plot2D_Aplikom-094.png}
\begin{eulercomment}
Poin utama dari bahasa matriks adalah bahwa bahasa ini memungkinkan
untuk menghasilkan tabel fungsi dengan mudah.
\end{eulercomment}
\begin{eulerprompt}
>t=linspace(0,2pi,1000); x=cos(3*t); y=sin(4*t);
\end{eulerprompt}
\begin{eulercomment}
Kita sekarang memiliki vektor nilai x dan y. plot2d() dapat memplot
nilai-nilai ini sebagai sebuah kurva yang menghubungkan titik-titik.
Plot dapat diisi. Dalam kasus ini, hal ini memberikan hasil yang bagus
karena aturan penggulungan, yang digunakan untuk\\
isi.
\end{eulercomment}
\begin{eulerprompt}
>plot2d(x,y,<grid,<frame,>filled):
\end{eulerprompt}
\eulerimg{14}{images/Pekan 5-6_Fanny Erina Dewi_22305141005_EMT00-Plot2D_Aplikom-095.png}
\begin{eulercomment}
Vektor interval diplot terhadap nilai x sebagai wilayah yang terisi
antara nilai bawah dan atas interval.

Hal ini dapat berguna untuk memplot kesalahan perhitungan. Tetapi juga
dapat digunakan untuk memplot kesalahan statistik.
\end{eulercomment}
\begin{eulerprompt}
>t=0:0.1:1; ...
> plot2d(t,interval(t-random(size(t)),t+random(size(t))),style="|");  ...
> plot2d(t,t,add=true):
\end{eulerprompt}
\eulerimg{14}{images/Pekan 5-6_Fanny Erina Dewi_22305141005_EMT00-Plot2D_Aplikom-096.png}
\begin{eulercomment}
Jika x adalah vektor yang diurutkan, dan y adalah vektor interval,
maka plot2d akan memplot rentang interval yang terisi di bidang, gaya
isian sama dengan gaya poligon.
\end{eulercomment}
\begin{eulerprompt}
>t=-1:0.01:1; x=~t-0.01,t+0.01~; y=x^3-x;
>plot2d(t,y):
\end{eulerprompt}
\eulerimg{14}{images/Pekan 5-6_Fanny Erina Dewi_22305141005_EMT00-Plot2D_Aplikom-097.png}
\begin{eulercomment}
Dimungkinkan untuk mengisi wilayah nilai untuk fungsi tertentu. Untuk
ini, level harus berupa matriks 2xn. Baris pertama adalah batas bawah
dan baris kedua berisi batas atas.
\end{eulercomment}
\begin{eulerprompt}
>expr := "2*x^2+x*y+3*y^4+y"; // mendefinisikan sebuah ekspresi f(x,y)
>plot2d(expr,level=[0;1],style="-",color=blue): // 0 <= f(x,y) <= 1
\end{eulerprompt}
\eulerimg{14}{images/Pekan 5-6_Fanny Erina Dewi_22305141005_EMT00-Plot2D_Aplikom-098.png}
\begin{eulercomment}
Kita juga dapat mengisi rentang nilai seperti

\end{eulercomment}
\begin{eulerformula}
\[
-1 \le (x^2+y^2)^2-x^2+y^2 \le 0.
\]
\end{eulerformula}
\begin{eulercomment}
\end{eulercomment}
\begin{eulerprompt}
>plot2d("(x^2+y^2)^2-x^2+y^2",r=1.2,level=[-1;0],style="/"):
\end{eulerprompt}
\eulerimg{14}{images/Pekan 5-6_Fanny Erina Dewi_22305141005_EMT00-Plot2D_Aplikom-099.png}
\begin{eulerprompt}
>plot2d("cos(x)","sin(x)^3",xmin=0,xmax=2pi,>filled,style="/"):
\end{eulerprompt}
\eulerimg{14}{images/Pekan 5-6_Fanny Erina Dewi_22305141005_EMT00-Plot2D_Aplikom-100.png}
\eulerheading{Grafik Fungsi Parametrik}
\begin{eulercomment}
Nilai x tidak perlu diurutkan. (x,y) hanya menggambarkan sebuah kurva.
Jika x diurutkan, kurva tersebut adalah grafik dari sebuah fungsi.

Dalam contoh berikut, kami memplot spiral


\end{eulercomment}
\begin{eulerformula}
\[
\gamma(t) = t \cdot (\cos(2\pi t),\sin(2\pi t))
\]
\end{eulerformula}
\begin{eulercomment}
Kita perlu menggunakan sangat banyak titik untuk tampilan yang halus
atau fungsi adaptive() untuk mengevaluasi ekspresi (lihat fungsi
adaptive() untuk lebih jelasnya).
\end{eulercomment}
\begin{eulerprompt}
>t=linspace(0,1,1000); ...
>plot2d(t*cos(2*pi*t),t*sin(2*pi*t),r=1):
\end{eulerprompt}
\eulerimg{14}{images/Pekan 5-6_Fanny Erina Dewi_22305141005_EMT00-Plot2D_Aplikom-101.png}
\begin{eulercomment}
Sebagai alternatif, Anda dapat menggunakan dua ekspresi untuk kurva.
Berikut ini adalah plot kurva yang sama seperti di atas.
\end{eulercomment}
\begin{eulerprompt}
>plot2d("x*cos(2*pi*x)","x*sin(2*pi*x)",xmin=0,xmax=1,r=1):
\end{eulerprompt}
\eulerimg{14}{images/Pekan 5-6_Fanny Erina Dewi_22305141005_EMT00-Plot2D_Aplikom-102.png}
\begin{eulerprompt}
>t=linspace(0,1,1000); r=exp(-t); x=r*cos(2pi*t); y=r*sin(2pi*t);
>plot2d(x,y,r=1):
\end{eulerprompt}
\eulerimg{14}{images/Pekan 5-6_Fanny Erina Dewi_22305141005_EMT00-Plot2D_Aplikom-103.png}
\begin{eulercomment}
Dalam contoh berikut, kami memplot\\
kurva

\end{eulercomment}
\begin{eulerformula}
\[
\gamma(t) = (r(t) \cos(t), r(t) \sin(t))
\]
\end{eulerformula}
\begin{eulercomment}
dengan

\end{eulercomment}
\begin{eulerformula}
\[
r(t) = 1 + \dfrac{\sin(3t)}{2}.
\]
\end{eulerformula}
\begin{eulerprompt}
>t=linspace(0,2pi,1000); r=1+sin(3*t)/2; x=r*cos(t); y=r*sin(t); ...
>plot2d(x,y,>filled,fillcolor=red,style="/",r=1.5):
\end{eulerprompt}
\eulerimg{14}{images/Pekan 5-6_Fanny Erina Dewi_22305141005_EMT00-Plot2D_Aplikom-104.png}
\eulerheading{Menggambar Grafik Bilangan Kompleks}
\begin{eulercomment}
Larik bilangan kompleks juga dapat diplot. Kemudian titik-titik kisi
akan dihubungkan. Jika sejumlah garis kisi ditentukan (atau vektor 1x2
garis kisi) pada argumen cgrid, hanya garis-garis kisi tersebut yang
akan terlihat.

Matriks bilangan kompleks akan secara otomatis diplot sebagai
kisi-kisi pada bidang kompleks.

Pada contoh berikut, kami memplot gambar lingkaran satuan di bawah
fungsi eksponensial. Parameter cgrid menyembunyikan beberapa kurva
kisi-kisi.
\end{eulercomment}
\begin{eulerprompt}
>aspect(); r=linspace(0,1,50); a=linspace(0,2pi,80)'; z=r*exp(I*a);...
>plot2d(z,a=-1.25,b=1.25,c=-1.25,d=1.25,cgrid=10):
\end{eulerprompt}
\eulerimg{27}{images/Pekan 5-6_Fanny Erina Dewi_22305141005_EMT00-Plot2D_Aplikom-105.png}
\begin{eulerprompt}
>aspect(1.25); r=linspace(0,1,50); a=linspace(0,2pi,200)'; z=r*exp(I*a);
>plot2d(exp(z),cgrid=[40,10]):
\end{eulerprompt}
\eulerimg{21}{images/Pekan 5-6_Fanny Erina Dewi_22305141005_EMT00-Plot2D_Aplikom-106.png}
\begin{eulerprompt}
>r=linspace(0,1,10); a=linspace(0,2pi,40)'; z=r*exp(I*a);
>plot2d(exp(z),>points,>add):
\end{eulerprompt}
\eulerimg{21}{images/Pekan 5-6_Fanny Erina Dewi_22305141005_EMT00-Plot2D_Aplikom-107.png}
\begin{eulercomment}
Vektor bilangan kompleks secara otomatis diplot sebagai kurva pada
bidang kompleks dengan bagian riil dan bagian imajiner.

Dalam contoh, kami memplot lingkaran satuan dengan


\end{eulercomment}
\begin{eulerformula}
\[
\gamma(t) = e^{it}
\]
\end{eulerformula}
\begin{eulerprompt}
>t=linspace(0,2pi,1000); ...
>plot2d(exp(I*t)+exp(4*I*t),r=2):
\end{eulerprompt}
\eulerimg{21}{images/Pekan 5-6_Fanny Erina Dewi_22305141005_EMT00-Plot2D_Aplikom-108.png}
\eulerheading{Plot Statistik}
\begin{eulercomment}
Terdapat banyak fungsi yang dikhususkan pada plot statistik. Salah
satu plot yang sering digunakan adalah plot kolom.

Jumlah kumulatif dari nilai berdistribusi normal 0-1 menghasilkan
jalan acak.
\end{eulercomment}
\begin{eulerprompt}
>plot2d(cumsum(randnormal(1,1000))):
\end{eulerprompt}
\eulerimg{21}{images/Pekan 5-6_Fanny Erina Dewi_22305141005_EMT00-Plot2D_Aplikom-109.png}
\begin{eulercomment}
Dengan menggunakan dua baris, ini menunjukkan jalan kaki dalam dua
dimensi.
\end{eulercomment}
\begin{eulerprompt}
>X=cumsum(randnormal(2,1000)); plot2d(X[1],X[2]):
\end{eulerprompt}
\eulerimg{21}{images/Pekan 5-6_Fanny Erina Dewi_22305141005_EMT00-Plot2D_Aplikom-110.png}
\begin{eulerprompt}
>columnsplot(cumsum(random(10)),style="/",color=blue):
\end{eulerprompt}
\eulerimg{21}{images/Pekan 5-6_Fanny Erina Dewi_22305141005_EMT00-Plot2D_Aplikom-111.png}
\begin{eulercomment}
Ini juga dapat menampilkan string sebagai label.
\end{eulercomment}
\begin{eulerprompt}
>months=["Jan","Feb","Mar","Apr","May","Jun", ...
>  "Jul","Aug","Sep","Oct","Nov","Dec"];
>values=[10,12,12,18,22,28,30,26,22,18,12,8];
>columnsplot(values,lab=months,color=red,style="-");
>title("Temperature"):
\end{eulerprompt}
\eulerimg{21}{images/Pekan 5-6_Fanny Erina Dewi_22305141005_EMT00-Plot2D_Aplikom-112.png}
\begin{eulerprompt}
>k=0:10;
>plot2d(k,bin(10,k),>bar):
\end{eulerprompt}
\eulerimg{21}{images/Pekan 5-6_Fanny Erina Dewi_22305141005_EMT00-Plot2D_Aplikom-113.png}
\begin{eulerprompt}
>plot2d(k,bin(10,k)); plot2d(k,bin(10,k),>points,>add):
\end{eulerprompt}
\eulerimg{21}{images/Pekan 5-6_Fanny Erina Dewi_22305141005_EMT00-Plot2D_Aplikom-114.png}
\begin{eulerprompt}
>plot2d(normal(1000),normal(1000),>points,grid=6,style=".."):
\end{eulerprompt}
\eulerimg{21}{images/Pekan 5-6_Fanny Erina Dewi_22305141005_EMT00-Plot2D_Aplikom-115.png}
\begin{eulerprompt}
>plot2d(normal(1,1000),>distribution,style="O"):
\end{eulerprompt}
\eulerimg{21}{images/Pekan 5-6_Fanny Erina Dewi_22305141005_EMT00-Plot2D_Aplikom-116.png}
\begin{eulerprompt}
>plot2d("qnormal",0,5;2.5,0.5,>filled):
\end{eulerprompt}
\eulerimg{21}{images/Pekan 5-6_Fanny Erina Dewi_22305141005_EMT00-Plot2D_Aplikom-117.png}
\begin{eulercomment}
Untuk memplot distribusi statistik eksperimental, Anda dapat
menggunakan distribution=n dengan plot2d.
\end{eulercomment}
\begin{eulerprompt}
>w=randexponential(1,1000); // distribusi eksponsial
>plot2d(w,>distribution): // atau distribusi=n dengan n interval 
\end{eulerprompt}
\eulerimg{21}{images/Pekan 5-6_Fanny Erina Dewi_22305141005_EMT00-Plot2D_Aplikom-118.png}
\begin{eulercomment}
Atau Anda dapat menghitung distribusi dari data dan memplot hasilnya
dengan \textgreater{}bar di plot3d, atau dengan plot kolom.
\end{eulercomment}
\begin{eulerprompt}
>w=normal(1000); // distribusi 0-1-normal
>\{x,y\}=histo(w,10,v=[-6,-4,-2,-1,0,1,2,4,6]); // batas-batas interval v
>plot2d(x,y,>bar):
\end{eulerprompt}
\eulerimg{21}{images/Pekan 5-6_Fanny Erina Dewi_22305141005_EMT00-Plot2D_Aplikom-119.png}
\begin{eulercomment}
Fungsi statplot() menetapkan gaya dengan string sederhana.
\end{eulercomment}
\begin{eulerprompt}
>statplot(1:10,cumsum(random(10)),"b"):
\end{eulerprompt}
\eulerimg{21}{images/Pekan 5-6_Fanny Erina Dewi_22305141005_EMT00-Plot2D_Aplikom-120.png}
\begin{eulerprompt}
>n=10; i=0:n; ...
>plot2d(i,bin(n,i)/2^n,a=0,b=10,c=0,d=0.3); ...
>plot2d(i,bin(n,i)/2^n,points=true,style="ow",add=true,color=blue):
\end{eulerprompt}
\eulerimg{21}{images/Pekan 5-6_Fanny Erina Dewi_22305141005_EMT00-Plot2D_Aplikom-121.png}
\begin{eulercomment}
Selain itu, data dapat diplot sebagai batang. Dalam hal ini, x harus
diurutkan dan satu elemen lebih panjang dari y. Batang akan memanjang
dari x[i] ke x[i+1] dengan nilai y[i]. Jika x memiliki ukuran yang
sama dengan y, maka x akan diperpanjang satu elemen dengan jarak
terakhir.

Gaya isian dapat digunakan seperti di atas.
\end{eulercomment}
\begin{eulerprompt}
>n=10; k=bin(n,0:n); ...
>plot2d(-0.5:n+0.5,k,bar=true,fillcolor=lightgray):
\end{eulerprompt}
\eulerimg{21}{images/Pekan 5-6_Fanny Erina Dewi_22305141005_EMT00-Plot2D_Aplikom-122.png}
\begin{eulercomment}
Data untuk plot batang (batang = 1) dan histogram (histogram = 1)
dapat diberikan secara eksplisit dalam xv dan yv, atau dapat dihitung
dari distribusi empiris dalam xv dengan \textgreater{}distribusi (atau distribusi =
n). Histogram dari nilai xv akan dihitung secara otomatis dengan
\textgreater{}histogram. Jika \textgreater{}even ditentukan, nilai xv akan dihitung dalam
interval bilangan bulat.
\end{eulercomment}
\begin{eulerprompt}
>plot2d(normal(10000),distribution=50):
\end{eulerprompt}
\eulerimg{21}{images/Pekan 5-6_Fanny Erina Dewi_22305141005_EMT00-Plot2D_Aplikom-123.png}
\begin{eulerprompt}
>k=0:10; m=bin(10,k); x=(0:11)-0.5; plot2d(x,m,>bar):
\end{eulerprompt}
\eulerimg{21}{images/Pekan 5-6_Fanny Erina Dewi_22305141005_EMT00-Plot2D_Aplikom-124.png}
\begin{eulerprompt}
>columnsplot(m,k):
\end{eulerprompt}
\eulerimg{21}{images/Pekan 5-6_Fanny Erina Dewi_22305141005_EMT00-Plot2D_Aplikom-125.png}
\begin{eulerprompt}
>plot2d(random(600)*6,histogram=6):
\end{eulerprompt}
\eulerimg{21}{images/Pekan 5-6_Fanny Erina Dewi_22305141005_EMT00-Plot2D_Aplikom-126.png}
\begin{eulercomment}
Untuk distribusi, ada parameter distribution=n, yang menghitung nilai
secara otomatis dan mencetak distribusi relatif dengan n sub-interval.
\end{eulercomment}
\begin{eulerprompt}
>plot2d(normal(1,1000),distribution=10,style="\(\backslash\)/"):
\end{eulerprompt}
\eulerimg{21}{images/Pekan 5-6_Fanny Erina Dewi_22305141005_EMT00-Plot2D_Aplikom-127.png}
\begin{eulercomment}
Dengan parameter even=true, ini akan menggunakan interval bilangan
bulat.
\end{eulercomment}
\begin{eulerprompt}
>plot2d(intrandom(1,1000,10),distribution=10,even=true):
\end{eulerprompt}
\eulerimg{21}{images/Pekan 5-6_Fanny Erina Dewi_22305141005_EMT00-Plot2D_Aplikom-128.png}
\begin{eulercomment}
Perhatikan bahwa ada banyak plot statistik yang mungkin berguna.
Lihatlah tutorial tentang statistik.
\end{eulercomment}
\begin{eulerprompt}
>columnsplot(getmultiplicities(1:6,intrandom(1,6000,6))):
\end{eulerprompt}
\eulerimg{21}{images/Pekan 5-6_Fanny Erina Dewi_22305141005_EMT00-Plot2D_Aplikom-129.png}
\begin{eulerprompt}
>plot2d(normal(1,1000),>distribution); ...
>  plot2d("qnormal(x)",color=red,thickness=2,>add):
\end{eulerprompt}
\eulerimg{21}{images/Pekan 5-6_Fanny Erina Dewi_22305141005_EMT00-Plot2D_Aplikom-130.png}
\begin{eulercomment}
Ada juga banyak plot khusus untuk statistik. Boxplot menunjukkan
kuartil dari distribusi ini dan banyak pencilan. Menurut definisi,
pencilan dalam boxplot adalah data yang melebihi 1,5 kali kisaran 50\%
tengah plot.
\end{eulercomment}
\begin{eulerprompt}
>M=normal(5,1000); boxplot(quartiles(M)):
\end{eulerprompt}
\eulerimg{21}{images/Pekan 5-6_Fanny Erina Dewi_22305141005_EMT00-Plot2D_Aplikom-131.png}
\eulerheading{Fungsi Implisit}
\begin{eulercomment}
Plot implisit menunjukkan garis level yang menyelesaikan f(x,y)=level,
di mana "level" dapat berupa nilai tunggal atau vektor nilai. Jika
level = "auto", akan ada nc garis level, yang akan menyebar di antara
nilai minimum dan maksimum fungsi secara merata. Warna yang lebih
gelap atau lebih terang dapat ditambahkan dengan \textgreater{}hue untuk
mengindikasikan nilai fungsi. Untuk fungsi implisit, xv haruslah
sebuah fungsi atau ekspresi dari parameter x dan y, atau, sebagai
alternatif, xv dapat berupa sebuah matriks nilai.

Euler dapat menandai garis level


\end{eulercomment}
\begin{eulerformula}
\[
f(x,y) = c
\]
\end{eulerformula}
\begin{eulercomment}
dari fungsi apa pun.

Untuk menggambar himpunan f(x,y)=c untuk satu atau lebih konstanta c,
Anda dapat menggunakan plot2d() dengan plot implisitnya pada bidang.
Parameter untuk c adalah level = c, di mana c dapat berupa vektor
garis level. Sebagai tambahan, sebuah skema warna dapat digambar pada
latar belakang untuk mengindikasikan nilai fungsi untuk setiap titik
pada plot. Parameter "n" menentukan kehalusan plot.
\end{eulercomment}
\begin{eulerprompt}
>aspect(1.5); 
>plot2d("x^2+y^2-x*y-x",r=1.5,level=0,contourcolor=red):
\end{eulerprompt}
\eulerimg{17}{images/Pekan 5-6_Fanny Erina Dewi_22305141005_EMT00-Plot2D_Aplikom-132.png}
\begin{eulerprompt}
>expr := "2*x^2+x*y+3*y^4+y"; // mendefinisikan sebuah ekpresi f(x,y)
>plot2d(expr,level=0): // Solusi dari f(x,y)=0
\end{eulerprompt}
\eulerimg{17}{images/Pekan 5-6_Fanny Erina Dewi_22305141005_EMT00-Plot2D_Aplikom-133.png}
\begin{eulerprompt}
>plot2d(expr,level=0:0.5:20,>hue,contourcolor=white,n=200): // bagus
\end{eulerprompt}
\eulerimg{17}{images/Pekan 5-6_Fanny Erina Dewi_22305141005_EMT00-Plot2D_Aplikom-134.png}
\begin{eulerprompt}
>plot2d(expr,level=0:0.5:20,>hue,>spectral,n=200,grid=4): // lebih bagus
\end{eulerprompt}
\eulerimg{17}{images/Pekan 5-6_Fanny Erina Dewi_22305141005_EMT00-Plot2D_Aplikom-135.png}
\begin{eulercomment}
Hal ini juga berlaku untuk plot data. Tetapi Anda harus menentukan
rentang untuk label sumbu.
\end{eulercomment}
\begin{eulerprompt}
>x=-2:0.05:1; y=x'; z=expr(x,y);
>plot2d(z,level=0,a=-1,b=2,c=-2,d=1,>hue):
\end{eulerprompt}
\eulerimg{17}{images/Pekan 5-6_Fanny Erina Dewi_22305141005_EMT00-Plot2D_Aplikom-136.png}
\begin{eulerprompt}
>plot2d("x^3-y^2",>contour,>hue,>spectral):
\end{eulerprompt}
\eulerimg{17}{images/Pekan 5-6_Fanny Erina Dewi_22305141005_EMT00-Plot2D_Aplikom-137.png}
\begin{eulerprompt}
>plot2d("x^3-y^2",level=0,contourwidth=3,>add,contourcolor=red):
\end{eulerprompt}
\eulerimg{17}{images/Pekan 5-6_Fanny Erina Dewi_22305141005_EMT00-Plot2D_Aplikom-138.png}
\begin{eulerprompt}
>z=z+normal(size(z))*0.2;
>plot2d(z,level=0.5,a=-1,b=2,c=-2,d=1):
\end{eulerprompt}
\eulerimg{17}{images/Pekan 5-6_Fanny Erina Dewi_22305141005_EMT00-Plot2D_Aplikom-139.png}
\begin{eulerprompt}
>plot2d(expr,level=[0:0.2:5;0.05:0.2:5.05],color=lightgray):
\end{eulerprompt}
\eulerimg{17}{images/Pekan 5-6_Fanny Erina Dewi_22305141005_EMT00-Plot2D_Aplikom-140.png}
\begin{eulerprompt}
>plot2d("x^2+y^3+x*y",level=1,r=4,n=100):
\end{eulerprompt}
\eulerimg{17}{images/Pekan 5-6_Fanny Erina Dewi_22305141005_EMT00-Plot2D_Aplikom-141.png}
\begin{eulerprompt}
>plot2d("x^2+2*y^2-x*y",level=0:0.1:10,n=100,contourcolor=white,>hue):
\end{eulerprompt}
\eulerimg{17}{images/Pekan 5-6_Fanny Erina Dewi_22305141005_EMT00-Plot2D_Aplikom-142.png}
\begin{eulercomment}
Dimungkinkan juga untuk\\
mengisi set

\end{eulercomment}
\begin{eulerformula}
\[
a \le f(x,y) \le b
\]
\end{eulerformula}
\begin{eulercomment}
dengan rentang level.

Dimungkinkan untuk mengisi wilayah nilai untuk fungsi tertentu. Untuk
ini, level harus berupa matriks 2xn. Baris pertama adalah batas bawah
dan baris kedua berisi batas atas.
\end{eulercomment}
\begin{eulerprompt}
>plot2d(expr,level=[0;1],style="-",color=blue): // 0 <= f(x,y) <= 1
\end{eulerprompt}
\eulerimg{17}{images/Pekan 5-6_Fanny Erina Dewi_22305141005_EMT00-Plot2D_Aplikom-143.png}
\begin{eulercomment}
Plot implisit juga dapat menunjukkan rentang level. Maka level harus
berupa matriks 2xn interval level, di mana baris pertama berisi awal
dan baris kedua adalah akhir dari setiap interval. Sebagai alternatif,
vektor baris sederhana dapat digunakan untuk level, dan parameter dl
memperluas nilai level ke interval.
\end{eulercomment}
\begin{eulerprompt}
>plot2d("x^4+y^4",r=1.5,level=[0;1],color=blue,style="/"):
\end{eulerprompt}
\eulerimg{17}{images/Pekan 5-6_Fanny Erina Dewi_22305141005_EMT00-Plot2D_Aplikom-144.png}
\begin{eulerprompt}
>plot2d("x^2+y^3+x*y",level=[0,2,4;1,3,5],style="/",r=2,n=100):
\end{eulerprompt}
\eulerimg{17}{images/Pekan 5-6_Fanny Erina Dewi_22305141005_EMT00-Plot2D_Aplikom-145.png}
\begin{eulerprompt}
>plot2d("x^2+y^3+x*y",level=-10:20,r=2,style="-",dl=0.1,n=100):
\end{eulerprompt}
\eulerimg{17}{images/Pekan 5-6_Fanny Erina Dewi_22305141005_EMT00-Plot2D_Aplikom-146.png}
\begin{eulerprompt}
>plot2d("sin(x)*cos(y)",r=pi,>hue,>levels,n=100):
\end{eulerprompt}
\eulerimg{17}{images/Pekan 5-6_Fanny Erina Dewi_22305141005_EMT00-Plot2D_Aplikom-147.png}
\begin{eulercomment}
Anda juga dapat menandai suatu\\
wilayah 

\end{eulercomment}
\begin{eulerformula}
\[
a \le f(x,y) \le b.
\]
\end{eulerformula}
\begin{eulercomment}
Hal ini dilakukan dengan menambahkan level dengan dua baris.
\end{eulercomment}
\begin{eulerprompt}
>plot2d("(x^2+y^2-1)^3-x^2*y^3",r=1.3, ...
>  style="#",color=red,<outline, ...
>  level=[-2;0],n=100):
\end{eulerprompt}
\eulerimg{17}{images/Pekan 5-6_Fanny Erina Dewi_22305141005_EMT00-Plot2D_Aplikom-148.png}
\begin{eulercomment}
Dimungkinkan untuk menentukan level tertentu. Misalnya, kita dapat
memplot solusi dari persamaan seperti

\end{eulercomment}
\begin{eulerformula}
\[
x^3-xy+x^2y^2=6
\]
\end{eulerformula}
\begin{eulerprompt}
>plot2d("x^3-x*y+x^2*y^2",r=6,level=1,n=100):
\end{eulerprompt}
\eulerimg{17}{images/Pekan 5-6_Fanny Erina Dewi_22305141005_EMT00-Plot2D_Aplikom-149.png}
\begin{eulerprompt}
>function starplot1 (v, style="/", color=green, lab=none) ...
\end{eulerprompt}
\begin{eulerudf}
    if !holding() then clg; endif;
    w=window(); window(0,0,1024,1024);
    h=holding(1);
    r=max(abs(v))*1.2;
    setplot(-r,r,-r,r);
    n=cols(v); t=linspace(0,2pi,n);
    v=v|v[1]; c=v*cos(t); s=v*sin(t);
    cl=barcolor(color); st=barstyle(style);
    loop 1 to n
      polygon([0,c[#],c[#+1]],[0,s[#],s[#+1]],1);
      if lab!=none then
        rlab=v[#]+r*0.1;
        \{col,row\}=toscreen(cos(t[#])*rlab,sin(t[#])*rlab);
        ctext(""+lab[#],col,row-textheight()/2);
      endif;
    end;
    barcolor(cl); barstyle(st);
    holding(h);
    window(w);
  endfunction
\end{eulerudf}
\begin{eulercomment}
Tidak ada kisi-kisi atau kutu sumbu di sini. Selain itu, kami
menggunakan jendela penuh untuk plot. 

Kami memanggil reset sebelum kami menguji plot ini untuk mengembalikan
default grafis. Hal ini tidak perlu dilakukan, jika Anda yakin bahwa
plot Anda berfungsi.
\end{eulercomment}
\begin{eulerprompt}
>reset; starplot1(normal(1,10)+5,color=red,lab=1:10):
\end{eulerprompt}
\eulerimg{27}{images/Pekan 5-6_Fanny Erina Dewi_22305141005_EMT00-Plot2D_Aplikom-150.png}
\begin{eulercomment}
Terkadang, Anda mungkin ingin merencanakan sesuatu yang tidak dapat
dilakukan oleh plot2d, tetapi hampir. 

Pada fungsi berikut, kita melakukan plot impuls logaritmik. plot2d
dapat melakukan plot logaritmik, tetapi tidak untuk batang impuls.
\end{eulercomment}
\begin{eulerprompt}
>function logimpulseplot1 (x,y) ...
\end{eulerprompt}
\begin{eulerudf}
    \{x0,y0\}=makeimpulse(x,log(y)/log(10));
    plot2d(x0,y0,>bar,grid=0);
    h=holding(1);
    frame();
    xgrid(ticks(x));
    p=plot();
    for i=-10 to 10;
      if i<=p[4] and i>=p[3] then
         ygrid(i,yt="10^"+i);
      endif;
    end;
    holding(h);
  endfunction
\end{eulerudf}
\begin{eulercomment}
Mari kita uji dengan nilai yang terdistribusi secara eksponensial.
\end{eulercomment}
\begin{eulerprompt}
>aspect(1.5); x=1:10; y=-log(random(size(x)))*200; ...
>logimpulseplot1(x,y):
\end{eulerprompt}
\eulerimg{17}{images/Pekan 5-6_Fanny Erina Dewi_22305141005_EMT00-Plot2D_Aplikom-151.png}
\begin{eulercomment}
Mari kita menghidupkan kurva 2D dengan menggunakan plot langsung.
Perintah plot(x,y) hanya memplot kurva ke dalam jendela plot.
setplot(a,b,c,d) mengatur jendela ini.

Fungsi wait(0) memaksa plot untuk muncul pada jendela grafis. Kalau
tidak, penggambaran ulang dilakukan dalam interval waktu yang jarang.
\end{eulercomment}
\begin{eulerprompt}
>function animliss (n,m) ...
\end{eulerprompt}
\begin{eulerudf}
  t=linspace(0,2pi,500);
  f=0;
  c=framecolor(0);
  l=linewidth(2);
  setplot(-1,1,-1,1);
  repeat
    clg;
    plot(sin(n*t),cos(m*t+f));
    wait(0);
    if testkey() then break; endif;
    f=f+0.02;
  end;
  framecolor(c);
  linewidth(l);
  endfunction
\end{eulerudf}
\begin{eulercomment}
Tekan sembarang tombol untuk menghentikan animasi ini.
\end{eulercomment}
\begin{eulerprompt}
>animliss(2,3); // lihat hasilnya, jika sudah puas, tekan ENTER
\end{eulerprompt}
\eulerheading{Plot Logaritmik}
\begin{eulercomment}
EMT menggunakan parameter "logplot" untuk skala logaritmik.\\
Plot logaritmik dapat diplot menggunakan skala logaritmik dalam y
dengan logplot = 1, atau menggunakan skala logaritmik dalam x dan y
dengan logplot = 2, atau dalam x dengan logplot = 3.

\end{eulercomment}
\begin{eulerttcomment}
 - logplot=1: y-logarithmic
 - logplot=2: x-y-logarithmic
 - logplot=3: x-logarithmic
\end{eulerttcomment}
\begin{eulerprompt}
>plot2d("exp(x^3-x)*x^2",1,5,logplot=1):
\end{eulerprompt}
\eulerimg{17}{images/Pekan 5-6_Fanny Erina Dewi_22305141005_EMT00-Plot2D_Aplikom-152.png}
\begin{eulerprompt}
>plot2d("exp(x+sin(x))",0,100,logplot=1):
\end{eulerprompt}
\eulerimg{17}{images/Pekan 5-6_Fanny Erina Dewi_22305141005_EMT00-Plot2D_Aplikom-153.png}
\begin{eulerprompt}
>plot2d("exp(x+sin(x))",10,100,logplot=2):
\end{eulerprompt}
\eulerimg{17}{images/Pekan 5-6_Fanny Erina Dewi_22305141005_EMT00-Plot2D_Aplikom-154.png}
\begin{eulerprompt}
>plot2d("gamma(x)",1,10,logplot=1):
\end{eulerprompt}
\eulerimg{17}{images/Pekan 5-6_Fanny Erina Dewi_22305141005_EMT00-Plot2D_Aplikom-155.png}
\begin{eulerprompt}
>plot2d("log(x*(2+sin(x/100)))",10,1000,logplot=3):
\end{eulerprompt}
\eulerimg{17}{images/Pekan 5-6_Fanny Erina Dewi_22305141005_EMT00-Plot2D_Aplikom-156.png}
\begin{eulercomment}
Hal ini juga berlaku pada plot data
\end{eulercomment}
\begin{eulerprompt}
>x=10^(1:20); y=x^2-x;
>plot2d(x,y,logplot=2):
\end{eulerprompt}
\eulerimg{17}{images/Pekan 5-6_Fanny Erina Dewi_22305141005_EMT00-Plot2D_Aplikom-157.png}
\begin{eulerprompt}
>reset;
\end{eulerprompt}
\begin{eulercomment}
CONTOH CONTOH SOAL PEKAN 5-6\\
1. Persamaan parabola yang didefinisikan dengan\\
\end{eulercomment}
\begin{eulerformula}
\[
x^2+2x+y=4
\]
\end{eulerformula}
\begin{eulercomment}
Tentukan persamaan parametrik dari persamaan tersebut!

Cara:\\
Misalkan x = 2t. Maka:\\
\end{eulercomment}
\begin{eulerformula}
\[
(2t)^2+2(2t)+y=4
\]
\end{eulerformula}
\begin{eulerformula}
\[
4t^4+4t+y=4
\]
\end{eulerformula}
\begin{eulerformula}
\[
y=4-4t-4t^2
\]
\end{eulerformula}
\begin{eulercomment}
Jadi persamaan parametrik dari parabola di atas adalah \\
\end{eulercomment}
\begin{eulerformula}
\[
x=2t, y=4-4t-4t^2
\]
\end{eulerformula}
\begin{eulerprompt}
>t=linspace(0,1,1000);...
>plot2d(t,-t^2-2*t+4):
\end{eulerprompt}
\eulerimg{27}{images/Pekan 5-6_Fanny Erina Dewi_22305141005_EMT00-Plot2D_Aplikom-158.png}
\begin{eulerprompt}
>reset;
\end{eulerprompt}
\begin{eulercomment}
2. Plot dasar\\
buatlah kurva grafik x\textasciicircum{}2-x
\end{eulercomment}
\begin{eulerprompt}
>plot2d("x^2-x");
>xw=100; yw=200; ww=100; hw=50
\end{eulerprompt}
\begin{euleroutput}
  50
\end{euleroutput}
\begin{eulerprompt}
>ow=window();
>window(xw,yw,xw+ww,yw+hw);
>hold on;
>barclear(xw-10,yw-50,ww+0,hw+50);
>plot2d("x^8+x",grid=5):
\end{eulerprompt}
\eulerimg{27}{images/Pekan 5-6_Fanny Erina Dewi_22305141005_EMT00-Plot2D_Aplikom-159.png}
\begin{eulerprompt}
>reset;
\end{eulerprompt}
\begin{eulercomment}
3. Aspek plot
\end{eulercomment}
\begin{eulerprompt}
>aspect(3); // rasio panjang dan lebar 2:1
>plot2d(["cos(2x)","sin(x)"],0,2pi):
\end{eulerprompt}
\eulerimg{8}{images/Pekan 5-6_Fanny Erina Dewi_22305141005_EMT00-Plot2D_Aplikom-160.png}
\begin{eulerprompt}
>aspect();
>reset;
\end{eulerprompt}
\begin{eulercomment}
4. Plot 2D di euler
\end{eulercomment}
\begin{eulerprompt}
>plot2d("x^3-2x",>square):
\end{eulerprompt}
\eulerimg{27}{images/Pekan 5-6_Fanny Erina Dewi_22305141005_EMT00-Plot2D_Aplikom-161.png}
\begin{eulercomment}
5. Interaksi Pengguna
\end{eulercomment}
\begin{eulerprompt}
>plot2d(\{\{"x^3+a*x",a=5\}\},>user,title="Ketik tulian apapun!"):
\end{eulerprompt}
\eulerimg{27}{images/Pekan 5-6_Fanny Erina Dewi_22305141005_EMT00-Plot2D_Aplikom-162.png}
\begin{eulercomment}
6. Fungsi dalam satu Parameter
\end{eulercomment}
\begin{eulerprompt}
>function f(x,a) := x^2/a+a*x^2-x; // mendefinisikan sebuah fungsi
>a=0.3; plot2d("f",0,1;a): // plot dengan a=0.3
\end{eulerprompt}
\eulerimg{27}{images/Pekan 5-6_Fanny Erina Dewi_22305141005_EMT00-Plot2D_Aplikom-163.png}
\begin{eulerprompt}
>reset;
>function f(x) := x^3+2x;...
>plot2d("f",r=1):
\end{eulerprompt}
\eulerimg{27}{images/Pekan 5-6_Fanny Erina Dewi_22305141005_EMT00-Plot2D_Aplikom-164.png}
\begin{eulercomment}
7. Menggambar Beberapa kurva pada bidang koordinat yang sama
\end{eulercomment}
\begin{eulerprompt}
>plot2d("cos(2x)",0,pi); plot2d(2,sin(2),>points,>add):
\end{eulerprompt}
\eulerimg{27}{images/Pekan 5-6_Fanny Erina Dewi_22305141005_EMT00-Plot2D_Aplikom-165.png}
\begin{eulercomment}
8. Label teks
\end{eulercomment}
\begin{eulerprompt}
>plot2d("x^3",0,2,title="y=x^3",yl="y",xl="x"):
\end{eulerprompt}
\eulerimg{27}{images/Pekan 5-6_Fanny Erina Dewi_22305141005_EMT00-Plot2D_Aplikom-166.png}
\begin{eulerprompt}
>reset;
\end{eulerprompt}
\begin{eulercomment}
9. Label teks
\end{eulercomment}
\begin{eulerprompt}
>plot2d("sinc(x)",0,2pi,grid=2,<ticks);...
>ygrid(2:1.5:2,grid=4);
>xgrid([0:2]*pi,<ticks,grid=2);
>xtrick([0,pi,2pi],["0","\(\backslash\)pi","2\(\backslash\)pi"],>latex):
\end{eulerprompt}
\begin{euleroutput}
  Function xtrick not found.
  Try list ... to find functions!
  Error in:
  xtrick([0,pi,2pi],["0","\(\backslash\)pi","2\(\backslash\)pi"],>latex): ...
                                              ^
\end{euleroutput}
\begin{eulerprompt}
>statplot(1:10,random(2,10),color=[red,blue]):
\end{eulerprompt}
\eulerimg{27}{images/Pekan 5-6_Fanny Erina Dewi_22305141005_EMT00-Plot2D_Aplikom-167.png}
\begin{eulerprompt}
>function f(x) &= x^2*exp(-x^2);  ...
>plot2d(&f(x),a=-3,b=3,c=-1,d=1);  ...
>plot2d(&diff(f(x),x),>add,color=blue,style="--"); ...
>labelbox(["function","derivative"],styles=["-","--"], ...
>colors=[black,blue],w=0.4):
\end{eulerprompt}
\eulerimg{27}{images/Pekan 5-6_Fanny Erina Dewi_22305141005_EMT00-Plot2D_Aplikom-168.png}
\eulerheading{Rujukan Lengkap Fungsi plot2d()}
\begin{eulercomment}
\end{eulercomment}
\begin{eulerttcomment}
  function plot2d (xv, yv, btest, a, b, c, d, xmin, xmax, r, n,  ..
  logplot, grid, frame, framecolor, square, color, thickness, style, ..
  auto, add, user, delta, points, addpoints, pointstyle, bar, histogram,  ..
  distribution, even, steps, own, adaptive, hue, level, contour,  ..
  nc, filled, fillcolor, outline, title, xl, yl, maps, contourcolor, ..
  contourwidth, ticks, margin, clipping, cx, cy, insimg, spectral,  ..
  cgrid, vertical, smaller, dl, niveau, levels)
\end{eulerttcomment}
\begin{eulercomment}
Multipurpose plot function for plots in the plane (2D plots). This function can do
plots of functions of one variables, data plots, curves in the plane, bar plots, grids
of complex numbers, and implicit plots of functions of two variables.

Parameters
\\
x,y       : equations, functions or data vectors\\
a,b,c,d   : Plot area (default a=-2,b=2)\\
r         : if r is set, then a=cx-r, b=cx+r, c=cy-r, d=cy+r\\
\end{eulercomment}
\begin{eulerttcomment}
            r can be a vector [rx,ry] or a vector [rx1,rx2,ry1,ry2].
\end{eulerttcomment}
\begin{eulercomment}
xmin,xmax : range of the parameter for curves\\
auto      : Determine y-range automatically (default)\\
square    : if true, try to keep square x-y-ranges\\
n         : number of intervals (default is adaptive)\\
grid      : 0 = no grid and labels,\\
\end{eulercomment}
\begin{eulerttcomment}
            1 = axis only,
            2 = normal grid (see below for the number of grid lines)
            3 = inside axis
            4 = no grid
            5 = full grid including margin
            6 = ticks at the frame
            7 = axis only
            8 = axis only, sub-ticks
\end{eulerttcomment}
\begin{eulercomment}
frame     : 0 = no frame\\
framecolor: color of the frame and the grid\\
margin    : number between 0 and 0.4 for the margin around the plot\\
color     : Color of curves. If this is a vector of colors,\\
\end{eulercomment}
\begin{eulerttcomment}
            it will be used for each row of a matrix of plots. In the case of
            point plots, it should be a column vector. If a row vector or a
            full matrix of colors is used for point plots, it will be used for
            each data point.
\end{eulerttcomment}
\begin{eulercomment}
thickness : line thickness for curves\\
\end{eulercomment}
\begin{eulerttcomment}
            This value can be smaller than 1 for very thin lines.
\end{eulerttcomment}
\begin{eulercomment}
style     : Plot style for lines, markers, and fills.\\
\end{eulercomment}
\begin{eulerttcomment}
            For points use
            "[]", "<>", ".", "..", "...",
            "*", "+", "|", "-", "o"
            "[]#", "<>#", "o#" (filled shapes)
            "[]w", "<>w", "ow" (non-transparent)
            For lines use
            "-", "--", "-.", ".", ".-.", "-.-", "->"
            For filled polygons or bar plots use
            "#", "#O", "O", "/", "\(\backslash\)", "\(\backslash\)/",
            "+", "|", "-", "t"
\end{eulerttcomment}
\begin{eulercomment}
points    : plot single points instead of line segments\\
addpoints : if true, plots line segments and points\\
add       : add the plot to the existing plot\\
user      : enable user interaction for functions\\
delta     : step size for user interaction\\
bar       : bar plot (x are the interval bounds, y the interval values)\\
histogram : plots the frequencies of x in n subintervals\\
distribution=n : plots the distribution of x with n subintervals\\
even      : use inter values for automatic histograms.\\
steps     : plots the function as a step function (steps=1,2)\\
adaptive  : use adaptive plots (n is the minimal number of steps)\\
level     : plot level lines of an implicit function of two variables\\
outline   : draws boundary of level ranges.
\\
If the level value is a 2xn matrix, ranges of levels will be drawn\\
in the color using the given fill style. If outline is true, it\\
will be drawn in the contour color. Using this feature, regions of\\
f(x,y) between limits can be marked.
\\
hue       : add hue color to the level plot to indicate the function\\
\end{eulercomment}
\begin{eulerttcomment}
            value
\end{eulerttcomment}
\begin{eulercomment}
contour   : Use level plot with automatic levels\\
nc        : number of automatic level lines\\
title     : plot title (default "")\\
xl, yl    : labels for the x- and y-axis\\
smaller   : if \textgreater{}0, there will be more space to the left for labels.\\
vertical  :\\
\end{eulercomment}
\begin{eulerttcomment}
  Turns vertical labels on or off. This changes the global variable
  verticallabels locally for one plot. The value 1 sets only vertical
  text, the value 2 uses vertical numerical labels on the y axis.
\end{eulerttcomment}
\begin{eulercomment}
filled    : fill the plot of a curve\\
fillcolor : fill color for bar and filled curves\\
outline   : boundary for filled polygons\\
logplot   : set logarithmic plots\\
\end{eulercomment}
\begin{eulerttcomment}
            1 = logplot in y,
            2 = logplot in xy,
            3 = logplot in x
\end{eulerttcomment}
\begin{eulercomment}
own       :\\
\end{eulercomment}
\begin{eulerttcomment}
  A string, which points to an own plot routine. With >user, you get
  the same user interaction as in plot2d. The range will be set
  before each call to your function.
\end{eulerttcomment}
\begin{eulercomment}
maps      : map expressions (0 is faster), functions are always mapped.\\
contourcolor : color of contour lines\\
contourwidth : width of contour lines\\
clipping  : toggles the clipping (default is true)\\
title     :\\
\end{eulercomment}
\begin{eulerttcomment}
  This can be used to describe the plot. The title will appear above
  the plot. Moreover, a label for the x and y axis can be added with
  xl="string" or yl="string". Other labels can be added with the
  functions label() or labelbox(). The title can be a unicode
  string or an image of a Latex formula.
\end{eulerttcomment}
\begin{eulercomment}
cgrid     :\\
\end{eulercomment}
\begin{eulerttcomment}
  Determines the number of grid lines for plots of complex grids.
  Should be a divisor of the the matrix size minus 1 (number of
  subintervals). cgrid can be a vector [cx,cy].
\end{eulerttcomment}
\begin{eulercomment}

Overview

The function can plot

- expressions, call collections or functions of one variable,\\
- parametric curves,\\
- x data against y data,\\
- implicit functions,\\
- bar plots,\\
- complex grids,\\
- polygons.

If a function or expression for xv is given, plot2d() will compute\\
values in the given range using the function or expression. The\\
expression must be an expression in the variable x. The range must\\
be defined in the parameters a and b unless the default range\\
[-2,2] should be used. The y-range will be computed automatically,\\
unless c and d are specified, or a radius r, which yields the range\\
[-r,r] for x and y. For plots of functions, plot2d will use an\\
adaptive evaluation of the function by default. To speed up the\\
plot for complicated functions, switch this off with \textless{}adaptive, and\\
optionally decrease the number of intervals n. Moreover, plot2d()\\
will by default use mapping. I.e., it will compute the plot element\\
for element. If your expression or your functions can handle a\\
vector x, you can switch that off with \textless{}maps for faster evaluation.

Note that adaptive plots are always computed element for element. \\
If functions or expressions for both xv and for yv are specified,\\
plot2d() will compute a curve with the xv values as x-coordinates\\
and the yv values as y-coordinates. In this case, a range should be\\
defined for the parameter using xmin, xmax. Expressions contained\\
in strings must always be expressions in the parameter variable x.
\end{eulercomment}
\end{eulernotebook}

\chapter{EMT plot 3D}
\begin{eulernotebook}
\eulerheading{Menggambar Plot 3D dengan EMT}
\begin{eulercomment}
Sama seperti plot 2D yang memplot grafik planar, fungsi plot3d memplot
fungsi variabel dan objek lain dalam grafik 3D menggunakan proyeksi
pusat (proyeksi garis lenyap).

Ada beberapa tipe dasar plot 3D berikut ini dalam EMT.

1. Plot padat. Memplot grafik fungsi dalam dua variabel, atau
permukaan yang diakhiri dengan tiga matriks koordinat x-y dan z.
Permukaan dapat memiliki dua sisi warna atau bayangan yang berbeda.
Beberapa jenis menghitung bayangan dengan sumber cahaya, yang lain
dengan koordinat z. Bayangan dapat berupa bayangan dari satu warna,
atau berbagai jenis bayangan spektral.

2. Plot garis. Plot ini hanya menampilkan garis dalam 3D. Garis-garis
tersebut juga dapat membentuk kisi-kisi.

3. Plot titik. Plot ini menunjukkan awan titik-titik di ruang angkasa.

\end{eulercomment}
\eulersubheading{}
\begin{eulercomment}
PLOT DASAR\\
\end{eulercomment}
\eulersubheading{}
\begin{eulercomment}
Plot garis sebuah fungsi dua variabel\\
\end{eulercomment}
\begin{eulerformula}
\[
 G = {(x,y,f(x,y)):a<=x<=b,c<=y<=d}
\]
\end{eulerformula}
\begin{eulerprompt}
>plot3d("x^2+y^2", a=-2, b=1, c=-2, d=1,>user);
\end{eulerprompt}
\begin{eulercomment}
Jenis kedua menunjukkan permukaan dengan rona warna dan garis-garis
yang rata. Rona warna dapat bergantung pada koordinat z dan bisa
berupa warna spektral atau warna sederhana dengan bayangan. Atau,
bayangan dapat bergantung pada jatuhnya cahaya pada plot. Jenis plot
ini juga bisa berisi garis level. Garis level dari level c adalah

\end{eulercomment}
\begin{eulerformula}
\[
Nc={(x,y,f(x,y)):f(x,y)=c}
\]
\end{eulerformula}
\begin{eulercomment}
\end{eulercomment}
\begin{eulerprompt}
>plot3d("y^2-x^2*sin(x)+x/2",>hue,>levels,n=100,...
>hue,cp=1,cpcolor=spectral,cpdelta=0.2,zoom=2.8):
\end{eulerprompt}
\begin{euleroutput}
  Variable or function hue not found.
  Error in:
  hue,cp=1,cpcolor=spectral,cpdelta=0.2,zoom=2.8): ...
     ^
\end{euleroutput}
\begin{eulerprompt}
>aspect(1.5); plot3d("x^2+sin(y)",-5,5,0,6*pi):
\end{eulerprompt}
\eulerimg{17}{images/Pekan 7-8_Fanny Erina Dewi_22305141005_EMT00-Plot3D_Aplikom-001.png}
\begin{eulercomment}
Ini adalah pengenalan plot 3D di Euler. Kita memerlukan plot 3D untuk
memvisualisasikan fungsi dari dua variabel.

Euler menggambar fungsi-fungsi tersebut dengan menggunakan algoritme
pengurutan untuk menyembunyikan bagian-bagian di latar belakang.
Secara umum, Euler menggunakan proyeksi pusat. Standarnya adalah dari
kuadran x-y positif ke arah asal x=y=z=0, tetapi sudut=0 terlihat dari
arah sumbu-y. Sudut pandang dan ketinggian dapat diubah.

Euler dapat merencanakan

-   permukaan dengan bayangan dan garis level atau rentang level,\\
-   awan titik-titik,\\
-   kurva parametrik,\\
-   permukaan implisit.

Plot 3D dari sebuah fungsi menggunakan plot3d. Cara termudah adalah
dengan memplot ekspresi dalam x dan y. Parameter r mengatur rentang
plot di sekitar (0,0).
\end{eulercomment}
\begin{eulerprompt}
>plot3d("x^2+x*sin(y)",-5,5,0,6*pi):
\end{eulerprompt}
\eulerimg{17}{images/Pekan 7-8_Fanny Erina Dewi_22305141005_EMT00-Plot3D_Aplikom-002.png}
\begin{eulercomment}
Silakan lakukan modifikasi agar gambar "talang bergelombang" tersebut tidak lurus melainkan melengkung/melingkar, baik
melingkar secara mendatar maupun melingkar turun/naik (seperti papan peluncur pada kolam renang. Temukan rumusnya.
\end{eulercomment}
\eulerheading{Functions of two Variables}
\begin{eulercomment}
Untuk grafik fungsi, gunakan

-   ekspresi sederhana dalam x dan y,\\
-   nama fungsi dari dua variabell\\
-   atau matriks data.

Standarnya adalah kisi-kisi kawat yang terisi dengan warna yang
berbeda pada kedua sisinya. Perhatikan, bahwa jumlah interval
kisi-kisi default adalah 10, tetapi plot menggunakan jumlah default
40x40 persegi panjang untuk membangun permukaan. Hal ini dapat diubah.

-   n = 40, n = [40,40]: jumlah garis kisi di setiap arah\\
-   grid=10, grid=[10,10]: jumlah garis kisi di setiap arah. Kami
menggunakan default n=40 dan grid=10.
\end{eulercomment}
\begin{eulerprompt}
>plot3d("x^2+y^2"):
\end{eulerprompt}
\eulerimg{17}{images/Pekan 7-8_Fanny Erina Dewi_22305141005_EMT00-Plot3D_Aplikom-003.png}
\begin{eulercomment}
Interaksi pengguna dapat dilakukan dengan parameter \textgreater{}user. Pengguna
dapat menekan tombol berikut ini.

-   kiri, kanan, atas, bawah: putar sudut pandang\\
-   +,-: memperbesar atau memperkecil\\
-   a: menghasilkan anaglyph (lihat di bawah)\\
-   l: sakelar untuk memutar sumber cahaya (lihat di bawah)\\
-   spasi: setel ulang ke default\\
-   kembali: mengakhiri interaksi
\end{eulercomment}
\begin{eulerprompt}
>plot3d("exp(-x^2+y^2)",>user, ...
>  title="Turn with the vector keys (press return to finish)"):
\end{eulerprompt}
\eulerimg{17}{images/Pekan 7-8_Fanny Erina Dewi_22305141005_EMT00-Plot3D_Aplikom-004.png}
\begin{eulercomment}
Rentang plot untuk fungsi dapat ditentukan dengan

-   a, b: rentang x\\
-   c, d: rentang y\\
-   r: bujur sangkar simetris di sekitar (0,0).\\
-   n: jumlah subinterval untuk plot.

Ada beberapa parameter untuk menskalakan fungsi atau mengubah tampilan
grafik.\\
\end{eulercomment}
\begin{eulerttcomment}
 
\end{eulerttcomment}
\begin{eulercomment}
fscale: skala ke nilai fungsi (defaultnya adalah \textless{}fscale).\\
scale: angka atau vektor 1x2 untuk menskalakan ke arah x dan\\
y. \\
frame: jenis frame (default 1).
\end{eulercomment}
\begin{eulerprompt}
>plot3d("exp(-(x^2+y^2)/5)",r=10,n=80,fscale=4,scale=1.2,frame=3,>user):
\end{eulerprompt}
\eulerimg{17}{images/Pekan 7-8_Fanny Erina Dewi_22305141005_EMT00-Plot3D_Aplikom-005.png}
\begin{eulercomment}
Tampilan dapat diubah dengan berbagai cara.

-   jarak: jarak pandang ke plot.\\
-   zoom: nilai zoom.\\
-   sudut: sudut ke sumbu y negatif dalam radian.\\
-   height: ketinggian tampilan dalam radian.

Nilai default dapat diperiksa atau diubah dengan fungsi view(). Fungsi
ini mengembalikan parameter sesuai urutan di atas.
\end{eulercomment}
\begin{eulerprompt}
>view
\end{eulerprompt}
\begin{euleroutput}
  [5,  2.6,  2,  0.4]
\end{euleroutput}
\begin{eulercomment}
Jarak yang lebih dekat membutuhkan zoom yang lebih sedikit. Efeknya
lebih seperti lensa sudut lebar.

Pada contoh berikut ini, sudut = 0 dan tinggi = 0 terlihat dari sumbu
y negatif. Label sumbu untuk y disembunyikan dalam kasus ini.
\end{eulercomment}
\begin{eulerprompt}
>plot3d("x^2+y",distance=3,zoom=1,angle=pi/2,height=0):
\end{eulerprompt}
\eulerimg{17}{images/Pekan 7-8_Fanny Erina Dewi_22305141005_EMT00-Plot3D_Aplikom-006.png}
\begin{eulercomment}
Plot terlihat selalu ke bagian tengah kubus plot. Anda dapat
memindahkan bagian tengah dengan parameter center.

\end{eulercomment}
\begin{eulerprompt}
>plot3d("x^4+y^2",a=0,b=1,c=-1,d=1,angle=-20°,height=20°, ...
>  center=[0.4,0,0],zoom=5):
\end{eulerprompt}
\eulerimg{17}{images/Pekan 7-8_Fanny Erina Dewi_22305141005_EMT00-Plot3D_Aplikom-007.png}
\begin{eulercomment}
Plot diskalakan agar sesuai dengan kubus satuan untuk dilihat. Jadi,
tidak perlu mengubah jarak atau melakukan zoom, tergantung pada ukuran
plot. Namun demikian, label mengacu ke ukuran yang sesungguhnya.

Jika Anda menonaktifkannya dengan scale=false, Anda harus
berhati-hati, agar plot tetap muat ke dalam jendela plotting, dengan
mengubah jarak pandang atau zoom, dan memindahkan bagian tengahnya.
\end{eulercomment}
\begin{eulerprompt}
>plot3d("5*exp(-x^2-y^2)",r=2,<fscale,<scale,distance=13,height=50°, ...
>  center=[0,0,-2],frame=3):
\end{eulerprompt}
\eulerimg{17}{images/Pekan 7-8_Fanny Erina Dewi_22305141005_EMT00-Plot3D_Aplikom-008.png}
\begin{eulercomment}
Plot polar juga tersedia. Parameter polar=true menggambar plot polar.
Fungsi harus tetap merupakan fungsi dari x dan y. Parameter "fscale"
menskalakan fungsi dengan skala sendiri. Jika tidak, fungsi  akan
diskalakan agar sesuai dengan kubus.
\end{eulercomment}
\begin{eulerprompt}
>plot3d("1/(x^2+y^2+1)",r=5,>polar, ...
>fscale=2,>hue,n=100,zoom=4,>contour,color=blue):
\end{eulerprompt}
\eulerimg{17}{images/Pekan 7-8_Fanny Erina Dewi_22305141005_EMT00-Plot3D_Aplikom-009.png}
\begin{eulerprompt}
>function f(r) := exp(-r/2)*cos(r); ...
>plot3d("f(x^2+y^2)",>polar,scale=[1,1,0.4],r=pi,frame=3,zoom=4):
\end{eulerprompt}
\eulerimg{17}{images/Pekan 7-8_Fanny Erina Dewi_22305141005_EMT00-Plot3D_Aplikom-010.png}
\begin{eulercomment}
Parameter rotate memutar fungsi dalam x di sekitar sumbu x.

-   rotate = 1: Menggunakan sumbu x\\
-   rotate = 2: Menggunakan sumbu z
\end{eulercomment}
\begin{eulerprompt}
>plot3d("x^2+1",a=-1,b=1,rotate=true,grid=5):
\end{eulerprompt}
\eulerimg{17}{images/Pekan 7-8_Fanny Erina Dewi_22305141005_EMT00-Plot3D_Aplikom-011.png}
\begin{eulerprompt}
>plot3d("x^2+1",a=-1,b=1,rotate=2,grid=5):
\end{eulerprompt}
\eulerimg{17}{images/Pekan 7-8_Fanny Erina Dewi_22305141005_EMT00-Plot3D_Aplikom-012.png}
\begin{eulerprompt}
>plot3d("sqrt(25-x^2)",a=0,b=5,rotate=1):
\end{eulerprompt}
\eulerimg{17}{images/Pekan 7-8_Fanny Erina Dewi_22305141005_EMT00-Plot3D_Aplikom-013.png}
\begin{eulerprompt}
>plot3d("x*sin(x)",a=0,b=6pi,rotate=2):
\end{eulerprompt}
\eulerimg{17}{images/Pekan 7-8_Fanny Erina Dewi_22305141005_EMT00-Plot3D_Aplikom-014.png}
\begin{eulercomment}
Berikut ini adalah plot dengan tiga fungsi.
\end{eulercomment}
\begin{eulerprompt}
>plot3d("x","x^2+y^2","y",r=2,zoom=3.5,frame=3):
\end{eulerprompt}
\eulerimg{17}{images/Pekan 7-8_Fanny Erina Dewi_22305141005_EMT00-Plot3D_Aplikom-015.png}
\eulerheading{Plot Kontur}
\begin{eulercomment}
Untuk plot, Euler menambahkan garis kisi-kisi. Sebagai gantinya,
dimungkinkan untuk menggunakan garis level dan rona satu warna atau
rona berwarna spektral. Euler dapat menggambar ketinggian fungsi pada
plot dengan bayangan. Di semua plot 3D, Euler dapat menghasilkan
anaglyph merah / cyan.

-   \textgreater{} Rona: Mengaktifkan bayangan cahaya, bukan kabel.\\
-   \textgreater{}kontur: Memplot garis kontur otomatis pada plot.\\
-   level=... (atau level): Vektor nilai untuk garis kontur.

Standarnya adalah level = "auto", yang menghitung beberapa garis level
secara otomatis. Seperti yang Anda lihat dalam plot, level sebenarnya
adalah kisaran level.

Gaya default dapat diubah. Untuk plot kontur berikut ini, kami
menggunakan grid yang lebih halus untuk titik-titik 100x100, skala
fungsi dan plot, dan menggunakan sudut pandang yang berbeda.
\end{eulercomment}
\begin{eulerprompt}
>plot3d("exp(-x^2-y^2)",r=2,n=100,level="thin", ...
> >contour,>spectral,fscale=1,scale=1.1,angle=45°,height=20°):
\end{eulerprompt}
\eulerimg{17}{images/Pekan 7-8_Fanny Erina Dewi_22305141005_EMT00-Plot3D_Aplikom-016.png}
\begin{eulerprompt}
>plot3d("exp(x*y)",angle=100°,>contour,color=green):
\end{eulerprompt}
\eulerimg{17}{images/Pekan 7-8_Fanny Erina Dewi_22305141005_EMT00-Plot3D_Aplikom-017.png}
\begin{eulercomment}
Bayangan default menggunakan warna abu-abu. Tetapi, kisaran warna
spektral juga tersedia.

-   \textgreater{}spektral: Menggunakan skema spektral default\\
-   color =...: Menggunakan warna khusus atau skema spektral

Untuk plot berikut ini, kami menggunakan skema spektral default dan
menambah jumlah titik untuk mendapatkan tampilan yang sangat mulus.
\end{eulercomment}
\begin{eulerprompt}
>plot3d("x^2+y^2",>spectral,>contour,n=100):
\end{eulerprompt}
\eulerimg{17}{images/Pekan 7-8_Fanny Erina Dewi_22305141005_EMT00-Plot3D_Aplikom-018.png}
\begin{eulercomment}
Alih-alih garis level otomatis, kita juga dapat menetapkan nilai garis
level. Hal ini akan menghasilkan garis level yang tipis, alih-alih
rentang level.

\end{eulercomment}
\begin{eulerprompt}
>plot3d("x^2-y^2",0,5,0,5,level=-1:0.1:1,color=redgreen):
\end{eulerprompt}
\eulerimg{17}{images/Pekan 7-8_Fanny Erina Dewi_22305141005_EMT00-Plot3D_Aplikom-019.png}
\begin{eulercomment}
Pada plot berikut, kami menggunakan dua pita level yang sangat luas
dari -0,1 hingga 1, dan dari 0,9 hingga 1. Ini dimasukkan sebagai
matriks dengan batas-batas level sebagai kolom.

Selain itu, kami menghamparkan kisi-kisi dengan 10 interval di setiap
arah.

\end{eulercomment}
\begin{eulerprompt}
>plot3d("x^2+y^3",level=[-0.1,0.9;0,1], ...
>  >spectral,angle=30°,grid=10,contourcolor=gray):
\end{eulerprompt}
\eulerimg{17}{images/Pekan 7-8_Fanny Erina Dewi_22305141005_EMT00-Plot3D_Aplikom-020.png}
\begin{eulercomment}
Pada contoh berikut, kami memplot himpunan, di mana

\end{eulercomment}
\begin{eulerformula}
\[
f(x,y) = x^y-y^x = 0
\]
\end{eulerformula}
\begin{eulercomment}
Kami menggunakan satu garis tipis untuk garis level.

\end{eulercomment}
\begin{eulerprompt}
>plot3d("x^y-y^x",level=0,a=0,b=6,c=0,d=6,contourcolor=red,n=100):
\end{eulerprompt}
\eulerimg{17}{images/Pekan 7-8_Fanny Erina Dewi_22305141005_EMT00-Plot3D_Aplikom-021.png}
\begin{eulercomment}
Dimungkinkan untuk menampilkan bidang kontur di bawah plot. Warna dan
jarak ke plot dapat ditentukan.
\end{eulercomment}
\begin{eulerprompt}
>plot3d("x^2+y^4",>cp,cpcolor=green,cpdelta=0.2):
\end{eulerprompt}
\eulerimg{17}{images/Pekan 7-8_Fanny Erina Dewi_22305141005_EMT00-Plot3D_Aplikom-022.png}
\begin{eulercomment}
Berikut ini beberapa gaya lainnya. Kami selalu mematikan bingkai, dan
menggunakan berbagai skema warna untuk plot dan kisi-kisi.
\end{eulercomment}
\begin{eulerprompt}
>figure(2,2); ...
>expr="y^3-x^2"; ...
>figure(1);  ...
>  plot3d(expr,<frame,>cp,cpcolor=spectral); ...
>figure(2);  ...
>  plot3d(expr,<frame,>spectral,grid=10,cp=2); ...
>figure(3);  ...
>  plot3d(expr,<frame,>contour,color=gray,nc=5,cp=3,cpcolor=greenred); ...
>figure(4);  ...
>  plot3d(expr,<frame,>hue,grid=10,>transparent,>cp,cpcolor=gray); ...
>figure(0):
\end{eulerprompt}
\eulerimg{17}{images/Pekan 7-8_Fanny Erina Dewi_22305141005_EMT00-Plot3D_Aplikom-023.png}
\begin{eulercomment}
Ada beberapa skema spektral lainnya, yang diberi nomor dari 1 hingga
9. Tetapi Anda juga dapat menggunakan color=value, di mana value

-   spektral: untuk rentang dari biru ke merah\\
-   putih: untuk rentang yang lebih redup\\
-   kuningbiru, ungu-hijau, biru-kuning, hijau-merah\\
-   biru-kuning, hijau-ungu, kuning-biru, merah-hijau
\end{eulercomment}
\begin{eulerprompt}
>figure(3,3); ...
>for i=1:9;  ...
>  figure(i); plot3d("x^2+y^2",spectral=i,>contour,>cp,<frame,zoom=4);  ...
>end; ...
>figure(0):
\end{eulerprompt}
\eulerimg{17}{images/Pekan 7-8_Fanny Erina Dewi_22305141005_EMT00-Plot3D_Aplikom-024.png}
\begin{eulercomment}
Sumber cahaya dapat diubah dengan l dan tombol kursor selama interaksi
pengguna. Ini juga dapat ditetapkan dengan parameter.

-   cahaya: arah untuk cahaya\\
-   amb: cahaya sekitar antara 0 dan 1

Perhatikan, bahwa program ini tidak membuat perbedaan di antara
sisi-sisi plot. Tidak ada bayangan. Untuk ini, Anda memerlukan Povray.
\end{eulercomment}
\begin{eulerprompt}
>plot3d("-x^2-y^2", ...
>  hue=true,light=[0,1,1],amb=0,user=true, ...
>  title="Press l and cursor keys (return to exit)"):
\end{eulerprompt}
\eulerimg{17}{images/Pekan 7-8_Fanny Erina Dewi_22305141005_EMT00-Plot3D_Aplikom-025.png}
\begin{eulercomment}
Parameter warna mengubah warna permukaan. Warna garis level juga dapat
diubah.
\end{eulercomment}
\begin{eulerprompt}
>plot3d("-x^2-y^2",color=rgb(0.2,0.2,0),hue=true,frame=false, ...
>  zoom=3,contourcolor=red,level=-2:0.1:1,dl=0.01):
\end{eulerprompt}
\eulerimg{17}{images/Pekan 7-8_Fanny Erina Dewi_22305141005_EMT00-Plot3D_Aplikom-026.png}
\begin{eulercomment}
Warna 0 memberikan efek pelangi yang istimewa.
\end{eulercomment}
\begin{eulerprompt}
>plot3d("x^2/(x^2+y^2+1)",color=0,hue=true,grid=10):
\end{eulerprompt}
\eulerimg{17}{images/Pekan 7-8_Fanny Erina Dewi_22305141005_EMT00-Plot3D_Aplikom-027.png}
\begin{eulercomment}
Permukaannya juga bisa transparan.
\end{eulercomment}
\begin{eulerprompt}
>plot3d("x^2+y^2",>transparent,grid=10,wirecolor=red):
\end{eulerprompt}
\eulerimg{17}{images/Pekan 7-8_Fanny Erina Dewi_22305141005_EMT00-Plot3D_Aplikom-028.png}
\eulerheading{Plot Implisit}
\begin{eulercomment}
Ada juga plot implisit dalam tiga dimensi. Euler menghasilkan potongan
melalui objek. Fitur plot3d termasuk plot implisit. Plot-plot ini
menunjukkan himpunan nol dari sebuah fungsi dalam tiga variabel.
Solusi dari

\end{eulercomment}
\begin{eulerformula}
\[
f(x,y,z) = 0
\]
\end{eulerformula}
\begin{eulercomment}
dapat divisualisasikan dalam potongan yang sejajar dengan bidang x-y,
bidang x-z dan bidang y-z.

-   implisit = 1: potong sejajar dengan bidang y-z\\
-   implicit=2: potong sejajar dengan bidang x-z\\
-   implicit = 4: potong sejajar dengan bidang x-y

Tambahkan nilai-nilai ini, jika Anda mau. Dalam contoh, kami memplot

\end{eulercomment}
\begin{eulerformula}
\[
M = \{ (x,y,z) : x^2+y^3+zy=1 \}
\]
\end{eulerformula}
\begin{eulerprompt}
>plot2d("2*x^2+y^2+x*y+x+2*y",r=3,levels=[1;2],...
>style = "/", color=green, grid=6):
\end{eulerprompt}
\eulerimg{17}{images/Pekan 7-8_Fanny Erina Dewi_22305141005_EMT00-Plot3D_Aplikom-029.png}
\begin{eulerprompt}
>sytle = "/", color= green, grid =6):
\end{eulerprompt}
\begin{euleroutput}
  /
  3
  Found too many closing brackets, excessive )
  Space between commands expected!
  Found: ): (character 41)
  You can disable this in the Options menu.
  Error in:
  sytle = "/", color= green, grid =6): ...
                                    ^
\end{euleroutput}
\begin{eulerprompt}
>plot3d("x^2+y^3+z*y-1",r=5,implicit=3):
\end{eulerprompt}
\eulerimg{17}{images/Pekan 7-8_Fanny Erina Dewi_22305141005_EMT00-Plot3D_Aplikom-030.png}
\begin{eulerprompt}
>c=1; d=1;
>plot3d("((x^2+y^2-c^2)^2+(z^2-1)^2)*((y^2+z^2-c^2)^2+(x^2-1)^2)*((z^2+x^2-c^2)^2+(y^2-1)^2)-d",r=2,<frame,>implicit,>user): 
\end{eulerprompt}
\eulerimg{17}{images/Pekan 7-8_Fanny Erina Dewi_22305141005_EMT00-Plot3D_Aplikom-031.png}
\begin{eulerprompt}
>plot3d("x^2+y^2+4*x*z+z^3",>implicit,r=2,zoom=2.5):
\end{eulerprompt}
\eulerimg{17}{images/Pekan 7-8_Fanny Erina Dewi_22305141005_EMT00-Plot3D_Aplikom-032.png}
\begin{eulercomment}
Contoh \\
Selidiki fungsi f(x,y)=x\textasciicircum{}y y\textasciicircum{}x untuk x;y\textgreater{}0. Pertama, kira memplot
solusi dari persamaan x\textasciicircum{}y=y\textasciicircum{}x dalam rentang 0 x;y 5

\end{eulercomment}
\begin{eulerprompt}
>fungsi f(x,y) := x^y-y^x;
\end{eulerprompt}
\begin{euleroutput}
  Variable fungsi not found!
  Error in:
  fungsi f(x,y) := x^y-y^x; ...
         ^
\end{euleroutput}
\begin{eulerprompt}
>plot2d("f",a=0,b=5,c=0,d=5,n=100, ...
>level=0,>hue,>spectral,contourcolor=red,contourwidth=3):
\end{eulerprompt}
\begin{euleroutput}
  Function f needs only 1 arguments (got 2)!
  Use: f (r) 
  Error in map.
  %ploteval2:
      return map(f$,x,y;args());
  fcontour:
      Z=%ploteval2(f$,X,Y,maps;args());
  Try "trace errors" to inspect local variables after errors.
  plot2d:
      =style,=outline,=frame);
\end{euleroutput}
\begin{eulerprompt}
> 
\end{eulerprompt}
\eulerheading{Memplot Data 3D}
\begin{eulercomment}
Sama seperti plot2d, plot3d menerima data. Untuk objek 3D, Anda perlu
menyediakan matriks nilai x, y, dan z, atau tiga fungsi atau ekspresi
fx(x,y), fy(x,y), fz(x,y).

\end{eulercomment}
\begin{eulerformula}
\[
\gamma(t,s) = (x(t,s),y(t,s),z(t,s))
\]
\end{eulerformula}
\begin{eulercomment}
Karena x, y, z adalah matriks, kita asumsikan bahwa (t, s) berjalan
melalui kisi-kisi persegi. Hasilnya, Anda dapat memplot gambar persegi
panjang dalam ruang.

Anda dapat menggunakan bahasa matriks Euler untuk menghasilkan
koordinat secara efektif.

Pada contoh berikut, kita menggunakan vektor nilai t dan vektor kolom
nilai s untuk memparameterkan permukaan bola. Pada gambar kita dapat
menandai daerah, dalam kasus kita daerah kutub.
\end{eulercomment}
\begin{eulerprompt}
>t=linspace(0,2pi,180); s=linspace(-pi/2,pi/2,90)'; ...
>x=cos(s)*cos(t); y=cos(s)*sin(t); z=sin(s); ...
>plot3d(x,y,z,>hue, ...
>color=blue,<frame,grid=[10,20], ...
>values=s,contourcolor=red,level=[90°-24°;90°-22°], ...
>scale=1.4,height=50°):
\end{eulerprompt}
\eulerimg{17}{images/Pekan 7-8_Fanny Erina Dewi_22305141005_EMT00-Plot3D_Aplikom-033.png}
\begin{eulercomment}
Berikut ini adalah contoh, yang merupakan grafik suatu fungsi.
\end{eulercomment}
\begin{eulerprompt}
>t=-1:0.1:1; s=(-1:0.1:1)'; plot3d(t,s,t*s,grid=10):
\end{eulerprompt}
\eulerimg{17}{images/Pekan 7-8_Fanny Erina Dewi_22305141005_EMT00-Plot3D_Aplikom-034.png}
\begin{eulercomment}
Namun demikian, kita bisa membuat segala macam permukaan. Berikut ini
adalah permukaan yang sama dengan suatu fungsi

\end{eulercomment}
\begin{eulerformula}
\[
x = y \, z
\]
\end{eulerformula}
\begin{eulerprompt}
>plot3d(t*s,t,s,angle=180°,grid=10):
\end{eulerprompt}
\eulerimg{17}{images/Pekan 7-8_Fanny Erina Dewi_22305141005_EMT00-Plot3D_Aplikom-035.png}
\begin{eulercomment}
Dengan lebih banyak upaya, kita bisa menghasilkan banyak permukaan.

Dalam contoh berikut ini, kami membuat tampilan berbayang dari bola
yang terdistorsi. Koordinat biasa untuk bola adalah

\end{eulercomment}
\begin{eulerformula}
\[
\gamma(t,s) = (\cos(t)\cos(s),\sin(t)\sin(s),\cos(s))
\]
\end{eulerformula}
\begin{eulercomment}
dengan

\end{eulercomment}
\begin{eulerformula}
\[
0 \le t \le 2\pi, \quad \frac{-\pi}{2} \le s \le \frac{\pi}{2}.
\]
\end{eulerformula}
\begin{eulercomment}
Kami mengurangi hal ini dengan faktor

\end{eulercomment}
\begin{eulerformula}
\[
d(t,s) = \frac{\cos(4t)+\cos(8s)}{4}.
\]
\end{eulerformula}
\begin{eulerprompt}
>t=linspace(0,2pi,320); s=linspace(-pi/2,pi/2,160)'; ...
>d=1+0.2*(cos(4*t)+cos(8*s)); ...
>plot3d(cos(t)*cos(s)*d,sin(t)*cos(s)*d,sin(s)*d,hue=1, ...
>  light=[1,0,1],frame=0,zoom=5):
\end{eulerprompt}
\eulerimg{17}{images/Pekan 7-8_Fanny Erina Dewi_22305141005_EMT00-Plot3D_Aplikom-036.png}
\begin{eulercomment}
Tentu saja, awan titik juga dimungkinkan. Untuk memplot data titik
dalam ruang, kita memerlukan tiga vektor untuk koordinat titik.

Gaya sama seperti di plot2d dengan poin=true;

\end{eulercomment}
\begin{eulerprompt}
>n=500;  ...
>  plot3d(normal(1,n),normal(1,n),normal(1,n),points=true,style="."):
\end{eulerprompt}
\eulerimg{17}{images/Pekan 7-8_Fanny Erina Dewi_22305141005_EMT00-Plot3D_Aplikom-037.png}
\begin{eulercomment}
Anda juga dapat memplot kurva dalam bentuk 3D. Dalam hal ini, akan
lebih mudah untuk menghitung titik-titik kurva. Untuk kurva pada
bidang, kami menggunakan urutan koordinat dan parameter wire = true.
\end{eulercomment}
\begin{eulerprompt}
>t=linspace(0,8pi,500); ...
>plot3d(sin(t),cos(t),t/10,>wire,zoom=3):
\end{eulerprompt}
\eulerimg{17}{images/Pekan 7-8_Fanny Erina Dewi_22305141005_EMT00-Plot3D_Aplikom-038.png}
\begin{eulerprompt}
>t=linspace(0,4pi,1000); plot3d(cos(t),sin(t),t/2pi,>wire, ...
>linewidth=3,wirecolor=blue):
\end{eulerprompt}
\eulerimg{17}{images/Pekan 7-8_Fanny Erina Dewi_22305141005_EMT00-Plot3D_Aplikom-039.png}
\begin{eulerprompt}
>X=cumsum(normal(3,100)); ...
> plot3d(X[1],X[2],X[3],>anaglyph,>wire):
\end{eulerprompt}
\eulerimg{17}{images/Pekan 7-8_Fanny Erina Dewi_22305141005_EMT00-Plot3D_Aplikom-040.png}
\begin{eulercomment}
EMT juga dapat membuat plot dalam mode anaglyph. Untuk melihat plot
semacam itu, Anda memerlukan kacamata merah/cyan.
\end{eulercomment}
\begin{eulerprompt}
> plot3d("x^2+y^3",>anaglyph,>contour,angle=30°):
\end{eulerprompt}
\eulerimg{17}{images/Pekan 7-8_Fanny Erina Dewi_22305141005_EMT00-Plot3D_Aplikom-041.png}
\begin{eulercomment}
Sering kali, skema warna spektral digunakan untuk plot. Hal ini
menekankan ketinggian fungsi.
\end{eulercomment}
\begin{eulerprompt}
>plot3d("x^2*y^3-y",>spectral,>contour,zoom=3.2):
\end{eulerprompt}
\eulerimg{17}{images/Pekan 7-8_Fanny Erina Dewi_22305141005_EMT00-Plot3D_Aplikom-042.png}
\begin{eulercomment}
Euler juga dapat memplot permukaan yang diparameterkan, apabila
parameternya adalah nilai x, y, dan z dari gambar kisi-kisi persegi
panjang di dalam ruang.

Untuk demo berikut ini, kita akan menyiapkan parameter u dan v, dan
menghasilkan koordinat ruang dari parameter ini.
\end{eulercomment}
\begin{eulerprompt}
>u=linspace(-1,1,10); v=linspace(0,2*pi,50)'; ...
>X=(3+u*cos(v/2))*cos(v); Y=(3+u*cos(v/2))*sin(v); Z=u*sin(v/2); ...
>plot3d(X,Y,Z,>anaglyph,<frame,>wire,scale=2.3):
\end{eulerprompt}
\eulerimg{17}{images/Pekan 7-8_Fanny Erina Dewi_22305141005_EMT00-Plot3D_Aplikom-043.png}
\begin{eulercomment}
Berikut ini contoh yang lebih rumit, yang tampak megah dengan kacamata
merah/cyan.
\end{eulercomment}
\begin{eulerprompt}
>u:=linspace(-pi,pi,160); v:=linspace(-pi,pi,400)';  ...
>x:=(4*(1+.25*sin(3*v))+cos(u))*cos(2*v); ...
>y:=(4*(1+.25*sin(3*v))+cos(u))*sin(2*v); ...
> z=sin(u)+2*cos(3*v); ...
>plot3d(x,y,z,frame=0,scale=1.5,hue=1,light=[1,0,-1],zoom=2.8,>anaglyph):
\end{eulerprompt}
\eulerimg{17}{images/Pekan 7-8_Fanny Erina Dewi_22305141005_EMT00-Plot3D_Aplikom-044.png}
\eulerheading{Plot Statistik}
\begin{eulercomment}
Petak batang juga dimungkinkan. Untuk ini, kita harus menyediakan

-   x: vektor baris dengan n+1 elemen\\
-   y: vektor kolom dengan n+1 elemen\\
-   z: matriks nilai nxn.

z dapat lebih besar, tetapi hanya nilai nxn yang akan digunakan.

Dalam contoh, pertama-tama kita menghitung nilainya. Kemudian kita
menyesuaikan x dan y, sehingga vektor berpusat pada nilai yang
digunakan.
\end{eulercomment}
\begin{eulerprompt}
>x=-1:0.1:1; y=x'; z=x^2+y^2; ...
>xa=(x|1.1)-0.05; ya=(y_1.1)-0.05; ...
>plot3d(xa,ya,z,bar=true):
\end{eulerprompt}
\eulerimg{17}{images/Pekan 7-8_Fanny Erina Dewi_22305141005_EMT00-Plot3D_Aplikom-045.png}
\begin{eulercomment}
Hal ini memungkinkan untuk membagi plot permukaan menjadi dua bagian
atau lebih.
\end{eulercomment}
\begin{eulerprompt}
>x=-1:0.1:1; y=x'; z=x+y; d=zeros(size(x)); ...
>plot3d(x,y,z,disconnect=2:2:20):
\end{eulerprompt}
\eulerimg{17}{images/Pekan 7-8_Fanny Erina Dewi_22305141005_EMT00-Plot3D_Aplikom-046.png}
\begin{eulercomment}
Jika memuat atau menghasilkan matriks data M dari file dan perlu
memplotnya dalam 3D, Anda dapat menskalakan matriks ke [-1,1] dengan
scale(M), atau menskalakan matriks dengan \textgreater{}zscale. Hal ini dapat
dikombinasikan dengan faktor penskalaan individual yang diterapkan
sebagai tambahan.
\end{eulercomment}
\begin{eulerprompt}
>i=1:20; j=i'; ...
>plot3d(i*j^2+100*normal(20,20),>zscale,scale=[1,1,1.5],angle=-40°,zoom=1.8):
\end{eulerprompt}
\eulerimg{17}{images/Pekan 7-8_Fanny Erina Dewi_22305141005_EMT00-Plot3D_Aplikom-047.png}
\begin{eulerprompt}
>Z=intrandom(5,100,6); v=zeros(5,6); ...
>loop 1 to 5; v[#]=getmultiplicities(1:6,Z[#]); end; ...
>columnsplot3d(v',scols=1:5,ccols=[1:5]):
\end{eulerprompt}
\eulerimg{17}{images/Pekan 7-8_Fanny Erina Dewi_22305141005_EMT00-Plot3D_Aplikom-048.png}
\eulerheading{Permukaan Benda Putar}
\begin{eulerprompt}
>plot2d("(x^2+y^2-1)^3-x^2*y^3",r=1.3, ...
>style="#",color=red,<outline, ...
>level=[-2;0],n=100):
\end{eulerprompt}
\eulerimg{17}{images/Pekan 7-8_Fanny Erina Dewi_22305141005_EMT00-Plot3D_Aplikom-049.png}
\begin{eulerprompt}
>ekspresi &= (x^2+y^2-1)^3-x^2*y^3; $ekspresi
\end{eulerprompt}
\begin{eulerformula}
\[
\left(y^2+x^2-1\right)^3-x^2\,y^3
\]
\end{eulerformula}
\begin{eulercomment}
Kami ingin memutar kurva jantung di sekitar sumbu y. Inilah ekspresi
yang mendefinisikan jantung:

\end{eulercomment}
\begin{eulerformula}
\[
f(x,y)=(x^2+y^2-1)^3-x^2.y^3.
\]
\end{eulerformula}
\begin{eulercomment}
Selanjutnya kami menetapkan

\end{eulercomment}
\begin{eulerformula}
\[
x=r.cos(a),\quad y=r.sin(a).
\]
\end{eulerformula}
\begin{eulerprompt}
>function fr(r,a) &= ekspresi with [x=r*cos(a),y=r*sin(a)] | trigreduce; $fr(r,a)
\end{eulerprompt}
\begin{eulerformula}
\[
\left(r^2-1\right)^3+\frac{\left(\sin \left(5\,a\right)-\sin \left(
 3\,a\right)-2\,\sin a\right)\,r^5}{16}
\]
\end{eulerformula}
\begin{eulercomment}
Hal ini memungkinkan untuk mendefinisikan fungsi numerik, yang
menyelesaikan untuk r, jika a diberikan. Dengan fungsi tersebut kita
dapat memplotkan jantung yang diputar sebagai permukaan parametrik.
\end{eulercomment}
\begin{eulerprompt}
>function map f(a) := bisect("fr",0,2;a); ...
>t=linspace(-pi/2,pi/2,100); r=f(t);  ...
>s=linspace(pi,2pi,100)'; ...
>plot3d(r*cos(t)*sin(s),r*cos(t)*cos(s),r*sin(t), ...
>>hue,<frame,color=red,zoom=4,amb=0,max=0.7,grid=12,height=50°):
\end{eulerprompt}
\eulerimg{17}{images/Pekan 7-8_Fanny Erina Dewi_22305141005_EMT00-Plot3D_Aplikom-052.png}
\begin{eulercomment}
Berikut ini adalah plot 3D dari gambar di atas yang diputar
mengelilingi sumbu-z. Kami mendefinisikan fungsi, yang menggambarkan
objek.

\end{eulercomment}
\begin{eulerprompt}
>function f(x,y,z) ...
\end{eulerprompt}
\begin{eulerudf}
  r=x^2+y^2;
  return (r+z^2-1)^3-r*z^3;
   endfunction
\end{eulerudf}
\begin{eulerprompt}
>plot3d("f(x,y,z)", ...
>xmin=0,xmax=1.2,ymin=-1.2,ymax=1.2,zmin=-1.2,zmax=1.4, ...
>implicit=1,angle=-30°,zoom=2.5,n=[10,100,60],>anaglyph):
\end{eulerprompt}
\eulerimg{17}{images/Pekan 7-8_Fanny Erina Dewi_22305141005_EMT00-Plot3D_Aplikom-053.png}
\eulerheading{Plot 3D Khusus}
\begin{eulercomment}
Fungsi plot3d memang bagus untuk dimiliki, tetapi tidak memenuhi semua
kebutuhan. Di samping rutinitas yang lebih mendasar, Anda bisa
mendapatkan plot berbingkai dari objek apa pun yang Anda sukai.

Meskipun Euler bukan program 3D, namun dapat menggabungkan beberapa
objek dasar. Kami mencoba memvisualisasikan parabola dan garis
singgungnya.

\end{eulercomment}
\begin{eulerprompt}
>function myplot ...
\end{eulerprompt}
\begin{eulerudf}
    y=-1:0.01:1; x=(-1:0.01:1)';
    plot3d(x,y,0.2*(x-0.1)/2,<scale,<frame,>hue, ..
      hues=0.5,>contour,color=orange);
    h=holding(1);
    plot3d(x,y,(x^2+y^2)/2,<scale,<frame,>contour,>hue);
    holding(h);
  endfunction
\end{eulerudf}
\begin{eulercomment}
Sekarang framedplot() menyediakan frame, dan mengatur tampilan.

\end{eulercomment}
\begin{eulerprompt}
>framedplot("myplot",[-1,1,-1,1,0,1],height=0,angle=-30°, ...
>  center=[0,0,-0.7],zoom=3):
\end{eulerprompt}
\eulerimg{17}{images/Pekan 7-8_Fanny Erina Dewi_22305141005_EMT00-Plot3D_Aplikom-054.png}
\begin{eulercomment}
Dengan cara yang sama, Anda dapat memplot bidang kontur secara manual.
Perhatikan bahwa plot3d() mengatur jendela ke fullwindow() secara
default, namun plotcontourplane() mengasumsikannya.
\end{eulercomment}
\begin{eulerprompt}
>x=-1:0.02:1.1; y=x'; z=x^2-y^4;
>function myplot (x,y,z) ...
\end{eulerprompt}
\begin{eulerudf}
    zoom(2);
    wi=fullwindow();
    plotcontourplane(x,y,z,level="auto",<scale);
    plot3d(x,y,z,>hue,<scale,>add,color=white,level="thin");
    window(wi);
    reset();
  endfunction
\end{eulerudf}
\begin{eulerprompt}
>myplot(x,y,z):
\end{eulerprompt}
\eulerimg{27}{images/Pekan 7-8_Fanny Erina Dewi_22305141005_EMT00-Plot3D_Aplikom-055.png}
\eulerheading{Animasi}
\begin{eulercomment}
Euler dapat menggunakan frame untuk melakukan pra-komputasi animasi.

Salah satu fungsi yang memanfaatkan teknik ini adalah rotate. Fungsi
ini dapat mengubah sudut pandang dan menggambar ulang plot 3D. Fungsi
ini memanggil addpage() untuk setiap plot baru. Terakhir, fungsi ini
menganimasikan plot-plot tersebut.

Silakan pelajari sumber rotasi untuk mengetahui detail selengkapnya.

\end{eulercomment}
\begin{eulerprompt}
>function testplot () := plot3d("x^2+y^3"); ...
>rotate("testplot"); testplot():
\end{eulerprompt}
\eulerimg{27}{images/Pekan 7-8_Fanny Erina Dewi_22305141005_EMT00-Plot3D_Aplikom-056.png}
\eulerheading{Menggambar Povray}
\begin{eulercomment}
Dengan bantuan file Euler povray.e, Euler dapat menghasilkan file
Povray. Hasilnya sangat bagus untuk dilihat.

Anda perlu menginstal Povray (32bit atau 64bit) dari
http://www.povray.org/, dan meletakkan sub- direktori "bin" dari Povray ke dalam jalur lingkungan, atau mengatur variabel "defaultpovray" dengan jalur penuh yang mengarah ke "pvengine.exe".

Antarmuka Povray dari Euler menghasilkan file Povray di direktori home
pengguna, dan memanggil Povray untuk mengurai file-file ini. Nama file
default adalah current.pov, dan direktori defaultnya adalah
eulerhome(), biasanya c:\textbackslash{}Users\textbackslash{}Username\textbackslash{}Euler. Povray menghasilkan
sebuah file PNG, yang dapat dimuat oleh Euler ke dalam notebook. Untuk
membersihkan berkas-berkas ini, gunakan povclear().

Fungsi pov3d memiliki semangat yang sama dengan plot3d. Fungsi ini
dapat menghasilkan grafik fungsi f(x,y), atau permukaan dengan
koordinat X,Y,Z dalam matriks, termasuk garis level opsional. Fungsi
ini memulai raytracer secara otomatis, dan memuat adegan ke dalam
notebook Euler.

Selain pov3d(), ada banyak fungsi yang menghasilkan objek Povray.
Fungsi-fungsi ini mengembalikan string, yang berisi kode Povray untuk
objek. Untuk menggunakan fungsi-fungsi ini, mulai file Povray dengan
povstart(). Kemudian gunakan writeln(...) untuk menulis objek ke file
scene. Terakhir, akhiri file dengan povend(). Secara default,
raytracer akan dimulai, dan PNG akan dimasukkan ke dalam notebook
Euler.

Fungsi objek memiliki parameter yang disebut "look", yang membutuhkan
string dengan kode Povray untuk tekstur dan hasil akhir objek. Fungsi
povlook() dapat digunakan untuk menghasilkan string ini. Fungsi ini
memiliki parameter untuk warna, transparansi, Phong Shading, dll.

Perhatikan bahwa alam semesta Povray memiliki sistem koordinat lain.
Antarmuka ini menerjemahkan semua koordinat ke sistem Povray. Jadi,
Anda dapat terus berpikir dalam sistem koordinat Euler dengan z
menunjuk vertikal ke atas, dan sumbu x, y, z di tangan kanan.

Anda perlu memuat file povray.

\end{eulercomment}
\begin{eulerprompt}
>load povray;
\end{eulerprompt}
\begin{eulercomment}
Pastikan direktori bin povray berada di dalam path. Jika tidak, edit
variabel berikut sehingga berisi jalur ke povray yang dapat
dieksekusi.
\end{eulercomment}
\begin{eulerprompt}
>defaultpovray="C:\(\backslash\)Program Files\(\backslash\)POV-Ray\(\backslash\)v3.7\(\backslash\)bin\(\backslash\)pvengine.exe"
\end{eulerprompt}
\begin{euleroutput}
  C:\(\backslash\)Program Files\(\backslash\)POV-Ray\(\backslash\)v3.7\(\backslash\)bin\(\backslash\)pvengine.exe
\end{euleroutput}
\begin{eulercomment}
Untuk kesan pertama, kita plot sebuah fungsi sederhana. Perintah
berikut ini menghasilkan file povray di direktori pengguna Anda, dan
menjalankan Povray untuk melacak sinar pada file ini.

Jika Anda menjalankan perintah berikut, GUI Povray akan membuka,
menjalankan file, dan menutup secara otomatis. Karena alasan keamanan,
Anda akan ditanya apakah Anda ingin mengizinkan file exe untuk
dijalankan. Anda dapat menekan cancel untuk menghentikan pertanyaan
lebih lanjut. Anda mungkin harus menekan OK pada jendela Povray untuk
mengetahui dialog awal Povray.

\end{eulercomment}
\begin{eulerprompt}
>plot3d("x^2+y^2",zoom=2):
\end{eulerprompt}
\eulerimg{27}{images/Pekan 7-8_Fanny Erina Dewi_22305141005_EMT00-Plot3D_Aplikom-057.png}
\begin{eulerprompt}
>pov3d("x^2+y^2",zoom=3);
\end{eulerprompt}
\eulerimg{28}{images/Pekan 7-8_Fanny Erina Dewi_22305141005_EMT00-Plot3D_Aplikom-058.png}
\begin{eulercomment}
Kita dapat membuat fungsi menjadi transparan dan menambahkan hasil
akhir lainnya. Kita juga dapat menambahkan garis level ke plot fungsi.

\end{eulercomment}
\begin{eulerprompt}
>pov3d("x^2+y^3",axiscolor=red,angle=-45°,>anaglyph, ...
>  look=povlook(cyan,0.2),level=-1:0.5:1,zoom=3.8);
\end{eulerprompt}
\eulerimg{27}{images/Pekan 7-8_Fanny Erina Dewi_22305141005_EMT00-Plot3D_Aplikom-059.png}
\begin{eulercomment}
Kadang-kadang perlu untuk mencegah penskalaan fungsi, dan menskalakan
fungsi dengan tangan.

Kami memplot kumpulan titik pada bidang kompleks, di mana hasil kali
jarak ke 1 dan -1 sama dengan 1.
\end{eulercomment}
\begin{eulerprompt}
>pov3d("((x-1)^2+y^2)*((x+1)^2+y^2)/40",r=2, ...
>  angle=-120°,level=1/40,dlevel=0.005,light=[-1,1,1],height=10°,n=50, ...
>  <fscale,zoom=3.8);
\end{eulerprompt}
\eulerimg{28}{images/Pekan 7-8_Fanny Erina Dewi_22305141005_EMT00-Plot3D_Aplikom-060.png}
\eulerheading{Merencanakan dengan Koordinat}
\begin{eulercomment}
Alih-alih menggunakan fungsi, kita dapat membuat plot dengan
koordinat. Seperti pada plot3d, kita memerlukan tiga matriks untuk
mendefinisikan objek.

Dalam contoh, kita memutar fungsi di sekitar sumbu z.
\end{eulercomment}
\begin{eulerprompt}
>function f(x) := x^3-x+1; ...
>x=-1:0.01:1; t=linspace(0,2pi,50)'; ...
>Z=x; X=cos(t)*f(x); Y=sin(t)*f(x); ...
>pov3d(X,Y,Z,angle=40°,look=povlook(red,0.1),height=50°,axis=0,zoom=4,light=[10,5,15]);
\end{eulerprompt}
\eulerimg{28}{images/Pekan 7-8_Fanny Erina Dewi_22305141005_EMT00-Plot3D_Aplikom-061.png}
\begin{eulercomment}
Pada contoh berikut, kita memplot gelombang teredam. Kita menghasilkan
gelombang dengan bahasa matriks Euler.

Kami juga menunjukkan, bagaimana objek tambahan dapat ditambahkan ke
adegan pov3d. Untuk pembuatan objek, lihat contoh berikut. Perhatikan
bahwa plot3d menskalakan plot, sehingga sesuai dengan kubus satuan.
\end{eulercomment}
\begin{eulerprompt}
>r=linspace(0,1,80); phi=linspace(0,2pi,80)'; ...
>x=r*cos(phi); y=r*sin(phi); z=exp(-5*r)*cos(8*pi*r)/3;  ...
>pov3d(x,y,z,zoom=6,axis=0,height=30°,add=povsphere([0.5,0,0.25],0.15,povlook(red)), ...
>  w=500,h=300);
\end{eulerprompt}
\eulerimg{16}{images/Pekan 7-8_Fanny Erina Dewi_22305141005_EMT00-Plot3D_Aplikom-062.png}
\begin{eulercomment}
Dengan metode bayangan canggih Povray, hanya sedikit titik yang bisa
menghasilkan permukaan yang sangat halus. Hanya pada batas-batas dan
bayangan, trik ini bisa terlihat jelas.

Untuk itu, kita perlu menambahkan vektor normal di setiap titik
matriks.

\end{eulercomment}
\begin{eulerprompt}
>Z &= x^2*y^3
\end{eulerprompt}
\begin{euleroutput}
  
                                   2  3
                                  x  y
  
\end{euleroutput}
\begin{eulercomment}
Persamaan permukaannya adalah [x,y,Z]. Kami menghitung dua turunan
terhadap x dan y dari persamaan ini dan mengambil hasil perkalian
silang sebagai normal.
\end{eulercomment}
\begin{eulerprompt}
>dx &= diff([x,y,Z],x); dy &= diff([x,y,Z],y);
\end{eulerprompt}
\begin{eulercomment}
Kami mendefinisikan normal sebagai hasil kali silang dari turunan ini,
dan mendefinisikan fungsi koordinat.
\end{eulercomment}
\begin{eulerprompt}
>N &= crossproduct(dx,dy); NX &= N[1]; NY &= N[2]; NZ &= N[3]; N,
\end{eulerprompt}
\begin{euleroutput}
  
                                 3       2  2
                         [- 2 x y , - 3 x  y , 1]
  
\end{euleroutput}
\begin{eulercomment}
Kami hanya menggunakan 25 poin.
\end{eulercomment}
\begin{eulerprompt}
>x=-1:0.5:1; y=x';
>pov3d(x,y,Z(x,y),angle=10°, ...
>  xv=NX(x,y),yv=NY(x,y),zv=NZ(x,y),<shadow);
\end{eulerprompt}
\eulerimg{28}{images/Pekan 7-8_Fanny Erina Dewi_22305141005_EMT00-Plot3D_Aplikom-063.png}
\begin{eulercomment}
Berikut ini adalah simpul Trefoil yang dibuat oleh A. Busser di
Povray. Ada versi yang lebih baik dari ini dalam contoh.

See: Examples\textbackslash{}Trefoil Knot \textbar{} Trefoil Knot

Untuk tampilan yang bagus dengan tidak terlalu banyak titik, kami
menambahkan vektor normal di sini. Kami menggunakan Maxima untuk
menghitung normal untuk kami. Pertama, tiga fungsi untuk koordinat
sebagai ekspresi simbolis.
\end{eulercomment}
\begin{eulerprompt}
>X &= ((4+sin(3*y))+cos(x))*cos(2*y); ...
>Y &= ((4+sin(3*y))+cos(x))*sin(2*y); ...
>Z &= sin(x)+2*cos(3*y);
\end{eulerprompt}
\begin{eulercomment}
Kemudian dua vektor turunan terhadap x dan y.
\end{eulercomment}
\begin{eulerprompt}
>dx &= diff([X,Y,Z],x); dy &= diff([X,Y,Z],y);
\end{eulerprompt}
\begin{eulercomment}
Sekarang yang normal, yang merupakan produk silang dari dua turunan.
\end{eulercomment}
\begin{eulerprompt}
>dn &= crossproduct(dx,dy);
\end{eulerprompt}
\begin{eulercomment}
Kami sekarang mengevaluasi semua ini secara numerik.
\end{eulercomment}
\begin{eulerprompt}
>x:=linspace(-%pi,%pi,40); y:=linspace(-%pi,%pi,100)';
\end{eulerprompt}
\begin{eulercomment}
Vektor normal adalah evaluasi dari ekspresi simbolik dn[i] untuk
i=1,2,3. Sintaks untuk ini adalah \&"ekspresi"(parameter). Ini adalah
sebuah alternatif dari metode pada contoh sebelumnya, di mana kita
mendefinisikan ekspresi simbolik NX, NY, NZ terlebih dahulu.
\end{eulercomment}
\begin{eulerprompt}
>pov3d(X(x,y),Y(x,y),Z(x,y),>anaglyph,axis=0,zoom=5,w=450,h=350, ...
>  <shadow,look=povlook(blue), ...
>  xv=&"dn[1]"(x,y), yv=&"dn[2]"(x,y), zv=&"dn[3]"(x,y));
\end{eulerprompt}
\eulerimg{21}{images/Pekan 7-8_Fanny Erina Dewi_22305141005_EMT00-Plot3D_Aplikom-064.png}
\begin{eulercomment}
Kami juga dapat menghasilkan kisi-kisi dalam bentuk 3D.
\end{eulercomment}
\begin{eulerprompt}
>povstart(zoom=4); ...
>x=-1:0.5:1; r=1-(x+1)^2/6; ...
>t=(0°:30°:360°)'; y=r*cos(t); z=r*sin(t); ...
>writeln(povgrid(x,y,z,d=0.02,dballs=0.05)); ...
>povend();
\end{eulerprompt}
\eulerimg{28}{images/Pekan 7-8_Fanny Erina Dewi_22305141005_EMT00-Plot3D_Aplikom-065.png}
\begin{eulercomment}
Dengan povgrid(), kurva dapat dibuat.
\end{eulercomment}
\begin{eulerprompt}
>povstart(center=[0,0,1],zoom=3.6); ...
>t=linspace(0,2,1000); r=exp(-t); ...
>x=cos(2*pi*10*t)*r; y=sin(2*pi*10*t)*r; z=t; ...
>writeln(povgrid(x,y,z,povlook(red))); ...
>writeAxis(0,2,axis=3); ...
>povend();
\end{eulerprompt}
\eulerimg{28}{images/Pekan 7-8_Fanny Erina Dewi_22305141005_EMT00-Plot3D_Aplikom-066.png}
\eulerheading{Object Povray}
\begin{eulercomment}
Di atas, kami menggunakan pov3d untuk memplot permukaan. Antarmuka
povray di Euler juga dapat menghasilkan objek Povray. Objek-objek ini
disimpan sebagai string di Euler, dan perlu ditulis ke file Povray.

Kita memulai output dengan povstart().
\end{eulercomment}
\begin{eulerprompt}
>povstart(zoom=4);
\end{eulerprompt}
\begin{eulercomment}
Pertama, kita mendefinisikan tiga silinder, dan menyimpannya dalam
string di Euler.

Fungsi povx() dll. hanya mengembalikan vektor [1,0,0], yang dapat
digunakan sebagai gantinya.

\end{eulercomment}
\begin{eulerprompt}
>c1=povcylinder(-povx,povx,1,povlook(red)); ...
>c2=povcylinder(-povy,povy,1,povlook(yellow)); ...
>c3=povcylinder(-povz,povz,1,povlook(blue)); ...
\end{eulerprompt}
\begin{eulercomment}
String berisi kode Povray, yang tidak perlu kita pahami pada saat itu.
\end{eulercomment}
\begin{eulerprompt}
>c2
\end{eulerprompt}
\begin{euleroutput}
  cylinder \{ <0,0,-1>, <0,0,1>, 1
   texture \{ pigment \{ color rgb <0.941176,0.941176,0.392157> \}  \} 
   finish \{ ambient 0.2 \} 
   \}
\end{euleroutput}
\begin{eulercomment}
Seperti yang Anda lihat, kami menambahkan tekstur ke objek dalam tiga
warna berbeda.

Hal itu dilakukan dengan povlook(), yang mengembalikan sebuah string
dengan kode Povray yang relevan. Kita dapat menggunakan warna default
Euler, atau menentukan warna kita sendiri. Kita juga dapat menambahkan
transparansi, atau mengubah cahaya sekitar.
\end{eulercomment}
\begin{eulerprompt}
>povlook(rgb(0.1,0.2,0.3),0.1,0.5)
\end{eulerprompt}
\begin{euleroutput}
   texture \{ pigment \{ color rgbf <0.101961,0.2,0.301961,0.1> \}  \} 
   finish \{ ambient 0.5 \} 
  
\end{euleroutput}
\begin{eulercomment}
Sekarang kita mendefinisikan objek perpotongan, dan menulis hasilnya
ke file.
\end{eulercomment}
\begin{eulerprompt}
>writeln(povintersection([c1,c2,c3]));
\end{eulerprompt}
\begin{eulercomment}
Perpotongan tiga silinder sulit dibayangkan, jika Anda belum pernah
melihatnya.
\end{eulercomment}
\begin{eulerprompt}
>povend;
\end{eulerprompt}
\eulerimg{28}{images/Pekan 7-8_Fanny Erina Dewi_22305141005_EMT00-Plot3D_Aplikom-067.png}
\begin{eulercomment}
Fungsi-fungsi berikut ini menghasilkan fraktal secara rekursif.

Fungsi pertama menunjukkan, bagaimana Euler menangani objek Povray
sederhana. Fungsi povbox() mengembalikan sebuah string, yang berisi
koordinat kotak, tekstur dan hasil akhir.
\end{eulercomment}
\begin{eulerprompt}
>function onebox(x,y,z,d) := povbox([x,y,z],[x+d,y+d,z+d],povlook());
>function fractal (x,y,z,h,n) ...
\end{eulerprompt}
\begin{eulerudf}
   if n==1 then writeln(onebox(x,y,z,h));
   else
     h=h/3;
     fractal(x,y,z,h,n-1);
     fractal(x+2*h,y,z,h,n-1);
     fractal(x,y+2*h,z,h,n-1);
     fractal(x,y,z+2*h,h,n-1);
     fractal(x+2*h,y+2*h,z,h,n-1);
     fractal(x+2*h,y,z+2*h,h,n-1);
     fractal(x,y+2*h,z+2*h,h,n-1);
     fractal(x+2*h,y+2*h,z+2*h,h,n-1);
     fractal(x+h,y+h,z+h,h,n-1);
   endif;
  endfunction
\end{eulerudf}
\begin{eulerprompt}
>povstart(fade=10,<shadow);
>fractal(-1,-1,-1,2,4);
>povend();
\end{eulerprompt}
\eulerimg{28}{images/Pekan 7-8_Fanny Erina Dewi_22305141005_EMT00-Plot3D_Aplikom-068.png}
\begin{eulercomment}
Perbedaan memungkinkan pemotongan satu objek dari objek lainnya.
Seperti persimpangan, ada bagian dari objek CSG Povray.
\end{eulercomment}
\begin{eulerprompt}
>povstart(light=[5,-5,5],fade=10);
\end{eulerprompt}
\begin{eulercomment}
Untuk demonstrasi ini, kita akan mendefinisikan sebuah objek di
Povray, alih-alih menggunakan sebuah string di Euler. Definisi akan
langsung dituliskan ke file.

Koordinat kotak -1 berarti [-1,-1,-1].
\end{eulercomment}
\begin{eulerprompt}
>povdefine("mycube",povbox(-1,1));
\end{eulerprompt}
\begin{eulercomment}
Kita dapat menggunakan objek ini dalam povobject(), yang mengembalikan
sebuah string seperti biasa
\end{eulercomment}
\begin{eulerprompt}
>c1=povobject("mycube",povlook(red));
\end{eulerprompt}
\begin{eulercomment}
Kami menghasilkan kubus kedua, dan memutar serta menskalakannya
sedikit.
\end{eulercomment}
\begin{eulerprompt}
>c2=povobject("mycube",povlook(yellow),translate=[1,1,1], ...
>  rotate=xrotate(10°)+yrotate(10°), scale=1.2);
\end{eulerprompt}
\begin{eulercomment}
Kemudian kita ambil selisih dari kedua objek tersebut.
\end{eulercomment}
\begin{eulerprompt}
>writeln(povdifference(c1,c2));
\end{eulerprompt}
\begin{eulercomment}
Sekarang tambahkan tiga sumbu
\end{eulercomment}
\begin{eulerprompt}
>writeAxis(-1.2,1.2,axis=1); ...
>writeAxis(-1.2,1.2,axis=2); ...
>writeAxis(-1.2,1.2,axis=4); ...
>povend();
\end{eulerprompt}
\eulerimg{28}{images/Pekan 7-8_Fanny Erina Dewi_22305141005_EMT00-Plot3D_Aplikom-069.png}
\eulerheading{Fungsi Implisit}
\begin{eulercomment}
Povray dapat memplot himpunan di mana f(x,y,z)=0, seperti parameter
implisit pada plot3d. Namun, hasilnya terlihat jauh lebih baik.

Sintaks untuk fungsi-fungsi tersebut sedikit berbeda. Anda tidak dapat
menggunakan output dari ekspresi Maxima atau Euler

\end{eulercomment}
\begin{eulerformula}
\[
((x^2+y^2-c^2)^2+(z^2-1)^2)*((y^2+z^2-c^2)^2+(x^2-1)^2)*((z^2+x^2-c^2)^2+(y^2-1)^2)=d
\]
\end{eulerformula}
\begin{eulerprompt}
>povstart(angle=70°,height=50°,zoom=4);
>c=0.1; d=0.1; ...
>writeln(povsurface("(pow(pow(x,2)+pow(y,2)-pow(c,2),2)+pow(pow(z,2)-1,2))*(pow(pow(y,2)+pow(z,2)-pow(c,2),2)+pow(pow(x,2)-1,2))*(pow(pow(z,2)+pow(x,2)-pow(c,2),2)+pow(pow(y,2)-1,2))-d",povlook(red))); ...
>povend();
\end{eulerprompt}
\begin{euleroutput}
  Error : Povray error!
  
  Error generated by error() command
  
  povray:
      error("Povray error!");
  Try "trace errors" to inspect local variables after errors.
  povend:
      povray(file,w,h,aspect,exit); 
\end{euleroutput}
\begin{eulerprompt}
>povstart(angle=25°,height=10°); 
>writeln(povsurface("pow(x,2)+pow(y,2)*pow(z,2)-1",povlook(blue),povbox(-2,2,"")));
>povend();
\end{eulerprompt}
\eulerimg{28}{images/Pekan 7-8_Fanny Erina Dewi_22305141005_EMT00-Plot3D_Aplikom-070.png}
\begin{eulerprompt}
>povstart(angle=70°,height=50°,zoom=4);
\end{eulerprompt}
\begin{eulercomment}
Membuat permukaan implisit. Perhatikan sintaks yang berbeda dalam
ekspresi.
\end{eulercomment}
\begin{eulerprompt}
>writeln(povsurface("pow(x,2)*y-pow(y,3)-pow(z,2)",povlook(green))); ...
>writeAxes(); ...
>povend();
\end{eulerprompt}
\eulerimg{28}{images/Pekan 7-8_Fanny Erina Dewi_22305141005_EMT00-Plot3D_Aplikom-071.png}
\eulerheading{Objek Jaring}
\begin{eulercomment}
Dalam contoh ini, kami menunjukkan cara membuat objek mesh, dan
menggambarnya dengan informasi tambahan.

Kami ingin memaksimalkan xy di bawah kondisi x+y = 1 dan
mendemonstrasikan sentuhan tangensial dari garis level.
\end{eulercomment}
\begin{eulerprompt}
>povstart(angle=-10°,center=[0.5,0.5,0.5],zoom=7);
\end{eulerprompt}
\begin{eulercomment}
Kita tidak dapat menyimpan objek dalam sebuah string seperti
sebelumnya, karena ukurannya terlalu besar. Jadi kita mendefinisikan
objek dalam file Povray menggunakan deklarasikan. Fungsi povtriangle()
melakukan hal ini secara otomatis. Fungsi ini dapat menerima vektor
normal seperti halnya pov3d().

Berikut ini mendefinisikan objek mesh, dan langsung menuliskannya ke
dalam file.
\end{eulercomment}
\begin{eulerprompt}
>px=0:0.02:1; y=x'; z=x*y; vx=-y; vy=-x; vz=1;
>mesh=povtriangles(x,y,z,"",vx,vy,vz);
\end{eulerprompt}
\begin{eulercomment}
Sekarang kita tentukan dua cakram, yang akan berpotongan dengan
permukaan.
\end{eulercomment}
\begin{eulerprompt}
>cl=povdisc([0.5,0.5,0],[1,1,0],2); ...
>ll=povdisc([0,0,1/4],[0,0,1],2);
\end{eulerprompt}
\begin{eulercomment}
Tuliskan permukaan dikurangi kedua cakram.
\end{eulercomment}
\begin{eulerprompt}
>writeln(povdifference(mesh,povunion([cl,ll]),povlook(green)));
\end{eulerprompt}
\begin{eulercomment}
Tuliskan kedua perpotongan tersebut.
\end{eulercomment}
\begin{eulerprompt}
>writeln(povintersection([mesh,cl],povlook(red))); ...
>writeln(povintersection([mesh,ll],povlook(gray)));
\end{eulerprompt}
\begin{eulercomment}
Tulislah satu titik secara maksimal.
\end{eulercomment}
\begin{eulerprompt}
>writeln(povpoint([1/2,1/2,1/4],povlook(gray),size=2*defaultpointsize));
\end{eulerprompt}
\begin{eulercomment}
Tambahkan sumbu dan selesaikan.
\end{eulercomment}
\begin{eulerprompt}
>writeAxes(0,1,0,1,0,1,d=0.015); ...
>povend();
\end{eulerprompt}
\eulerimg{28}{images/Pekan 7-8_Fanny Erina Dewi_22305141005_EMT00-Plot3D_Aplikom-072.png}
\eulerheading{Anaglyph di Povray}
\begin{eulercomment}
Untuk menghasilkan anaglyph untuk kacamata merah/cyan, Povray harus
dijalankan dua kali dari posisi kamera yang berbeda. Ini menghasilkan
dua file Povray dan dua file PNG, yang dimuat dengan fungsi
loadanaglyph().

Tentu saja, Anda memerlukan kacamata merah/cyan untuk melihat contoh
berikut ini dengan benar. 

Fungsi pov3d() memiliki saklar sederhana untuk menghasilkan anaglyph.
\end{eulercomment}
\begin{eulerprompt}
>pov3d("-exp(-x^2-y^2)/2",r=2,height=45°,>anaglyph, ...
>  center=[0,0,0.5],zoom=3.5);
\end{eulerprompt}
\begin{euleroutput}
  Command was not allowed!
  exec:
      return _exec(program,param,dir,print,hidden,wait);
  povray:
      exec(program,params,defaulthome);
  Try "trace errors" to inspect local variables after errors.
  pov3d:
      if povray then povray(currentfile,w,h,w/h); endif;
\end{euleroutput}
\begin{eulercomment}
Jika Anda membuat scene dengan objek, Anda harus menempatkan pembuatan
scene ke dalam fungsi, dan menjalankannya dua kali dengan nilai yang
berbeda untuk parameter anaglyph.
\end{eulercomment}
\begin{eulerprompt}
>function myscene ...
\end{eulerprompt}
\begin{eulerudf}
    s=povsphere(povc,1);
    cl=povcylinder(-povz,povz,0.5);
    clx=povobject(cl,rotate=xrotate(90°));
    cly=povobject(cl,rotate=yrotate(90°));
    c=povbox([-1,-1,0],1);
    un=povunion([cl,clx,cly,c]);
    obj=povdifference(s,un,povlook(red));
    writeln(obj);
    writeAxes();
  endfunction
\end{eulerudf}
\begin{eulercomment}
Fungsi povanaglyph() melakukan semua ini. Parameter-parameternya
seperti pada povstart() dan povend() yang digabungkan.
\end{eulercomment}
\begin{eulerprompt}
>povanaglyph("myscene",zoom=4.5);
\end{eulerprompt}
\begin{euleroutput}
  Command was not allowed!
  exec:
      return _exec(program,param,dir,print,hidden,wait);
  povray:
      exec(program,params,defaulthome);
  Try "trace errors" to inspect local variables after errors.
  povanaglyph:
      povray(currentfile,w,h,aspect,exit); 
\end{euleroutput}
\eulerheading{Mendifinisikan Objek Sendiri}
\begin{eulercomment}
Antarmuka povray Euler berisi banyak sekali objek. Tetapi Anda tidak
dibatasi pada objek-objek tersebut. Anda dapat membuat objek sendiri,
yang menggabungkan objek lain, atau objek yang benar-benar baru.

Kami mendemonstrasikan sebuah torus. Perintah Povray untuk ini adalah
"torus". Jadi kita mengembalikan sebuah string dengan perintah ini dan
parameternya. Perhatikan bahwa torus selalu berpusat di titik asal.
\end{eulercomment}
\begin{eulerprompt}
>function povdonat (r1,r2,look="") ...
\end{eulerprompt}
\begin{eulerudf}
    return "torus \{"+r1+","+r2+look+"\}";
  endfunction
\end{eulerudf}
\begin{eulercomment}
Inilah torus pertama kami.
\end{eulercomment}
\begin{eulerprompt}
>t1=povdonat(0.8,0.2)
\end{eulerprompt}
\begin{euleroutput}
  torus \{0.8,0.2\}
\end{euleroutput}
\begin{eulercomment}
Mari kita gunakan objek ini untuk membuat torus kedua, diterjemahkan
dan diputar.
\end{eulercomment}
\begin{eulerprompt}
>t2=povobject(t1,rotate=xrotate(90°),translate=[0.8,0,0])
\end{eulerprompt}
\begin{euleroutput}
  object \{ torus \{0.8,0.2\}
   rotate 90 *x 
   translate <0.8,0,0>
   \}
\end{euleroutput}
\begin{eulercomment}
Sekarang, kita tempatkan semua benda ini ke dalam suatu pemandangan.
Untuk tampilannya, kami menggunakan Phong Shading.
\end{eulercomment}
\begin{eulerprompt}
>povstart(center=[0.4,0,0],angle=0°,zoom=3.8,aspect=1.5); ...
>writeln(povobject(t1,povlook(green,phong=1))); ...
>writeln(povobject(t2,povlook(green,phong=1))); ...
\end{eulerprompt}
\begin{eulerttcomment}
 >povend();
\end{eulerttcomment}
\begin{eulercomment}
memanggil program Povray. Namun, jika terjadi kesalahan, program ini
tidak menampilkan kesalahan. Oleh karena itu, Anda harus menggunakan

\end{eulercomment}
\begin{eulerttcomment}
 >povend(<exit);
\end{eulerttcomment}
\begin{eulercomment}

jika ada yang tidak berhasil. Ini akan membiarkan jendela Povray
terbuka.
\end{eulercomment}
\begin{eulerprompt}
>povend(h=320,w=480);
\end{eulerprompt}
\eulerimg{18}{images/Pekan 7-8_Fanny Erina Dewi_22305141005_EMT00-Plot3D_Aplikom-073.png}
\begin{eulercomment}
Berikut adalah contoh yang lebih rumit. Kami menyelesaikan

\end{eulercomment}
\begin{eulerformula}
\[
Ax \le b, \quad x \ge 0, \quad c.x \to \text{Max.}
\]
\end{eulerformula}
\begin{eulercomment}
dan menunjukkan titik-titik yang layak dan optimal dalam plot 3D.
\end{eulercomment}
\begin{eulerprompt}
>A=[10,8,4;5,6,8;6,3,2;9,5,6];
>b=[10,10,10,10]';
>c=[1,1,1];
\end{eulerprompt}
\begin{eulercomment}
Pertama, mari kita periksa, apakah contoh ini memiliki solusi atau
tidak.
\end{eulercomment}
\begin{eulerprompt}
>x=simplex(A,b,c,>max,>check)'
\end{eulerprompt}
\begin{euleroutput}
  [0,  1,  0.5]
\end{euleroutput}
\begin{eulercomment}
Ya, benar.

Selanjutnya kita mendefinisikan dua objek. Yang pertama adalah pesawat

\end{eulercomment}
\begin{eulerformula}
\[
a \cdot x \le b
\]
\end{eulerformula}
\begin{eulerprompt}
>function oneplane (a,b,look="") ...
\end{eulerprompt}
\begin{eulerudf}
    return povplane(a,b,look)
  endfunction
\end{eulerudf}
\begin{eulercomment}
Kemudian kita mendefinisikan perpotongan semua setengah ruang dan
kubus.
\end{eulercomment}
\begin{eulerprompt}
>function adm (A, b, r, look="") ...
\end{eulerprompt}
\begin{eulerudf}
    ol=[];
    loop 1 to rows(A); ol=ol|oneplane(A[#],b[#]); end;
    ol=ol|povbox([0,0,0],[r,r,r]);
    return povintersection(ol,look);
  endfunction
\end{eulerudf}
\begin{eulercomment}
Sekarang, kita bisa merencanakan adegan tersebut.
\end{eulercomment}
\begin{eulerprompt}
>povstart(angle=120°,center=[0.5,0.5,0.5],zoom=3.5); ...
>writeln(adm(A,b,2,povlook(green,0.4))); ...
>writeAxes(0,1.3,0,1.6,0,1.5); ...
\end{eulerprompt}
\begin{eulercomment}
Berikut ini adalah lingkaran di sekeliling optimal.
\end{eulercomment}
\begin{eulerprompt}
>writeln(povintersection([povsphere(x,0.5),povplane(c,c.x')], ...
>  povlook(red,0.9)));
\end{eulerprompt}
\begin{eulercomment}
Dan kesalahan pada arah yang optimal.
\end{eulercomment}
\begin{eulerprompt}
>writeln(povarrow(x,c*0.5,povlook(red)));
\end{eulerprompt}
\begin{eulercomment}
Kami menambahkan teks ke layar. Teks hanyalah sebuah objek 3D. Kita
perlu menempatkan dan memutarnya sesuai dengan pandangan kita.
\end{eulercomment}
\begin{eulerprompt}
>writeln(povtext("Linear Problem",[0,0.2,1.3],size=0.05,rotate=5°)); ...
>povend();
\end{eulerprompt}
\eulerimg{28}{images/Pekan 7-8_Fanny Erina Dewi_22305141005_EMT00-Plot3D_Aplikom-074.png}
\eulerheading{Contoh Lainnya}
\begin{eulercomment}
Anda dapat menemukan beberapa contoh lain untuk Povray di Euler dalam
file

See: Examples/Dandelin Spheres\\
See: Examples/Donat Math\\
See: Examples/Trefoil Knot\\
See: Examples/Optimization by Affine Scaling
\end{eulercomment}
\begin{eulercomment}
SOAL SOAL PLOT 3D\\
\end{eulercomment}
\eulersubheading{}
\begin{eulercomment}
1. Selesaikan sistem persamaan berikut ini\\
\end{eulercomment}
\begin{eulerformula}
\[
x^2+y^2=1; y=e^xy:
\]
\end{eulerformula}
\begin{eulerprompt}
>f1 &= x^2+y^2-1; f2 &= y-exp(-x*y);
>plot2d(f1,r=1.2,level=0); plot2d(f2,level=0,>add):
\end{eulerprompt}
\eulerimg{27}{images/Pekan 7-8_Fanny Erina Dewi_22305141005_EMT00-Plot3D_Aplikom-075.png}
\begin{eulercomment}
2. Selidiki fungsi f(x;y)=x\textasciicircum{}y-y\textasciicircum{}x untuk x;y\textgreater{}0.
\end{eulercomment}
\begin{eulerprompt}
>function f(x,y) := x^y-y^x;
>plot2d("f",a=0,b=5,c=0,d=5,n=100,...
>level=0,>hue,>spectral,contourcolor=red,contourwidth=3):
\end{eulerprompt}
\eulerimg{27}{images/Pekan 7-8_Fanny Erina Dewi_22305141005_EMT00-Plot3D_Aplikom-076.png}
\begin{eulercomment}
3. sin (x\textasciicircum{}2+y\textasciicircum{}2)
\end{eulercomment}
\begin{eulerprompt}
>aspect(2); plot3d("sin(x^2+y^2)*exp((-x^2-y^2)/5)",r=4,>polar,...
><frame,n=200,fscale=0.8,>hue,scale=3,>anaglyph,center=[0,0,0.5]):
\end{eulerprompt}
\eulerimg{13}{images/Pekan 7-8_Fanny Erina Dewi_22305141005_EMT00-Plot3D_Aplikom-077.png}
\begin{eulerprompt}
>plot3d(cos(x)*(1+y/2*cos(x/2)),sin(x*(1+y/2*cos(x/2)),y/2*sin(x/2),  ...
>	<frame,hue=2,max=0.9,scale=2.7):
\end{eulerprompt}
\begin{euleroutput}
  Built-in function sin needs 1 argument (got 6)!
  Error in:
  ... ),y/2*sin(x/2),     <frame,hue=2,max=0.9,scale=2.7): ...
                                                       ^
\end{euleroutput}
\begin{eulercomment}
4. cos(x)*(1+y/2*cos(x/2)),sin(x)*(1+y/2*cos(x/2)),y/2*sin(x/2)\\
dengan y (-1:0.1:1)
\end{eulercomment}
\begin{eulerprompt}
>x:=linspace(0,2*pi,100);  y:=(-1:0.1:1)';  ...
>plot3d(cos(x)*(1+y/2*cos(x/2)),sin(x)*(1+y/2*cos(x/2)),y/2*sin(x/2),  ...
><frame,hue=2,max=0.9,scale=2.7):
\end{eulerprompt}
\eulerimg{13}{images/Pekan 7-8_Fanny Erina Dewi_22305141005_EMT00-Plot3D_Aplikom-078.png}
\begin{eulerprompt}
>reset;...
>plot3d("sin(x)","cos(x)","x/2Pi","lines",xmin=0,xmax=10pi,n=100,"user"):
\end{eulerprompt}
\begin{euleroutput}
  Illegal parameter after named parameter!
  Error in:
  ... x)","x/2Pi","lines",xmin=0,xmax=10pi,n=100,"user"): ...
                                                       ^
\end{euleroutput}
\begin{eulerprompt}
>defaultpovray="C:\(\backslash\)Program Files\(\backslash\)POV-Ray\(\backslash\)v3.7\(\backslash\)bin\(\backslash\)pvengine.exe"
\end{eulerprompt}
\begin{euleroutput}
  C:\(\backslash\)Program Files\(\backslash\)POV-Ray\(\backslash\)v3.7\(\backslash\)bin\(\backslash\)pvengine.exe
\end{euleroutput}
\begin{eulercomment}
5. Buatlah grafik 3d\\
\end{eulercomment}
\begin{eulerformula}
\[
x^2+y^2
\]
\end{eulerformula}
\begin{eulerprompt}
>plot3d("exp(x^2+y^2)",>user,...
>title="x^2+y^2"):
\end{eulerprompt}
\eulerimg{27}{images/Pekan 7-8_Fanny Erina Dewi_22305141005_EMT00-Plot3D_Aplikom-079.png}
\begin{eulercomment}
6. Buatlah plot 3d \\
\end{eulercomment}
\begin{eulerformula}
\[
-x^2+y^2
\]
\end{eulerformula}
\begin{eulercomment}
dengan distance =3, zoom=1, dan angle = pi/2
\end{eulercomment}
\begin{eulerprompt}
>plot3d("-x^2+y^2",distance=3,zoom=1,angle=pi/2,height=1):
\end{eulerprompt}
\eulerimg{27}{images/Pekan 7-8_Fanny Erina Dewi_22305141005_EMT00-Plot3D_Aplikom-080.png}
\begin{eulercomment}
7. Buatlah plot 3d\\
\end{eulercomment}
\begin{eulerformula}
\[
1/x^2+y^2+1
\]
\end{eulerformula}
\begin{eulercomment}
dengan angle -20°, a=0, b=1,c=-3,d=4 
\end{eulercomment}
\begin{eulerprompt}
>plot3d("1/(x^2+y^2+1)",a=0,b=1,c=-3,d=4,angle=-20°,height=10°, ...
>center=[0.2,0,0],zoom=5):
\end{eulerprompt}
\eulerimg{27}{images/Pekan 7-8_Fanny Erina Dewi_22305141005_EMT00-Plot3D_Aplikom-081.png}
\begin{eulercomment}
8. Buat plot 3D \\
\end{eulercomment}
\begin{eulerformula}
\[
x*cos(x)
\]
\end{eulerformula}
\begin{eulercomment}
dengan ketentuan a=0, b=3 dan rotasi =2
\end{eulercomment}
\begin{eulerprompt}
>plot3d("x*cos(x)",a=0,b=3,rotate=2):
\end{eulerprompt}
\eulerimg{27}{images/Pekan 7-8_Fanny Erina Dewi_22305141005_EMT00-Plot3D_Aplikom-082.png}
\begin{eulercomment}
9. Buatlah plot contour\\
\end{eulercomment}
\begin{eulerformula}
\[
x*sin(x)
\]
\end{eulerformula}
\begin{eulercomment}
dengan angle 30° dan warna plot merah
\end{eulercomment}
\begin{eulerprompt}
>plot3d("exp(x*sin(x))",angle=30°,>contour,color=red):
\end{eulerprompt}
\eulerimg{27}{images/Pekan 7-8_Fanny Erina Dewi_22305141005_EMT00-Plot3D_Aplikom-083.png}
\begin{eulerprompt}
>plot3d("x*sin(x)",r=5,implicit=2):
\end{eulerprompt}
\eulerimg{27}{images/Pekan 7-8_Fanny Erina Dewi_22305141005_EMT00-Plot3D_Aplikom-084.png}
\begin{eulerprompt}
>t=linspace(0,2pi,60); s=linspace(-pi/2,pi/2,90)'; ...
>x=cos(s)*cos(t); y=cos(s)*sin(t); z=sin(s); ...
>plot3d(x,y,z,>hue, ...
>color=yellow,<frame,grid=[10,30], ...
>values=s,contourcolor=green,level=[90°-14°;90°-12°], ...
>scale=1.2,height=50°):
\end{eulerprompt}
\eulerimg{27}{images/Pekan 7-8_Fanny Erina Dewi_22305141005_EMT00-Plot3D_Aplikom-085.png}
\begin{eulerprompt}
>t=linspace(0,2pi,140); s=linspace(-pi/4,pi/4,120)'; ...
>d=1+0.2*(sin(2*t)+cos(4*s)); ...
>plot3d(sin(t)*cos(s)*d,sin(t)*cos(s)*d,sin(s)*d,hue=1, ...
>  light=[1,0,1],frame=0,zoom=5):
\end{eulerprompt}
\eulerimg{27}{images/Pekan 7-8_Fanny Erina Dewi_22305141005_EMT00-Plot3D_Aplikom-086.png}
\begin{eulerprompt}
>t=linspace(0,2pi,700); plot3d(tan(t),cos(t),t/2pi,>wire, ...
>linewidth=40,wirecolor=green):
\end{eulerprompt}
\eulerimg{27}{images/Pekan 7-8_Fanny Erina Dewi_22305141005_EMT00-Plot3D_Aplikom-087.png}
\begin{eulerprompt}
>x=-1:0.4:1; y=x'; z=x+y; d=zeros(size(x)); ...
>plot3d(x,y,z,disconnect=2:20:20):
\end{eulerprompt}
\eulerimg{27}{images/Pekan 7-8_Fanny Erina Dewi_22305141005_EMT00-Plot3D_Aplikom-088.png}
\begin{eulerprompt}
>povstart(angle=70°,height=50°,zoom=4);
\end{eulercomment}
\end{eulernotebook}

\chapter{EMT Kalkulus}
\begin{eulernotebook}
\eulerheading{Kalkulus dengan EMT }
\begin{eulercomment}
Materi Kalkulus mencakup di antaranya: - Fungsi (fungsi aljabar,
trigonometri, eksponensial, logaritma, komposisi fungsi)\\
- Limit Fungsi,\\
- Turunan Fungsi,\\
- Integral Tak Tentu,\\
- Integral Tentu dan Aplikasinya,\\
- Barisan dan Deret (kekonvergenan barisan dan deret).\\
EMT (bersama Maxima) dapat digunakan untuk melakukan semua perhitungan
di dalam kalkulus, baik secara numerik maupun analitik (eksak).

\begin{eulercomment}
\eulerheading{Mendifinisikan Fungsi }
\begin{eulercomment}
Terdapat beberapa cara mendefinisikan fungsi pada EMT, yakni:\\
- Menggunakan format nama\_fungsi := rumus fungsi (untuk fungsi
numerik),\\
- Menggunakan format nama\_fungsi \&= rumus fungsi (untuk fungsi
simbolik, namun dapat dihitung secara numerik),\\
- Menggunakan format nama\_fungsi \&\&= rumus fungsi (untuk fungsi
simbolik murni, tidak dapat dihitung langsung),\\
- Fungsi sebagai program EMT.\\
Setiap format harus diawali dengan perintah function (bukan sebagai
ekspresi).\\
Berikut adalah adalah beberapa contoh cara mendefinisikan fungsi.
\end{eulercomment}
\begin{eulerprompt}
>function f(x) := 2*x^2+exp(sin(x)) // fungsi numerik
>f(0), f(1), f(pi)
\end{eulerprompt}
\begin{euleroutput}
  1
  4.31977682472
  20.7392088022
\end{euleroutput}
\begin{eulerprompt}
>function g(x) := sqrt(x^2-3*x)/(x+1)
>g(3)
\end{eulerprompt}
\begin{euleroutput}
  0
\end{euleroutput}
\begin{eulerprompt}
>g(0)
\end{eulerprompt}
\begin{euleroutput}
  0
\end{euleroutput}
\begin{eulerprompt}
>g(1)
\end{eulerprompt}
\begin{euleroutput}
  Floating point error!
  Error in sqrt
  Try "trace errors" to inspect local variables after errors.
  g:
      useglobal; return sqrt(x^2-3*x)/(x+1) 
  Error in:
  g(1) ...
      ^
\end{euleroutput}
\begin{eulerprompt}
>f(g(5)) // komposisi fungsi
\end{eulerprompt}
\begin{euleroutput}
  2.20920171961
\end{euleroutput}
\begin{eulerprompt}
>g(f(5))
\end{eulerprompt}
\begin{euleroutput}
  0.950898070639
\end{euleroutput}
\begin{eulerprompt}
>function h(x) := f(g(x)) // definisi komposisi fungsi
>h(5) // sama dengan f(g(5))
\end{eulerprompt}
\begin{euleroutput}
  2.20920171961
\end{euleroutput}
\begin{eulerprompt}
>f(0:10) // nilai-nilai f(1), f(2), ..., f(10)
\end{eulerprompt}
\begin{euleroutput}
  [1,  4.31978,  10.4826,  19.1516,  32.4692,  50.3833,  72.7562,
  99.929,  130.69,  163.51,  200.58]
\end{euleroutput}
\begin{eulerprompt}
>fmap(0:10) // sama dengan f(0:10), berlaku untuk semua fungsi
\end{eulerprompt}
\begin{euleroutput}
  [1,  4.31978,  10.4826,  19.1516,  32.4692,  50.3833,  72.7562,
  99.929,  130.69,  163.51,  200.58]
\end{euleroutput}
\begin{eulercomment}
Misalkan kita akan mendifinisikan fungsi\\
\end{eulercomment}
\begin{eulerformula}
\[
f(x)= \begin{cases} x^3 & x>0 \\ x^2 & x\le0 \end{cases}
\]
\end{eulerformula}
\begin{eulercomment}
Fungsi tersebut tidak dapat didefinisikan sebagai fungsi numerik
secara "inline" menggunakan format :=, melainkan didefinisikan sebagai
program.\\
Perhatikan, kata "map" digunakan agar fungsi dapat menerima vektor
sebagai input, dan hasilnya berupa vektor. Jika tanpa kata "map"
fungsinya hanya dapat menerima input satu nilai.
\end{eulercomment}
\begin{eulerprompt}
>function map f(x) ...
\end{eulerprompt}
\begin{eulerudf}
  if x>0 then return x^3
  else return x^2
  endif;
  endfunction
\end{eulerudf}
\begin{eulerprompt}
>f(1)
\end{eulerprompt}
\begin{euleroutput}
  1
\end{euleroutput}
\begin{eulerprompt}
>f(-2)
\end{eulerprompt}
\begin{euleroutput}
  4
\end{euleroutput}
\begin{eulerprompt}
>f(-5:5)
\end{eulerprompt}
\begin{euleroutput}
  [25,  16,  9,  4,  1,  0,  1,  8,  27,  64,  125]
\end{euleroutput}
\begin{eulerprompt}
>aspect(1.5); plot2d("f(x)",-5,5):
\end{eulerprompt}
\eulerimg{17}{images/Pekan 9-10_Fanny Erina Dewi_22305141005_EMT00-Kalkulus_Aplikom-001.png}
\begin{eulerprompt}
>function f(x) &= 2*E^x // fungsi simbolik
\end{eulerprompt}
\begin{euleroutput}
  
                                      x
                                   2 E
  
\end{euleroutput}
\begin{eulerprompt}
>function g(x) &= 3*x+1
\end{eulerprompt}
\begin{euleroutput}
  
                                 3 x + 1
  
\end{euleroutput}
\begin{eulerprompt}
>function h(x) &= f(g(x)) // komposisi fungsi
\end{eulerprompt}
\begin{euleroutput}
  
                                   3 x + 1
                                2 E
  
\end{euleroutput}
\begin{eulercomment}
\begin{eulercomment}
\eulerheading{Latihan }
\begin{eulercomment}
Bukalah buku Kalkulus. Cari dan pilih beberapa (paling sedikit 5
fungsi berbeda tipe/bentuk/jenis) fungsi dari buku tersebut, kemudian
definisikan di EMT pada baris-baris perintah berikut (jika perlu
tambahkan lagi). Untuk setiap fungsi, hitung beberapa nilainya, baik
untuk satu nilai maupun vektor. Gambar grafik tersebut.

Juga, carilah fungsi beberapa (dua) variabel. Lakukan hal sama seperti
di atas.

1. Untuk fungsi

\end{eulercomment}
\begin{eulerformula}
\[
k(x)=3x^2-2
\]
\end{eulerformula}
\begin{eulercomment}
tentukan nilai\\
a. k(-3)\\
b. k(3)
\end{eulercomment}
\begin{eulerprompt}
>function k(x) := 3*x^2 -2
>k(-3), k(3)
\end{eulerprompt}
\begin{euleroutput}
  25
  25
\end{euleroutput}
\begin{eulerprompt}
>plot2d("k",-3,3):
\end{eulerprompt}
\eulerimg{17}{images/Pekan 9-10_Fanny Erina Dewi_22305141005_EMT00-Kalkulus_Aplikom-002.png}
\begin{eulercomment}
2. Untuk fungsi

\end{eulercomment}
\begin{eulerformula}
\[
z(x)=\frac{x^2-25}{x-5}
\]
\end{eulerformula}
\begin{eulercomment}
hitunglah masing-masing nilai.\\
a. z(4)\\
b. z(6)
\end{eulercomment}
\begin{eulerprompt}
>function z(x) := (x^2-25)/(x-5)
>z(4), z(6)
\end{eulerprompt}
\begin{euleroutput}
  9
  11
\end{euleroutput}
\begin{eulerprompt}
>plot2d("z",-6,6):
\end{eulerprompt}
\eulerimg{17}{images/Pekan 9-10_Fanny Erina Dewi_22305141005_EMT00-Kalkulus_Aplikom-003.png}
\begin{eulercomment}
3. untuk fungsi

\end{eulercomment}
\begin{eulerformula}
\[
r(x)=2x^3-x^2+6x-2
\]
\end{eulerformula}
\begin{eulercomment}
tentukan nilai r(2), r(-4), r(6)
\end{eulercomment}
\begin{eulerprompt}
>function f(x) := 2x^3-x^2+6*x-2
>f(2), f(-4), f(6)
\end{eulerprompt}
\begin{euleroutput}
  22
  -170
  430
\end{euleroutput}
\begin{eulerprompt}
>plot2d("2x^3-x^2+6*x-2",-2,9):
\end{eulerprompt}
\eulerimg{17}{images/Pekan 9-10_Fanny Erina Dewi_22305141005_EMT00-Kalkulus_Aplikom-004.png}
\begin{eulercomment}
4. Tentukan nilai f(80) dari fungsi berikut

\end{eulercomment}
\begin{eulerformula}
\[
f(x)=\sqrt{x+16}
\]
\end{eulerformula}
\begin{eulerprompt}
>function f(x) := sqrt(x+16)
>f(80)
\end{eulerprompt}
\begin{euleroutput}
  9.79795897113
\end{euleroutput}
\begin{eulerprompt}
>plot2d("sqrt(x+16)",0,100):
\end{eulerprompt}
\eulerimg{17}{images/Pekan 9-10_Fanny Erina Dewi_22305141005_EMT00-Kalkulus_Aplikom-005.png}
\begin{eulercomment}
5. Untuk fungsi

\end{eulercomment}
\begin{eulerformula}
\[
f(x)=x^2+2x-6
\]
\end{eulerformula}
\begin{eulercomment}
dan\\
\end{eulercomment}
\begin{eulerformula}
\[
g(x)=2x+2
\]
\end{eulerformula}
\begin{eulercomment}
cari nilai fog(4), gof(-2)
\end{eulercomment}
\begin{eulerprompt}
>function f(x) := x^2+2*x-6; $f(x)
>function g(x) := 2*x+2; $g(x)
>f(g(4)), g(f(-2))
\end{eulerprompt}
\begin{euleroutput}
  114
  -10
\end{euleroutput}
\eulerheading{Menghitung Limit }
\begin{eulercomment}
Perhitungan limit pada EMT dapat dilakukan dengan menggunakan fungsi
Maxima, yakni "limit". Fungsi "limit" dapat digunakan untuk menghitung
limit fungsi dalam bentuk ekspresi maupun fungsi yang sudah
didefinisikan sebelumnya. Nilai limit dapat dihitung pada sebarang
nilai atau pada tak hingga (-inf, minf, dan inf). Limit kiri dan limit
kanan juga dapat dihitung, dengan cara memberi opsi "plus" atau
"minus". Hasil limit dapat berupa nilai, "und’ (tak definisi), "ind"
(tak tentu namun terbatas), "infinity" (kompleks tak hingga).

Perhatikan beberapa contoh berikut. Perhatikan cara menampilkan
perhitungan secara lengkap, tidak hanya menampilkan hasilnya saja.
\end{eulercomment}
\begin{eulerprompt}
>$limit((x^3-13*x^2+51*x-63)/(x^3-4*x^2-3*x+18),x,3)
\end{eulerprompt}
\begin{eulerformula}
\[
-\frac{4}{5}
\]
\end{eulerformula}
\begin{eulerformula}
\[
\lim_{x\to 3}\frac{x^3-13x^2+15x-63}{x^3-4x^2-3x+18}=-\frac{4}{5}
\]
\end{eulerformula}
\begin{eulercomment}
Fungsi tersebut diskontinu di titik x=3. Berikut adalah grafik
fungsinya.
\end{eulercomment}
\begin{eulerprompt}
>aspect(1.5); plot2d("(x^3-13*x^2+51*x-63)/(x^3-4*x^2-3*x+18)",0,4); plot2d(3,-4/5,>points,style="ow",>add):
\end{eulerprompt}
\eulerimg{17}{images/Pekan 9-10_Fanny Erina Dewi_22305141005_EMT00-Kalkulus_Aplikom-007.png}
\begin{eulerprompt}
>$limit(2*x*sin(x)/(1-cos(x)),x,0)
\end{eulerprompt}
\begin{eulerformula}
\[
4
\]
\end{eulerformula}
\begin{eulerformula}
\[
2(\lim_{x\to 0}\frac{x\sin(x)}{1-\cos(x)})=4
\]
\end{eulerformula}
\begin{eulercomment}
Fungsi tersebut diskontinu di titik x=3. Berikut adalah grafik
fungsinya.
\end{eulercomment}
\begin{eulerprompt}
>plot2d("2*x*sin(x)/(1-cos(x))",-pi,pi); plot2d(0,4,>points,style="ow",>add):
\end{eulerprompt}
\eulerimg{17}{images/Pekan 9-10_Fanny Erina Dewi_22305141005_EMT00-Kalkulus_Aplikom-009.png}
\begin{eulerprompt}
>$limit(cot(7*h)/cot(5*h),h,0)
\end{eulerprompt}
\begin{eulerformula}
\[
\frac{5}{7}
\]
\end{eulerformula}
\begin{eulerformula}
\[
\lim_{h\to 0}\frac{\cot(7h)}{\cot(5h)}=\frac{5}{7}
\]
\end{eulerformula}
\begin{eulercomment}
Fungsi tersebut juga diskontinu (karena tidak terdefinisi) di x=0.
Berikut adalah grafiknya.
\end{eulercomment}
\begin{eulerprompt}
>plot2d("cot(7*x)/cot(5*x)",-0.001,0.001); plot2d(0,5/7,>points,style="ow",>add):
\end{eulerprompt}
\eulerimg{17}{images/Pekan 9-10_Fanny Erina Dewi_22305141005_EMT00-Kalkulus_Aplikom-011.png}
\begin{eulerprompt}
>$showev('limit(((x/8)^(1/3)-1)/(x-8),x,8))
\end{eulerprompt}
\begin{eulerformula}
\[
\lim_{x\rightarrow 8}{\frac{\frac{x^{\frac{1}{3}}}{2}-1}{x-8}}=
 \frac{1}{24}
\]
\end{eulerformula}
\begin{eulercomment}
Tunjukkan limit tersebut dengan grafik, seperti contoh-contoh
sebelumnya.
\end{eulercomment}
\begin{eulerprompt}
>$showev('limit(1/(2*x-1),x,0))
\end{eulerprompt}
\begin{eulerformula}
\[
\lim_{x\rightarrow 0}{\frac{1}{2\,x-1}}=-1
\]
\end{eulerformula}
\begin{eulerprompt}
>$showev('limit((x^2-3*x-10)/(x-5),x,5))
\end{eulerprompt}
\begin{eulerformula}
\[
\lim_{x\rightarrow 5}{\frac{x^2-3\,x-10}{x-5}}=7
\]
\end{eulerformula}
\begin{eulercomment}
Tunjukkan limit tersebut dengan grafik, seperti contoh-contoh
sebelumnya.
\end{eulercomment}
\begin{eulerprompt}
>$showev('limit(sqrt(x^2+x)-x,x,inf))
\end{eulerprompt}
\begin{eulerformula}
\[
\lim_{x\rightarrow \infty }{\sqrt{x^2+x}-x}=\frac{1}{2}
\]
\end{eulerformula}
\begin{eulerprompt}
>$showev('limit(abs(x-1)/(x-1),x,1,minus))
\end{eulerprompt}
\begin{eulerformula}
\[
\lim_{x\uparrow 1}{\frac{\left| x-1\right| }{x-1}}=-1
\]
\end{eulerformula}
\begin{eulercomment}
Tunjukkan limit tersebut dengan grafik, seperti contoh-contoh
sebelumnya.
\end{eulercomment}
\begin{eulerprompt}
>$showev('limit(sin(x)/x,x,0))
\end{eulerprompt}
\begin{eulerformula}
\[
\lim_{x\rightarrow 0}{\frac{\sin x}{x}}=1
\]
\end{eulerformula}
\begin{eulerprompt}
>plot2d("sin(x)/x",-pi,pi); plot2d(0,1,>points,style="ow",>add):
\end{eulerprompt}
\eulerimg{17}{images/Pekan 9-10_Fanny Erina Dewi_22305141005_EMT00-Kalkulus_Aplikom-018.png}
\begin{eulerprompt}
>$showev('limit(sin(x^3)/x,x,0))
\end{eulerprompt}
\begin{eulerformula}
\[
\lim_{x\rightarrow 0}{\frac{\sin x^3}{x}}=0
\]
\end{eulerformula}
\begin{eulercomment}
Tunjukkan limit tersebut dengan grafik, seperti contoh-contoh
sebelumnya.
\end{eulercomment}
\begin{eulerprompt}
>$showev('limit(log(x), x, minf))
\end{eulerprompt}
\begin{eulerformula}
\[
\lim_{x\rightarrow  -\infty }{\log x}={\it infinity}
\]
\end{eulerformula}
\begin{eulerprompt}
>$showev('limit(t-sqrt(2-t),t,2,minus))
\end{eulerprompt}
\begin{eulerformula}
\[
\lim_{t\uparrow 2}{t-\sqrt{2-t}}=2
\]
\end{eulerformula}
\begin{eulerprompt}
>$showev('limit(t-sqrt(2-t),t,5,plus)) // Perhatikan hasilnya
\end{eulerprompt}
\begin{eulerformula}
\[
\lim_{t\downarrow 5}{t-\sqrt{2-t}}=5-\sqrt{3}\,i
\]
\end{eulerformula}
\begin{eulerprompt}
>plot2d("x-sqrt(2-x)",0,2):
\end{eulerprompt}
\eulerimg{17}{images/Pekan 9-10_Fanny Erina Dewi_22305141005_EMT00-Kalkulus_Aplikom-023.png}
\begin{eulerprompt}
>$showev('limit((x^2-9)/(2*x^2-5*x-3),x,3))
\end{eulerprompt}
\begin{eulerformula}
\[
\lim_{x\rightarrow 3}{\frac{x^2-9}{2\,x^2-5\,x-3}}=\frac{6}{7}
\]
\end{eulerformula}
\begin{eulercomment}
Tunjukkan limit tersebut dengan grafik, seperti contoh-contoh
sebelumnya.
\end{eulercomment}
\begin{eulerprompt}
>$showev('limit((1-cos(x))/x,x,0))
\end{eulerprompt}
\begin{eulerformula}
\[
\lim_{x\rightarrow 0}{\frac{1-\cos x}{x}}=0
\]
\end{eulerformula}
\begin{eulercomment}
Tunjukkan limit tersebut dengan grafik, seperti contoh-contoh
sebelumnya.
\end{eulercomment}
\begin{eulerprompt}
>$showev('limit((x^2+abs(x))/(x^2-abs(x)),x,0))
\end{eulerprompt}
\begin{eulerformula}
\[
\lim_{x\rightarrow 0}{\frac{\left| x\right| +x^2}{x^2-\left| x
 \right| }}=-1
\]
\end{eulerformula}
\begin{eulercomment}
Tunjukkan limit tersebut dengan grafik, seperti contoh-contoh
sebelumnya.
\end{eulercomment}
\begin{eulerprompt}
>$showev('limit((1+1/x)^x,x,inf))
\end{eulerprompt}
\begin{eulerformula}
\[
\lim_{x\rightarrow \infty }{\left(\frac{1}{x}+1\right)^{x}}=e
\]
\end{eulerformula}
\begin{eulerprompt}
>plot2d("(1+1/x)^x",0,1000):
\end{eulerprompt}
\eulerimg{17}{images/Pekan 9-10_Fanny Erina Dewi_22305141005_EMT00-Kalkulus_Aplikom-028.png}
\begin{eulerprompt}
>$showev('limit((1+k/x)^x,x,inf))
\end{eulerprompt}
\begin{eulerformula}
\[
\lim_{x\rightarrow \infty }{\left(\frac{k}{x}+1\right)^{x}}=e^{k}
\]
\end{eulerformula}
\begin{eulerprompt}
>$showev('limit((1+x)^(1/x),x,0))
\end{eulerprompt}
\begin{eulerformula}
\[
\lim_{x\rightarrow 0}{\left(x+1\right)^{\frac{1}{x}}}=e
\]
\end{eulerformula}
\begin{eulercomment}
Tunjukkan limit tersebut dengan grafik, seperti contoh-contoh
sebelumnya.
\end{eulercomment}
\begin{eulerprompt}
>$showev('limit((x/(x+k))^x,x,inf))
\end{eulerprompt}
\begin{eulerformula}
\[
\lim_{x\rightarrow \infty }{\left(\frac{x}{x+k}\right)^{x}}=e^ {- k
  }
\]
\end{eulerformula}
\begin{eulerprompt}
>$showev('limit((E^x-E^2)/(x-2),x,2))
\end{eulerprompt}
\begin{eulerformula}
\[
\lim_{x\rightarrow 2}{\frac{e^{x}-e^2}{x-2}}=e^2
\]
\end{eulerformula}
\begin{eulercomment}
Tunjukkan limit tersebut dengan grafik, seperti contoh-contoh
sebelumnya.
\end{eulercomment}
\begin{eulerprompt}
>$showev('limit(sin(1/x),x,0))
\end{eulerprompt}
\begin{eulerformula}
\[
\lim_{x\rightarrow 0}{\sin \left(\frac{1}{x}\right)}={\it ind}
\]
\end{eulerformula}
\begin{eulerprompt}
>$showev('limit(sin(1/x),x,inf))
\end{eulerprompt}
\begin{eulerformula}
\[
\lim_{x\rightarrow \infty }{\sin \left(\frac{1}{x}\right)}=0
\]
\end{eulerformula}
\begin{eulerprompt}
>plot2d("sin(1/x)",-5,5):
\end{eulerprompt}
\eulerimg{17}{images/Pekan 9-10_Fanny Erina Dewi_22305141005_EMT00-Kalkulus_Aplikom-035.png}
\begin{eulercomment}
\begin{eulercomment}
\eulerheading{Latihan }
\begin{eulercomment}
Bukalah buku Kalkulus. Cari dan pilih beberapa (paling sedikit 5
fungsi berbeda tipe/bentuk/jenis) fungsi dari buku tersebut, kemudian
definisikan di EMT pada baris-baris perintah berikut (jika perlu
tambahkan lagi). Untuk setiap fungsi, hitung nilai limit fungsi
tersebut di beberapa nilai dan di tak hingga. Gambar grafik fungsi
tersebut untuk mengkonfirmasi nilai-nilai limit tersebut.

1. Hitunglah nilai limit berikut.

\end{eulercomment}
\begin{eulerformula}
\[
\lim_{x\to 2}(x+3)
\]
\end{eulerformula}
\begin{eulerprompt}
>$showev('limit((x+3),x,2))
\end{eulerprompt}
\begin{eulerformula}
\[
\lim_{x\rightarrow 2}{x+3}=5
\]
\end{eulerformula}
\begin{eulercomment}
2. Hitunglah nilai limit berikut.\\
\end{eulercomment}
\begin{eulerformula}
\[
\lim_{x\to 4}\frac{2x^2+1}{3x-4}
\]
\end{eulerformula}
\begin{eulerprompt}
>$showev('limit((2*x^2+1)/(3*x-4=4),x,4))
\end{eulerprompt}
\begin{eulerformula}
\[
\lim_{x\rightarrow 4}{\frac{2\,x^2+1}{3\,x-4}=\frac{2\,x^2+1}{4}}=
 \left(\frac{33}{8}=\frac{33}{4}\right)
\]
\end{eulerformula}
\begin{eulercomment}
3. Hitunglah nilai limit berikut dan gambarlah grafiknya.\\
\end{eulercomment}
\begin{eulerformula}
\[
\lim_{t\to 1}\frac{t^2+31}{cos(t+2)}
\]
\end{eulerformula}
\begin{eulerprompt}
>$showev('limit((t^2+31)/cos(t+2),t,2))
\end{eulerprompt}
\begin{eulerformula}
\[
\lim_{t\rightarrow 2}{\frac{t^2+31}{\cos \left(t+2\right)}}=\frac{
 35}{\cos 4}
\]
\end{eulerformula}
\begin{eulerprompt}
>(plot2d("(x^2+31)/cos(x+2)", -10,10)):
\end{eulerprompt}
\eulerimg{17}{images/Pekan 9-10_Fanny Erina Dewi_22305141005_EMT00-Kalkulus_Aplikom-039.png}
\begin{eulercomment}
4. Tentukan nilai limit berikut.\\
\end{eulercomment}
\begin{eulerformula}
\[
\lim_{x\to -2}\frac{\sqrt{1-5x}}{(x+6)^4}
\]
\end{eulerformula}
\begin{eulerprompt}
>$showev('limit((sqrt(1-5*x))/((x+6)^4), x, -1))
\end{eulerprompt}
\begin{eulerformula}
\[
\lim_{x\rightarrow -1}{\frac{\sqrt{1-5\,x}}{\left(x+6\right)^4}}=
 \frac{\sqrt{6}}{625}
\]
\end{eulerformula}
\begin{eulercomment}
5. Tentukan nilai limit berikut.\\
\end{eulercomment}
\begin{eulerformula}
\[
\lim_{t\to 0}\frac{(t-sin(t))^2}{t^2}
\]
\end{eulerformula}
\begin{eulerprompt}
>$showev('limit(((t-sin(t))^2)/(t^2),t,0))
\end{eulerprompt}
\begin{eulerformula}
\[
\lim_{t\rightarrow 0}{\frac{\left(t-\sin t\right)^2}{t^2}}=0
\]
\end{eulerformula}
\begin{eulerprompt}
>(plot2d("((x-sin(x))^2)/(x^2)",-5,5)):
\end{eulerprompt}
\eulerimg{17}{images/Pekan 9-10_Fanny Erina Dewi_22305141005_EMT00-Kalkulus_Aplikom-042.png}
\eulerheading{Turunan Fungsi }
\begin{eulercomment}
Definisi turunan: latex: f'(x)=\textbackslash{}lim\_\{h\textbackslash{}to 0\}\textbackslash{}frac\{f(x+h)-f(x)\}\{h\}\\
Berikut adalah contoh-contoh menentukan turunan fungsi dengan
menggunakan definisi turunan (limit).
\end{eulercomment}
\begin{eulerprompt}
>$showev('limit(((x+h)^n-x^n)/h,h,0)) // turunan x^n
\end{eulerprompt}
\begin{eulerformula}
\[
\lim_{h\rightarrow 0}{\frac{\left(x+h\right)^{n}-x^{n}}{h}}=n\,x^{n
 -1}
\]
\end{eulerformula}
\begin{eulercomment}
Mengapa hasilnya seperti itu? Tuliskan atau tunjukkan bahwa hasil
limit tersebut benar, sehingga benar turunan fungsinya benar. Tulis
penjelasan Anda di komentar ini.

Sebagai petunjuk, ekspansikan (x+h)\textasciicircum{}n dengan menggunakan teorema
binomial

\end{eulercomment}
\eulersubheading{BUKTI}
\begin{eulercomment}
\end{eulercomment}
\begin{eulerformula}
\[
f'(x) = \lim_{h\to 0} \frac{f(x+h)-f(x)}{h}
\]
\end{eulerformula}
\begin{eulercomment}
Untuk\\
\end{eulercomment}
\begin{eulerformula}
\[
f(x)=x^{n}
\]
\end{eulerformula}
\begin{eulerformula}
\[
\frac{d}{dx}sin(x) = \lim_{h\to 0} \frac{(x+h)^{n}-x^{n}}{h}
\]
\end{eulerformula}
\begin{eulercomment}
Dengan\\
\end{eulercomment}
\begin{eulerformula}
\[
(a+b)^{n}=\sum_{k=0}^n a^{k}b^{n-k}
\]
\end{eulerformula}
\begin{eulercomment}
maka\\
\end{eulercomment}
\begin{eulerformula}
\[
= \lim_{h\to 0} \frac{(x^{n}+\frac{n}{1!}x^{n-1}h+\frac{n(n-1)}{2!}x^{n-2}h^2+\frac{n(n-1)(n-2)}{3!}x^{n-3}h^{3}+...)-x^{n}}{h}
\]
\end{eulerformula}
\begin{eulerformula}
\[
= \lim_{h\to 0} \frac{n.x^{n-1}h+\frac{n(n-1)}{2!}x^{n-2}h^2+\frac{n(n-1)(n-2)}{3!}x^{n-3}h^{3}+...}{h}
\]
\end{eulerformula}
\begin{eulerformula}
\[
= \lim_{h\to 0} n.x^{n-1}+\frac{n(n-1)}{2!}.x^{n-2}h+\frac{n(n-1)(n-2)}{3!}.x^{n-3}h^{2}+...
\]
\end{eulerformula}
\begin{eulerformula}
\[
= n.x^{n-1}+0+0+...+0
\]
\end{eulerformula}
\begin{eulerformula}
\[
= n.x^{n-1}
\]
\end{eulerformula}
\begin{eulercomment}
Jadi, terbukti benar bahwa\\
\end{eulercomment}
\begin{eulerformula}
\[
f'(x^n) = n.x^{n-1}
\]
\end{eulerformula}
\eulersubheading{}
\begin{eulerprompt}
>$showev('limit((sin(x+h)-sin(x))/h,h,0)) // turunan sin(x)
\end{eulerprompt}
\begin{eulerformula}
\[
\lim_{h\rightarrow 0}{\frac{\sin \left(x+h\right)-\sin x}{h}}=\cos 
 x
\]
\end{eulerformula}
\begin{eulercomment}
Mengapa hasilnya seperti itu? Tuliskan atau tunjukkan bahwa hasil
limit tersebut benar, sehingga benar turunan fungsinya benar. Tulis
penjelasan Anda di komentar ini. 

Sebagai petunjuk, ekspansikan sin(x+h) dengan menggunakan rumus jumlah
dua sudut.

\end{eulercomment}
\eulersubheading{Bukti}
\begin{eulercomment}
\end{eulercomment}
\begin{eulerformula}
\[
f'(x) = \lim_{h\to 0} \frac{sin(x+h)-sin(x)}{h}
\]
\end{eulerformula}
\begin{eulercomment}
\end{eulercomment}
\begin{eulerformula}
\[
sin(a+b)=sin(a)cos(a)+cos(a)sin(b)
\]
\end{eulerformula}
\begin{eulercomment}
\end{eulercomment}
\begin{eulerformula}
\[
= \lim_{h\to 0} \frac{sin(x)cos(h)+cos(x)sin(h)-sin(x)}{h}
\]
\end{eulerformula}
\begin{eulercomment}
\end{eulercomment}
\begin{eulerformula}
\[
= \lim_{h\to 0} sinx.\frac{cos(h)-1}{h}+\lim_{h\to 0} cos(x).\frac{sin(h)}{h}
\]
\end{eulerformula}
\begin{eulerformula}
\[
= sin(x).0+cos(x).1
\]
\end{eulerformula}
\begin{eulercomment}
\end{eulercomment}
\begin{eulerformula}
\[
= cos(x)
\]
\end{eulerformula}
\begin{eulercomment}
Jadi, terbukti benar bahwa

\end{eulercomment}
\begin{eulerformula}
\[
f'(sin(x)) = cos(x)
\]
\end{eulerformula}
\eulersubheading{}
\begin{eulerprompt}
>$showev('limit((log(x+h)-log(x))/h,h,0)) // turunan log(x)
\end{eulerprompt}
\begin{eulerformula}
\[
\lim_{h\rightarrow 0}{\frac{\log \left(x+h\right)-\log x}{h}}=
 \frac{1}{x}
\]
\end{eulerformula}
\begin{eulercomment}
Mengapa hasilnya seperti itu? Tuliskan atau tunjukkan bahwa hasil
limit tersebut benar, sehingga benar turunan fungsinya benar. Tulis
penjelasan Anda di komentar ini. 

Sebagai petunjuk, gunakan sifat-sifat logaritma dan hasil limit pada
bagian sebelumnya di atas.

\end{eulercomment}
\eulersubheading{Bukti}
\begin{eulercomment}
\end{eulercomment}
\begin{eulerformula}
\[
f'(x) = \lim_{h\to 0} \frac{log(x+h)-log x}{h}
\]
\end{eulerformula}
\begin{eulercomment}
\end{eulercomment}
\begin{eulerformula}
\[
=\lim_{h\to 0} \frac{\frac{d}{dh}(log(x+h)-log x)}{\frac{d}{dh}(h)}
\]
\end{eulerformula}
\begin{eulerformula}
\[
=\lim_{h\to 0} \frac{\frac{1}{x+h}}{1}
\]
\end{eulerformula}
\begin{eulerformula}
\[
=\lim_{h\to 0} \frac{1}{x+h}
\]
\end{eulerformula}
\begin{eulerformula}
\[
=\frac{1}{x}
\]
\end{eulerformula}
\begin{eulercomment}
Jadi, terbukti benar bahwa\\
\end{eulercomment}
\begin{eulerformula}
\[
f'(x) = \lim_{h\to 0} \frac{log(x+h)-log x}{h} = \frac{1}{x}
\]
\end{eulerformula}
\eulersubheading{}
\begin{eulerprompt}
>$showev('limit((1/(x+h)-1/x)/h,h,0)) // turunan 1/x
\end{eulerprompt}
\begin{eulerformula}
\[
\lim_{h\rightarrow 0}{\frac{\frac{1}{x+h}-\frac{1}{x}}{h}}=-\frac{1
 }{x^2}
\]
\end{eulerformula}
\begin{eulerprompt}
>$showev('limit((E^(x+h)-E^x)/h,h,0)) // turunan f(x)=e^x
\end{eulerprompt}
\begin{euleroutput}
  Answering "Is x an integer?" with "integer"
  Answering "Is x an integer?" with "integer"
  Answering "Is x an integer?" with "integer"
  Answering "Is x an integer?" with "integer"
  Answering "Is x an integer?" with "integer"
  Maxima is asking
  Acceptable answers are: yes, y, Y, no, n, N, unknown, uk
  Is x an integer?
  
  Use assume!
  Error in:
  $showev('limit((E^(x+h)-E^x)/h,h,0)) // turunan f(x)=e^x ...
                                       ^
\end{euleroutput}
\begin{eulercomment}
Maxima bermasalah dengan limit:\\
\end{eulercomment}
\begin{eulerformula}
\[
\lim_{h\to 0}\frac{e^{x+h}-e^x}{h}
\]
\end{eulerformula}
\begin{eulercomment}
Oleh karena itu diperlukan trik khusus agar hasilnya benar.
\end{eulercomment}
\begin{eulerprompt}
>$showev('limit((E^h-1)/h,h,0))
\end{eulerprompt}
\begin{eulerformula}
\[
\lim_{h\rightarrow 0}{\frac{e^{h}-1}{h}}=1
\]
\end{eulerformula}
\begin{eulerprompt}
>$factor(E^(x+h)-E^x)
\end{eulerprompt}
\begin{eulerformula}
\[
\left(e^{h}-1\right)\,e^{x}
\]
\end{eulerformula}
\begin{eulerprompt}
>$showev('limit(factor((E^(x+h)-E^x)/h),h,0)) // turunan f(x)=e^x
\end{eulerprompt}
\begin{eulerformula}
\[
\left(\lim_{h\rightarrow 0}{\frac{e^{h}-1}{h}}\right)\,e^{x}=e^{x}
\]
\end{eulerformula}
\begin{eulerprompt}
>function f(x) &= x^x
\end{eulerprompt}
\begin{euleroutput}
  
                                     x
                                    x
  
\end{euleroutput}
\begin{eulerprompt}
>$showev('limit((f(x+h)-f(x))/h,h,0)) // turunan f(x)=x^x
\end{eulerprompt}
\begin{eulerformula}
\[
\lim_{h\rightarrow 0}{\frac{\left(x+h\right)^{x+h}-x^{x}}{h}}=
 {\it infinity}
\]
\end{eulerformula}
\begin{eulercomment}
Di sini Maxima juga bermasalah terkait limit:\\
\end{eulercomment}
\begin{eulerformula}
\[
\lim_{h\to 0}\frac{(x+h)^{x+h}-x^x}{h}
\]
\end{eulerformula}
\begin{eulercomment}
Dalam hal ini diperlukan asumsi nilai x.
\end{eulercomment}
\begin{eulerprompt}
>&assume(x>0); $showev('limit((f(x+h)-f(x))/h,h,0)) // turunan f(x)=x^x
\end{eulerprompt}
\begin{eulerformula}
\[
\lim_{h\rightarrow 0}{\frac{\left(x+h\right)^{x+h}-x^{x}}{h}}=x^{x}
 \,\left(\log x+1\right)
\]
\end{eulerformula}
\begin{eulerprompt}
>&forget(x>0) // jangan lupa, lupakan asumsi untuk kembali ke semula
\end{eulerprompt}
\begin{euleroutput}
  
                                 [x > 0]
  
\end{euleroutput}
\begin{eulerprompt}
>&forget(x<0)
\end{eulerprompt}
\begin{euleroutput}
  
                                 [x < 0]
  
\end{euleroutput}
\begin{eulerprompt}
>&facts()
\end{eulerprompt}
\begin{euleroutput}
  
                                    []
  
\end{euleroutput}
\begin{eulerprompt}
>$showev('limit((asin(x+h)-asin(x))/h,h,0)) // turunan arcsin(x)
\end{eulerprompt}
\begin{eulerformula}
\[
\lim_{h\rightarrow 0}{\frac{\arcsin \left(x+h\right)-\arcsin x}{h}}=
 \frac{1}{\sqrt{1-x^2}}
\]
\end{eulerformula}
\begin{eulerprompt}
>$showev('limit((tan(x+h)-tan(x))/h,h,0)) // turunan tan(x)
\end{eulerprompt}
\begin{eulerformula}
\[
\lim_{h\rightarrow 0}{\frac{\tan \left(x+h\right)-\tan x}{h}}=
 \frac{1}{\cos ^2x}
\]
\end{eulerformula}
\begin{eulerprompt}
>function f(x) &= sinh(x) // definisikan f(x)=sinh(x)
\end{eulerprompt}
\begin{euleroutput}
  
                                 sinh(x)
  
\end{euleroutput}
\begin{eulerprompt}
>function df(x) &= limit((f(x+h)-f(x))/h,h,0); $df(x) // df(x) = f’(x)
\end{eulerprompt}
\begin{eulerformula}
\[
\frac{e^ {- x }\,\left(e^{2\,x}+1\right)}{2}
\]
\end{eulerformula}
\begin{eulercomment}
Hasilnya adalah cos(x), karena\\
\end{eulercomment}
\begin{eulerformula}
\[
\frac{e^x+e^{-x}}{2}=\cos(x)
\]
\end{eulerformula}
\begin{eulerprompt}
>plot2d(["f(x)","df(x)"],-pi,pi,color=[blue,red]):
\end{eulerprompt}
\eulerimg{17}{images/Pekan 9-10_Fanny Erina Dewi_22305141005_EMT00-Kalkulus_Aplikom-055.png}
\begin{eulerprompt}
>function f(x) &= sin(3*x^5+7)^2
\end{eulerprompt}
\begin{euleroutput}
  
                                 2    5
                              sin (3 x  + 7)
  
\end{euleroutput}
\begin{eulerprompt}
>diff(f,3), diffc(f,3)
\end{eulerprompt}
\begin{euleroutput}
  1198.32948904
  1198.72863721
\end{euleroutput}
\begin{eulercomment}
Apakah perbedaan diff dan diffc?

diff adalah fungsi diferensiasi numerik yang menghitung turunan suatu
fungsi tertentu menggunakan perbedaan hingga. Ini dapat digunakan
untuk memperkirakan turunan suatu fungsi pada titik tertentu atau pada
interval 

diffc adalah fungsi diferensiasi numerik lain yang menghitung turunan
dari fungsi tertentu menggunakan diferensiasi langkah kompleks. Ini
lebih akurat daripada diff dan dapat digunakan untuk menghitung
turunan suatu fungsi pada titik tertentu atau pada interval
\end{eulercomment}
\begin{eulerprompt}
>$showev('diff(f(x),x))
\end{eulerprompt}
\begin{eulerformula}
\[
\frac{d}{d\,x}\,\sin ^2\left(3\,x^5+7\right)=30\,x^4\,\cos \left(3
 \,x^5+7\right)\,\sin \left(3\,x^5+7\right)
\]
\end{eulerformula}
\begin{eulerprompt}
>$% with x=3
\end{eulerprompt}
\begin{eulerformula}
\[
{\it \%at}\left(\frac{d}{d\,x}\,\sin ^2\left(3\,x^5+7\right) , x=3
 \right)=2430\,\cos 736\,\sin 736
\]
\end{eulerformula}
\begin{eulerprompt}
>$float(%)
\end{eulerprompt}
\begin{eulerformula}
\[
{\it \%at}\left(\frac{d^{1.0}}{d\,x^{1.0}}\,\sin ^2\left(3.0\,x^5+
 7.0\right) , x=3.0\right)=1198.728637211748
\]
\end{eulerformula}
\begin{eulerprompt}
>plot2d(f,0,3.1):
\end{eulerprompt}
\eulerimg{17}{images/Pekan 9-10_Fanny Erina Dewi_22305141005_EMT00-Kalkulus_Aplikom-059.png}
\begin{eulerprompt}
>function f(x) &=5*cos(2*x)-2*x*sin(2*x) // mendifinisikan fungsi f
\end{eulerprompt}
\begin{euleroutput}
  
                        5 cos(2 x) - 2 x sin(2 x)
  
\end{euleroutput}
\begin{eulerprompt}
>function df(x) &=diff(f(x),x) // fd(x) = f'(x)
\end{eulerprompt}
\begin{euleroutput}
  
                       - 12 sin(2 x) - 4 x cos(2 x)
  
\end{euleroutput}
\begin{eulerprompt}
>$'f(1)=f(1), $float(f(1)), $'f(2)=f(2), $float(f(2)) // nilai f(1) dan f(2)
\end{eulerprompt}
\begin{eulerformula}
\[
f\left(1\right)=5\,\cos 2-2\,\sin 2
\]
\end{eulerformula}
\begin{eulerformula}
\[
-3.899329036387075
\]
\end{eulerformula}
\begin{eulerformula}
\[
f\left(2\right)=5\,\cos 4-4\,\sin 4
\]
\end{eulerformula}
\begin{eulerformula}
\[
-0.2410081230863468
\]
\end{eulerformula}
\begin{eulerprompt}
>xp=solve("df(x)",1,2,0) // solusi f'(x)=0 pada interval [1, 2]
\end{eulerprompt}
\begin{euleroutput}
  1.35822987384
\end{euleroutput}
\begin{eulerprompt}
>df(xp), f(xp) // cek bahwa f'(xp)=0 dan nilai ekstrim di titik tersebut
\end{eulerprompt}
\begin{euleroutput}
  0
  -5.67530133759
\end{euleroutput}
\begin{eulerprompt}
>plot2d(["f(x)","df(x)"],0,2*pi,color=[blue,red]): //grafik fungsi dan turunannya
\end{eulerprompt}
\eulerimg{17}{images/Pekan 9-10_Fanny Erina Dewi_22305141005_EMT00-Kalkulus_Aplikom-064.png}
\begin{eulercomment}
Perhatikan titik-titik "puncak" grafik y=f(x) dan nilai turunan pada
saat grafik fungsinya mencapai titik "puncak" tersebut.

\begin{eulercomment}
\eulerheading{Latihan }
\begin{eulercomment}
Bukalah buku Kalkulus. Cari dan pilih beberapa (paling sedikit 5
fungsi berbeda tipe/bentuk/jenis) fungsi dari buku tersebut, kemudian
definisikan di EMT pada baris-baris perintah berikut (jika perlu
tambahkan lagi). Untuk setiap fungsi, tentukan turunannya dengan
menggunakan definisi turunan (limit), seperti contohcontoh tersebut.
Gambar grafik fungsi asli dan fungsi turunannya pada sumbu koordinat
yang sama.

1. Tentukan nilai turunan berikut dan sketsakan grafiknya.\\
\end{eulercomment}
\begin{eulerformula}
\[
f(x)=2x^2-6
\]
\end{eulerformula}
\begin{eulerprompt}
>function f(x) &= 2*x^2-6; $f(x)
\end{eulerprompt}
\begin{eulerformula}
\[
2\,x^2-6
\]
\end{eulerformula}
\begin{eulerprompt}
>function df(x) &= limit((f(x+h)-f(x))/h,h,0); &df(x)//df(x)=f'(x)
\end{eulerprompt}
\begin{euleroutput}
  
                                   4 x
  
\end{euleroutput}
\begin{eulerprompt}
>plot2d(["f(x)","df(x)"],-pi,pi,color=[red,green]):
\end{eulerprompt}
\eulerimg{17}{images/Pekan 9-10_Fanny Erina Dewi_22305141005_EMT00-Kalkulus_Aplikom-066.png}
\begin{eulercomment}
2. Carilah turunan dari fungsi berikut\\
\end{eulercomment}
\begin{eulerformula}
\[
f(x)=\frac{3x+3}{2x-1}
\]
\end{eulerformula}
\begin{eulerprompt}
>function f(x) &= (3*x+3)/(2*x-1); $f(x)
\end{eulerprompt}
\begin{eulerformula}
\[
\frac{3\,x+3}{2\,x-1}
\]
\end{eulerformula}
\begin{eulerprompt}
>function df(x) &= limit((f(x+h)-f(x))/h,h,0); $df(x) // df(x) = f'(x)
\end{eulerprompt}
\begin{eulerformula}
\[
-\frac{9}{4\,x^2-4\,x+1}
\]
\end{eulerformula}
\begin{eulerprompt}
>plot2d(["f(x)","df(x)"],-10,10,color=[blue,red]):
\end{eulerprompt}
\eulerimg{17}{images/Pekan 9-10_Fanny Erina Dewi_22305141005_EMT00-Kalkulus_Aplikom-069.png}
\begin{eulercomment}
3. Carilah turunan dari fungsi berikut\\
\end{eulercomment}
\begin{eulerformula}
\[
f(x)= \frac{2}{\sqrt{2x+3}}
\]
\end{eulerformula}
\begin{eulerprompt}
>function f(x) &= 2/sqrt(2*x+3); $f(x)
\end{eulerprompt}
\begin{eulerformula}
\[
\frac{2}{\sqrt{2\,x+3}}
\]
\end{eulerformula}
\begin{eulerprompt}
>function df(x) &= limit((f(x+h)-f(x))/h,h,0); $df(x) // df(x) = f'(x)function f(x) &= 3/sqrt(x-2); $f(x)
\end{eulerprompt}
\begin{eulerformula}
\[
-\frac{2}{\left(2\,x+3\right)^{\frac{3}{2}}}
\]
\end{eulerformula}
\begin{eulerprompt}
>plot2d(["f(x)","df(x)"],-10,10,color=[yellow,red]):
\end{eulerprompt}
\eulerimg{17}{images/Pekan 9-10_Fanny Erina Dewi_22305141005_EMT00-Kalkulus_Aplikom-072.png}
\begin{eulercomment}
4. Carilah turunan fungsi berikut.\\
\end{eulercomment}
\begin{eulerformula}
\[
f(x) = 3cos(x)+sin(2x)
\]
\end{eulerformula}
\begin{eulerprompt}
>function f(x) &= (3*cos(x)+sin(2*x)); $f(x)
\end{eulerprompt}
\begin{eulerformula}
\[
\sin \left(2\,x\right)+3\,\cos x
\]
\end{eulerformula}
\begin{eulerprompt}
>function df(x) &= limit((f(x+h)-f(x))/h,h,0); &df(x)
\end{eulerprompt}
\begin{euleroutput}
  
                          2 cos(2 x) - 3 sin(x)
  
\end{euleroutput}
\begin{eulerprompt}
>plot2d(["f(x)","df(x)"],-pi,pi,color=[blue,yellow]):
\end{eulerprompt}
\eulerimg{17}{images/Pekan 9-10_Fanny Erina Dewi_22305141005_EMT00-Kalkulus_Aplikom-074.png}
\begin{eulercomment}
5. Tentukan turunan dan grafik fungsi berikut.\\
\end{eulercomment}
\begin{eulerformula}
\[
f(x) = \frac{sin(3x)-cos(2x)}{sin(x)}
\]
\end{eulerformula}
\begin{eulerprompt}
>function f(x) &= (sin(3*x)-cos(2*x))/(sin(x)); $f(x)
\end{eulerprompt}
\begin{eulerformula}
\[
\frac{\sin \left(3\,x\right)-\cos \left(2\,x\right)}{\sin x}
\]
\end{eulerformula}
\begin{eulerprompt}
>function df(x) &= limit((f(x+h)-f(x))/h,h,0); $df(x) // df(x) = f'(x)
\end{eulerprompt}
\begin{eulerformula}
\[
\frac{-\cos x\,\sin \left(3\,x\right)-\sin x\,\left(-3\,\cos \left(
 3\,x\right)-2\,\sin \left(2\,x\right)\right)+\cos x\,\cos \left(2\,x
 \right)}{\sin ^2x}
\]
\end{eulerformula}
\begin{eulerprompt}
>plot2d(["f(x)","df(x)"],-pi,pi,color=[blue,yellow]):
\end{eulerprompt}
\eulerimg{17}{images/Pekan 9-10_Fanny Erina Dewi_22305141005_EMT00-Kalkulus_Aplikom-077.png}
\eulerheading{Integral }
\begin{eulercomment}
EMT dapat digunakan untuk menghitung integral, baik integral tak tentu
maupun integral tentu. Untuk integral tak tentu (simbolik) sudah
tentu EMT menggunakan Maxima, sedangkan untuk perhitungan integral
tentu EMT sudah menyediakan beberapa fungsi yang mengimplementasikan
algoritma kuadratur (perhitungan integral tentu menggunakan metode
numerik).\\
Pada notebook ini akan ditunjukkan perhitungan integral tentu dengan
menggunakan Teorema Dasar Kalkulus:

\end{eulercomment}
\begin{eulerformula}
\[
\int_{a}^{b}f(x)dx=F(b)-F(a), \text{ dengan } F'(x)=f(x)
\]
\end{eulerformula}
\begin{eulercomment}
Fungsi untuk menentukan integral adalah integrate. Fungsi ini dapat
digunakan untuk menentukan, baik integral tentu maupun tak tentu
(jika fungsinya memiliki antiderivatif). Untuk perhitungan integral
tentu fungsi integrate menggunakan metode numerik (kecuali fungsinya
tidak integrabel, kita tidak akan menggunakan metode ini).
\end{eulercomment}
\begin{eulerprompt}
>$showev('integrate(x^n,x))
\end{eulerprompt}
\begin{euleroutput}
  Answering "Is n equal to -1?" with "no"
\end{euleroutput}
\begin{eulerformula}
\[
\int {x^{n}}{\;dx}=\frac{x^{n+1}}{n+1}
\]
\end{eulerformula}
\begin{eulerprompt}
>$showev('integrate(1/(1+x),x))
\end{eulerprompt}
\begin{eulerformula}
\[
\int {\frac{1}{x+1}}{\;dx}=\log \left(x+1\right)
\]
\end{eulerformula}
\begin{eulerprompt}
>$showev('integrate(1/(1+x^2),x))
\end{eulerprompt}
\begin{eulerformula}
\[
\int {\frac{1}{x^2+1}}{\;dx}=\arctan x
\]
\end{eulerformula}
\begin{eulerprompt}
>$showev('integrate(1/sqrt(1-x^2),x))
\end{eulerprompt}
\begin{eulerformula}
\[
\int {\frac{1}{\sqrt{1-x^2}}}{\;dx}=\arcsin x
\]
\end{eulerformula}
\begin{eulerprompt}
>$showev('integrate(sin(x),x,0,pi))
\end{eulerprompt}
\begin{eulerformula}
\[
\int_{0}^{\pi}{\sin x\;dx}=2
\]
\end{eulerformula}
\begin{eulerprompt}
>$showev('integrate(sin(x),x,a,b))
\end{eulerprompt}
\begin{eulerformula}
\[
\int_{a}^{b}{\sin x\;dx}=\cos a-\cos b
\]
\end{eulerformula}
\begin{eulerprompt}
>$showev('integrate(x^n,x,a,b))
\end{eulerprompt}
\begin{euleroutput}
  Answering "Is n positive, negative or zero?" with "positive"
\end{euleroutput}
\begin{eulerformula}
\[
\int_{a}^{b}{x^{n}\;dx}=\frac{b^{n+1}}{n+1}-\frac{a^{n+1}}{n+1}
\]
\end{eulerformula}
\begin{eulerprompt}
>$showev('integrate(x^2*sqrt(2*x+1),x))
\end{eulerprompt}
\begin{eulerformula}
\[
\int {x^2\,\sqrt{2\,x+1}}{\;dx}=\frac{\left(2\,x+1\right)^{\frac{7
 }{2}}}{28}-\frac{\left(2\,x+1\right)^{\frac{5}{2}}}{10}+\frac{\left(
 2\,x+1\right)^{\frac{3}{2}}}{12}
\]
\end{eulerformula}
\begin{eulerprompt}
>$showev('integrate(x^2*sqrt(2*x+1),x,0,2))
\end{eulerprompt}
\begin{eulerformula}
\[
\int_{0}^{2}{x^2\,\sqrt{2\,x+1}\;dx}=\frac{2\,5^{\frac{5}{2}}}{21}-
 \frac{2}{105}
\]
\end{eulerformula}
\begin{eulerprompt}
>$ratsimp(%)
\end{eulerprompt}
\begin{eulerformula}
\[
\int_{0}^{2}{x^2\,\sqrt{2\,x+1}\;dx}=\frac{2\,5^{\frac{7}{2}}-2}{
 105}
\]
\end{eulerformula}
\begin{eulerprompt}
>$showev('integrate((sin(sqrt(x)+a)*E^sqrt(x))/sqrt(x),x,0,pi^2))
\end{eulerprompt}
\begin{eulerformula}
\[
\int_{0}^{\pi^2}{\frac{\sin \left(\sqrt{x}+a\right)\,e^{\sqrt{x}}}{
 \sqrt{x}}\;dx}=\left(-e^{\pi}-1\right)\,\sin a+\left(e^{\pi}+1
 \right)\,\cos a
\]
\end{eulerformula}
\begin{eulerprompt}
>$factor(%)
\end{eulerprompt}
\begin{eulerformula}
\[
\int_{0}^{\pi^2}{\frac{\sin \left(\sqrt{x}+a\right)\,e^{\sqrt{x}}}{
 \sqrt{x}}\;dx}=\left(-e^{\pi}-1\right)\,\left(\sin a-\cos a\right)
\]
\end{eulerformula}
\begin{eulerprompt}
>function map f(x) &= E^(-x^2)
\end{eulerprompt}
\begin{euleroutput}
  
                                      2
                                   - x
                                  E
  
\end{euleroutput}
\begin{eulerprompt}
>$showev('integrate(f(x),x))
\end{eulerprompt}
\begin{eulerformula}
\[
\int {e^ {- x^2 }}{\;dx}=\frac{\sqrt{\pi}\,\mathrm{erf}\left(x
 \right)}{2}
\]
\end{eulerformula}
\begin{eulercomment}
Fungsi f tidak memiliki antiturunan, integralnya masih memuat integral
lain.

\end{eulercomment}
\begin{eulerformula}
\[
erf(x)=\int \frac{e^{-x^2}}{\sqrt{\pi}}dx
\]
\end{eulerformula}
\begin{eulercomment}
Kita tidak dapat menggunakan teorema Dasar kalkulus untuk menghitung
integral tentu fungsi tersebut jika semua batasnya berhingga. Dalam
hal ini dapat digunakan metode numerik (rumus kuadratur).\\
Misalkan kita akan menghitung:

\end{eulercomment}
\begin{eulerformula}
\[
\int_{0}^{\pi}e^{-x^2}dx
\]
\end{eulerformula}
\begin{eulerprompt}
>x=0:0.1:pi-0.1; plot2d(x,f(x+0.1),>bar); plot2d("f(x)",0,pi,>add):
\end{eulerprompt}
\eulerimg{17}{images/Pekan 9-10_Fanny Erina Dewi_22305141005_EMT00-Kalkulus_Aplikom-091.png}
\begin{eulercomment}
Integral tentu

\end{eulercomment}
\begin{eulerformula}
\[
\int_{0}^{\pi}e^{-x^2}dx
\]
\end{eulerformula}
\begin{eulercomment}
dapat dihampiri dengan jumlah luas persegi-persegi panjang di bawah
kurva y=f(x) tersebut. Langkahlangkahnya adalah sebagai berikut
\end{eulercomment}
\begin{eulerprompt}
>t &= makelist(a,a,0,pi-0.1,0.1); // t sebagai list untuk menyimpan nilai-nilai x
>fx &= makelist(f(t[i]+0.1),i,1,length(t)); // simpan nilai-nilai f(x)
>// jangan menggunakan x sebagai list, kecuali Anda pakar Maxima!
\end{eulerprompt}
\begin{eulercomment}
Hasilnya adalah:

\end{eulercomment}
\begin{eulerformula}
\[
\int_{0}^{\pi}x^x dx=0.8362196102528469
\]
\end{eulerformula}
\begin{eulercomment}
Jumlah tersebut diperoleh dari hasil kali lebar sub-subinterval (=0.1)
dan jumlah nilai-nilai f(x) untuk x = 0.1, 0.2, 0.3, ..., 3.2.
\end{eulercomment}
\begin{eulerprompt}
>0.1*sum(f(x+0.1)) // cek langsung dengan perhitungan numerik EMT
\end{eulerprompt}
\begin{euleroutput}
  0.836219610253
\end{euleroutput}
\begin{eulercomment}
Untuk mendapatkan nilai integral tentu yang mendekati nilai
sebenarnya, lebar sub-intervalnya dapat diperkecil lagi, sehingga
daerah di bawah kurva tertutup semuanya, misalnya dapat digunakan
lebar subinterval\\
0.001. (Silakan dicoba!)

Meskipun Maxima tidak dapat menghitung integral tentu fungsi tersebut
untuk batas-batas yang berhingga, namun integral tersebut dapat
dihitung secara eksak jika batas-batasnya tak hingga. Ini adalah salah
satu keajaiban di dalam matematika, yang terbatas tidak dapat dihitung
secara eksak, namun yang tak hingga malah dapat dihitung secara eksak.
\end{eulercomment}
\begin{eulerprompt}
>$showev('integrate(f(x),x,0,inf))
\end{eulerprompt}
\begin{eulerformula}
\[
\int_{0}^{\infty }{e^ {- x^2 }\;dx}=\frac{\sqrt{\pi}}{2}
\]
\end{eulerformula}
\begin{eulercomment}
Berikut adalah contoh lain fungsi yang tidak memiliki antiderivatif,
sehingga integral tentunya hanya dapat dihitung dengan metode numerik.
\end{eulercomment}
\begin{eulerprompt}
>function f(x) &= x^x
\end{eulerprompt}
\begin{euleroutput}
  
                                     x
                                    x
  
\end{euleroutput}
\begin{eulerprompt}
>$showev('integrate(f(x),x,0,1))
\end{eulerprompt}
\begin{eulerformula}
\[
\int_{0}^{1}{x^{x}\;dx}=\int_{0}^{1}{x^{x}\;dx}
\]
\end{eulerformula}
\begin{eulerprompt}
>x=0:0.1:1-0.01; plot2d(x,f(x+0.01),>bar); plot2d("f(x)",0,1,>add):
\end{eulerprompt}
\eulerimg{17}{images/Pekan 9-10_Fanny Erina Dewi_22305141005_EMT00-Kalkulus_Aplikom-094.png}
\begin{eulercomment}
Maxima gagal menghitung integral tentu tersebut secara langsung
menggunakan perintah integrate. Berikut kita lakukan seperti contoh
sebelumnya untuk mendapat hasil atau pendekatan nilai integral tentu
tersebut.
\end{eulercomment}
\begin{eulerprompt}
>t &= makelist(a,a,0,1-0.01,0.01);
>fx &= makelist(f(t[i]+0.01),i,1,length(t));
\end{eulerprompt}
\begin{eulerformula}
\[
\int_{0}^{1}x^x dx=0.7834935879025506
\]
\end{eulerformula}
\begin{eulercomment}
Apakah hasil tersebut cukup baik? perhatikan gambarnya.
\end{eulercomment}
\begin{eulerprompt}
>function f(x) &= sin(3*x^5+7)^2
\end{eulerprompt}
\begin{euleroutput}
  
                                 2    5
                              sin (3 x  + 7)
  
\end{euleroutput}
\begin{eulerprompt}
>integrate(f,0,1)
\end{eulerprompt}
\begin{euleroutput}
  0.542581176074
\end{euleroutput}
\begin{eulerprompt}
>&showev('integrate(f(x),x,0,1))
\end{eulerprompt}
\begin{euleroutput}
  
           1                           1              pi
          /                      gamma(-) sin(14) sin(--)
          [     2    5                 5              10
          I  sin (3 x  + 7) dx = ------------------------
          ]                                  1/5
          /                              10 6
           0
         4/5                  1          4/5                  1
   - (((6    gamma_incomplete(-, 6 I) + 6    gamma_incomplete(-, - 6 I))
                              5                               5
               4/5                    1
   sin(14) + (6    I gamma_incomplete(-, 6 I)
                                      5
      4/5                    1                       pi
   - 6    I gamma_incomplete(-, - 6 I)) cos(14)) sin(--) - 60)/120
                             5                       10
  
\end{euleroutput}
\begin{eulerprompt}
>&float(%)
\end{eulerprompt}
\begin{euleroutput}
  
           1.0
          /
          [       2      5
          I    sin (3.0 x  + 7.0) dx = 
          ]
          /
           0.0
  0.09820784258795788 - 0.008333333333333333
   (0.3090169943749474 (0.1367372182078336
   (4.192962712629476 I gamma__incomplete(0.2, 6.0 I)
   - 4.192962712629476 I gamma__incomplete(0.2, - 6.0 I))
   + 0.9906073556948704 (4.192962712629476 gamma__incomplete(0.2, 6.0 I)
   + 4.192962712629476 gamma__incomplete(0.2, - 6.0 I))) - 60.0)
  
\end{euleroutput}
\begin{eulerprompt}
>$showev('integrate(x*exp(-x),x,0,1)) // Integral tentu (eksak)
\end{eulerprompt}
\begin{eulerformula}
\[
\int_{0}^{1}{x\,e^ {- x }\;dx}=1-2\,e^ {- 1 }
\]
\end{eulerformula}
\begin{eulercomment}
\begin{eulercomment}
\eulerheading{Aplikasi Integral Tentu}
\begin{eulerprompt}
>plot2d("x^3-x",-0.1,1.1); plot2d("-x^2",>add); ...
>b=solve("x^3-x+x^2",0.5); x=linspace(0,b,200); xi=flipx(x); ...
>plot2d(x|xi,x^3-x|-xi^2,>filled,style="|",fillcolor=1,>add): // Plot daerah antara 2 kurva
\end{eulerprompt}
\eulerimg{17}{images/Pekan 9-10_Fanny Erina Dewi_22305141005_EMT00-Kalkulus_Aplikom-096.png}
\begin{eulerprompt}
>a=solve("x^3-x+x^2",0), b=solve("x^3-x+x^2",1) // absis titik-titik potong kedua kurva
\end{eulerprompt}
\begin{euleroutput}
  0
  0.61803398875
\end{euleroutput}
\begin{eulerprompt}
>integrate("(-x^2)-(x^3-x)",a,b) // luas daerah yang diarsir
\end{eulerprompt}
\begin{euleroutput}
  0.0758191713542
\end{euleroutput}
\begin{eulercomment}
Hasil tersebut akan kita bandingkan dengan perhitungan secara
analitik.
\end{eulercomment}
\begin{eulerprompt}
>a &= solve((-x^2)-(x^3-x),x); $a // menentukan absis titik potong kedua kurva secara eksak
\end{eulerprompt}
\begin{eulerformula}
\[
\left[ x=\frac{-\sqrt{5}-1}{2} , x=\frac{\sqrt{5}-1}{2} , x=0
  \right] 
\]
\end{eulerformula}
\begin{eulerprompt}
>$showev('integrate(-x^2-x^3+x,x,0,(sqrt(5)-1)/2)) // Nilai integral secara eksak
\end{eulerprompt}
\begin{eulerformula}
\[
\int_{0}^{\frac{\sqrt{5}-1}{2}}{-x^3-x^2+x\;dx}=\frac{13-5^{\frac{3
 }{2}}}{24}
\]
\end{eulerformula}
\begin{eulerprompt}
>$float(%)
\end{eulerprompt}
\begin{eulerformula}
\[
\int_{0.0}^{0.6180339887498949}{-1.0\,x^3-1.0\,x^2+x\;dx}=
 0.07581917135421037
\]
\end{eulerformula}
\begin{eulercomment}
\begin{eulercomment}
\eulerheading{Panjang Kurva}
\begin{eulercomment}
Hitunglah panjang kurva berikut ini dan luas daerah di dalam kurva
tersebut.

\end{eulercomment}
\begin{eulerformula}
\[
\gamma(t)=(r(t)\cos(t),r(t)\sin(t))
\]
\end{eulerformula}
\begin{eulercomment}
dengan

\end{eulercomment}
\begin{eulerformula}
\[
r(t)=1+\frac{\sin(3t)}{2},\text{  }0\le t \le 2\pi
\]
\end{eulerformula}
\begin{eulerprompt}
>t=linspace(0,2pi,1000); r=1+sin(3*t)/2; x=r*cos(t); y=r*sin(t); ...
>plot2d(x,y,>filled,fillcolor=red,style="/",r=1.5): // Kita gambar kurvanya terlebih dahulu
\end{eulerprompt}
\eulerimg{17}{images/Pekan 9-10_Fanny Erina Dewi_22305141005_EMT00-Kalkulus_Aplikom-100.png}
\begin{eulerprompt}
>function r(t) &= 1+sin(3*t)/2; $'r(t)=r(t)
\end{eulerprompt}
\begin{eulerformula}
\[
r\left(\left[ 0 , 0.01 , 0.02 , 0.03 , 0.04 , 0.05 , 0.06 , 0.07 , 
 0.08 , 0.09 , 0.1 , 0.11 , 0.12 , 0.13 , 0.14 , 0.15 , 0.16 , 0.17
  , 0.18 , 0.19 , 0.2 , 0.21 , 0.2200000000000001 , 
 0.2300000000000001 , 0.2400000000000001 , 0.2500000000000001 , 
 0.2600000000000001 , 0.2700000000000001 , 0.2800000000000001 , 
 0.2900000000000001 , 0.3000000000000001 , 0.3100000000000001 , 
 0.3200000000000001 , 0.3300000000000001 , 0.3400000000000001 , 
 0.3500000000000001 , 0.3600000000000002 , 0.3700000000000002 , 
 0.3800000000000002 , 0.3900000000000002 , 0.4000000000000002 , 
 0.4100000000000002 , 0.4200000000000002 , 0.4300000000000002 , 
 0.4400000000000002 , 0.4500000000000002 , 0.4600000000000002 , 
 0.4700000000000003 , 0.4800000000000003 , 0.4900000000000003 , 
 0.5000000000000002 , 0.5100000000000002 , 0.5200000000000002 , 
 0.5300000000000002 , 0.5400000000000003 , 0.5500000000000003 , 
 0.5600000000000003 , 0.5700000000000003 , 0.5800000000000003 , 
 0.5900000000000003 , 0.6000000000000003 , 0.6100000000000003 , 
 0.6200000000000003 , 0.6300000000000003 , 0.6400000000000003 , 
 0.6500000000000004 , 0.6600000000000004 , 0.6700000000000004 , 
 0.6800000000000004 , 0.6900000000000004 , 0.7000000000000004 , 
 0.7100000000000004 , 0.7200000000000004 , 0.7300000000000004 , 
 0.7400000000000004 , 0.7500000000000004 , 0.7600000000000005 , 
 0.7700000000000005 , 0.7800000000000005 , 0.7900000000000005 , 
 0.8000000000000005 , 0.8100000000000005 , 0.8200000000000005 , 
 0.8300000000000005 , 0.8400000000000005 , 0.8500000000000005 , 
 0.8600000000000005 , 0.8700000000000006 , 0.8800000000000006 , 
 0.8900000000000006 , 0.9000000000000006 , 0.9100000000000006 , 
 0.9200000000000006 , 0.9300000000000006 , 0.9400000000000006 , 
 0.9500000000000006 , 0.9600000000000006 , 0.9700000000000006 , 
 0.9800000000000006 , 0.9900000000000007 \right] \right)=\left[ 1 , 
 1.014997750101248 , 1.029982003239722 , 1.044939274599006 , 
 1.05985610364446 , 1.0747190662368 , 1.089514786712912 , 
 1.10422994992305 , 1.118851313213567 , 1.133365718344415 , 
 1.14776010333067 , 1.162021514197434 , 1.176137116637545 , 
 1.190094207561581 , 1.203880226529785 , 1.217482767055615 , 
 1.230889587770742 , 1.244088623441454 , 1.257067995826556 , 
 1.269816024366985 , 1.282321236697518 , 1.294572378971135 , 
 1.306558425986717 , 1.318268591110984 , 1.329692335985737 , 
 1.340819380011667 , 1.351639709600205 , 1.362143587185071 , 
 1.37232155998543 , 1.382164468512753 , 1.391663454813742 , 
 1.400809970441889 , 1.409595784150499 , 1.41801298930026 , 
 1.426054010974682 , 1.433711612797009 , 1.440978903442474 , 
 1.447849342840024 , 1.454316748057942 , 1.460375298868068 , 
 1.466019542983613 , 1.471244400965849 , 1.476045170795258 , 
 1.480417532103036 , 1.484357550059133 , 1.48786167891333 , 
 1.49092676518618 , 1.493550050506925 , 1.495729174095843 , 
 1.49746217488879 , 1.498747493302027 , 1.499583972635738 , 
 1.499970860114983 , 1.499907807567145 , 1.499394871735262 , 
 1.498432514226959 , 1.497021601099038 , 1.495163402078079 , 
 1.492859589417777 , 1.490112236394023 , 1.486923815439098 , 
 1.483297195916649 , 1.479235641539457 , 1.474742807432315 , 
 1.469822736842662 , 1.464479857501934 , 1.458718977640905 , 
 1.4525452816626 , 1.44596432547669 , 1.438982031499539 , 
 1.431604683324436 , 1.423838920066784 , 1.415691730389341 , 
 1.407170446212898 , 1.398282736118043 , 1.38903659844396 , 
 1.379440354090461 , 1.369502639029735 , 1.359232396534563 , 
 1.348638869129968 , 1.337731590275575 , 1.326520375786132 , 
 1.315015314997945 , 1.303226761689157 , 1.29116532476204 , 
 1.278841858695708 , 1.26626745377781 , 1.253453426124026 , 
 1.240411307494323 , 1.227152834915152 , 1.213689940116914 , 
 1.200034738796209 , 1.186199519712527 , 1.172196733629194 , 
 1.158038982108526 , 1.143739006171271 , 1.129309674830555 , 
 1.114763973510631 , 1.100114992360884 , 1.085375914475572 \right] 
\]
\end{eulerformula}
\begin{eulerprompt}
>function fx(t) &= r(t)*cos(t); $'fx(t)=fx(t)
\end{eulerprompt}
\begin{eulerformula}
\[
{\it fx}\left(\left[ 0 , 0.01 , 0.02 , 0.03 , 0.04 , 0.05 , 0.06 , 
 0.07 , 0.08 , 0.09 , 0.1 , 0.11 , 0.12 , 0.13 , 0.14 , 0.15 , 0.16
  , 0.17 , 0.18 , 0.19 , 0.2 , 0.21 , 0.2200000000000001 , 
 0.2300000000000001 , 0.2400000000000001 , 0.2500000000000001 , 
 0.2600000000000001 , 0.2700000000000001 , 0.2800000000000001 , 
 0.2900000000000001 , 0.3000000000000001 , 0.3100000000000001 , 
 0.3200000000000001 , 0.3300000000000001 , 0.3400000000000001 , 
 0.3500000000000001 , 0.3600000000000002 , 0.3700000000000002 , 
 0.3800000000000002 , 0.3900000000000002 , 0.4000000000000002 , 
 0.4100000000000002 , 0.4200000000000002 , 0.4300000000000002 , 
 0.4400000000000002 , 0.4500000000000002 , 0.4600000000000002 , 
 0.4700000000000003 , 0.4800000000000003 , 0.4900000000000003 , 
 0.5000000000000002 , 0.5100000000000002 , 0.5200000000000002 , 
 0.5300000000000002 , 0.5400000000000003 , 0.5500000000000003 , 
 0.5600000000000003 , 0.5700000000000003 , 0.5800000000000003 , 
 0.5900000000000003 , 0.6000000000000003 , 0.6100000000000003 , 
 0.6200000000000003 , 0.6300000000000003 , 0.6400000000000003 , 
 0.6500000000000004 , 0.6600000000000004 , 0.6700000000000004 , 
 0.6800000000000004 , 0.6900000000000004 , 0.7000000000000004 , 
 0.7100000000000004 , 0.7200000000000004 , 0.7300000000000004 , 
 0.7400000000000004 , 0.7500000000000004 , 0.7600000000000005 , 
 0.7700000000000005 , 0.7800000000000005 , 0.7900000000000005 , 
 0.8000000000000005 , 0.8100000000000005 , 0.8200000000000005 , 
 0.8300000000000005 , 0.8400000000000005 , 0.8500000000000005 , 
 0.8600000000000005 , 0.8700000000000006 , 0.8800000000000006 , 
 0.8900000000000006 , 0.9000000000000006 , 0.9100000000000006 , 
 0.9200000000000006 , 0.9300000000000006 , 0.9400000000000006 , 
 0.9500000000000006 , 0.9600000000000006 , 0.9700000000000006 , 
 0.9800000000000006 , 0.9900000000000007 \right] \right)=\left[ 1 , 
 1.014947000636657 , 1.029776013705529 , 1.044469087191079 , 
 1.059008331806833 , 1.073375947255439 , 1.087554248364218 , 
 1.101525691055367 , 1.11527289811021 , 1.128778684687222 , 
 1.142026083553954 , 1.154998369993414 , 1.16767908634602 , 
 1.180052066148761 , 1.192101457833886 , 1.203811747950136 , 
 1.215167783870255 , 1.226154795949382 , 1.236758419099762 , 
 1.246964713748154 , 1.256760186143285 , 1.266131807981756 , 
 1.275067035321848 , 1.283553826755846 , 1.29158066081265 , 
 1.29913655256367 , 1.306211069406282 , 1.312794346000405 , 
 1.318877098335118 , 1.324450636903608 , 1.329506878966172 , 
 1.334038359882425 , 1.338038243495345 , 1.341500331551311 , 
 1.344419072141793 , 1.346789567153917 , 1.348607578718725 , 
 1.349869534647481 , 1.350572532848044 , 1.350714344714907 , 
 1.350293417488142 , 1.349308875578123 , 1.347760520854542 , 
 1.345648831899879 , 1.342974962229111 , 1.339740737479097 , 
 1.335948651572729 , 1.331601861864506 , 1.326704183275865 , 
 1.321260081430156 , 1.315274664798767 , 1.308753675871437 , 
 1.301703481365363 , 1.294131061489226 , 1.286043998279732 , 
 1.277450463029762 , 1.268359202828647 , 1.25877952623647 , 
 1.248721288115691 , 1.238194873644713 , 1.227211181539273 , 
 1.215781606508839 , 1.203918020976346 , 1.191632756090801 , 
 1.17893858206338 , 1.165848687858719 , 1.152376660274093 , 
 1.138536462440146 , 1.124342411777761 , 1.10980915744646 , 
 1.094951657320579 , 1.079785154530145 , 1.064325153604093 , 
 1.04858739625406 , 1.032587836837555 , 1.0163426175398 , 
 0.999868043313951 , 0.9831805566197906 , 0.9662967120012925 , 
 0.9492331505436565 , 0.932006574250646 , 0.9146337203831 , 
 0.897131335799599 , 0.8795161513401855 , 0.8618048562939812 , 
 0.8440140729913906 , 0.8261603315613344 , 0.8082600448937051 , 
 0.7903294838468643 , 0.7723847527396025 , 0.754441765166499 , 
 0.7365162201750889 , 0.7186235788426429 , 0.7007790412897039 , 
 0.6829975241668103 , 0.6652936386500562 , 0.6476816689803099 , 
 0.6301755515800127 , 0.6127888547805567 , 0.595534759192214 \right] 
\]
\end{eulerformula}
\begin{eulerprompt}
>function fy(t) &= r(t)*sin(t); $'fy(t)=fy(t)
\end{eulerprompt}
\begin{eulerformula}
\[
{\it fy}\left(\left[ 0 , 0.01 , 0.02 , 0.03 , 0.04 , 0.05 , 0.06 , 
 0.07 , 0.08 , 0.09 , 0.1 , 0.11 , 0.12 , 0.13 , 0.14 , 0.15 , 0.16
  , 0.17 , 0.18 , 0.19 , 0.2 , 0.21 , 0.2200000000000001 , 
 0.2300000000000001 , 0.2400000000000001 , 0.2500000000000001 , 
 0.2600000000000001 , 0.2700000000000001 , 0.2800000000000001 , 
 0.2900000000000001 , 0.3000000000000001 , 0.3100000000000001 , 
 0.3200000000000001 , 0.3300000000000001 , 0.3400000000000001 , 
 0.3500000000000001 , 0.3600000000000002 , 0.3700000000000002 , 
 0.3800000000000002 , 0.3900000000000002 , 0.4000000000000002 , 
 0.4100000000000002 , 0.4200000000000002 , 0.4300000000000002 , 
 0.4400000000000002 , 0.4500000000000002 , 0.4600000000000002 , 
 0.4700000000000003 , 0.4800000000000003 , 0.4900000000000003 , 
 0.5000000000000002 , 0.5100000000000002 , 0.5200000000000002 , 
 0.5300000000000002 , 0.5400000000000003 , 0.5500000000000003 , 
 0.5600000000000003 , 0.5700000000000003 , 0.5800000000000003 , 
 0.5900000000000003 , 0.6000000000000003 , 0.6100000000000003 , 
 0.6200000000000003 , 0.6300000000000003 , 0.6400000000000003 , 
 0.6500000000000004 , 0.6600000000000004 , 0.6700000000000004 , 
 0.6800000000000004 , 0.6900000000000004 , 0.7000000000000004 , 
 0.7100000000000004 , 0.7200000000000004 , 0.7300000000000004 , 
 0.7400000000000004 , 0.7500000000000004 , 0.7600000000000005 , 
 0.7700000000000005 , 0.7800000000000005 , 0.7900000000000005 , 
 0.8000000000000005 , 0.8100000000000005 , 0.8200000000000005 , 
 0.8300000000000005 , 0.8400000000000005 , 0.8500000000000005 , 
 0.8600000000000005 , 0.8700000000000006 , 0.8800000000000006 , 
 0.8900000000000006 , 0.9000000000000006 , 0.9100000000000006 , 
 0.9200000000000006 , 0.9300000000000006 , 0.9400000000000006 , 
 0.9500000000000006 , 0.9600000000000006 , 0.9700000000000006 , 
 0.9800000000000006 , 0.9900000000000007 \right] \right)=\left[ 0 , 
 0.01014980833556662 , 0.02059826678292271 , 0.03134347622283015 , 
 0.04238293991838228 , 0.05371356612987439 , 0.06533167172990376 , 
 0.07723298681299934 , 0.08941266029246918 , 0.1018652664755576 , 
 0.1145848126064173 , 0.1275647473648353 , 0.1407979703071057 , 
 0.1542768422339107 , 0.1679931964685752 , 0.1819383510275811 , 
 0.1961031216637831 , 0.2104778357613507 , 0.2250523470600841 , 
 0.2398160511854019 , 0.2547579019589912 , 0.2698664284638497 , 
 0.2851297528362152 , 0.3005356087557041 , 0.3160713606038417 , 
 0.3317240232600813 , 0.3474802825033731 , 0.3633265159863522 , 
 0.3792488147482899 , 0.3952330052320643 , 0.411264671769591 , 
 0.4273291794993832 , 0.4434116976792021 , 0.4594972233561165 , 
 0.4755706053556919 , 0.4916165685515136 , 0.5076197383757777 , 
 0.5235646655312819 , 0.5394358508648145 , 0.5552177703616642 , 
 0.5708949002207642 , 0.5864517419698421 , 0.6018728475798654 , 
 0.6171428445380648 , 0.6322464608388652 , 0.6471685498521687 , 
 0.6618941150286309 , 0.6764083344018014 , 0.6906965848473219 , 
 0.704744466059751 , 0.7185378242080237 , 0.7320627752310482 , 
 0.7453057277355214 , 0.7582534054586558 , 0.7708928692592016 , 
 0.7832115386008901 , 0.7951972124932317 , 0.8068380898554457 , 
 0.8181227892702304 , 0.8290403680950348 , 0.8395803408995157 , 
 0.8497326971989371 , 0.8594879184543822 , 0.8688369943118147 , 
 0.877771438053233 , 0.8862833012344233 , 0.894365187485098 , 
 0.9020102654485477 , 0.9092122808393135 , 0.91596556759876 , 
 0.9222650581299157 , 0.9281062925943645 , 0.9334854272555032 , 
 0.9383992418539865 , 0.9428451460027243 , 0.9468211845903713 , 
 0.9503260421838114 , 0.9533590464217597 , 0.9559201703932094 , 
 0.9580100339960551 , 0.9596299042728891 , 0.9607816947225576 , 
 0.9614679635877484 , 0.9616919111204768 , 0.9614573758289937 , 
 0.9607688297112769 , 0.9596313724818526 , 0.9580507248003547 , 
 0.9560332205117796 , 0.9535857979100135 , 0.950715990037748 , 
 0.9474319140374602 , 0.9437422595696462 , 0.9396562763159917 , 
 0.9351837605866338 , 0.9303350410521015 , 0.9251209636219332 , 
 0.9195528754933222 , 0.9136426083945087 , 0.9074024610488752
  \right] 
\]
\end{eulerformula}
\begin{eulerprompt}
>function ds(t) &= trigreduce(radcan(sqrt(diff(fx(t),t)^2+diff(fy(t),t)^2))); $'ds(t)=ds(t)
\end{eulerprompt}
\begin{euleroutput}
  Maxima said:
  diff: second argument must be a variable; found errexp1
   -- an error. To debug this try: debugmode(true);
  
  Error in:
  ... e(radcan(sqrt(diff(fx(t),t)^2+diff(fy(t),t)^2))); $'ds(t)=ds(t ...
                                                       ^
\end{euleroutput}
\begin{eulerprompt}
>$integrate(ds(x),x,0,2*pi) //panjang (keliling) kurva
\end{eulerprompt}
\begin{eulerformula}
\[
\int_{0}^{2\,\pi}{{\it ds}\left(x\right)\;dx}
\]
\end{eulerformula}
\begin{eulercomment}
Maxima gagal melakukan perhitungan eksak integral tersebut.\\
Berikut kita hitung integralnya secara umerik dengan perintah EMT
\end{eulercomment}
\begin{eulerprompt}
>integrate("ds(x)",0,2*pi)
\end{eulerprompt}
\begin{euleroutput}
  Function ds not found.
  Try list ... to find functions!
  Error in expression: ds(x)
  %mapexpression1:
      return expr(x,args());
  Error in map.
  %evalexpression:
      if maps then return %mapexpression1(x,f$;args());
  gauss:
      if maps then y=%evalexpression(f$,a+h-(h*xn)',maps;args());
  adaptivegauss:
      t1=gauss(f$,c,c+h;args(),=maps);
  Try "trace errors" to inspect local variables after errors.
  integrate:
      return adaptivegauss(f$,a,b,eps*1000;args(),=maps);
\end{euleroutput}
\begin{eulercomment}
Spiral Logaritmik\\
\end{eulercomment}
\begin{eulerformula}
\[
x=e^{ax}\cos(x),y=e^{ax}\sin(x)
\]
\end{eulerformula}
\begin{eulerprompt}
>a=0.1; plot2d("exp(a*x)*cos(x)","exp(a*x)*sin(x)",r=2,xmin=0,xmax=2*pi):
\end{eulerprompt}
\eulerimg{17}{images/Pekan 9-10_Fanny Erina Dewi_22305141005_EMT00-Kalkulus_Aplikom-105.png}
\begin{eulerprompt}
>&kill(a) // hapus expresi a
\end{eulerprompt}
\begin{euleroutput}
  
                                   done
  
\end{euleroutput}
\begin{eulerprompt}
>function fx(t) &= exp(a*t)*cos(t); $'fx(t)=fx(t)
\end{eulerprompt}
\begin{eulerformula}
\[
{\it fx}\left(\left[ 0 , 0.01 , 0.02 , 0.03 , 0.04 , 0.05 , 0.06 , 
 0.07 , 0.08 , 0.09 , 0.1 , 0.11 , 0.12 , 0.13 , 0.14 , 0.15 , 0.16
  , 0.17 , 0.18 , 0.19 , 0.2 , 0.21 , 0.2200000000000001 , 
 0.2300000000000001 , 0.2400000000000001 , 0.2500000000000001 , 
 0.2600000000000001 , 0.2700000000000001 , 0.2800000000000001 , 
 0.2900000000000001 , 0.3000000000000001 , 0.3100000000000001 , 
 0.3200000000000001 , 0.3300000000000001 , 0.3400000000000001 , 
 0.3500000000000001 , 0.3600000000000002 , 0.3700000000000002 , 
 0.3800000000000002 , 0.3900000000000002 , 0.4000000000000002 , 
 0.4100000000000002 , 0.4200000000000002 , 0.4300000000000002 , 
 0.4400000000000002 , 0.4500000000000002 , 0.4600000000000002 , 
 0.4700000000000003 , 0.4800000000000003 , 0.4900000000000003 , 
 0.5000000000000002 , 0.5100000000000002 , 0.5200000000000002 , 
 0.5300000000000002 , 0.5400000000000003 , 0.5500000000000003 , 
 0.5600000000000003 , 0.5700000000000003 , 0.5800000000000003 , 
 0.5900000000000003 , 0.6000000000000003 , 0.6100000000000003 , 
 0.6200000000000003 , 0.6300000000000003 , 0.6400000000000003 , 
 0.6500000000000004 , 0.6600000000000004 , 0.6700000000000004 , 
 0.6800000000000004 , 0.6900000000000004 , 0.7000000000000004 , 
 0.7100000000000004 , 0.7200000000000004 , 0.7300000000000004 , 
 0.7400000000000004 , 0.7500000000000004 , 0.7600000000000005 , 
 0.7700000000000005 , 0.7800000000000005 , 0.7900000000000005 , 
 0.8000000000000005 , 0.8100000000000005 , 0.8200000000000005 , 
 0.8300000000000005 , 0.8400000000000005 , 0.8500000000000005 , 
 0.8600000000000005 , 0.8700000000000006 , 0.8800000000000006 , 
 0.8900000000000006 , 0.9000000000000006 , 0.9100000000000006 , 
 0.9200000000000006 , 0.9300000000000006 , 0.9400000000000006 , 
 0.9500000000000006 , 0.9600000000000006 , 0.9700000000000006 , 
 0.9800000000000006 , 0.9900000000000007 \right] \right)=\left[ 1 , 
 0.9999500004166653\,e^{0.01\,a} , 0.9998000066665778\,e^{0.02\,a} , 
 0.9995500337489875\,e^{0.03\,a} , 0.9992001066609779\,e^{0.04\,a} , 
 0.9987502603949663\,e^{0.05\,a} , 0.9982005399352042\,e^{0.06\,a} , 
 0.9975510002532796\,e^{0.07\,a} , 0.9968017063026194\,e^{0.08\,a} , 
 0.9959527330119943\,e^{0.09\,a} , 0.9950041652780258\,e^{0.1\,a} , 
 0.9939560979566968\,e^{0.11\,a} , 0.9928086358538663\,e^{0.12\,a} , 
 0.9915618937147881\,e^{0.13\,a} , 0.9902159962126372\,e^{0.14\,a} , 
 0.9887710779360422\,e^{0.15\,a} , 0.9872272833756269\,e^{0.16\,a} , 
 0.9855847669095608\,e^{0.17\,a} , 0.9838436927881214\,e^{0.18\,a} , 
 0.9820042351172703\,e^{0.19\,a} , 0.9800665778412416\,e^{0.2\,a} , 
 0.9780309147241483\,e^{0.21\,a} , 0.9758974493306055\,e^{
 0.2200000000000001\,a} , 0.9736663950053748\,e^{0.2300000000000001\,
 a} , 0.9713379748520296\,e^{0.2400000000000001\,a} , 
 0.9689124217106447\,e^{0.2500000000000001\,a} , 0.9663899781345132\,
 e^{0.2600000000000001\,a} , 0.9637708963658905\,e^{
 0.2700000000000001\,a} , 0.9610554383107709\,e^{0.2800000000000001\,
 a} , 0.9582438755126972\,e^{0.2900000000000001\,a} , 
 0.955336489125606\,e^{0.3000000000000001\,a} , 0.9523335698857134\,e
 ^{0.3100000000000001\,a} , 0.9492354180824408\,e^{0.3200000000000001
 \,a} , 0.9460423435283869\,e^{0.3300000000000001\,a} , 
 0.9427546655283462\,e^{0.3400000000000001\,a} , 0.9393727128473789\,
 e^{0.3500000000000001\,a} , 0.9358968236779348\,e^{
 0.3600000000000002\,a} , 0.9323273456060344\,e^{0.3700000000000002\,
 a} , 0.9286646355765101\,e^{0.3800000000000002\,a} , 
 0.924909059857313\,e^{0.3900000000000002\,a} , 0.921060994002885\,e
 ^{0.4000000000000002\,a} , 0.917120822816605\,e^{0.4100000000000002
 \,a} , 0.9130889403123081\,e^{0.4200000000000002\,a} , 
 0.9089657496748851\,e^{0.4300000000000002\,a} , 0.9047516632199634\,
 e^{0.4400000000000002\,a} , 0.9004471023526768\,e^{
 0.4500000000000002\,a} , 0.8960524975255252\,e^{0.4600000000000002\,
 a} , 0.8915682881953289\,e^{0.4700000000000003\,a} , 
 0.886994922779284\,e^{0.4800000000000003\,a} , 0.8823328586101213\,e
 ^{0.4900000000000003\,a} , 0.8775825618903726\,e^{0.5000000000000002
 \,a} , 0.8727445076457512\,e^{0.5100000000000002\,a} , 
 0.8678191796776498\,e^{0.5200000000000002\,a} , 0.8628070705147609\,
 e^{0.5300000000000002\,a} , 0.857708681363824\,e^{0.5400000000000003
 \,a} , 0.8525245220595056\,e^{0.5500000000000003\,a} , 
 0.847255111013416\,e^{0.5600000000000003\,a} , 0.8419009751622686\,e
 ^{0.5700000000000003\,a} , 0.8364626499151868\,e^{0.5800000000000003
 \,a} , 0.8309406791001633\,e^{0.5900000000000003\,a} , 
 0.8253356149096781\,e^{0.6000000000000003\,a} , 0.8196480178454794\,
 e^{0.6100000000000003\,a} , 0.8138784566625338\,e^{
 0.6200000000000003\,a} , 0.8080275083121516\,e^{0.6300000000000003\,
 a} , 0.8020957578842924\,e^{0.6400000000000003\,a} , 
 0.7960837985490556\,e^{0.6500000000000004\,a} , 0.7899922314973649\,
 e^{0.6600000000000004\,a} , 0.783821665880849\,e^{0.6700000000000004
 \,a} , 0.7775727187509277\,e^{0.6800000000000004\,a} , 
 0.7712460149971063\,e^{0.6900000000000004\,a} , 0.7648421872844882\,
 e^{0.7000000000000004\,a} , 0.7583618759905079\,e^{
 0.7100000000000004\,a} , 0.7518057291408947\,e^{0.7200000000000004\,
 a} , 0.7451744023448701\,e^{0.7300000000000004\,a} , 
 0.7384685587295876\,e^{0.7400000000000004\,a} , 0.7316888688738206\,
 e^{0.7500000000000004\,a} , 0.7248360107409049\,e^{
 0.7600000000000005\,a} , 0.7179106696109431\,e^{0.7700000000000005\,
 a} , 0.7109135380122771\,e^{0.7800000000000005\,a} , 
 0.7038453156522357\,e^{0.7900000000000005\,a} , 0.696706709347165\,e
 ^{0.8000000000000005\,a} , 0.6894984329517466\,e^{0.8100000000000005
 \,a} , 0.6822212072876132\,e^{0.8200000000000005\,a} , 
 0.6748757600712667\,e^{0.8300000000000005\,a} , 0.6674628258413078\,
 e^{0.8400000000000005\,a} , 0.6599831458849817\,e^{
 0.8500000000000005\,a} , 0.6524374681640515\,e^{0.8600000000000005\,
 a} , 0.6448265472400008\,e^{0.8700000000000006\,a} , 
 0.6371511441985798\,e^{0.8800000000000006\,a} , 0.6294120265736964\,
 e^{0.8900000000000006\,a} , 0.6216099682706641\,e^{
 0.9000000000000006\,a} , 0.6137457494888111\,e^{0.9100000000000006\,
 a} , 0.6058201566434623\,e^{0.9200000000000006\,a} , 
 0.5978339822872978\,e^{0.9300000000000006\,a} , 0.5897880250310977\,
 e^{0.9400000000000006\,a} , 0.581683089463883\,e^{0.9500000000000006
 \,a} , 0.5735199860724561\,e^{0.9600000000000006\,a} , 
 0.5652995311603538\,e^{0.9700000000000006\,a} , 0.5570225467662168\,
 e^{0.9800000000000006\,a} , 0.548689860581587\,e^{0.9900000000000007
 \,a} \right] 
\]
\end{eulerformula}
\begin{eulerprompt}
>function fy(t) &= exp(a*t)*sin(t); $'fy(t)=fy(t)
\end{eulerprompt}
\begin{eulerformula}
\[
{\it fy}\left(\left[ 0 , 0.01 , 0.02 , 0.03 , 0.04 , 0.05 , 0.06 , 
 0.07 , 0.08 , 0.09 , 0.1 , 0.11 , 0.12 , 0.13 , 0.14 , 0.15 , 0.16
  , 0.17 , 0.18 , 0.19 , 0.2 , 0.21 , 0.2200000000000001 , 
 0.2300000000000001 , 0.2400000000000001 , 0.2500000000000001 , 
 0.2600000000000001 , 0.2700000000000001 , 0.2800000000000001 , 
 0.2900000000000001 , 0.3000000000000001 , 0.3100000000000001 , 
 0.3200000000000001 , 0.3300000000000001 , 0.3400000000000001 , 
 0.3500000000000001 , 0.3600000000000002 , 0.3700000000000002 , 
 0.3800000000000002 , 0.3900000000000002 , 0.4000000000000002 , 
 0.4100000000000002 , 0.4200000000000002 , 0.4300000000000002 , 
 0.4400000000000002 , 0.4500000000000002 , 0.4600000000000002 , 
 0.4700000000000003 , 0.4800000000000003 , 0.4900000000000003 , 
 0.5000000000000002 , 0.5100000000000002 , 0.5200000000000002 , 
 0.5300000000000002 , 0.5400000000000003 , 0.5500000000000003 , 
 0.5600000000000003 , 0.5700000000000003 , 0.5800000000000003 , 
 0.5900000000000003 , 0.6000000000000003 , 0.6100000000000003 , 
 0.6200000000000003 , 0.6300000000000003 , 0.6400000000000003 , 
 0.6500000000000004 , 0.6600000000000004 , 0.6700000000000004 , 
 0.6800000000000004 , 0.6900000000000004 , 0.7000000000000004 , 
 0.7100000000000004 , 0.7200000000000004 , 0.7300000000000004 , 
 0.7400000000000004 , 0.7500000000000004 , 0.7600000000000005 , 
 0.7700000000000005 , 0.7800000000000005 , 0.7900000000000005 , 
 0.8000000000000005 , 0.8100000000000005 , 0.8200000000000005 , 
 0.8300000000000005 , 0.8400000000000005 , 0.8500000000000005 , 
 0.8600000000000005 , 0.8700000000000006 , 0.8800000000000006 , 
 0.8900000000000006 , 0.9000000000000006 , 0.9100000000000006 , 
 0.9200000000000006 , 0.9300000000000006 , 0.9400000000000006 , 
 0.9500000000000006 , 0.9600000000000006 , 0.9700000000000006 , 
 0.9800000000000006 , 0.9900000000000007 \right] \right)=\left[ 0 , 
 0.009999833334166664\,e^{0.01\,a} , 0.01999866669333308\,e^{0.02\,a}
  , 0.02999550020249566\,e^{0.03\,a} , 0.03998933418663416\,e^{0.04\,
 a} , 0.04997916927067833\,e^{0.05\,a} , 0.0599640064794446\,e^{0.06
 \,a} , 0.06994284733753277\,e^{0.07\,a} , 0.0799146939691727\,e^{
 0.08\,a} , 0.08987854919801104\,e^{0.09\,a} , 0.09983341664682814\,e
 ^{0.1\,a} , 0.1097783008371748\,e^{0.11\,a} , 0.1197122072889193\,e
 ^{0.12\,a} , 0.1296341426196948\,e^{0.13\,a} , 0.1395431146442365\,e
 ^{0.14\,a} , 0.1494381324735992\,e^{0.15\,a} , 0.159318206614246\,e
 ^{0.16\,a} , 0.169182349066996\,e^{0.17\,a} , 0.1790295734258242\,e
 ^{0.18\,a} , 0.1888588949765006\,e^{0.19\,a} , 0.1986693307950612\,e
 ^{0.2\,a} , 0.2084598998460996\,e^{0.21\,a} , 0.2182296230808694\,e
 ^{0.2200000000000001\,a} , 0.2279775235351885\,e^{0.2300000000000001
 \,a} , 0.2377026264271347\,e^{0.2400000000000001\,a} , 
 0.247403959254523\,e^{0.2500000000000001\,a} , 0.2570805518921552\,e
 ^{0.2600000000000001\,a} , 0.2667314366888312\,e^{0.2700000000000001
 \,a} , 0.2763556485641138\,e^{0.2800000000000001\,a} , 
 0.2859522251048356\,e^{0.2900000000000001\,a} , 0.2955202066613397\,
 e^{0.3000000000000001\,a} , 0.3050586364434436\,e^{
 0.3100000000000001\,a} , 0.3145665606161179\,e^{0.3200000000000001\,
 a} , 0.3240430283948685\,e^{0.3300000000000001\,a} , 
 0.3334870921408145\,e^{0.3400000000000001\,a} , 0.3428978074554515\,
 e^{0.3500000000000001\,a} , 0.3522742332750901\,e^{
 0.3600000000000002\,a} , 0.3616154319649622\,e^{0.3700000000000002\,
 a} , 0.3709204694129828\,e^{0.3800000000000002\,a} , 
 0.3801884151231616\,e^{0.3900000000000002\,a} , 0.3894183423086507\,
 e^{0.4000000000000002\,a} , 0.3986093279844231\,e^{
 0.4100000000000002\,a} , 0.4077604530595704\,e^{0.4200000000000002\,
 a} , 0.416870802429211\,e^{0.4300000000000002\,a} , 
 0.4259394650659998\,e^{0.4400000000000002\,a} , 0.4349655341112304\,
 e^{0.4500000000000002\,a} , 0.44394810696552\,e^{0.4600000000000002
 \,a} , 0.4528862853790685\,e^{0.4700000000000003\,a} , 
 0.4617791755414831\,e^{0.4800000000000003\,a} , 0.4706258881711582\,
 e^{0.4900000000000003\,a} , 0.4794255386042032\,e^{
 0.5000000000000002\,a} , 0.4881772468829077\,e^{0.5100000000000002\,
 a} , 0.4968801378437369\,e^{0.5200000000000002\,a} , 
 0.5055333412048472\,e^{0.5300000000000002\,a} , 0.5141359916531133\,
 e^{0.5400000000000003\,a} , 0.5226872289306594\,e^{
 0.5500000000000003\,a} , 0.5311861979208836\,e^{0.5600000000000003\,
 a} , 0.5396320487339695\,e^{0.5700000000000003\,a} , 
 0.5480239367918738\,e^{0.5800000000000003\,a} , 0.556361022912784\,e
 ^{0.5900000000000003\,a} , 0.5646424733950356\,e^{0.6000000000000003
 \,a} , 0.5728674601004815\,e^{0.6100000000000003\,a} , 
 0.5810351605373053\,e^{0.6200000000000003\,a} , 0.5891447579422698\,
 e^{0.6300000000000003\,a} , 0.5971954413623923\,e^{
 0.6400000000000003\,a} , 0.6051864057360399\,e^{0.6500000000000004\,
 a} , 0.6131168519734341\,e^{0.6600000000000004\,a} , 
 0.6209859870365599\,e^{0.6700000000000004\,a} , 0.6287930240184688\,
 e^{0.6800000000000004\,a} , 0.6365371822219682\,e^{
 0.6900000000000004\,a} , 0.6442176872376913\,e^{0.7000000000000004\,
 a} , 0.651833771021537\,e^{0.7100000000000004\,a} , 
 0.6593846719714734\,e^{0.7200000000000004\,a} , 0.6668696350036982\,
 e^{0.7300000000000004\,a} , 0.6742879116281454\,e^{
 0.7400000000000004\,a} , 0.6816387600233345\,e^{0.7500000000000004\,
 a} , 0.6889214451105516\,e^{0.7600000000000005\,a} , 
 0.696135238627357\,e^{0.7700000000000005\,a} , 0.7032794192004105\,e
 ^{0.7800000000000005\,a} , 0.7103532724176082\,e^{0.7900000000000005
 \,a} , 0.7173560908995231\,e^{0.8000000000000005\,a} , 
 0.7242871743701429\,e^{0.8100000000000005\,a} , 0.7311458297268962\,
 e^{0.8200000000000005\,a} , 0.7379313711099631\,e^{
 0.8300000000000005\,a} , 0.7446431199708596\,e^{0.8400000000000005\,
 a} , 0.751280405140293\,e^{0.8500000000000005\,a} , 
 0.7578425628952773\,e^{0.8600000000000005\,a} , 0.7643289370255054\,
 e^{0.8700000000000006\,a} , 0.7707388788989696\,e^{
 0.8800000000000006\,a} , 0.7770717475268242\,e^{0.8900000000000006\,
 a} , 0.7833269096274837\,e^{0.9000000000000006\,a} , 
 0.7895037396899508\,e^{0.9100000000000006\,a} , 0.7956016200363664\,
 e^{0.9200000000000006\,a} , 0.8016199408837775\,e^{
 0.9300000000000006\,a} , 0.8075581004051147\,e^{0.9400000000000006\,
 a} , 0.8134155047893741\,e^{0.9500000000000006\,a} , 
 0.8191915683009986\,e^{0.9600000000000006\,a} , 0.8248857133384504\,
 e^{0.9700000000000006\,a} , 0.8304973704919708\,e^{
 0.9800000000000006\,a} , 0.8360259786005209\,e^{0.9900000000000007\,
 a} \right] 
\]
\end{eulerformula}
\begin{eulerprompt}
>function df(t) &= trigreduce(radcan(sqrt(diff(fx(t),t)^2+diff(fy(t),t)^2))); $'df(t)=df(t)
\end{eulerprompt}
\begin{euleroutput}
  Maxima said:
  diff: second argument must be a variable; found errexp1
   -- an error. To debug this try: debugmode(true);
  
  Error in:
  ... e(radcan(sqrt(diff(fx(t),t)^2+diff(fy(t),t)^2))); $'df(t)=df(t ...
                                                       ^
\end{euleroutput}
\begin{eulerprompt}
>S &=integrate(df(t),t,0,2*%pi); $S // panjang kurva (spiral)
\end{eulerprompt}
\begin{euleroutput}
  Maxima said:
  expt: undefined: 0 to a negative exponent.
  #0: df(x=[0,0.01,0.02,0.03,0.04,0.05,0.06,0.07,0.08,0.09,0.1,0.11,0.12,0.13,0.14,0.15,0.16,0.17,0.18,0.19,0.2...)
   -- an error. To debug this try: debugmode(true);
  
  Error in:
  S &=integrate(df(t),t,0,2*%pi); $S // panjang kurva (spiral) ...
                                ^
\end{euleroutput}
\begin{eulerprompt}
>S(a=0.1) // Panjang kurva untuk a=0.1
\end{eulerprompt}
\begin{euleroutput}
  Function S not found.
  Try list ... to find functions!
  Error in:
  S(a=0.1) // Panjang kurva untuk a=0.1 ...
          ^
\end{euleroutput}
\begin{eulercomment}
Berikut adalah contoh menghitung panjang parabola.
\end{eulercomment}
\begin{eulerprompt}
>plot2d("x^2",xmin=-1,xmax=1):
\end{eulerprompt}
\eulerimg{17}{images/Pekan 9-10_Fanny Erina Dewi_22305141005_EMT00-Kalkulus_Aplikom-108.png}
\begin{eulerprompt}
>$showev('integrate(sqrt(1+diff(x^2,x)^2),x,-1,1))
\end{eulerprompt}
\begin{eulerformula}
\[
\int_{-1}^{1}{\sqrt{4\,x^2+1}\;dx}=\frac{{\rm asinh}\; 2+2\,\sqrt{5
 }}{2}
\]
\end{eulerformula}
\begin{eulerprompt}
>$float(%)
\end{eulerprompt}
\begin{eulerformula}
\[
\int_{-1.0}^{1.0}{\sqrt{4.0\,x^2+1.0}\;dx}=2.957885715089195
\]
\end{eulerformula}
\begin{eulerprompt}
>x=-1:0.2:1; y=x^2; plot2d(x,y); ...
>plot2d(x,y,points=1,style="o#",add=1):
\end{eulerprompt}
\eulerimg{17}{images/Pekan 9-10_Fanny Erina Dewi_22305141005_EMT00-Kalkulus_Aplikom-111.png}
\begin{eulercomment}
Panjang tersebut dapat dihampiri dengan menggunakan jumlah panjang
ruas-ruas garis yang menghubungkan titik-titik pada parabola tersebut.
\end{eulercomment}
\begin{eulerprompt}
>i=1:cols(x)-1; sum(sqrt((x[i+1]-x[i])^2+(y[i+1]-y[i])^2))
\end{eulerprompt}
\begin{euleroutput}
  2.95191957027
\end{euleroutput}
\begin{eulercomment}
Hasilnya mendekati panjang yang dihitung secara eksak. Untuk
mendapatkan hampiran yang cukup akurat, jarak antar titik dapat
diperkecil, misalnya 0.1, 0.05, 0.01, dan seterusnya. Cobalah Anda
ulangi perhitungannya dengan nilai-nilai tersebut.


\begin{eulercomment}
\eulerheading{Koordinat Kartesius}
\begin{eulercomment}
Berikut diberikan contoh perhitungan panjang kurva menggunakan
koordinat Kartesius. Kita akan hitung panjang kurva dengan persamaan
implisit:\\
\end{eulercomment}
\begin{eulerformula}
\[
x^3+y^3-3xy=0
\]
\end{eulerformula}
\begin{eulerprompt}
>z &= x^3+y^3-3*x*y; $z
\end{eulerprompt}
\begin{eulerformula}
\[
y^3-3\,x\,y+x^3
\]
\end{eulerformula}
\begin{eulerprompt}
>plot2d(z,r=2,level=0,n=100):
\end{eulerprompt}
\eulerimg{17}{images/Pekan 9-10_Fanny Erina Dewi_22305141005_EMT00-Kalkulus_Aplikom-113.png}
\begin{eulercomment}
Kita tertarik pada kurva di kuadran pertama.
\end{eulercomment}
\begin{eulerprompt}
>plot2d(z,a=0,b=2,c=0,d=2,level=[-10;0],n=100,contourwidth=3,style="/"):
\end{eulerprompt}
\eulerimg{17}{images/Pekan 9-10_Fanny Erina Dewi_22305141005_EMT00-Kalkulus_Aplikom-114.png}
\begin{eulercomment}
Kita selesaikan persamaannya untuk x.\\
tama.
\end{eulercomment}
\begin{eulerprompt}
>$z with y=l*x, sol &= solve(%,x); $sol
\end{eulerprompt}
\begin{eulerformula}
\[
l^3\,x^3+x^3-3\,l\,x^2
\]
\end{eulerformula}
\begin{eulerformula}
\[
\left[ x=\frac{3\,l}{l^3+1} , x=0 \right] 
\]
\end{eulerformula}
\begin{eulercomment}
Kita gunakan solusi tersebut untuk mendefinisikan fungsi dengan
Maxima.
\end{eulercomment}
\begin{eulerprompt}
>function f(l) &= rhs(sol[1]); $'f(l)=f(l)
\end{eulerprompt}
\begin{eulerformula}
\[
f\left(l\right)=\frac{3\,l}{l^3+1}
\]
\end{eulerformula}
\begin{eulercomment}
Fungsi tersebut juga dapat digunaka untuk menggambar kurvanya. Ingat,
bahwa fungsi tersebut adalah nilai x dan nilai y=l*x, yakni x=f(l) dan
y=l*f(l).
\end{eulercomment}
\begin{eulerprompt}
>plot2d(&f(x),&x*f(x),xmin=-0.5,xmax=2,a=0,b=2,c=0,d=2,r=1.5):
\end{eulerprompt}
\eulerimg{17}{images/Pekan 9-10_Fanny Erina Dewi_22305141005_EMT00-Kalkulus_Aplikom-118.png}
\begin{eulercomment}
Elemen panjang kurva adalah:\\
\end{eulercomment}
\begin{eulerformula}
\[
ds=\sqrt{f'(l)^2+(lf'(l)+f(l))^2}
\]
\end{eulerformula}
\begin{eulerprompt}
>function ds(l) &= ratsimp(sqrt(diff(f(l),l)^2+diff(l*f(l),l)^2)); $'ds(l)=ds(l)
\end{eulerprompt}
\begin{eulerformula}
\[
{\it ds}\left(l\right)=\frac{\sqrt{9\,l^8+36\,l^6-36\,l^5-36\,l^3+
 36\,l^2+9}}{\sqrt{l^{12}+4\,l^9+6\,l^6+4\,l^3+1}}
\]
\end{eulerformula}
\begin{eulerprompt}
>$integrate(ds(l),l,0,1)
\end{eulerprompt}
\begin{eulerformula}
\[
\int_{0}^{1}{\frac{\sqrt{9\,l^8+36\,l^6-36\,l^5-36\,l^3+36\,l^2+9}
 }{\sqrt{l^{12}+4\,l^9+6\,l^6+4\,l^3+1}}\;dl}
\]
\end{eulerformula}
\begin{eulercomment}
Integral tersebut tidak dapat dihitung secara eksak menggunakan
Maxima. Kita hitung integral etrsebut secara numerik dengan Euler.
Karena kurva simetris, kita hitung untuk nilai variabel integrasi dari
0 sampai 1, kemudian hasilnya dikalikan 2.
\end{eulercomment}
\begin{eulerprompt}
>2*integrate("ds(x)",0,1)
\end{eulerprompt}
\begin{euleroutput}
  4.91748872168
\end{euleroutput}
\begin{eulerprompt}
>2*romberg(&ds(x),0,1)// perintah Euler lain untuk menghitung nilai hampiran integral yang
\end{eulerprompt}
\begin{euleroutput}
  4.91748872168
\end{euleroutput}
\begin{eulercomment}
Perhitungan di datas dapat dilakukan untuk sebarang fungsi x dan y
dengan mendefinisikan fungsi EMT, misalnya kita beri nama
panjangkurva. Fungsi ini selalu memanggil Maxima untuk menurunkan
fungsi yang diberikan.
\end{eulercomment}
\begin{eulerprompt}
>function panjangkurva(fx,fy,a,b) ...
\end{eulerprompt}
\begin{eulerudf}
  ds=mxm("sqrt(diff(@fx,x)^2+diff(@fy,x)^2)");
  return romberg(ds,a,b);
  endfunction
\end{eulerudf}
\begin{eulerprompt}
>panjangkurva("x","x^2",-1,1) // cek untuk menghitung panjang kurva parabola sebelumnya
\end{eulerprompt}
\begin{euleroutput}
  2.95788571509
\end{euleroutput}
\begin{eulercomment}
Bandingkan dengan nilai eksak di atas.
\end{eulercomment}
\begin{eulerprompt}
>2*panjangkurva(mxm("f(x)"),mxm("x*f(x)"),0,1) // cek contoh terakhir, bandingkan hasilnya
\end{eulerprompt}
\begin{euleroutput}
  4.91748872168
\end{euleroutput}
\begin{eulercomment}
Kita hitung panjang spiral Archimides berikut ini dengan fungsi
tersebut.
\end{eulercomment}
\begin{eulerprompt}
>plot2d("x*cos(x)","x*sin(x)",xmin=0,xmax=2*pi,square=1):
\end{eulerprompt}
\eulerimg{17}{images/Pekan 9-10_Fanny Erina Dewi_22305141005_EMT00-Kalkulus_Aplikom-121.png}
\begin{eulerprompt}
>panjangkurva("x*cos(x)","x*sin(x)",0,2*pi)
\end{eulerprompt}
\begin{euleroutput}
  21.2562941482
\end{euleroutput}
\begin{eulercomment}
Berikut kita definisikan fungsi yang sama namun dengan Maxima, untuk
perhitungan eksak.
\end{eulercomment}
\begin{eulerprompt}
>&kill(ds,x,fx,fy)
\end{eulerprompt}
\begin{euleroutput}
  
                                   done
  
\end{euleroutput}
\begin{eulerprompt}
>function ds(fx,fy) &&= sqrt(diff(fx,x)^2+diff(fy,x)^2)
\end{eulerprompt}
\begin{euleroutput}
  
                             2              2
                    sqrt(diff (fy, x) + diff (fx, x))
  
\end{euleroutput}
\begin{eulerprompt}
>sol &= ds(x*cos(x),x*sin(x)); $sol // Kita gunakan untuk menghitung panjang kurva terakhi
\end{eulerprompt}
\begin{eulerformula}
\[
\sqrt{\left(\cos x-x\,\sin x\right)^2+\left(\sin x+x\,\cos x\right)
 ^2}
\]
\end{eulerformula}
\begin{eulerprompt}
>$sol | trigreduce | expand, $integrate(%,x,0,2*pi), %()
\end{eulerprompt}
\begin{eulerformula}
\[
\sqrt{x^2+1}
\]
\end{eulerformula}
\begin{eulerformula}
\[
\frac{{\rm asinh}\; \left(2\,\pi\right)+2\,\pi\,\sqrt{4\,\pi^2+1}}{
 2}
\]
\end{eulerformula}
\begin{euleroutput}
  21.2562941482
\end{euleroutput}
\begin{eulercomment}
Hasilnya sama dengan perhitungan menggunakan fungsi EMT.\\
Berikut adalah contoh lain penggunaan fungsi Maxima tersebut.
\end{eulercomment}
\begin{eulerprompt}
>plot2d("3*x^2-1","3*x^3-1",xmin=-1/sqrt(3),xmax=1/sqrt(3),square=1):
\end{eulerprompt}
\eulerimg{17}{images/Pekan 9-10_Fanny Erina Dewi_22305141005_EMT00-Kalkulus_Aplikom-125.png}
\begin{eulerprompt}
>sol &= radcan(ds(3*x^2-1,3*x^3-1)); $sol
\end{eulerprompt}
\begin{eulerformula}
\[
3\,x\,\sqrt{9\,x^2+4}
\]
\end{eulerformula}
\begin{eulerprompt}
>$showev('integrate(sol,x,0,1/sqrt(3))), $2*float(%) // panjang kurva di atas
\end{eulerprompt}
\begin{eulerformula}
\[
3\,\int_{0}^{\frac{1}{\sqrt{3}}}{x\,\sqrt{9\,x^2+4}\;dx}=3\,\left(
 \frac{7^{\frac{3}{2}}}{27}-\frac{8}{27}\right)
\]
\end{eulerformula}
\begin{eulerformula}
\[
6.0\,\int_{0.0}^{0.5773502691896258}{x\,\sqrt{9.0\,x^2+4.0}\;dx}=
 2.337835372767141
\]
\end{eulerformula}
\begin{eulercomment}
\begin{eulercomment}
\eulerheading{Sikloid }
\begin{eulercomment}
Berikut kita akan menghitung panjang kurva lintasan (sikloid) suatu
titik pada lingkaran yang berputar ke kanan pada permukaan datar.
Misalkan jari-jari lingkaran tersebut adalah r. Posisi titik pusat
lingkaran pada\\
saat t adalah:

\end{eulercomment}
\begin{eulerformula}
\[
(rt,r)
\]
\end{eulerformula}
\begin{eulercomment}
Misalkan posisi titik pada lingkaran tersebut mula-mula (0,0) dan
posisinya pada saat t adalah:

\end{eulercomment}
\begin{eulerformula}
\[
(r(t-\sin(t)),r(1-\cos(t)))
\]
\end{eulerformula}
\begin{eulercomment}
Berikut kita plot lintasan tersebut dan beberapa posisi lingkaran
ketika t=0, t=pi/2, t=r*pi.
\end{eulercomment}
\begin{eulerprompt}
>x &= r*(t-sin(t))
\end{eulerprompt}
\begin{euleroutput}
  
          [0, 1.66665833335744e-7 r, 1.33330666692022e-6 r, 
  4.499797504338432e-6 r, 1.066581336583994e-5 r, 
  2.083072932167196e-5 r, 3.599352055540239e-5 r, 
  5.71526624672386e-5 r, 8.530603082730626e-5 r, 
  1.214508019889565e-4 r, 1.665833531718508e-4 r, 
  2.216991628251896e-4 r, 2.877927110806339e-4 r, 
  3.658573803051457e-4 r, 4.568853557635201e-4 r, 
  5.618675264007778e-4 r, 6.817933857540259e-4 r, 
  8.176509330039827e-4 r, 9.704265741758145e-4 r, 
  0.001141105023499428 r, 0.001330669204938795 r, 
  0.001540100153900437 r, 0.001770376919130678 r, 
  0.002022476464811601 r, 0.002297373572865413 r, 
  0.002596040745477063 r, 0.002919448107844891 r, 
  0.003268563311168871 r, 0.003644351435886262 r, 
  0.004047774895164447 r, 0.004479793338660443 r, 0.0049413635565565 r, 
  0.005433439383882244 r, 0.005956971605131645 r, 
  0.006512907859185624 r, 0.007102192544548636 r, 
  0.007725766724910044 r, 0.00838456803503801 r, 
  0.009079530587017326 r, 0.009811584876838586 r, 0.0105816576913495 r, 
  0.01139067201557714 r, 0.01223954694042984 r, 0.01312919757078923 r, 
  0.01406053493400045 r, 0.01503446588876983 r, 0.01605189303448024 r, 
  0.01711371462093175 r, 0.01822082445851714 r, 0.01937411182884202 r, 
  0.02057446139579705 r, 0.02182275311709253 r, 0.02311986215626333 r, 
  0.02446665879515308 r, 0.02586400834688696 r, 0.02731277106934082 r, 
  0.02881380207911666 r, 0.03036795126603076 r, 0.03197606320812652 r, 
  0.0336389770872163 r, 0.03535752660496472 r, 0.03713253989951881 r, 
  0.03896483946269502 r, 0.0408552420577305 r, 0.04280455863760801 r, 
  0.04481359426396048 r, 0.04688314802656623 r, 0.04901401296344043 r, 
  0.05120697598153157 r, 0.05346281777803219 r, 0.05578231276230905 r, 
  0.05816622897846346 r, 0.06061532802852698 r, 0.0631303649963022 r, 
  0.06571208837185505 r, 0.06836123997666599 r, 0.07107855488944881 r, 
  0.07386476137264342 r, 0.07672058079958999 r, 0.07964672758239233 r, 
  0.08264390910047736 r, 0.0857128256298576 r, 0.08885417027310427 r, 
  0.09206862889003742 r, 0.09535688002914089 r, 0.0987195948597075 r, 
  0.1021574371047232 r, 0.1056710629744951 r, 0.1092611211010309 r, 
  0.1129282524731764 r, 0.1166730903725168 r, 0.1204962603100498 r, 
  0.1243983799636342 r, 0.1283800591162231 r, 0.1324418995948859 r, 
  0.1365844952106265 r, 0.140808431699002 r, 0.1451142866615502 r, 
  0.1495026295080298 r, 0.1539740213994798 r]
  
\end{euleroutput}
\begin{eulerprompt}
>y &= r*(1-cos(t))
\end{eulerprompt}
\begin{euleroutput}
  
          [0, 4.999958333473664e-5 r, 1.999933334222437e-4 r, 
  4.499662510124569e-4 r, 7.998933390220841e-4 r, 
  0.001249739605033717 r, 0.00179946006479581 r, 
  0.002448999746720415 r, 0.003198293697380561 r, 
  0.004047266988005727 r, 0.004995834721974179 r, 
  0.006043902043303184 r, 0.00719136414613375 r, 0.00843810628521191 r, 
  0.009784003787362772 r, 0.01122892206395776 r, 0.01277271662437307 r, 
  0.01441523309043924 r, 0.01615630721187855 r, 0.01799576488272969 r, 
  0.01993342215875837 r, 0.02196908527585173 r, 0.02410255066939448 r, 
  0.02633360499462523 r, 0.02866202514797045 r, 0.03108757828935527 r, 
  0.03361002186548678 r, 0.03622910363410947 r, 0.03894456168922911 r, 
  0.04175612448730281 r, 0.04466351087439402 r, 0.04766643011428662 r, 
  0.05076458191755917 r, 0.0539576564716131 r, 0.05724533447165381 r, 
  0.06062728715262111 r, 0.06410317632206519 r, 0.06767265439396564 r, 
  0.07133536442348987 r, 0.07509094014268702 r, 0.07893900599711501 r, 
  0.08287917718339499 r, 0.08691105968769186 r, 0.09103425032511492 r, 
  0.09524833678003664 r, 0.09955289764732322 r, 0.1039475024744748 r, 
  0.1084317118046711 r, 0.113005077220716 r, 0.1176671413898787 r, 
  0.1224174381096274 r, 0.1272554923542488 r, 0.1321808203223502 r, 
  0.1371929294852391 r, 0.1422913186361759 r, 0.1474754779404944 r, 
  0.152744888986584 r, 0.1580990248377314 r, 0.1635373500848132 r, 
  0.1690593208998367 r, 0.1746643850903219 r, 0.1803519821545206 r, 
  0.1861215433374662 r, 0.1919724916878484 r, 0.1979042421157076 r, 
  0.2039162014509444 r, 0.2100077685026351 r, 0.216178334119151 r, 
  0.2224272812490723 r, 0.2287539850028937 r, 0.2351578127155118 r, 
  0.2416381240094921 r, 0.2481942708591053 r, 0.2548255976551299 r, 
  0.2615314412704124 r, 0.2683111311261794 r, 0.2751639892590951 r, 
  0.2820893303890569 r, 0.2890864619877229 r, 0.2961546843477643 r, 
  0.3032932906528349 r, 0.3105015670482534 r, 0.3177787927123868 r, 
  0.3251242399287333 r, 0.3325371741586922 r, 0.3400168541150183 r, 
  0.3475625318359485 r, 0.3551734527599992 r, 0.3628488558014202 r, 
  0.3705879734263036 r, 0.3783900317293359 r, 0.3862542505111889 r, 
  0.3941798433565377 r, 0.4021660177127022 r, 0.4102119749689023 r, 
  0.418316910536117 r, 0.4264800139275439 r, 0.4347004688396462 r, 
  0.4429774532337832 r, 0.451310139418413 r]
  
\end{euleroutput}
\begin{eulercomment}
Berikut kita gambar sikloid untuk r=1.
\end{eulercomment}
\begin{eulerprompt}
>ex &= x-sin(x); ey &= 1-cos(x); aspect(1);
>plot2d(ex,ey,xmin=0,xmax=4pi,square=1); ...
>plot2d("2+cos(x)","1+sin(x)",xmin=0,xmax=2pi,>add,color=blue); ...
>plot2d([2,ex(2)],[1,ey(2)],color=red,>add); ...
>plot2d(ex(2),ey(2),>points,>add,color=red); ...
>plot2d("2pi+cos(x)","1+sin(x)",xmin=0,xmax=2pi,>add,color=blue); ...
>plot2d([2pi,ex(2pi)],[1,ey(2pi)],color=red,>add); ...
>plot2d(ex(2pi),ey(2pi),>points,>add,color=red):
\end{eulerprompt}
\begin{euleroutput}
  Error : [0,1.66665833335744e-7*r-sin(1.66665833335744e-7*r),1.33330666692022e-6*r-sin(1.33330666692022e-6*r),4.499797504338432e-6*r-sin(4.499797504338432e-6*r),1.066581336583994e-5*r-sin(1.066581336583994e-5*r),2.083072932167196e-5*r-sin(2.083072932167196e-5*r),3.599352055540239e-5*r-sin(3.599352055540239e-5*r),5.71526624672386e-5*r-sin(5.71526624672386e-5*r),8.530603082730626e-5*r-sin(8.530603082730626e-5*r),1.214508019889565e-4*r-sin(1.214508019889565e-4*r),1.665833531718508e-4*r-sin(1.665833531718508e-4*r),2.216991628251896e-4*r-sin(2.216991628251896e-4*r),2.877927110806339e-4*r-sin(2.877927110806339e-4*r),3.658573803051457e-4*r-sin(3.658573803051457e-4*r),4.5688535576352e-4*r-sin(4.5688535576352e-4*r),5.618675264007778e-4*r-sin(5.618675264007778e-4*r),6.817933857540259e-4*r-sin(6.817933857540259e-4*r),8.176509330039827e-4*r-sin(8.176509330039827e-4*r),9.704265741758145e-4*r-sin(9.704265741758145e-4*r),0.001141105023499428*r-sin(0.001141105023499428*r),0.001330669204938795*r-sin(0.001330669204938795*r),0.001540100153900437*r-sin(0.001540100153900437*r),0.001770376919130678*r-sin(0.001770376919130678*r),0.002022476464811601*r-sin(0.002022476464811601*r),0.002297373572865413*r-sin(0.002297373572865413*r),0.002596040745477063*r-sin(0.002596040745477063*r),0.002919448107844891*r-sin(0.002919448107844891*r),0.003268563311168871*r-sin(0.003268563311168871*r),0.003644351435886262*r-sin(0.003644351435886262*r),0.004047774895164447*r-sin(0.004047774895164447*r),0.004479793338660443*r-sin(0.004479793338660443*r),0.0049413635565565*r-sin(0.0049413635565565*r),0.005433439383882244*r-sin(0.005433439383882244*r),0.005956971605131645*r-sin(0.005956971605131645*r),0.006512907859185624*r-sin(0.006512907859185624*r),0.007102192544548636*r-sin(0.007102192544548636*r),0.007725766724910044*r-sin(0.007725766724910044*r),0.00838456803503801*r-sin(0.00838456803503801*r),0.009079530587017326*r-sin(0.009079530587017326*r),0.009811584876838586*r-sin(0.009811584876838586*r),0.0105816576913495*r-sin(0.0105816576913495*r),0.01139067201557714*r-sin(0.01139067201557714*r),0.01223954694042984*r-sin(0.01223954694042984*r),0.01312919757078923*r-sin(0.01312919757078923*r),0.01406053493400045*r-sin(0.01406053493400045*r),0.01503446588876983*r-sin(0.01503446588876983*r),0.01605189303448024*r-sin(0.01605189303448024*r),0.01711371462093175*r-sin(0.01711371462093175*r),0.01822082445851714*r-sin(0.01822082445851714*r),0.01937411182884202*r-sin(0.01937411182884202*r),0.02057446139579705*r-sin(0.02057446139579705*r),0.02182275311709253*r-sin(0.02182275311709253*r),0.02311986215626333*r-sin(0.02311986215626333*r),0.02446665879515308*r-sin(0.02446665879515308*r),0.02586400834688696*r-sin(0.02586400834688696*r),0.02731277106934082*r-sin(0.02731277106934082*r),0.02881380207911666*r-sin(0.02881380207911666*r),0.03036795126603076*r-sin(0.03036795126603076*r),0.03197606320812652*r-sin(0.03197606320812652*r),0.0336389770872163*r-sin(0.0336389770872163*r),0.03535752660496472*r-sin(0.03535752660496472*r),0.03713253989951881*r-sin(0.03713253989951881*r),0.03896483946269502*r-sin(0.03896483946269502*r),0.0408552420577305*r-sin(0.0408552420577305*r),0.04280455863760801*r-sin(0.04280455863760801*r),0.04481359426396048*r-sin(0.04481359426396048*r),0.04688314802656623*r-sin(0.04688314802656623*r),0.04901401296344043*r-sin(0.04901401296344043*r),0.05120697598153157*r-sin(0.05120697598153157*r),0.05346281777803219*r-sin(0.05346281777803219*r),0.05578231276230905*r-sin(0.05578231276230905*r),0.05816622897846346*r-sin(0.05816622897846346*r),0.06061532802852698*r-sin(0.06061532802852698*r),0.0631303649963022*r-sin(0.0631303649963022*r),0.06571208837185505*r-sin(0.06571208837185505*r),0.06836123997666599*r-sin(0.06836123997666599*r),0.07107855488944881*r-sin(0.07107855488944881*r),0.07386476137264342*r-sin(0.07386476137264342*r),0.07672058079958999*r-sin(0.07672058079958999*r),0.07964672758239233*r-sin(0.07964672758239233*r),0.08264390910047736*r-sin(0.08264390910047736*r),0.0857128256298576*r-sin(0.0857128256298576*r),0.08885417027310427*r-sin(0.08885417027310427*r),0.09206862889003742*r-sin(0.09206862889003742*r),0.09535688002914089*r-sin(0.09535688002914089*r),0.0987195948597075*r-sin(0.0987195948597075*r),0.1021574371047232*r-sin(0.1021574371047232*r),0.1056710629744951*r-sin(0.1056710629744951*r),0.1092611211010309*r-sin(0.1092611211010309*r),0.1129282524731764*r-sin(0.1129282524731764*r),0.1166730903725168*r-sin(0.1166730903725168*r),0.1204962603100498*r-sin(0.1204962603100498*r),0.1243983799636342*r-sin(0.1243983799636342*r),0.1283800591162231*r-sin(0.1283800591162231*r),0.1324418995948859*r-sin(0.1324418995948859*r),0.1365844952106265*r-sin(0.1365844952106265*r),0.140808431699002*r-sin(0.140808431699002*r),0.1451142866615502*r-sin(0.1451142866615502*r),0.1495026295080298*r-sin(0.1495026295080298*r),0.1539740213994798*r-sin(0.1539740213994798*r)] does not produce a real or column vector
  
  Error generated by error() command
  
  adaptiveeval:
      error(f$|" does not produce a real or column vector"); 
  Try "trace errors" to inspect local variables after errors.
  plot2d:
      dw/n,dw/n^2,dw/n;args());
\end{euleroutput}
\begin{eulercomment}
Berikut dihitung panjang lintasan untuk 1 putaran penuh. (Jangan salah
menduga bahwa panjang lintasan 1 putaran penuh sama dengan keliling
lingkaran!)
\end{eulercomment}
\begin{eulerprompt}
>ds &= radcan(sqrt(diff(ex,x)^2+diff(ey,x)^2)); $ds=trigsimp(ds) // elemen panjang kurva sikloid 
\end{eulerprompt}
\begin{euleroutput}
  Maxima said:
  diff: second argument must be a variable; found errexp1
   -- an error. To debug this try: debugmode(true);
  
  Error in:
  ds &= radcan(sqrt(diff(ex,x)^2+diff(ey,x)^2)); $ds=trigsimp(ds ...
                                               ^
\end{euleroutput}
\begin{eulerprompt}
>ds &= trigsimp(ds); $ds
>$showev('integrate(ds,x,0,2*pi)) // hitung panjang sikloid satu putaran penuh
\end{eulerprompt}
\begin{euleroutput}
  Maxima said:
  defint: variable of integration must be a simple or subscripted variable.
  defint: found errexp1
  #0: showev(f='integrate(ds,[0,1.66665833335744e-7*r,1.33330666692022e-6*r,4.499797504338432e-6*r,1.06658133658399...)
   -- an error. To debug this try: debugmode(true);
  
  Error in:
  $showev('integrate(ds,x,0,2*pi)) // hitung panjang sikloid sat ...
                                   ^
\end{euleroutput}
\begin{eulerprompt}
>integrate(mxm("ds"),0,2*pi) // hitung secara numerik
\end{eulerprompt}
\begin{euleroutput}
  Illegal function result in map.
  %evalexpression:
      if maps then return %mapexpression1(x,f$;args());
  gauss:
      if maps then y=%evalexpression(f$,a+h-(h*xn)',maps;args());
  adaptivegauss:
      t1=gauss(f$,c,c+h;args(),=maps);
  Try "trace errors" to inspect local variables after errors.
  integrate:
      return adaptivegauss(f$,a,b,eps*1000;args(),=maps);
\end{euleroutput}
\begin{eulerprompt}
>romberg(mxm("ds"),0,2*pi) // cara lain hitung secara numerik
\end{eulerprompt}
\begin{euleroutput}
  Wrong argument!
  
  Cannot combine a symbolic expression here.
  Did you want to create a symbolic expression?
  Then start with &.
  
  Try "trace errors" to inspect local variables after errors.
  romberg:
      if cols(y)==1 then return y*(b-a); endif;
  Error in:
  romberg(mxm("ds"),0,2*pi) // cara lain hitung secara numerik ...
                           ^
\end{euleroutput}
\begin{eulercomment}
Perhatikan, seperti terlihat pada gambar, panjang sikloid lebih besar
daripada keliling lingkarannya, yakni:\\
\end{eulercomment}
\begin{eulerformula}
\[
2\pi
\]
\end{eulerformula}
\begin{eulercomment}
\begin{eulercomment}
\eulerheading{Kurvatur (Kelengkungan) Kurva}
\begin{eulercomment}
image: emtKalkulus.png

Aslinya, kelengkungan kurva diferensiabel (yakni, kurva mulus yang
tidak lancip) di titik P didefinisikan melalui lingkaran oskulasi
(yaitu, lingkaran yang melalui titik P dan terbaik memperkirakan,
paling banyak\\
menyinggung kurva di sekitar P). Pusat dan radius kelengkungan kurva
di P adalah pusat dan radius lingkaran oskulasi. Kelengkungan adalah
kebalikan dari radius kelengkungan:


\end{eulercomment}
\begin{eulerformula}
\[
\kappa = \frac{1}{R}
\]
\end{eulerformula}
\begin{eulercomment}
dengan R adalah radius kelengkungan. (Setiap lingkaran memiliki
kelengkungan ini pada setiap titiknya, dapat diartikan, setiap
lingkaran berputar 2pi sejauh 2piR.)\\
Definisi ini sulit dimanipulasi dan dinyatakan ke dalam rumus untuk
kurva umum. Oleh karena itu digunakan definisi lain yang ekivalen.

\begin{eulercomment}
\eulerheading{Definisi Kurvatur dengan Fungsi Parametik Panjang Kurva Setiap kurva}
\begin{eulercomment}
diferensiabel dapat dinyatakan dengan persamaan parametrik terhadap
panjang kurva s:

\end{eulercomment}
\begin{eulerformula}
\[
\gamma(s)=(x(s),y(s)),
\]
\end{eulerformula}
\begin{eulercomment}
dengan x dan y adalah fungsi riil yang diferensiabel, yang memenuhi:

\end{eulercomment}
\begin{eulerformula}
\[
||\gamma'(s)||=\sqrt{x'(s)^2+y'(s)^2}=1
\]
\end{eulerformula}
\begin{eulercomment}
ini berarti bahwa vektor singgung

\end{eulercomment}
\begin{eulerformula}
\[
T(s)=(x'(s),y'(s))
\]
\end{eulerformula}
\begin{eulercomment}
memiliki norm 1 dan merupakan vektor singgung satuan.\\
Apabila kurvanya memiliki turunan kedua, artinya turunan kedua x dan y
ada, maka T’(s) ada. Vektor ini merupakan normal kurva yang arahnya
menuju pusat kurvatur, norm-nya merupakan nilai kurvatur
(kelengkungan):

\end{eulercomment}
\begin{eulerformula}
\[
T(s)=\gamma'(s)
\]
\end{eulerformula}
\begin{eulerformula}
\[
T^2(s)=1(\text{konstanta})\Rightarrow T'(s)\cdot T(s)=0
\]
\end{eulerformula}
\begin{eulerformula}
\[
\kappa(s)=||T'(s)||=||\gamma''(s)||=\sqrt{x''(s)^2+y''(s)^2}
\]
\end{eulerformula}
\begin{eulercomment}
Nilai

\end{eulercomment}
\begin{eulerformula}
\[
R(s)=\frac{1}{\kappa(s)}
\]
\end{eulerformula}
\begin{eulercomment}
disebut jari-jari (radius) kelengkungan kurva.\\
Bilangan riil

\end{eulercomment}
\begin{eulerformula}
\[
k(s)=\pm \kappa(s)
\]
\end{eulerformula}
\begin{eulercomment}
disebut nilai kelengkungan bertanda.\\
Contoh:\\
Akan ditentukan kurvatur lingkaran

\end{eulercomment}
\begin{eulerformula}
\[
x=r\cos{t}, y=r\sin{t}
\]
\end{eulerformula}
\begin{eulerprompt}
>fx &= r*cos(t); fy &=r*sin(t);
>&assume(t>0,r>0); s &=integrate(sqrt(diff(fx,t)^2+diff(fy,t)^2),t,0,t); s // elemen panjang kurva, panjang busur lingkaran (s)
\end{eulerprompt}
\begin{euleroutput}
  Maxima said:
  diff: second argument must be a variable; found errexp1
   -- an error. To debug this try: debugmode(true);
  
  Error in:
  ... =integrate(sqrt(diff(fx,t)^2+diff(fy,t)^2),t,0,t); s // elemen ...
                                                       ^
\end{euleroutput}
\begin{eulerprompt}
>&kill(s); fx &= r*cos(s/r); fy &=r*sin(s/r); // definisi ulang persamaan parametrik terhadap s dengan substitusi t=s/r
>k &= trigsimp(sqrt(diff(fx,s,2)^2+diff(fy,s,2)^2)); $k // nilai kurvatur lingkaran dengan menggunakan definisi di atas
\end{eulerprompt}
\begin{eulerformula}
\[
\frac{1}{r}
\]
\end{eulerformula}
\begin{eulercomment}
Untuk representasi parametrik umum, misalkan

\end{eulercomment}
\begin{eulerformula}
\[
x=x(t), y=y(t)
\]
\end{eulerformula}
\begin{eulercomment}
merupakan persamaan parametrik untuk kurva bidang yang
terdiferensialkan dua kali. Kurvatur untuk kurva tersebut
didefinisikan sebagai

\end{eulercomment}
\begin{eulerformula}
\[
\kappa=\frac{d\phi}{ds}=\frac{\frac{d\phi}{dt}}{\frac{ds}{dt}}  (\phi \text{ adalah sudut kemiringan garis singgung dan } s \text{ adalah panjang kurva})
\]
\end{eulerformula}
\begin{eulercomment}
\end{eulercomment}
\begin{eulerformula}
\[
=\frac{\frac{d\phi}{dt}}{\sqrt{(\frac{dx}{dt})^2 + (\frac{dy}{dt})^2}} = \frac{\frac{d\phi}{dt}}{\sqrt{x'(t)^2+y'(t)^2}}
\]
\end{eulerformula}
\begin{eulercomment}
Selanjutnya, pembilang pada persamaan di atas dapat dicari sebagai
berikut.

\end{eulercomment}
\begin{eulerformula}
\[
\begin{aligned}\sec^2\phi\frac{d\phi}{dt} &= \frac{d}{dt}\left(\tan\phi\right)= \frac{d}{dt}\left(\frac{dy}{dx}\right)= \frac{d}{dt}\left(\frac{dy/dt}{dx/dt}\right)= \frac{d}{dt}\left(\frac{y'(t)}{x'(t)}\right)=\frac{x'(t)y''(t)-x''(t)y'(t)}{x'(t)^2}.\\ & \\ \frac{d\phi}{dt} &= \frac{1}{\sec^2\phi}\frac{x'(t)y''(t)-x''(t)y'(t)}{x'(t)^2}\\ &= \frac{1}{1+\tan^2\phi}\frac{x'(t)y''(t)-x''(t)y'(t)}{x'(t)^2}\\ &= \frac{1}{1+\left(\frac{y'(t)}{x'(t)}\right)^2}\frac{x'(t)y''(t)-x''(t)y'(t)}{x'(t)^2}\\ &= \frac{x'(t)y''(t)-x''(t)y'(t)}{x'(t)^2+y'(t)^2}.\end{aligned}
\]
\end{eulerformula}
\begin{eulercomment}
Jadi, rumus kurvatur untuk kurva parametrik

\end{eulercomment}
\begin{eulerformula}
\[
x=x(t),\ y=y(t)
\]
\end{eulerformula}
\begin{eulercomment}
adalah

\end{eulercomment}
\begin{eulerformula}
\[
\kappa(t) = \frac{x'(t)y''(t)-x''(t)y'(t)}{\left(x'(t)^2+y'(t)^2\right)^{3/2}}
\]
\end{eulerformula}
\begin{eulercomment}
Jika kurvanya dinyatakan dengan persamaan parametrik pada koordinat
kutub

\end{eulercomment}
\begin{eulerformula}
\[
x=r(\theta)\cos\theta,\ y=r(\theta)\sin\theta,
\]
\end{eulerformula}
\begin{eulercomment}
maka rumus kurvaturnya adalah

\end{eulercomment}
\begin{eulerformula}
\[
\kappa(\theta) = \frac{r(\theta)^2+2r'(\theta)^2-r(\theta)r''(\theta)}{\left(r'(\theta)^2+r'(\theta)^2\right)^{3/2}}
\]
\end{eulerformula}
\begin{eulercomment}
(Silakan Anda turunkan rumus tersebut!)

Contoh:\\
Lingkaran dengan pusat (0,0) dan jari-jari r dapat dinyatakan dengan
persamaan parametrik

\end{eulercomment}
\begin{eulerformula}
\[
x=r\cos t,\ y=r\sin t
\]
\end{eulerformula}
\begin{eulercomment}
Nilai kelengkungan lingkaran tersebut adalah

\end{eulercomment}
\begin{eulerformula}
\[
\kappa(t)=\frac{x'(t)y''(t)-x''(t)y'(t)}{\left(x'(t)^2+y'(t)^2\right)^{3/2}}=\frac{r^2}{r^3}=\frac 1 r
\]
\end{eulerformula}
\begin{eulercomment}
Hasil cocok dengan definisi kurvatur suatu kelengkungan.\\
Kurva

\end{eulercomment}
\begin{eulerformula}
\[
y=f(x)
\]
\end{eulerformula}
\begin{eulercomment}
dapat dinyatakan ke dalam persamaan parametrik

\end{eulercomment}
\begin{eulerformula}
\[
x=t,\ y=f(t),\ \text{ dengan } x'(t)=1,\ x''(t)=0,
\]
\end{eulerformula}
\begin{eulercomment}
sehingga kurvaturnya adalah

\end{eulercomment}
\begin{eulerformula}
\[
\kappa(t) = \frac{y''(t)}{\left(1+y'(t)^2\right)^{3/2}}
\]
\end{eulerformula}
\begin{eulercomment}
Contoh:\\
Akan ditentukan kurvatur parabola

\end{eulercomment}
\begin{eulerformula}
\[
y=ax^2+bx+c
\]
\end{eulerformula}
\begin{eulerprompt}
>function f(x) &= a*x^2+b*x+c; $y=f(x)
\end{eulerprompt}
\begin{eulerformula}
\[
\left[ 0 , 4.999958333473664 \times 10^{-5}\,r , 
 1.999933334222437 \times 10^{-4}\,r , 
 4.499662510124569 \times 10^{-4}\,r , 
 7.998933390220841 \times 10^{-4}\,r , 0.001249739605033717\,r , 
 0.00179946006479581\,r , 0.002448999746720415\,r , 
 0.003198293697380561\,r , 0.004047266988005727\,r , 
 0.004995834721974179\,r , 0.006043902043303184\,r , 
 0.00719136414613375\,r , 0.00843810628521191\,r , 
 0.009784003787362772\,r , 0.01122892206395776\,r , 
 0.01277271662437307\,r , 0.01441523309043924\,r , 
 0.01615630721187855\,r , 0.01799576488272969\,r , 
 0.01993342215875837\,r , 0.02196908527585173\,r , 
 0.02410255066939448\,r , 0.02633360499462523\,r , 
 0.02866202514797045\,r , 0.03108757828935527\,r , 
 0.03361002186548678\,r , 0.03622910363410947\,r , 
 0.03894456168922911\,r , 0.04175612448730281\,r , 
 0.04466351087439402\,r , 0.04766643011428662\,r , 
 0.05076458191755917\,r , 0.0539576564716131\,r , 0.05724533447165381
 \,r , 0.06062728715262111\,r , 0.06410317632206519\,r , 
 0.06767265439396564\,r , 0.07133536442348987\,r , 
 0.07509094014268702\,r , 0.07893900599711501\,r , 
 0.08287917718339499\,r , 0.08691105968769186\,r , 
 0.09103425032511492\,r , 0.09524833678003664\,r , 
 0.09955289764732322\,r , 0.1039475024744748\,r , 0.1084317118046711
 \,r , 0.113005077220716\,r , 0.1176671413898787\,r , 
 0.1224174381096274\,r , 0.1272554923542488\,r , 0.1321808203223502\,
 r , 0.1371929294852391\,r , 0.1422913186361759\,r , 
 0.1474754779404944\,r , 0.152744888986584\,r , 0.1580990248377314\,r
  , 0.1635373500848132\,r , 0.1690593208998367\,r , 
 0.1746643850903219\,r , 0.1803519821545206\,r , 0.1861215433374662\,
 r , 0.1919724916878484\,r , 0.1979042421157076\,r , 
 0.2039162014509444\,r , 0.2100077685026351\,r , 0.216178334119151\,r
  , 0.2224272812490723\,r , 0.2287539850028937\,r , 
 0.2351578127155118\,r , 0.2416381240094921\,r , 0.2481942708591053\,
 r , 0.2548255976551299\,r , 0.2615314412704124\,r , 
 0.2683111311261794\,r , 0.2751639892590951\,r , 0.2820893303890569\,
 r , 0.2890864619877229\,r , 0.2961546843477643\,r , 
 0.3032932906528349\,r , 0.3105015670482534\,r , 0.3177787927123868\,
 r , 0.3251242399287333\,r , 0.3325371741586922\,r , 
 0.3400168541150183\,r , 0.3475625318359485\,r , 0.3551734527599992\,
 r , 0.3628488558014202\,r , 0.3705879734263036\,r , 
 0.3783900317293359\,r , 0.3862542505111889\,r , 0.3941798433565377\,
 r , 0.4021660177127022\,r , 0.4102119749689023\,r , 
 0.418316910536117\,r , 0.4264800139275439\,r , 0.4347004688396462\,r
  , 0.4429774532337832\,r , 0.451310139418413\,r \right] =\left[ c , 
 2.7777500001498 \times 10^{-14}\,a\,r^2+
 1.66665833335744 \times 10^{-7}\,b\,r+c , 
 1.777706668053906 \times 10^{-12}\,a\,r^2+
 1.33330666692022 \times 10^{-6}\,b\,r+c , 
 2.024817758005038 \times 10^{-11}\,a\,r^2+
 4.499797504338432 \times 10^{-6}\,b\,r+c , 
 1.137595747549299 \times 10^{-10}\,a\,r^2+
 1.066581336583994 \times 10^{-5}\,b\,r+c , 
 4.339192840727639 \times 10^{-10}\,a\,r^2+
 2.083072932167196 \times 10^{-5}\,b\,r+c , 
 1.295533521972174 \times 10^{-9}\,a\,r^2+
 3.599352055540239 \times 10^{-5}\,b\,r+c , 
 3.266426827094104 \times 10^{-9}\,a\,r^2+
 5.71526624672386 \times 10^{-5}\,b\,r+c , 
 7.277118895509326 \times 10^{-9}\,a\,r^2+
 8.530603082730626 \times 10^{-5}\,b\,r+c , 
 1.475029730376073 \times 10^{-8}\,a\,r^2+
 1.214508019889565 \times 10^{-4}\,b\,r+c , 
 2.775001355397757 \times 10^{-8}\,a\,r^2+
 1.665833531718508 \times 10^{-4}\,b\,r+c , 
 4.915051879738995 \times 10^{-8}\,a\,r^2+
 2.216991628251896 \times 10^{-4}\,b\,r+c , 
 8.28246445511412 \times 10^{-8}\,a\,r^2+
 2.877927110806339 \times 10^{-4}\,b\,r+c , 
 1.33851622723744 \times 10^{-7}\,a\,r^2+
 3.658573803051457 \times 10^{-4}\,b\,r+c , 
 2.087442283111582 \times 10^{-7}\,a\,r^2+
 4.568853557635201 \times 10^{-4}\,b\,r+c , 
 3.156951172237287 \times 10^{-7}\,a\,r^2+
 5.618675264007778 \times 10^{-4}\,b\,r+c , 
 4.64842220857938 \times 10^{-7}\,a\,r^2+
 6.817933857540259 \times 10^{-4}\,b\,r+c , 
 6.685530482422835 \times 10^{-7}\,a\,r^2+
 8.176509330039827 \times 10^{-4}\,b\,r+c , 
 9.417277358666075 \times 10^{-7}\,a\,r^2+
 9.704265741758145 \times 10^{-4}\,b\,r+c , 
 1.30212067465563 \times 10^{-6}\,a\,r^2+0.001141105023499428\,b\,r+c
  , 1.770680532972444 \times 10^{-6}\,a\,r^2+0.001330669204938795\,b
 \,r+c , 2.371908484044149 \times 10^{-6}\,a\,r^2+
 0.001540100153900437\,b\,r+c , 3.134234435790633 \times 10^{-6}\,a\,
 r^2+0.001770376919130678\,b\,r+c , 4.090411050716832 \times 10^{-6}
 \,a\,r^2+0.002022476464811601\,b\,r+c , 
 5.277925333300395 \times 10^{-6}\,a\,r^2+0.002297373572865413\,b\,r+
 c , 6.739427552177103 \times 10^{-6}\,a\,r^2+0.002596040745477063\,b
 \,r+c , 8.523177254399114 \times 10^{-6}\,a\,r^2+
 0.002919448107844891\,b\,r+c , 1.068350611911921 \times 10^{-5}\,a\,
 r^2+0.003268563311168871\,b\,r+c , 1.328129738824626 \times 10^{-5}
 \,a\,r^2+0.003644351435886262\,b\,r+c , 
 1.638448160192355 \times 10^{-5}\,a\,r^2+0.004047774895164447\,b\,r+
 c , 2.006854835710647 \times 10^{-5}\,a\,r^2+0.004479793338660443\,b
 \,r+c , 2.44170737980647 \times 10^{-5}\,a\,r^2+0.0049413635565565\,
 b\,r+c , 2.952226353832265 \times 10^{-5}\,a\,r^2+
 0.005433439383882244\,b\,r+c , 3.548551070434468 \times 10^{-5}\,a\,
 r^2+0.005956971605131645\,b\,r+c , 4.241796878224187 \times 10^{-5}
 \,a\,r^2+0.006512907859185624\,b\,r+c , 
 5.044113893984222 \times 10^{-5}\,a\,r^2+0.007102192544548636\,b\,r+
 c , 5.968747148772726 \times 10^{-5}\,a\,r^2+0.007725766724910044\,b
 \,r+c , 7.030098113418114 \times 10^{-5}\,a\,r^2+0.00838456803503801
 \,b\,r+c , 8.243787568058321 \times 10^{-5}\,a\,r^2+
 0.009079530587017326\,b\,r+c , 9.626719779540763 \times 10^{-5}\,a\,
 r^2+0.009811584876838586\,b\,r+c , 1.11971479496896 \times 10^{-4}\,
 a\,r^2+0.0105816576913495\,b\,r+c , 1.297474089664522 \times 10^{-4}
 \,a\,r^2+0.01139067201557714\,b\,r+c , 
 1.498065093069853 \times 10^{-4}\,a\,r^2+0.01223954694042984\,b\,r+c
  , 1.723758288528179 \times 10^{-4}\,a\,r^2+0.01312919757078923\,b\,
 r+c , 1.976986426302469 \times 10^{-4}\,a\,r^2+0.01406053493400045\,
 b\,r+c , 2.260351645605837 \times 10^{-4}\,a\,r^2+
 0.01503446588876983\,b\,r+c , 2.576632699903951 \times 10^{-4}\,a\,r
 ^2+0.01605189303448024\,b\,r+c , 2.928792281266932 \times 10^{-4}\,a
 \,r^2+0.01711371462093175\,b\,r+c , 3.319984439480964 \times 10^{-4}
 \,a\,r^2+0.01822082445851714\,b\,r+c , 
 3.753562091564763 \times 10^{-4}\,a\,r^2+0.01937411182884202\,b\,r+c
  , 4.233084617271431 \times 10^{-4}\,a\,r^2+0.02057446139579705\,b\,
 r+c , 4.762325536095718 \times 10^{-4}\,a\,r^2+0.02182275311709253\,
 b\,r+c , 5.34528026124617 \times 10^{-4}\,a\,r^2+0.02311986215626333
 \,b\,r+c , 5.986173925984417 \times 10^{-4}\,a\,r^2+
 0.02446665879515308\,b\,r+c , 6.689469277678383 \times 10^{-4}\,a\,r
 ^2+0.02586400834688696\,b\,r+c , 7.459874634862211 \times 10^{-4}\,a
 \,r^2+0.02731277106934082\,b\,r+c , 8.302351902545073 \times 10^{-4}
 \,a\,r^2+0.02881380207911666\,b\,r+c , 
 9.222124640960191 \times 10^{-4}\,a\,r^2+0.03036795126603076\,b\,r+c
  , 0.001022468618290102\,a\,r^2+0.03197606320812652\,b\,r+c , 
 0.001131580779474263\,a\,r^2+0.0336389770872163\,b\,r+c , 
 0.001250154687620788\,a\,r^2+0.03535752660496472\,b\,r+c , 
 0.001378825519389357\,a\,r^2+0.03713253989951881\,b\,r+c , 
 0.001518258714353595\,a\,r^2+0.03896483946269502\,b\,r+c , 
 0.001669150803595751\,a\,r^2+0.0408552420577305\,b\,r+c , 
 0.001832230240160423\,a\,r^2+0.04280455863760801\,b\,r+c , 
 0.002008258230854871\,a\,r^2+0.04481359426396048\,b\,r+c , 
 0.002198029568880921\,a\,r^2+0.04688314802656623\,b\,r+c , 
 0.002402373466780307\,a\,r^2+0.04901401296344043\,b\,r+c , 
 0.002622154389173151\,a\,r^2+0.05120697598153157\,b\,r+c , 
 0.002858272884767075\,a\,r^2+0.05346281777803219\,b\,r+c , 
 0.003111666417112067\,a\,r^2+0.05578231276230905\,b\,r+c , 
 0.003383310193575043\,a\,r^2+0.05816622897846346\,b\,r+c , 
 0.003674217992005929\,a\,r^2+0.06061532802852698\,b\,r+c , 
 0.003985442984566339\,a\,r^2+0.0631303649963022\,b\,r+c , 
 0.004318078558190487\,a\,r^2+0.06571208837185505\,b\,r+c , 
 0.004673259131147316\,a\,r^2+0.06836123997666599\,b\,r+c , 
 0.005052160965172387\,a\,r^2+0.07107855488944881\,b\,r+c , 
 0.005456002972637555\,a\,r^2+0.07386476137264342\,b\,r+c , 
 0.005886047518226416\,a\,r^2+0.07672058079958999\,b\,r+c , 
 0.006343601214583815\,a\,r^2+0.07964672758239233\,b\,r+c , 
 0.006830015711407966\,a\,r^2+0.08264390910047736\,b\,r+c , 
 0.007346688477454374\,a\,r^2+0.0857128256298576\,b\,r+c , 
 0.007895063574921807\,a\,r^2+0.08885417027310427\,b\,r+c , 
 0.008476632425691433\,a\,r^2+0.09206862889003742\,b\,r+c , 
 0.009092934568891969\,a\,r^2+0.09535688002914089\,b\,r+c , 
 0.009745558409264787\,a\,r^2+0.0987195948597075\,b\,r+c , 
 0.01043614195580549\,a\,r^2+0.1021574371047232\,b\,r+c , 
 0.01116637355015972\,a\,r^2+0.1056710629744951\,b\,r+c , 
 0.01193799258425414\,a\,r^2+0.1092611211010309\,b\,r+c , 
 0.01275279020664547\,a\,r^2+0.1129282524731764\,b\,r+c , 
 0.01361261001707348\,a\,r^2+0.1166730903725168\,b\,r+c , 
 0.01451934874870728\,a\,r^2+0.1204962603100498\,b\,r+c , 
 0.01547495693757671\,a\,r^2+0.1243983799636342\,b\,r+c , 
 0.01648143957868493\,a\,r^2+0.1283800591162231\,b\,r+c , 
 0.01754085676830185\,a\,r^2+0.1324418995948859\,b\,r+c , 
 0.01865532433194167\,a\,r^2+0.1365844952106265\,b\,r+c , 
 0.01982701443753252\,a\,r^2+0.140808431699002\,b\,r+c , 
 0.02105815619329058\,a\,r^2+0.1451142866615502\,b\,r+c , 
 0.02235103622981523\,a\,r^2+0.1495026295080298\,b\,r+c , 
 0.02370799926592746\,a\,r^2+0.1539740213994798\,b\,r+c \right] 
\]
\end{eulerformula}
\begin{eulerprompt}
>function k(x) &= (diff(f(x),x,2))/(1+diff(f(x),x)^2)^(3/2); $'k(x)=k(x) // kelengkungan parabola
\end{eulerprompt}
\begin{euleroutput}
  Maxima said:
  diff: second argument must be a variable; found errexp1
   -- an error. To debug this try: debugmode(true);
  
  Error in:
  ... (x) &= (diff(f(x),x,2))/(1+diff(f(x),x)^2)^(3/2); $'k(x)=k(x)  ...
                                                       ^
\end{euleroutput}
\begin{eulerprompt}
>function f(x) &= x^2+x+1; $y=f(x) // akan kita plot kelengkungan parabola untuk a=b=c=1
\end{eulerprompt}
\begin{eulerformula}
\[
\left[ 0 , 4.999958333473664 \times 10^{-5}\,r , 
 1.999933334222437 \times 10^{-4}\,r , 
 4.499662510124569 \times 10^{-4}\,r , 
 7.998933390220841 \times 10^{-4}\,r , 0.001249739605033717\,r , 
 0.00179946006479581\,r , 0.002448999746720415\,r , 
 0.003198293697380561\,r , 0.004047266988005727\,r , 
 0.004995834721974179\,r , 0.006043902043303184\,r , 
 0.00719136414613375\,r , 0.00843810628521191\,r , 
 0.009784003787362772\,r , 0.01122892206395776\,r , 
 0.01277271662437307\,r , 0.01441523309043924\,r , 
 0.01615630721187855\,r , 0.01799576488272969\,r , 
 0.01993342215875837\,r , 0.02196908527585173\,r , 
 0.02410255066939448\,r , 0.02633360499462523\,r , 
 0.02866202514797045\,r , 0.03108757828935527\,r , 
 0.03361002186548678\,r , 0.03622910363410947\,r , 
 0.03894456168922911\,r , 0.04175612448730281\,r , 
 0.04466351087439402\,r , 0.04766643011428662\,r , 
 0.05076458191755917\,r , 0.0539576564716131\,r , 0.05724533447165381
 \,r , 0.06062728715262111\,r , 0.06410317632206519\,r , 
 0.06767265439396564\,r , 0.07133536442348987\,r , 
 0.07509094014268702\,r , 0.07893900599711501\,r , 
 0.08287917718339499\,r , 0.08691105968769186\,r , 
 0.09103425032511492\,r , 0.09524833678003664\,r , 
 0.09955289764732322\,r , 0.1039475024744748\,r , 0.1084317118046711
 \,r , 0.113005077220716\,r , 0.1176671413898787\,r , 
 0.1224174381096274\,r , 0.1272554923542488\,r , 0.1321808203223502\,
 r , 0.1371929294852391\,r , 0.1422913186361759\,r , 
 0.1474754779404944\,r , 0.152744888986584\,r , 0.1580990248377314\,r
  , 0.1635373500848132\,r , 0.1690593208998367\,r , 
 0.1746643850903219\,r , 0.1803519821545206\,r , 0.1861215433374662\,
 r , 0.1919724916878484\,r , 0.1979042421157076\,r , 
 0.2039162014509444\,r , 0.2100077685026351\,r , 0.216178334119151\,r
  , 0.2224272812490723\,r , 0.2287539850028937\,r , 
 0.2351578127155118\,r , 0.2416381240094921\,r , 0.2481942708591053\,
 r , 0.2548255976551299\,r , 0.2615314412704124\,r , 
 0.2683111311261794\,r , 0.2751639892590951\,r , 0.2820893303890569\,
 r , 0.2890864619877229\,r , 0.2961546843477643\,r , 
 0.3032932906528349\,r , 0.3105015670482534\,r , 0.3177787927123868\,
 r , 0.3251242399287333\,r , 0.3325371741586922\,r , 
 0.3400168541150183\,r , 0.3475625318359485\,r , 0.3551734527599992\,
 r , 0.3628488558014202\,r , 0.3705879734263036\,r , 
 0.3783900317293359\,r , 0.3862542505111889\,r , 0.3941798433565377\,
 r , 0.4021660177127022\,r , 0.4102119749689023\,r , 
 0.418316910536117\,r , 0.4264800139275439\,r , 0.4347004688396462\,r
  , 0.4429774532337832\,r , 0.451310139418413\,r \right] =\left[ 1 , 
 2.7777500001498 \times 10^{-14}\,r^2+1.66665833335744 \times 10^{-7}
 \,r+1 , 1.777706668053906 \times 10^{-12}\,r^2+
 1.33330666692022 \times 10^{-6}\,r+1 , 
 2.024817758005038 \times 10^{-11}\,r^2+
 4.499797504338432 \times 10^{-6}\,r+1 , 
 1.137595747549299 \times 10^{-10}\,r^2+
 1.066581336583994 \times 10^{-5}\,r+1 , 
 4.339192840727639 \times 10^{-10}\,r^2+
 2.083072932167196 \times 10^{-5}\,r+1 , 
 1.295533521972174 \times 10^{-9}\,r^2+
 3.599352055540239 \times 10^{-5}\,r+1 , 
 3.266426827094104 \times 10^{-9}\,r^2+
 5.71526624672386 \times 10^{-5}\,r+1 , 
 7.277118895509326 \times 10^{-9}\,r^2+
 8.530603082730626 \times 10^{-5}\,r+1 , 
 1.475029730376073 \times 10^{-8}\,r^2+
 1.214508019889565 \times 10^{-4}\,r+1 , 
 2.775001355397757 \times 10^{-8}\,r^2+
 1.665833531718508 \times 10^{-4}\,r+1 , 
 4.915051879738995 \times 10^{-8}\,r^2+
 2.216991628251896 \times 10^{-4}\,r+1 , 
 8.28246445511412 \times 10^{-8}\,r^2+
 2.877927110806339 \times 10^{-4}\,r+1 , 
 1.33851622723744 \times 10^{-7}\,r^2+
 3.658573803051457 \times 10^{-4}\,r+1 , 
 2.087442283111582 \times 10^{-7}\,r^2+
 4.568853557635201 \times 10^{-4}\,r+1 , 
 3.156951172237287 \times 10^{-7}\,r^2+
 5.618675264007778 \times 10^{-4}\,r+1 , 
 4.64842220857938 \times 10^{-7}\,r^2+
 6.817933857540259 \times 10^{-4}\,r+1 , 
 6.685530482422835 \times 10^{-7}\,r^2+
 8.176509330039827 \times 10^{-4}\,r+1 , 
 9.417277358666075 \times 10^{-7}\,r^2+
 9.704265741758145 \times 10^{-4}\,r+1 , 
 1.30212067465563 \times 10^{-6}\,r^2+0.001141105023499428\,r+1 , 
 1.770680532972444 \times 10^{-6}\,r^2+0.001330669204938795\,r+1 , 
 2.371908484044149 \times 10^{-6}\,r^2+0.001540100153900437\,r+1 , 
 3.134234435790633 \times 10^{-6}\,r^2+0.001770376919130678\,r+1 , 
 4.090411050716832 \times 10^{-6}\,r^2+0.002022476464811601\,r+1 , 
 5.277925333300395 \times 10^{-6}\,r^2+0.002297373572865413\,r+1 , 
 6.739427552177103 \times 10^{-6}\,r^2+0.002596040745477063\,r+1 , 
 8.523177254399114 \times 10^{-6}\,r^2+0.002919448107844891\,r+1 , 
 1.068350611911921 \times 10^{-5}\,r^2+0.003268563311168871\,r+1 , 
 1.328129738824626 \times 10^{-5}\,r^2+0.003644351435886262\,r+1 , 
 1.638448160192355 \times 10^{-5}\,r^2+0.004047774895164447\,r+1 , 
 2.006854835710647 \times 10^{-5}\,r^2+0.004479793338660443\,r+1 , 
 2.44170737980647 \times 10^{-5}\,r^2+0.0049413635565565\,r+1 , 
 2.952226353832265 \times 10^{-5}\,r^2+0.005433439383882244\,r+1 , 
 3.548551070434468 \times 10^{-5}\,r^2+0.005956971605131645\,r+1 , 
 4.241796878224187 \times 10^{-5}\,r^2+0.006512907859185624\,r+1 , 
 5.044113893984222 \times 10^{-5}\,r^2+0.007102192544548636\,r+1 , 
 5.968747148772726 \times 10^{-5}\,r^2+0.007725766724910044\,r+1 , 
 7.030098113418114 \times 10^{-5}\,r^2+0.00838456803503801\,r+1 , 
 8.243787568058321 \times 10^{-5}\,r^2+0.009079530587017326\,r+1 , 
 9.626719779540763 \times 10^{-5}\,r^2+0.009811584876838586\,r+1 , 
 1.11971479496896 \times 10^{-4}\,r^2+0.0105816576913495\,r+1 , 
 1.297474089664522 \times 10^{-4}\,r^2+0.01139067201557714\,r+1 , 
 1.498065093069853 \times 10^{-4}\,r^2+0.01223954694042984\,r+1 , 
 1.723758288528179 \times 10^{-4}\,r^2+0.01312919757078923\,r+1 , 
 1.976986426302469 \times 10^{-4}\,r^2+0.01406053493400045\,r+1 , 
 2.260351645605837 \times 10^{-4}\,r^2+0.01503446588876983\,r+1 , 
 2.576632699903951 \times 10^{-4}\,r^2+0.01605189303448024\,r+1 , 
 2.928792281266932 \times 10^{-4}\,r^2+0.01711371462093175\,r+1 , 
 3.319984439480964 \times 10^{-4}\,r^2+0.01822082445851714\,r+1 , 
 3.753562091564763 \times 10^{-4}\,r^2+0.01937411182884202\,r+1 , 
 4.233084617271431 \times 10^{-4}\,r^2+0.02057446139579705\,r+1 , 
 4.762325536095718 \times 10^{-4}\,r^2+0.02182275311709253\,r+1 , 
 5.34528026124617 \times 10^{-4}\,r^2+0.02311986215626333\,r+1 , 
 5.986173925984417 \times 10^{-4}\,r^2+0.02446665879515308\,r+1 , 
 6.689469277678383 \times 10^{-4}\,r^2+0.02586400834688696\,r+1 , 
 7.459874634862211 \times 10^{-4}\,r^2+0.02731277106934082\,r+1 , 
 8.302351902545073 \times 10^{-4}\,r^2+0.02881380207911666\,r+1 , 
 9.222124640960191 \times 10^{-4}\,r^2+0.03036795126603076\,r+1 , 
 0.001022468618290102\,r^2+0.03197606320812652\,r+1 , 
 0.001131580779474263\,r^2+0.0336389770872163\,r+1 , 
 0.001250154687620788\,r^2+0.03535752660496472\,r+1 , 
 0.001378825519389357\,r^2+0.03713253989951881\,r+1 , 
 0.001518258714353595\,r^2+0.03896483946269502\,r+1 , 
 0.001669150803595751\,r^2+0.0408552420577305\,r+1 , 
 0.001832230240160423\,r^2+0.04280455863760801\,r+1 , 
 0.002008258230854871\,r^2+0.04481359426396048\,r+1 , 
 0.002198029568880921\,r^2+0.04688314802656623\,r+1 , 
 0.002402373466780307\,r^2+0.04901401296344043\,r+1 , 
 0.002622154389173151\,r^2+0.05120697598153157\,r+1 , 
 0.002858272884767075\,r^2+0.05346281777803219\,r+1 , 
 0.003111666417112067\,r^2+0.05578231276230905\,r+1 , 
 0.003383310193575043\,r^2+0.05816622897846346\,r+1 , 
 0.003674217992005929\,r^2+0.06061532802852698\,r+1 , 
 0.003985442984566339\,r^2+0.0631303649963022\,r+1 , 
 0.004318078558190487\,r^2+0.06571208837185505\,r+1 , 
 0.004673259131147316\,r^2+0.06836123997666599\,r+1 , 
 0.005052160965172387\,r^2+0.07107855488944881\,r+1 , 
 0.005456002972637555\,r^2+0.07386476137264342\,r+1 , 
 0.005886047518226416\,r^2+0.07672058079958999\,r+1 , 
 0.006343601214583815\,r^2+0.07964672758239233\,r+1 , 
 0.006830015711407966\,r^2+0.08264390910047736\,r+1 , 
 0.007346688477454374\,r^2+0.0857128256298576\,r+1 , 
 0.007895063574921807\,r^2+0.08885417027310427\,r+1 , 
 0.008476632425691433\,r^2+0.09206862889003742\,r+1 , 
 0.009092934568891969\,r^2+0.09535688002914089\,r+1 , 
 0.009745558409264787\,r^2+0.0987195948597075\,r+1 , 
 0.01043614195580549\,r^2+0.1021574371047232\,r+1 , 
 0.01116637355015972\,r^2+0.1056710629744951\,r+1 , 
 0.01193799258425414\,r^2+0.1092611211010309\,r+1 , 
 0.01275279020664547\,r^2+0.1129282524731764\,r+1 , 
 0.01361261001707348\,r^2+0.1166730903725168\,r+1 , 
 0.01451934874870728\,r^2+0.1204962603100498\,r+1 , 
 0.01547495693757671\,r^2+0.1243983799636342\,r+1 , 
 0.01648143957868493\,r^2+0.1283800591162231\,r+1 , 
 0.01754085676830185\,r^2+0.1324418995948859\,r+1 , 
 0.01865532433194167\,r^2+0.1365844952106265\,r+1 , 
 0.01982701443753252\,r^2+0.140808431699002\,r+1 , 
 0.02105815619329058\,r^2+0.1451142866615502\,r+1 , 
 0.02235103622981523\,r^2+0.1495026295080298\,r+1 , 
 0.02370799926592746\,r^2+0.1539740213994798\,r+1 \right] 
\]
\end{eulerformula}
\begin{eulerprompt}
>function k(x) &= (diff(f(x),x,2))/(1+diff(f(x),x)^2)^(3/2); $'k(x)=k(x) // kelengkungan parabola
\end{eulerprompt}
\begin{euleroutput}
  Maxima said:
  diff: second argument must be a variable; found errexp1
   -- an error. To debug this try: debugmode(true);
  
  Error in:
  ... (x) &= (diff(f(x),x,2))/(1+diff(f(x),x)^2)^(3/2); $'k(x)=k(x)  ...
                                                       ^
\end{euleroutput}
\begin{eulercomment}
Berikut kita gambar parabola tersebut beserta kurva kelengkungan,
kurva jari-jari kelengkungan dan salah satu lingkaran oskulasi\\
di titik puncak parabola. Perhatikan, puncak parabola dan jari-jari
lingkaran oskulasi di puncak parabola adalah

\end{eulercomment}
\begin{eulerformula}
\[
(-1/2,3/4),\ 1/k(2)=1/2
\]
\end{eulerformula}
\begin{eulercomment}
sehingga pusat lingkaran oskulasi adalah (-1/2, 5/4)

\end{eulercomment}
\begin{eulerprompt}
>plot2d(["f(x)", "k(x)"],-2,1, color=[blue,red]); plot2d("1/k(x)",-1.5,1,color=green,>add); ...
>plot2d("-1/2+1/k(-1/2)*cos(x)","5/4+1/k(-1/2)*sin(x)",xmin=0,xmax=2pi,>add,color=blue):
\end{eulerprompt}
\begin{euleroutput}
  Error : f(x) does not produce a real or column vector
  
  Error generated by error() command
  
  %ploteval:
      error(f$|" does not produce a real or column vector"); 
  adaptiveevalone:
      s=%ploteval(g$,t;args());
  Try "trace errors" to inspect local variables after errors.
  plot2d:
      dw/n,dw/n^2,dw/n,auto;args());
\end{euleroutput}
\begin{eulercomment}
Untuk kurva yang dinyatakan dengan fungsi implisit

\end{eulercomment}
\begin{eulerformula}
\[
F(x,y)=0
\]
\end{eulerformula}
\begin{eulercomment}
dengan turunan-turunan parsial

\end{eulercomment}
\begin{eulerformula}
\[
F_x=\frac{\partial F}{\partial x},\ F_y=\frac{\partial F}{\partial y},\ F_{xy}=\frac{\partial}{\partial y}\left(\frac{\partial F}{\partial x}\right),\ F_{xx}=\frac{\partial}{\partial x}\left(\frac{\partial F}{\partial x}\right),\ F_{yy}=\frac{\partial}{\partial y}\left(\frac{\partial F}{\partial y}\right)
\]
\end{eulerformula}
\begin{eulercomment}
berlaku

\end{eulercomment}
\begin{eulerformula}
\[
F_x dx+ F_y dy = 0\text{ atau } \frac{dy}{dx}=-\frac{F_x}{F_y}
\]
\end{eulerformula}
\begin{eulercomment}
sehingga kurvaturnya adalah

\end{eulercomment}
\begin{eulerformula}
\[
\kappa =\frac {F_y^2F_{xx}-2F_xF_yF_{xy}+F_x^2F_{yy}}{\left(F_x^2+F_y^2\right)^{3/2}}
\]
\end{eulerformula}
\begin{eulercomment}
(Silakan Anda turunkan sendiri!)

Contoh 1:\\
Parabola

\end{eulercomment}
\begin{eulerformula}
\[
y=ax^2+bx+c
\]
\end{eulerformula}
\begin{eulercomment}
dapat dinyatakan ke dalam persamaan implisit

\end{eulercomment}
\begin{eulerformula}
\[
ax^2+bx+c-y=0
\]
\end{eulerformula}
\begin{eulerprompt}
>function F(x,y) &=a*x^2+b*x+c-y; $F(x,y)
\end{eulerprompt}
\begin{eulerformula}
\[
\left[ c , 2.7777500001498 \times 10^{-14}\,a\,r^2+
 1.66665833335744 \times 10^{-7}\,b\,r-
 4.999958333473664 \times 10^{-5}\,r+c , 
 1.777706668053906 \times 10^{-12}\,a\,r^2+
 1.33330666692022 \times 10^{-6}\,b\,r-
 1.999933334222437 \times 10^{-4}\,r+c , 
 2.024817758005038 \times 10^{-11}\,a\,r^2+
 4.499797504338432 \times 10^{-6}\,b\,r-
 4.499662510124569 \times 10^{-4}\,r+c , 
 1.137595747549299 \times 10^{-10}\,a\,r^2+
 1.066581336583994 \times 10^{-5}\,b\,r-
 7.998933390220841 \times 10^{-4}\,r+c , 
 4.339192840727639 \times 10^{-10}\,a\,r^2+
 2.083072932167196 \times 10^{-5}\,b\,r-0.001249739605033717\,r+c , 
 1.295533521972174 \times 10^{-9}\,a\,r^2+
 3.599352055540239 \times 10^{-5}\,b\,r-0.00179946006479581\,r+c , 
 3.266426827094104 \times 10^{-9}\,a\,r^2+
 5.71526624672386 \times 10^{-5}\,b\,r-0.002448999746720415\,r+c , 
 7.277118895509326 \times 10^{-9}\,a\,r^2+
 8.530603082730626 \times 10^{-5}\,b\,r-0.003198293697380561\,r+c , 
 1.475029730376073 \times 10^{-8}\,a\,r^2+
 1.214508019889565 \times 10^{-4}\,b\,r-0.004047266988005727\,r+c , 
 2.775001355397757 \times 10^{-8}\,a\,r^2+
 1.665833531718508 \times 10^{-4}\,b\,r-0.004995834721974179\,r+c , 
 4.915051879738995 \times 10^{-8}\,a\,r^2+
 2.216991628251896 \times 10^{-4}\,b\,r-0.006043902043303184\,r+c , 
 8.28246445511412 \times 10^{-8}\,a\,r^2+
 2.877927110806339 \times 10^{-4}\,b\,r-0.00719136414613375\,r+c , 
 1.33851622723744 \times 10^{-7}\,a\,r^2+
 3.658573803051457 \times 10^{-4}\,b\,r-0.00843810628521191\,r+c , 
 2.087442283111582 \times 10^{-7}\,a\,r^2+
 4.568853557635201 \times 10^{-4}\,b\,r-0.009784003787362772\,r+c , 
 3.156951172237287 \times 10^{-7}\,a\,r^2+
 5.618675264007778 \times 10^{-4}\,b\,r-0.01122892206395776\,r+c , 
 4.64842220857938 \times 10^{-7}\,a\,r^2+
 6.817933857540259 \times 10^{-4}\,b\,r-0.01277271662437307\,r+c , 
 6.685530482422835 \times 10^{-7}\,a\,r^2+
 8.176509330039827 \times 10^{-4}\,b\,r-0.01441523309043924\,r+c , 
 9.417277358666075 \times 10^{-7}\,a\,r^2+
 9.704265741758145 \times 10^{-4}\,b\,r-0.01615630721187855\,r+c , 
 1.30212067465563 \times 10^{-6}\,a\,r^2+0.001141105023499428\,b\,r-
 0.01799576488272969\,r+c , 1.770680532972444 \times 10^{-6}\,a\,r^2+
 0.001330669204938795\,b\,r-0.01993342215875837\,r+c , 
 2.371908484044149 \times 10^{-6}\,a\,r^2+0.001540100153900437\,b\,r-
 0.02196908527585173\,r+c , 3.134234435790633 \times 10^{-6}\,a\,r^2+
 0.001770376919130678\,b\,r-0.02410255066939448\,r+c , 
 4.090411050716832 \times 10^{-6}\,a\,r^2+0.002022476464811601\,b\,r-
 0.02633360499462523\,r+c , 5.277925333300395 \times 10^{-6}\,a\,r^2+
 0.002297373572865413\,b\,r-0.02866202514797045\,r+c , 
 6.739427552177103 \times 10^{-6}\,a\,r^2+0.002596040745477063\,b\,r-
 0.03108757828935527\,r+c , 8.523177254399114 \times 10^{-6}\,a\,r^2+
 0.002919448107844891\,b\,r-0.03361002186548678\,r+c , 
 1.068350611911921 \times 10^{-5}\,a\,r^2+0.003268563311168871\,b\,r-
 0.03622910363410947\,r+c , 1.328129738824626 \times 10^{-5}\,a\,r^2+
 0.003644351435886262\,b\,r-0.03894456168922911\,r+c , 
 1.638448160192355 \times 10^{-5}\,a\,r^2+0.004047774895164447\,b\,r-
 0.04175612448730281\,r+c , 2.006854835710647 \times 10^{-5}\,a\,r^2+
 0.004479793338660443\,b\,r-0.04466351087439402\,r+c , 
 2.44170737980647 \times 10^{-5}\,a\,r^2+0.0049413635565565\,b\,r-
 0.04766643011428662\,r+c , 2.952226353832265 \times 10^{-5}\,a\,r^2+
 0.005433439383882244\,b\,r-0.05076458191755917\,r+c , 
 3.548551070434468 \times 10^{-5}\,a\,r^2+0.005956971605131645\,b\,r-
 0.0539576564716131\,r+c , 4.241796878224187 \times 10^{-5}\,a\,r^2+
 0.006512907859185624\,b\,r-0.05724533447165381\,r+c , 
 5.044113893984222 \times 10^{-5}\,a\,r^2+0.007102192544548636\,b\,r-
 0.06062728715262111\,r+c , 5.968747148772726 \times 10^{-5}\,a\,r^2+
 0.007725766724910044\,b\,r-0.06410317632206519\,r+c , 
 7.030098113418114 \times 10^{-5}\,a\,r^2+0.00838456803503801\,b\,r-
 0.06767265439396564\,r+c , 8.243787568058321 \times 10^{-5}\,a\,r^2+
 0.009079530587017326\,b\,r-0.07133536442348987\,r+c , 
 9.626719779540763 \times 10^{-5}\,a\,r^2+0.009811584876838586\,b\,r-
 0.07509094014268702\,r+c , 1.11971479496896 \times 10^{-4}\,a\,r^2+
 0.0105816576913495\,b\,r-0.07893900599711501\,r+c , 
 1.297474089664522 \times 10^{-4}\,a\,r^2+0.01139067201557714\,b\,r-
 0.08287917718339499\,r+c , 1.498065093069853 \times 10^{-4}\,a\,r^2+
 0.01223954694042984\,b\,r-0.08691105968769186\,r+c , 
 1.723758288528179 \times 10^{-4}\,a\,r^2+0.01312919757078923\,b\,r-
 0.09103425032511492\,r+c , 1.976986426302469 \times 10^{-4}\,a\,r^2+
 0.01406053493400045\,b\,r-0.09524833678003664\,r+c , 
 2.260351645605837 \times 10^{-4}\,a\,r^2+0.01503446588876983\,b\,r-
 0.09955289764732322\,r+c , 2.576632699903951 \times 10^{-4}\,a\,r^2+
 0.01605189303448024\,b\,r-0.1039475024744748\,r+c , 
 2.928792281266932 \times 10^{-4}\,a\,r^2+0.01711371462093175\,b\,r-
 0.1084317118046711\,r+c , 3.319984439480964 \times 10^{-4}\,a\,r^2+
 0.01822082445851714\,b\,r-0.113005077220716\,r+c , 
 3.753562091564763 \times 10^{-4}\,a\,r^2+0.01937411182884202\,b\,r-
 0.1176671413898787\,r+c , 4.233084617271431 \times 10^{-4}\,a\,r^2+
 0.02057446139579705\,b\,r-0.1224174381096274\,r+c , 
 4.762325536095718 \times 10^{-4}\,a\,r^2+0.02182275311709253\,b\,r-
 0.1272554923542488\,r+c , 5.34528026124617 \times 10^{-4}\,a\,r^2+
 0.02311986215626333\,b\,r-0.1321808203223502\,r+c , 
 5.986173925984417 \times 10^{-4}\,a\,r^2+0.02446665879515308\,b\,r-
 0.1371929294852391\,r+c , 6.689469277678383 \times 10^{-4}\,a\,r^2+
 0.02586400834688696\,b\,r-0.1422913186361759\,r+c , 
 7.459874634862211 \times 10^{-4}\,a\,r^2+0.02731277106934082\,b\,r-
 0.1474754779404944\,r+c , 8.302351902545073 \times 10^{-4}\,a\,r^2+
 0.02881380207911666\,b\,r-0.152744888986584\,r+c , 
 9.222124640960191 \times 10^{-4}\,a\,r^2+0.03036795126603076\,b\,r-
 0.1580990248377314\,r+c , 0.001022468618290102\,a\,r^2+
 0.03197606320812652\,b\,r-0.1635373500848132\,r+c , 
 0.001131580779474263\,a\,r^2+0.0336389770872163\,b\,r-
 0.1690593208998367\,r+c , 0.001250154687620788\,a\,r^2+
 0.03535752660496472\,b\,r-0.1746643850903219\,r+c , 
 0.001378825519389357\,a\,r^2+0.03713253989951881\,b\,r-
 0.1803519821545206\,r+c , 0.001518258714353595\,a\,r^2+
 0.03896483946269502\,b\,r-0.1861215433374662\,r+c , 
 0.001669150803595751\,a\,r^2+0.0408552420577305\,b\,r-
 0.1919724916878484\,r+c , 0.001832230240160423\,a\,r^2+
 0.04280455863760801\,b\,r-0.1979042421157076\,r+c , 
 0.002008258230854871\,a\,r^2+0.04481359426396048\,b\,r-
 0.2039162014509444\,r+c , 0.002198029568880921\,a\,r^2+
 0.04688314802656623\,b\,r-0.2100077685026351\,r+c , 
 0.002402373466780307\,a\,r^2+0.04901401296344043\,b\,r-
 0.216178334119151\,r+c , 0.002622154389173151\,a\,r^2+
 0.05120697598153157\,b\,r-0.2224272812490723\,r+c , 
 0.002858272884767075\,a\,r^2+0.05346281777803219\,b\,r-
 0.2287539850028937\,r+c , 0.003111666417112067\,a\,r^2+
 0.05578231276230905\,b\,r-0.2351578127155118\,r+c , 
 0.003383310193575043\,a\,r^2+0.05816622897846346\,b\,r-
 0.2416381240094921\,r+c , 0.003674217992005929\,a\,r^2+
 0.06061532802852698\,b\,r-0.2481942708591053\,r+c , 
 0.003985442984566339\,a\,r^2+0.0631303649963022\,b\,r-
 0.2548255976551299\,r+c , 0.004318078558190487\,a\,r^2+
 0.06571208837185505\,b\,r-0.2615314412704124\,r+c , 
 0.004673259131147316\,a\,r^2+0.06836123997666599\,b\,r-
 0.2683111311261794\,r+c , 0.005052160965172387\,a\,r^2+
 0.07107855488944881\,b\,r-0.2751639892590951\,r+c , 
 0.005456002972637555\,a\,r^2+0.07386476137264342\,b\,r-
 0.2820893303890569\,r+c , 0.005886047518226416\,a\,r^2+
 0.07672058079958999\,b\,r-0.2890864619877229\,r+c , 
 0.006343601214583815\,a\,r^2+0.07964672758239233\,b\,r-
 0.2961546843477643\,r+c , 0.006830015711407966\,a\,r^2+
 0.08264390910047736\,b\,r-0.3032932906528349\,r+c , 
 0.007346688477454374\,a\,r^2+0.0857128256298576\,b\,r-
 0.3105015670482534\,r+c , 0.007895063574921807\,a\,r^2+
 0.08885417027310427\,b\,r-0.3177787927123868\,r+c , 
 0.008476632425691433\,a\,r^2+0.09206862889003742\,b\,r-
 0.3251242399287333\,r+c , 0.009092934568891969\,a\,r^2+
 0.09535688002914089\,b\,r-0.3325371741586922\,r+c , 
 0.009745558409264787\,a\,r^2+0.0987195948597075\,b\,r-
 0.3400168541150183\,r+c , 0.01043614195580549\,a\,r^2+
 0.1021574371047232\,b\,r-0.3475625318359485\,r+c , 
 0.01116637355015972\,a\,r^2+0.1056710629744951\,b\,r-
 0.3551734527599992\,r+c , 0.01193799258425414\,a\,r^2+
 0.1092611211010309\,b\,r-0.3628488558014202\,r+c , 
 0.01275279020664547\,a\,r^2+0.1129282524731764\,b\,r-
 0.3705879734263036\,r+c , 0.01361261001707348\,a\,r^2+
 0.1166730903725168\,b\,r-0.3783900317293359\,r+c , 
 0.01451934874870728\,a\,r^2+0.1204962603100498\,b\,r-
 0.3862542505111889\,r+c , 0.01547495693757671\,a\,r^2+
 0.1243983799636342\,b\,r-0.3941798433565377\,r+c , 
 0.01648143957868493\,a\,r^2+0.1283800591162231\,b\,r-
 0.4021660177127022\,r+c , 0.01754085676830185\,a\,r^2+
 0.1324418995948859\,b\,r-0.4102119749689023\,r+c , 
 0.01865532433194167\,a\,r^2+0.1365844952106265\,b\,r-
 0.418316910536117\,r+c , 0.01982701443753252\,a\,r^2+
 0.140808431699002\,b\,r-0.4264800139275439\,r+c , 
 0.02105815619329058\,a\,r^2+0.1451142866615502\,b\,r-
 0.4347004688396462\,r+c , 0.02235103622981523\,a\,r^2+
 0.1495026295080298\,b\,r-0.4429774532337832\,r+c , 
 0.02370799926592746\,a\,r^2+0.1539740213994798\,b\,r-
 0.451310139418413\,r+c \right] 
\]
\end{eulerformula}
\begin{eulerprompt}
>Fx &= diff(F(x,y),x), Fxx &=diff(F(x,y),x,2), Fy &=diff(F(x,y),y), Fxy &=diff(diff(F(x,y),x),y), Fyy &=diff(F(x,y),y,2)
\end{eulerprompt}
\begin{euleroutput}
  Maxima said:
  diff: second argument must be a variable; found errexp1
   -- an error. To debug this try: debugmode(true);
  
  Error in:
  Fx &= diff(F(x,y),x), Fxx &=diff(F(x,y),x,2), Fy &=diff(F(x,y) ...
                      ^
\end{euleroutput}
\begin{eulerprompt}
>function k(x) &= (Fy^2*Fxx-2*Fx*Fy*Fxy+Fx^2*Fyy)/(Fx^2+Fy^2)^(3/2); $'k(x)=k(x) // kurvatur parabola tersebut
\end{eulerprompt}
\begin{eulerformula}
\[
k\left(\left[ 0 , 1.66665833335744 \times 10^{-7}\,r , 
 1.33330666692022 \times 10^{-6}\,r , 
 4.499797504338432 \times 10^{-6}\,r , 
 1.066581336583994 \times 10^{-5}\,r , 
 2.083072932167196 \times 10^{-5}\,r , 
 3.599352055540239 \times 10^{-5}\,r , 
 5.71526624672386 \times 10^{-5}\,r , 
 8.530603082730626 \times 10^{-5}\,r , 
 1.214508019889565 \times 10^{-4}\,r , 
 1.665833531718508 \times 10^{-4}\,r , 
 2.216991628251896 \times 10^{-4}\,r , 
 2.877927110806339 \times 10^{-4}\,r , 
 3.658573803051457 \times 10^{-4}\,r , 
 4.568853557635201 \times 10^{-4}\,r , 
 5.618675264007778 \times 10^{-4}\,r , 
 6.817933857540259 \times 10^{-4}\,r , 
 8.176509330039827 \times 10^{-4}\,r , 
 9.704265741758145 \times 10^{-4}\,r , 0.001141105023499428\,r , 
 0.001330669204938795\,r , 0.001540100153900437\,r , 
 0.001770376919130678\,r , 0.002022476464811601\,r , 
 0.002297373572865413\,r , 0.002596040745477063\,r , 
 0.002919448107844891\,r , 0.003268563311168871\,r , 
 0.003644351435886262\,r , 0.004047774895164447\,r , 
 0.004479793338660443\,r , 0.0049413635565565\,r , 
 0.005433439383882244\,r , 0.005956971605131645\,r , 
 0.006512907859185624\,r , 0.007102192544548636\,r , 
 0.007725766724910044\,r , 0.00838456803503801\,r , 
 0.009079530587017326\,r , 0.009811584876838586\,r , 
 0.0105816576913495\,r , 0.01139067201557714\,r , 0.01223954694042984
 \,r , 0.01312919757078923\,r , 0.01406053493400045\,r , 
 0.01503446588876983\,r , 0.01605189303448024\,r , 
 0.01711371462093175\,r , 0.01822082445851714\,r , 
 0.01937411182884202\,r , 0.02057446139579705\,r , 
 0.02182275311709253\,r , 0.02311986215626333\,r , 
 0.02446665879515308\,r , 0.02586400834688696\,r , 
 0.02731277106934082\,r , 0.02881380207911666\,r , 
 0.03036795126603076\,r , 0.03197606320812652\,r , 0.0336389770872163
 \,r , 0.03535752660496472\,r , 0.03713253989951881\,r , 
 0.03896483946269502\,r , 0.0408552420577305\,r , 0.04280455863760801
 \,r , 0.04481359426396048\,r , 0.04688314802656623\,r , 
 0.04901401296344043\,r , 0.05120697598153157\,r , 
 0.05346281777803219\,r , 0.05578231276230905\,r , 
 0.05816622897846346\,r , 0.06061532802852698\,r , 0.0631303649963022
 \,r , 0.06571208837185505\,r , 0.06836123997666599\,r , 
 0.07107855488944881\,r , 0.07386476137264342\,r , 
 0.07672058079958999\,r , 0.07964672758239233\,r , 
 0.08264390910047736\,r , 0.0857128256298576\,r , 0.08885417027310427
 \,r , 0.09206862889003742\,r , 0.09535688002914089\,r , 
 0.0987195948597075\,r , 0.1021574371047232\,r , 0.1056710629744951\,
 r , 0.1092611211010309\,r , 0.1129282524731764\,r , 
 0.1166730903725168\,r , 0.1204962603100498\,r , 0.1243983799636342\,
 r , 0.1283800591162231\,r , 0.1324418995948859\,r , 
 0.1365844952106265\,r , 0.140808431699002\,r , 0.1451142866615502\,r
  , 0.1495026295080298\,r , 0.1539740213994798\,r \right] \right)=
 \frac{{\it Fx}^2\,{\it Fyy}+{\it Fxx}\,{\it Fy}^2-2\,{\it Fx}\,
 {\it Fxy}\,{\it Fy}}{\left({\it Fy}^2+{\it Fx}^2\right)^{\frac{3}{2}
 }}
\]
\end{eulerformula}
\begin{eulerformula}
\[
k\left(x\right)=\frac{2\,a}{\left(\left(2\,a\,x+b\right)^2+1\right)  ^{\frac{3}{2}}}
\]
\end{eulerformula}
\begin{eulercomment}
Hasilnya sama dengan sebelumnya yang menggunakan persamaan parabola
biasa.

\begin{eulercomment}
\eulerheading{Latihan}
\begin{eulercomment}
- Bukalah buku Kalkulus.\\
- Cari dan pilih beberapa (paling sedikit 5 fungsi berbeda
tipe/bentuk/jenis) fungsi dari buku tersebut, kemudian definisikan di\\
EMT pada baris-baris perintah berikut (jika perlu tambahkan lagi).\\
- Untuk setiap fungsi, tentukan anti turunannya (jika ada), hitunglah
integral tentu dengan batas-batas yang menarik (Anda\\
tentukan sendiri), seperti contoh-contoh tersebut.\\
- Lakukan hal yang sama untuk fungsi-fungsi yang tidak dapat
diintegralkan (cari sedikitnya 3 fungsi).\\
- Gambar grafik fungsi dan daerah integrasinya pada sumbu koordinat
yang sama.\\
- Gunakan integral tentu untuk mencari luas daerah yang dibatasi oleh
dua kurva yang berpotongan di dua titik. (Cari dan gambar\\
kedua kurva dan arsir (warnai) daerah yang dibatasi oleh keduanya.)\\
- Gunakan integral tentu untuk menghitung volume benda putar kurva y=
f(x) yang diputar mengelilingi sumbu x dari x=a sampai x=b,\\
yakni

\end{eulercomment}
\begin{eulerformula}
\[
V = \int_a^b \pi (f(x)^2\ dx
\]
\end{eulerformula}
\begin{eulercomment}
(Pilih fungsinya dan gambar kurva dan benda putar yang dihasilkan.
Anda dapat mencari contoh-contoh bagaimana cara menggambar\\
benda hasil perputaran suatu kurva.)\\
- Gunakan integral tentu untuk menghitung panjang kurva y=f(x) dari
x=a sampai x=b dengan menggunakan rumus:

\end{eulercomment}
\begin{eulerformula}
\[
S = \int_a^b \sqrt{1+(f'(x))^2} \ dx
\]
\end{eulerformula}
\begin{eulercomment}
(Pilih fungsi dan gambar kurvanya.)\\
- Apabila fungsi dinyatakan dalam koordinat kutub x=f(r,t), y=g(r,t),
r=h(t), x=a bersesuaian dengan t=t0 dan x=b bersesuian\\
dengan t=t1, maka rumus di atas akan menjadi:

\end{eulercomment}
\begin{eulerformula}
\[
S=\int_{t_0}^{t_1} \sqrt{x'(t)^2+y'(t)^2}\ dt
\]
\end{eulerformula}
\begin{eulercomment}
- Pilih beberapa kurva menarik (selain lingkaran dan parabola) dari
buku  kalkulus. Nyatakan setiap kurva tersebut dalam bentuk:

\end{eulercomment}
\begin{eulerttcomment}
  a. koordinat Kartesius (persamaan y=f(x))
  b. koordinat kutub ( r=r(theta))
  c. persamaan parametrik x=x(t), y=y(t)
  d. persamaan implit F(x,y)=0
\end{eulerttcomment}
\begin{eulercomment}

- Tentukan kurvatur masing-masing kurva dengan menggunakan keempat
representasi tersebut (hasilnya harus sama).\\
- Gambarlah kurva asli, kurva kurvatur, kurva jari-jari lingkaran
oskulasi, dan salah satu lingkaran oskulasinya.


\begin{eulercomment}
\eulerheading{Barisan dan Deret}
\begin{eulercomment}
(Catatan: bagian ini belum lengkap. Anda dapat membaca contoh-contoh
pengguanaan EMT dan\\
Maxima untuk menghitung limit barisan, rumus jumlah parsial suatu
deret, jumlah tak hingga\\
suatu deret konvergen, dan sebagainya. Anda dapat mengeksplor
contoh-contoh di EMT atau\\
perbagai panduan penggunaan Maxima di software Maxima atau dari
Internet.)

Barisan dapat didefinisikan dengan beberapa cara di dalam EMT, di
antaranya:

- dengan cara yang sama seperti mendefinisikan vektor dengan
elemen-elemen beraturan\\
(menggunakan titik dua ":");\\
- menggunakan perintah "sequence" dan rumus barisan (suku ke -n);\\
- menggunakan perintah "iterate" atau "niterate";\\
- menggunakan fungsi Maxima "create\textbackslash{}\_list" atau "makelist" untuk
menghasilkan barisan\\
simbolik;\\
- menggunakan fungsi biasa yang inputnya vektor atau barisan;\\
- menggunakan fungsi rekursif.

EMT menyediakan beberapa perintah (fungsi) terkait barisan, yakni:

- sum: menghitung jumlah semua elemen suatu barisan\\
- cumsum: jumlah kumulatif suatu barisan\\
- differences: selisih antar elemen-elemen berturutan

EMT juga dapat digunakan untuk menghitung jumlah deret berhingga
maupun deret tak hingga,\\
dengan menggunakan perintah (fungsi) "sum". Perhitungan dapat
dilakukan secara numerik\\
maupun simbolik dan eksak.

Berikut adalah beberapa contoh perhitungan barisan dan deret
menggunakan EMT.
\end{eulercomment}
\begin{eulerprompt}
>1:10 // barisan sederhana
\end{eulerprompt}
\begin{euleroutput}
  [1,  2,  3,  4,  5,  6,  7,  8,  9,  10]
\end{euleroutput}
\begin{eulerprompt}
>1:2:30
\end{eulerprompt}
\begin{euleroutput}
  [1,  3,  5,  7,  9,  11,  13,  15,  17,  19,  21,  23,  25,  27,  29]
\end{euleroutput}
\begin{eulercomment}
\begin{eulercomment}
\eulerheading{Iterasi dan Barisan EMT menyediakan fungsi iterate("g(x)", x0, n)}
\begin{eulercomment}
untuk melakukan iterasi

\end{eulercomment}
\begin{eulerformula}
\[
x_{k+1}=g(x_k), \ x_0=x_0, k= 1, 2, 3, ..., n
\]
\end{eulerformula}
\begin{eulercomment}
Berikut ini disajikan contoh-contoh penggunaan iterasi dan rekursi
dengan EMT. Contoh\\
pertama menunjukkan pertumbuhan dari nilai awal 1000 dengan laju
pertambahan 5\textbackslash{}\%, selama 10 periode.
\end{eulercomment}
\begin{eulerprompt}
>q=1.05; iterate("x*q",1000,n=10)'
\end{eulerprompt}
\begin{euleroutput}
           1000 
           1050 
         1102.5 
        1157.63 
        1215.51 
        1276.28 
         1340.1 
         1407.1 
        1477.46 
        1551.33 
        1628.89 
\end{euleroutput}
\begin{eulercomment}
Contoh berikutnya memperlihatkan bahaya menabung di bank pada masa
sekarang! Dengan bunga\\
tabungan sebesar 6\textbackslash{}\% per tahun atau 0.5\textbackslash{}\% per bulan dipotong pajak
20\textbackslash{}\%, dan biaya administrasi\\
10000 per bulan, tabungan sebesar 1 juta tanpa diambil selama sekitar
10 tahunan akan habis diambil oleh bank!
\end{eulercomment}
\begin{eulerprompt}
>r=0.005; plot2d(iterate("(1+0.8*r)*x-10000",1000000,n=130)):
\end{eulerprompt}
\eulerimg{27}{images/Pekan 9-10_Fanny Erina Dewi_22305141005_EMT00-Kalkulus_Aplikom-135.png}
\begin{eulercomment}
Silakan Anda coba-coba, dengan tabungan minimal berapa agar tidak akan
habis diambil oleh bank dengan ketentuan bunga dan biaya administrasi
seperti di atas.

Berikut adalah perhitungan minimal tabungan agar aman di bank dengan
bunga sebesar r dan biaya administrasi a, pajak bunga 20\textbackslash{}\%.
\end{eulercomment}
\begin{eulerprompt}
>$solve(0.8*r*A-a,A), $% with [r=0.005, a=10] 
\end{eulerprompt}
\begin{eulerformula}
\[
\left[ A=\frac{5\,a}{4\,r} \right] 
\]
\end{eulerformula}
\begin{eulerformula}
\[
\left[ A=2500.0 \right] 
\]
\end{eulerformula}
\begin{eulercomment}
Berikut didefinisikan fungsi untuk menghitung saldo tabungan, kemudian
dilakukan iterasi.
\end{eulercomment}
\begin{eulerprompt}
>function saldo(x,r,a) := round((1+0.8*r)*x-a,2);
>iterate(\{\{"saldo",0.005,10\}\},1000,n=6)
\end{eulerprompt}
\begin{euleroutput}
  [1000,  994,  987.98,  981.93,  975.86,  969.76,  963.64]
\end{euleroutput}
\begin{eulerprompt}
>iterate(\{\{"saldo",0.005,10\}\},2000,n=6)
\end{eulerprompt}
\begin{euleroutput}
  [2000,  1998,  1995.99,  1993.97,  1991.95,  1989.92,  1987.88]
\end{euleroutput}
\begin{eulerprompt}
>iterate(\{\{"saldo",0.005,10\}\},2500,n=6)
\end{eulerprompt}
\begin{euleroutput}
  [2500,  2500,  2500,  2500,  2500,  2500,  2500]
\end{euleroutput}
\begin{eulercomment}
Tabungan senilai 2,5 juta akan aman dan tidak akan berubah nilai (jika
tidak ada penarikan), sedangkan jika tabungan awal kurang dari 2,5
juta, lama kelamaan akan berkurang meskipun tidak pernah dilakukan
penarikan uang tabungan.
\end{eulercomment}
\begin{eulerprompt}
>iterate(\{\{"saldo",0.005,10\}\},3000,n=6)
\end{eulerprompt}
\begin{euleroutput}
  [3000,  3002,  3004.01,  3006.03,  3008.05,  3010.08,  3012.12]
\end{euleroutput}
\begin{eulercomment}
Tabungan yang lebih dari 2,5 juta baru akan bertambah jika tidak ada
penarikan.\\
Untuk barisan yang lebih kompleks dapat digunakan fungsi "sequence()".
Fungsi ini menghitung nilai-nilai\\
x[n] dari semua nilai sebelumnya, x[1],...,x[n-1] yang diketahui.\\
Berikut adalah contoh barisan Fibonacci.

\end{eulercomment}
\begin{eulerformula}
\[
x_n = x_{n-1}+x_{n-2}, \quad x_1=1, \quad x_2 =1
\]
\end{eulerformula}
\begin{eulerprompt}
>sequence("x[n-1]+x[n-2]",[1,1],15)
\end{eulerprompt}
\begin{euleroutput}
  [1,  1,  2,  3,  5,  8,  13,  21,  34,  55,  89,  144,  233,  377,  610]
\end{euleroutput}
\begin{eulercomment}
Barisan Fibonacci memiliki banyak sifat menarik, salah satunya adalah
akar pangkat ke-n suku ke-n akan konvergen ke pecahan emas:
\end{eulercomment}
\begin{eulerprompt}
>$'(1+sqrt(5))/2=float((1+sqrt(5))/2)
\end{eulerprompt}
\begin{eulerformula}
\[
\frac{\sqrt{5}+1}{2}=1.618033988749895
\]
\end{eulerformula}
\begin{eulerprompt}
>plot2d(sequence("x[n-1]+x[n-2]",[1,1],250)^(1/(1:250))):
\end{eulerprompt}
\eulerimg{27}{images/Pekan 9-10_Fanny Erina Dewi_22305141005_EMT00-Kalkulus_Aplikom-139.png}
\begin{eulercomment}
Barisan yang sama juga dapat dihasilkan dengan menggunakan loop.
\end{eulercomment}
\begin{eulerprompt}
>x=ones(500); for k=3 to 500; x[k]=x[k-1]+x[k-2]; end;
\end{eulerprompt}
\begin{eulercomment}
Rekursi dapat dilakukan dengan menggunakan rumus yang tergantung pada
semua elemen sebelumnya. Pada contoh berikut, elemen ke-n merupakan
jumlah (n-1) elemen sebelumnya, dimulai dengan 1 (elemen ke-1). Jelas,
nilai elemen ke-n adalah 2\textbackslash{}textasciicircum\{\}(n-2), untuk n=2, 4, 5,\\
....
\end{eulercomment}
\begin{eulerprompt}
>sequence("sum(x)",1,10)
\end{eulerprompt}
\begin{euleroutput}
  [1,  1,  2,  4,  8,  16,  32,  64,  128,  256]
\end{euleroutput}
\begin{eulercomment}
Selain menggunakan ekspresi dalam x dan n, kita juga dapat menggunakan
fungsi.

Pada contoh berikut, digunakan iterasi

\end{eulercomment}
\begin{eulerformula}
\[
x_n =A \cdot x_{n-1},
\]
\end{eulerformula}
\begin{eulercomment}
dengan A suatu matriks 2x2, dan setiap x[n] merupakan matriks/vektor
2x1.
\end{eulercomment}
\begin{eulerprompt}
>A=[1,1;1,2]; function suku(x,n) := A.x[,n-1]
>sequence("suku",[1;1],6)
\end{eulerprompt}
\begin{euleroutput}
  Real 2 x 6 matrix
  
              1             2             5            13     ...
              1             3             8            21     ...
\end{euleroutput}
\begin{eulercomment}
Hasil yang sama juga dapat diperoleh dengan menggunakan fungsi
perpangkatan matriks "matrixpower()". Cara ini lebih cepat, karena
hanya menggunakan perkalian matriks sebanyak log\textbackslash{}\_2(n).

\end{eulercomment}
\begin{eulerformula}
\[
x_n=A.x_{n-1}=A^2.x_{n-2}=A^3.x_{n-3}= ... = A^{n-1}.x_1.
\]
\end{eulerformula}
\begin{eulerprompt}
>sequence("matrixpower(A,n).[1;1]",1,6)
\end{eulerprompt}
\begin{euleroutput}
  Real 2 x 6 matrix
  
              1             5            13            34     ...
              1             8            21            55     ...
\end{euleroutput}
\begin{eulercomment}
\begin{eulercomment}
\eulerheading{Spiral Theodorus}
\begin{eulercomment}
image: theodorus.png

Spiral Theodorus (spiral segitiga siku-siku) dapat digambar secara
rekursif. Rumus rekursifnya adalah:

\end{eulercomment}
\begin{eulerformula}
\[
x_n = \left( 1 + \frac{i}{\sqrt{n-1}} \right) \, x_{n-1}, \quad x_1=1,
\]
\end{eulerformula}
\begin{eulercomment}
yang menghasilkan barisan bilangan kompleks.
\end{eulercomment}
\begin{eulerprompt}
>function g(n) := 1+I/sqrt(n)
\end{eulerprompt}
\begin{eulercomment}
Rekursinya dapat dijalankan sebanyak 17 untuk menghasilkan barisan 17
bilangan kompleks,kemudian digambar bilangan-bilangan kompleksnya.
\end{eulercomment}
\begin{eulerprompt}
>x=sequence("g(n-1)*x[n-1]",1,17);plot2d(x,r=3.5);textbox(latex("Spiral\(\backslash\)(\(\backslash\)backslash\(\backslash\))Theodorus"),0.4):
\end{eulerprompt}
\begin{euleroutput}
  Space between commands expected!
  Found: plot2d(x,r=3.5);textbox(latex("Spiral\(\backslash\)(\(\backslash\)backslash\(\backslash\))Theodorus"),0.4): (character 112)
  You can disable this in the Options menu.
  Error in:
  x=sequence("g(n-1)*x[n-1]",1,17);plot2d(x,r=3.5);textbox(latex ...
                                   ^
\end{euleroutput}
\begin{eulercomment}
Selanjutnya dihubungan titik 0 dengan titik-titik kompleks tersebut
menggunakan loop.
\end{eulercomment}
\begin{eulerprompt}
>for i=1:cols(x); plot2d([0,x[i]],>add); end:
\end{eulerprompt}
\eulerimg{27}{images/Pekan 9-10_Fanny Erina Dewi_22305141005_EMT00-Kalkulus_Aplikom-140.png}
\begin{eulercomment}
Spiral tersebut juga dapat didefinisikan menggunakan fungsi rekursif,
yang tidak memmerlukan indeks dan bilangan kompleks. Dalam hal ini
diigunakan vektor kolom pada bidang.
\end{eulercomment}
\begin{eulerprompt}
>function gstep (v) ...
\end{eulerprompt}
\begin{eulerudf}
  w=[-v[2];v[1]];
  return v+w/norm(w);
  endfunction
\end{eulerudf}
\begin{eulercomment}
Jika dilakukan iterasi 16 kali dimulai dari [1;0] akan didapatkan
matriks yang memuat vektor-vektor dari setiap iterasi.
\end{eulercomment}
\begin{eulerprompt}
>x=iterate("gstep",[1;0],16); plot2d(x[1],x[2],r=3.5,>points):
\end{eulerprompt}
\eulerimg{27}{images/Pekan 9-10_Fanny Erina Dewi_22305141005_EMT00-Kalkulus_Aplikom-141.png}
\begin{eulercomment}
\begin{eulercomment}
\eulerheading{Kekonvergenan}
\begin{eulercomment}
Terkadang kita ingin melakukan iterasi sampai konvergen. Apabila
iterasinya tidak konvergen setelah ditunggu lama, Anda dapat
menghentikannya dengan menekan tombol [ESC].
\end{eulercomment}
\begin{eulerprompt}
>iterate("cos(x)",1) // iterasi x(n+1)=cos(x(n)), dengan x(0)=1.
\end{eulerprompt}
\begin{euleroutput}
  0.739085133216
\end{euleroutput}
\begin{eulercomment}
Iterasi tersebut konvergen ke penyelesaian persamaan

\end{eulercomment}
\begin{eulerformula}
\[
x = \cos(x)
\]
\end{eulerformula}
\begin{eulercomment}
Iterasi ini juga dapat dilakukan pada interval, hasilnya adalah
barisan interval yang memuat akar tersebut.
\end{eulercomment}
\begin{eulerprompt}
>hasil := iterate("cos(x)",~1,2~) //iterasi x(n+1)=cos(x(n)), dengan interval awal (1, 2)
\end{eulerprompt}
\begin{euleroutput}
  ~0.739085133211,0.7390851332133~
\end{euleroutput}
\begin{eulercomment}
Jika interval hasil tersebut sedikit diperlebar, akan terlihat bahwa
interval tersebut memuat akar persamaan x=cos(x).
\end{eulercomment}
\begin{eulerprompt}
>h=expand(hasil,100), cos(h) << h
\end{eulerprompt}
\begin{euleroutput}
  ~0.73908513309,0.73908513333~
  1
\end{euleroutput}
\begin{eulercomment}
Iterasi juga dapat digunakan pada fungsi yang didefinisikan.
\end{eulercomment}
\begin{eulerprompt}
>function f(x) := (x+2/x)/2
\end{eulerprompt}
\begin{eulercomment}
Iterasi x(n+1)=f(x(n)) akan konvergen ke akar kuadrat 2.
\end{eulercomment}
\begin{eulerprompt}
>iterate("f",2), sqrt(2)
\end{eulerprompt}
\begin{euleroutput}
  1.41421356237
  1.41421356237
\end{euleroutput}
\begin{eulercomment}
Jika pada perintah iterate diberikan tambahan parameter n, maka hasil
iterasinya akan ditampilkan mulai dari iterasi pertama sampai ke-n.
\end{eulercomment}
\begin{eulerprompt}
>iterate("f",2,5)
\end{eulerprompt}
\begin{euleroutput}
  [2,  1.5,  1.41667,  1.41422,  1.41421,  1.41421]
\end{euleroutput}
\begin{eulercomment}
Untuk iterasi ini tidak dapat dilakukan terhadap interval.
\end{eulercomment}
\begin{eulerprompt}
>niterate("f",~1,2~,5)
\end{eulerprompt}
\begin{euleroutput}
  [ ~1,2~,  ~1,2~,  ~1,2~,  ~1,2~,  ~1,2~,  ~1,2~ ]
\end{euleroutput}
\begin{eulercomment}
Perhatikan, hasil iterasinya sama dengan interval awal. Alasannya
adalah perhitungan dengan interval bersifat terlalu longgar. Untuk
meingkatkan perhitungan pada ekspresi dapat digunakan pembagian
intervalnya, menggunakan fungsi ieval().
\end{eulercomment}
\begin{eulerprompt}
>function s(x) := ieval("(x+2/x)/2",x,10)
\end{eulerprompt}
\begin{eulercomment}
Selanjutnya dapat dilakukan iterasi hingga diperoleh hasil optimal,
dan intervalnya tidak semakin mengecil. Hasilnya berupa interval yang
memuat akar persamaan:

\end{eulercomment}
\begin{eulerformula}
\[
x = \frac{1}{2} \left( x + \frac{2}{x} \right)
\]
\end{eulerformula}
\begin{eulercomment}
Satu-satunya solusi adalah

\end{eulercomment}
\begin{eulerformula}
\[
x = \sqrt2.
\]
\end{eulerformula}
\begin{eulerprompt}
>iterate("s",~1,2~)
\end{eulerprompt}
\begin{euleroutput}
  ~1.41421356236,1.41421356239~
\end{euleroutput}
\begin{eulercomment}
Fungsi "iterate()" juga dapat bekerja pada vektor. Berikut adalah
contoh fungsi vektor, yang menghasilkan rata-rata aritmetika dan
rata-rata geometri.

\end{eulercomment}
\begin{eulerformula}
\[
(a_{n+1},b_{n+1}) = \left( \frac{a_n+b_n}{2}, \sqrt{a_nb_n} \right)
\]
\end{eulerformula}
\begin{eulercomment}
Iterasi ke-n disimpan pada vektor kolom x[n].
\end{eulercomment}
\begin{eulerprompt}
>function g(x) := [(x[1]+x[2])/2;sqrt(x[1]*x[2])]
\end{eulerprompt}
\begin{eulercomment}
Iterasi dengan menggunakan fungsi tersebut akan konvergen ke rata-rata
aritmetika dan geometri dari nilai-nilai awal. 
\end{eulercomment}
\begin{eulerprompt}
>iterate("g",[1;5])
\end{eulerprompt}
\begin{euleroutput}
        2.60401 
        2.60401 
\end{euleroutput}
\begin{eulercomment}
Hasil tersebut konvergen agak cepat, seperti kita cek sebagai berikut.
\end{eulercomment}
\begin{eulerprompt}
>iterate("g",[1;5],4)
\end{eulerprompt}
\begin{euleroutput}
              1             3       2.61803       2.60403       2.60401 
              5       2.23607       2.59002       2.60399       2.60401 
\end{euleroutput}
\begin{eulercomment}
Iterasi pada interval dapat dilakukan dan stabil, namun tidak
menunjukkan bahwa limitnya pada batas-batas yang dihitung.
\end{eulercomment}
\begin{eulerprompt}
>iterate("g",[~1~;~5~],4)
\end{eulerprompt}
\begin{euleroutput}
  Interval 2 x 5 matrix
  
  ~0.999999999999999778,1.00000000000000022~     ...
  ~4.99999999999999911,5.00000000000000089~     ...
\end{euleroutput}
\begin{eulercomment}
Iterasi berikut konvergen sangat lambat.

\end{eulercomment}
\begin{eulerformula}
\[
Iterasi berikut konvergen sangat lambat.
\]
\end{eulerformula}
\begin{eulerprompt}
>iterate("sqrt(x)",2,10)
\end{eulerprompt}
\begin{euleroutput}
  [2,  1.41421,  1.18921,  1.09051,  1.04427,  1.0219,  1.01089,
  1.00543,  1.00271,  1.00135,  1.00068]
\end{euleroutput}
\begin{eulercomment}
Kekonvergenan iterasi tersebut dapat dipercepatdengan percepatan
Steffenson:
\end{eulercomment}
\begin{eulerprompt}
>steffenson("sqrt(x)",2,10)
\end{eulerprompt}
\begin{euleroutput}
  [1.04888,  1.00028,  1,  1]
\end{euleroutput}
\begin{eulercomment}
\begin{eulercomment}
\eulerheading{Iterasi menggunakan Loop yang ditulis Langsung}
\begin{eulercomment}
Berikut adalah beberapa contoh penggunaan loop untuk melakukan iterasi
yang ditulis langsung pada baris perintah.
\end{eulercomment}
\begin{eulerprompt}
>x=2; repeat x=(x+2/x)/2; until x^2~=2; end; x,
\end{eulerprompt}
\begin{euleroutput}
  1.41421356237
\end{euleroutput}
\begin{eulercomment}
Penggabungan matriks menggunakan tanda "\textbackslash{}textbar\{\}" dapat digunakan
untuk menyimpan semua hasil iterasi.
\end{eulercomment}
\begin{eulerprompt}
>v=[1]; for i=2 to 8; v=v|(v[i-1]*i); end; v,
\end{eulerprompt}
\begin{euleroutput}
  [1,  2,  6,  24,  120,  720,  5040,  40320]
\end{euleroutput}
\begin{eulercomment}
hasil iterasi juga dapat disimpan pada vektor yang sudah ada.
\end{eulercomment}
\begin{eulerprompt}
>v=ones(1,100); for i=2 to cols(v); v[i]=v[i-1]*i; end; ...
>plot2d(v,logplot=1); textbox(latex(&log(n)),x=0.5):
\end{eulerprompt}
\eulerimg{27}{images/Pekan 9-10_Fanny Erina Dewi_22305141005_EMT00-Kalkulus_Aplikom-142.png}
\begin{eulerprompt}
>A =[0.5,0.2;0.7,0.1]; b=[2;2]; ...
>x=[1;1]; repeat xnew=A.x-b; until all(xnew~=x); x=xnew; end; ...
>x,
\end{eulerprompt}
\begin{euleroutput}
       -7.09677 
       -7.74194 
\end{euleroutput}
\begin{eulercomment}
\begin{eulercomment}
\eulerheading{Iterasi di dalam Fungsi}
\begin{eulercomment}
Fungsi atau program juga dapat menggunakan iterasi dan dapat digunakan
untuk melakukan iterasi. Berikut adalah beberapa contoh iterasi di
dalam fungsi.

Contoh berikut adalah suatu fungsi untuk menghitung berapa lama suatu
iterasi konvergen. Nilai fungsi tersebut adalah hasil akhir iterasi
dan banyak iterasi sampai konvergen.
\end{eulercomment}
\begin{eulerprompt}
>function map hiter(f$,x0) ...
\end{eulerprompt}
\begin{eulerudf}
  
  x=x0;
  maxiter=0;
  repeat
    xnew=f$(x);
    maxiter=maxiter+1;
    until xnew~=x;
    x=xnew;
  end;
  return maxiter;
  endfunction
\end{eulerudf}
\begin{eulercomment}
Misalnya, berikut adalah iterasi untuk mendapatkan hampiran akar
kuadrat 2, cukup cepat, konvergen pada iterasi ke-5, jika dimulai dari
hampiran awal 2.
\end{eulercomment}
\begin{eulerprompt}
>hiter("(x+2/x)/2",2)
\end{eulerprompt}
\begin{euleroutput}
  5
\end{euleroutput}
\begin{eulercomment}
Karena fungsinya didefinisikan menggunakan "map". maka nilai awalnya
dapat berupa vektor.
\end{eulercomment}
\begin{eulerprompt}
>x=1.5:0.1:10; hasil=hiter("(x+2/x)/2",x); ...
>plot2d(x,hasil):
\end{eulerprompt}
\eulerimg{27}{images/Pekan 9-10_Fanny Erina Dewi_22305141005_EMT00-Kalkulus_Aplikom-143.png}
\begin{eulercomment}
Dari gambar di atas terlihat bahwa kekonvergenan iterasinya semakin
lambat, untuk nilai awal semakin besar, namun penambahnnya tidak
kontinu. Kita dapat menemukan kapan maksimum iterasinya bertambah.
\end{eulercomment}
\begin{eulerprompt}
>hasil[1:10]
\end{eulerprompt}
\begin{euleroutput}
  [4,  5,  5,  5,  5,  5,  6,  6,  6,  6]
\end{euleroutput}
\begin{eulerprompt}
>x[nonzeros(differences(hasil))]
\end{eulerprompt}
\begin{euleroutput}
  [1.5,  2,  3.4,  6.6]
\end{euleroutput}
\begin{eulercomment}
maksimum iterasi sampai konvergen meningkat pada saat nilai awalnya
1.5, 2, 3.4, dan 6.6.

Contoh berikutnya adalah metode Newton pada polinomial kompleks
berderajat 3.
\end{eulercomment}
\begin{eulerprompt}
>p &= x^3-1; newton &= x-p/diff(p,x); $newton
\end{eulerprompt}
\begin{euleroutput}
  Maxima said:
  diff: second argument must be a variable; found errexp1
   -- an error. To debug this try: debugmode(true);
  
  Error in:
  p &= x^3-1; newton &= x-p/diff(p,x); $newton ...
                                     ^
\end{euleroutput}
\begin{eulercomment}
Selanjutnya didefinisikan fungsi untuk melakukan iterasi (aslinya 10
kali).
\end{eulercomment}
\begin{eulerudf}
  
  
  
  
  
\end{eulerudf}
\begin{eulerprompt}
>function iterasi(f$,x,n=10) ...
\end{eulerprompt}
\begin{eulerudf}
  loop 1 to n;x=f$(x);end
  return x;
  endfunction
\end{eulerudf}
\begin{eulercomment}
Kita mulai dengan menentukan titik-titik grid pada bidang kompleksnya.
\end{eulercomment}
\begin{eulerprompt}
>r=1.5; x=linspace(-r,r,501); Z=x+I*x'; W=iterasi(newton,Z);
\end{eulerprompt}
\begin{eulercomment}
Berikut adalah akar-akar polinomial di atas.
\end{eulercomment}
\begin{eulerprompt}
>z=&solve(p)()
\end{eulerprompt}
\begin{euleroutput}
  [ -0.5+0.866025i,  -0.5-0.866025i,  1+0i  ]
\end{euleroutput}
\begin{eulercomment}
Untuk menggambar hasil iterasinya, dihitung jarak dari hasil iterasi
ke-10 ke masing-masing akar, kemudian digunakan untuk menghitung warna
yang akan digambar, yang menunjukkan limit untuk masing-masing nilai
awal. 

Fungsi plotrgb() menggunakan jendela gambar terkini untuk menggambar
warna RGB sebagai matriks.
\end{eulercomment}
\begin{eulerprompt}
>C=rgb(max(abs(W-z[1]),1),max(abs(W-z[2]),1),max(abs(W-z[3]),1)); ...
>plot2d(none,-r,r,-r,r); plotrgb(C):
\end{eulerprompt}
\begin{eulercomment}
\begin{eulercomment}
\eulerheading{Iterasi Simbolik}
\begin{eulercomment}
Seperti sudah dibahas sebelumnya, untuk menghasilkan barisan ekspresi
simbolik dengan Maxima dapat digunakan fungsi makelist().
\end{eulercomment}
\begin{eulerprompt}
>deret &= makelist(taylor(exp(x),x,0,k),k,1,3); $deret // barisan deret Taylor untuk e^x
\end{eulerprompt}
\begin{eulercomment}
Untuk mengubah barisan deret tersebut menjadi vektor string di EMT
digunakan fungsi mxm2str(). Selanjutnya, vektor string/ekspresi
hasilnya dapat digambar seperti menggambar vektor eskpresi pada EMT.
\end{eulercomment}
\begin{eulerprompt}
>plot2d("exp(x)",0,3); // plot fungsi aslinya, e^x
>plot2d(mxm2str("deret"),>add,color=4:6): // plot ketiga deret taylor hampiran fungsi tersebut
\end{eulerprompt}
\begin{eulercomment}
Selain cara di atas dapat juga dengan cara menggunakan indeks pada
vektor/list yang dihasilkan.
\end{eulercomment}
\begin{eulerprompt}
>$deret[3]
>plot2d(["exp(x)",&deret[1],&deret[2],&deret[3]],0,3,color=1:4):
>$sum(sin(k*x)/k,k,1,5)
\end{eulerprompt}
\begin{eulercomment}
Berikut adalah cara menggambar kurva

\end{eulercomment}
\begin{eulerformula}
\[
y=\sin(x) + \dfrac{\sin 3x}{3} + \dfrac{\sin 5x}{5} + \ldots.
\]
\end{eulerformula}
\begin{eulerprompt}
>plot2d(&sum(sin((2*k+1)*x)/(2*k+1),k,0,20),0,2pi):
\end{eulerprompt}
\begin{eulercomment}
Hal serupa juga dapat dilakukan dengan menggunakan matriks, misalkan
kita akan menggambar kurva

\end{eulercomment}
\begin{eulerformula}
\[
y = \sum_{k=1}^{100} \dfrac{\sin(kx)}{k},\quad 0\le x\le 2\pi.
\]
\end{eulerformula}
\begin{eulerprompt}
>x=linspace(0,2pi,1000); k=1:100; y=sum(sin(k*x')/k)'; plot2d(x,y):
\end{eulerprompt}
\begin{eulercomment}
\begin{eulercomment}
\eulerheading{Tabel Fungsi}
\begin{eulercomment}
Terdapat cara menarik untuk menghasilkan barisan dengan ekspresi
Maxima. Perintah mxmtable() berguna untuk menampilkan dan menggambar
barisan dan menghasilkan barisan sebagai vektor kolom. 

Sebagai contoh berikut adalah barisan turunan ke-n
x\textbackslash{}textasciicircum\{\}x di x=1.
\end{eulercomment}
\begin{eulerprompt}
>mxmtable("diffat(x^x,x=1,n)","n",1,8,frac=1);
\end{eulerprompt}
\begin{euleroutput}
          1 
          2 
          3 
          8 
         10 
         54 
        -42 
        944 
\end{euleroutput}
\begin{eulerprompt}
>$'sum(k, k, 1, n) = factor(ev(sum(k, k, 1, n),simpsum=true)) // simpsum:menghitung deret secara simbolik
>$'sum(1/(3^k+k), k, 0, inf) = factor(ev(sum(1/(3^k+k), k, 0, inf),simpsum=true))
\end{eulerprompt}
\begin{eulercomment}
Di sini masih gagal, hasilnya tidak dihitung.
\end{eulercomment}
\begin{eulerprompt}
>$'sum(1/x^2, x, 1, inf)= ev(sum(1/x^2, x, 1, inf),simpsum=true) // ev: menghitung nilai ekspresi
>$'sum((-1)^(k-1)/k, k, 1, inf) = factor(ev(sum((-1)^(x-1)/x, x, 1, inf),simpsum=true))
\end{eulerprompt}
\begin{eulercomment}
Di sini masih gagal, hasilnya tidak dihitung.
\end{eulercomment}
\begin{eulerprompt}
>$'sum((-1)^k/(2*k-1), k, 1, inf) = factor(ev(sum((-1)^k/(2*k-1), k, 1, inf),simpsum=true))
>$ev(sum(1/n!, n, 0, inf),simpsum=true)
\end{eulerprompt}
\begin{eulercomment}
Di sini masih gagal, hasilnya tidak dihitung, harusnya hasilnya e.
\end{eulercomment}
\begin{eulerprompt}
>&assume(abs(x)<1); $'sum(a*x^k, k, 0, inf)=ev(sum(a*x^k, k, 0, inf),simpsum=true), &forget(abs(x)<1);
\end{eulerprompt}
\begin{eulercomment}
Deret geometri tak hingga, dengan asumsi rasional antara -1 dan 1.
\end{eulercomment}
\begin{eulerprompt}
>$'sum(x^k/k!,k,0,inf)=ev(sum(x^k/k!,k,0,inf),simpsum=true)
>$limit(sum(x^k/k!,k,0,n),n,inf)
>function d(n) &= sum(1/(k^2-k),k,2,n); $'d(n)=d(n)
>$d(10)=ev(d(10),simpsum=true)
>$d(100)=ev(d(100),simpsum=true)
\end{eulerprompt}
\begin{eulercomment}
\begin{eulercomment}
\eulerheading{Deret Taylor}
\begin{eulercomment}
Deret Taylor suatu fungsi f yang diferensiabel sampai tak hingga di
sekitar x=a adalah:

\end{eulercomment}
\begin{eulerformula}
\[
f(x) = \sum_{k=0}^\infty \frac{(x-a)^k f^{(k)}(a)}{k!}.
\]
\end{eulerformula}
\begin{eulerprompt}
>$'e^x =taylor(exp(x),x,0,10) // deret Taylor e^x di sekitar x=0, sampai suku ke-11
>$'log(x)=taylor(log(x),x,1,10)// deret log(x) di sekitar x=1
\end{eulerprompt}
\end{eulernotebook}







\documentclass[a4paper,10pt]{article}
\usepackage{eumat}


\begin{eulernotebook}
\eulersubheading{}
\begin{eulercomment}
Nama: Fanny Erina Dewi\\
NIM: 22305141005\\
Kelas: Matematika B 2022\\
\end{eulercomment}
\eulersubheading{}
\begin{eulercomment}
\begin{eulercomment}
\eulerheading{Visualisasi dan Perhitungan Geometri dengan EMT}
\begin{eulercomment}
Euler menyediakan beberapa fungsi untuk melakukan visualisasi dan
perhitungan geometri, baik secara numerik maupun analitik (seperti
biasanya tentunya, menggunakan Maxima). Fungsi-fungsi untuk
visualisasi dan perhitungan geometeri tersebut disimpan di dalam file
program "geometry.e", sehingga file tersebut harus dipanggil sebelum
menggunakan fungsi-fungsi atau perintah-perintah untuk geometri.
\end{eulercomment}
\begin{eulerprompt}
>load geometry
\end{eulerprompt}
\begin{euleroutput}
  Numerical and symbolic geometry.
\end{euleroutput}
\eulersubheading{Fungsi-fungsi Geometri}
\begin{eulercomment}
Fungsi-fungsi untuk Menggambar Objek Geometri:

\end{eulercomment}
\begin{eulerttcomment}
  defaultd:=textheight()*1.5: nilai asli untuk parameter d
  setPlotrange(x1,x2,y1,y2): menentukan rentang x dan y pada bidang koordinat
  setPlotRange(r): pusat bidang koordinat (0,0) dan batas-batas sumbu-x dan y adalah -r sd r
  plotPoint (P, "P"): menggambar titik P dan diberi label "P"
  plotSegment (A,B, "AB", d): menggambar ruas garis AB, diberi label "AB" sejauh d
  plotLine (g, "g", d): menggambar garis g diberi label "g" sejauh d
  plotCircle (c,"c",v,d): Menggambar lingkaran c dan diberi label "c"
  plotLabel (label, P, V, d): menuliskan label pada posisi P
\end{eulerttcomment}
\begin{eulercomment}

Fungsi-fungsi Geometri Analitik (numerik maupun simbolik):

\end{eulercomment}
\begin{eulerttcomment}
  turn(v, phi): memutar vektor v sejauh phi
  turnLeft(v):   memutar vektor v ke kiri
  turnRight(v):  memutar vektor v ke kanan
  normalize(v): normal vektor v
  crossProduct(v, w): hasil kali silang vektorv dan w.
  lineThrough(A, B): garis melalui A dan B, hasilnya [a,b,c] sdh. ax+by=c.
  lineWithDirection(A,v): garis melalui A searah vektor v
  getLineDirection(g): vektor arah (gradien) garis g
  getNormal(g): vektor normal (tegak lurus) garis g
  getPointOnLine(g):  titik pada garis g
  perpendicular(A, g):  garis melalui A tegak lurus garis g
  parallel (A, g):  garis melalui A sejajar garis g
  lineIntersection(g, h):  titik potong garis g dan h
  projectToLine(A, g):   proyeksi titik A pada garis g
  distance(A, B):  jarak titik A dan B
  distanceSquared(A, B):  kuadrat jarak A dan B
  quadrance(A, B): kuadrat jarak A dan B
  areaTriangle(A, B, C):  luas segitiga ABC
  computeAngle(A, B, C):   besar sudut <ABC
  angleBisector(A, B, C): garis bagi sudut <ABC
  circleWithCenter (A, r): lingkaran dengan pusat A dan jari-jari r
  getCircleCenter(c):  pusat lingkaran c
  getCircleRadius(c):  jari-jari lingkaran c
  circleThrough(A,B,C):  lingkaran melalui A, B, C
  middlePerpendicular(A, B): titik tengah AB
  lineCircleIntersections(g, c): titik potong garis g dan lingkran c
  circleCircleIntersections (c1, c2):  titik potong lingkaran c1 dan c2
  planeThrough(A, B, C):  bidang melalui titik A, B, C
\end{eulerttcomment}
\begin{eulercomment}

Fungsi-fungsi Khusus Untuk Geometri Simbolik:

\end{eulercomment}
\begin{eulerttcomment}
  getLineEquation (g,x,y): persamaan garis g dinyatakan dalam x dan y
  getHesseForm (g,x,y,A): bentuk Hesse garis g dinyatakan dalam x dan y dengan titik A pada
  sisi positif (kanan/atas) garis
  quad(A,B): kuadrat jarak AB
  spread(a,b,c): Spread segitiga dengan panjang sisi-sisi a,b,c, yakni sin(alpha)^2 dengan
  alpha sudut yang menghadap sisi a.
  crosslaw(a,b,c,sa): persamaan 3 quads dan 1 spread pada segitiga dengan panjang sisi a, b, c.
  triplespread(sa,sb,sc): persamaan 3 spread sa,sb,sc yang memebntuk suatu segitiga
  doublespread(sa): Spread sudut rangkap Spread 2*phi, dengan sa=sin(phi)^2 spread a.
\end{eulerttcomment}
\begin{eulercomment}

\end{eulercomment}
\eulersubheading{Contoh 1: Luas, Lingkaran Luar, Lingkaran Dalam Segitiga}
\begin{eulercomment}
Untuk menggambar objek-objek geometri, langkah pertama adalah menentukan rentang sumbu-sumbu
koordinat. Semua objek geometri akan digambar pada satu bidang koordinat, sampai didefinisikan
bidang koordinat yang baru.
\end{eulercomment}
\begin{eulerprompt}
>setPlotRange(-0.5,2.5,-0.5,2.5); // mendefinisikan bidang koordinat baru 
\end{eulerprompt}
\begin{eulercomment}
Sekarang atur tiga poin dan plot.
\end{eulercomment}
\begin{eulerprompt}
>A=[1,0]; plotPoint(A,"A"): // definisi dan gambar tiga titik
\end{eulerprompt}
\eulerimg{27}{images/Pekan 11-12_Fanny Erina Dewi_22305141005_EMT00-Geometry_Aplikom-001.png}
\begin{eulerprompt}
>B=[0,1]; plotPoint(B,"B"):
\end{eulerprompt}
\eulerimg{27}{images/Pekan 11-12_Fanny Erina Dewi_22305141005_EMT00-Geometry_Aplikom-002.png}
\begin{eulerprompt}
>C=[2,2]; plotPoint(C,"C"):
\end{eulerprompt}
\eulerimg{27}{images/Pekan 11-12_Fanny Erina Dewi_22305141005_EMT00-Geometry_Aplikom-003.png}
\begin{eulercomment}
Lalu tiga garis.
\end{eulercomment}
\begin{eulerprompt}
>plotSegment(A,B,"c"): // c=AB
\end{eulerprompt}
\eulerimg{27}{images/Pekan 11-12_Fanny Erina Dewi_22305141005_EMT00-Geometry_Aplikom-004.png}
\begin{eulerprompt}
>plotSegment(B,C,"a"): // a=BC
\end{eulerprompt}
\eulerimg{27}{images/Pekan 11-12_Fanny Erina Dewi_22305141005_EMT00-Geometry_Aplikom-005.png}
\begin{eulerprompt}
>plotSegment(A,C,"b"): // b=AC
\end{eulerprompt}
\eulerimg{27}{images/Pekan 11-12_Fanny Erina Dewi_22305141005_EMT00-Geometry_Aplikom-006.png}
\begin{eulercomment}
Fungsi geometri meliputi fungsi untuk membuat garis dan lingkaran.
Format untuk garis adalah [a, b, c], yang merepresentasikan garis
dengan persamaan ax + by = c.
\end{eulercomment}
\begin{eulerprompt}
>lineThrough(B,C) // garis yang melalui B dan C
\end{eulerprompt}
\begin{euleroutput}
  [-1,  2,  2]
\end{euleroutput}
\begin{eulercomment}
Hitung garis tegak lurus melalui A pada BC.
\end{eulercomment}
\begin{eulerprompt}
>h=perpendicular(A,lineThrough(B,C)): // garis h tegak lurus BC melalui A
\end{eulerprompt}
\eulerimg{27}{images/Pekan 11-12_Fanny Erina Dewi_22305141005_EMT00-Geometry_Aplikom-007.png}
\begin{eulercomment}
Dan perpotongannya dengan BC.
\end{eulercomment}
\begin{eulerprompt}
>D=lineIntersection(h,lineThrough(B,C)): // D adalah titik potong h dan BC
\end{eulerprompt}
\eulerimg{27}{images/Pekan 11-12_Fanny Erina Dewi_22305141005_EMT00-Geometry_Aplikom-008.png}
\begin{eulercomment}
Plot that.
\end{eulercomment}
\begin{eulerprompt}
>plotPoint(D,value=1); // koordinat D ditampilkan
>aspect(1); plotSegment(A,D): // tampilkan semua gambar hasil plot...()
\end{eulerprompt}
\eulerimg{27}{images/Pekan 11-12_Fanny Erina Dewi_22305141005_EMT00-Geometry_Aplikom-009.png}
\begin{eulercomment}
Hitung luas ABC:

\end{eulercomment}
\begin{eulerformula}
\[
L_{\triangle ABC}= \frac{1}{2}AD.BC.
\]
\end{eulerformula}
\begin{eulerprompt}
>norm(A-D)*norm(B-C)/2 // AD=norm(A-D), BC=norm(B-C)
\end{eulerprompt}
\begin{euleroutput}
  1.5
\end{euleroutput}
\begin{eulercomment}
Cara lain menghitung rumus determinan.\\
BC.
\end{eulercomment}
\begin{eulerprompt}
>areaTriangle(A,B,C) // hitung luas segitiga langusng dengan fungsi
\end{eulerprompt}
\begin{euleroutput}
  1.5
\end{euleroutput}
\begin{eulercomment}
Cara lain menghitung luas segitigas ABC:
\end{eulercomment}
\begin{eulerprompt}
>distance(A,D)*distance(B,C)/2
\end{eulerprompt}
\begin{euleroutput}
  1.5
\end{euleroutput}
\begin{eulercomment}
Sudut di C.
\end{eulercomment}
\begin{eulerprompt}
>degprint(computeAngle(B,C,A))
\end{eulerprompt}
\begin{euleroutput}
  36°52'11.63''
\end{euleroutput}
\begin{eulercomment}
Sekarang, lingkarilah segitiga tersebut.
\end{eulercomment}
\begin{eulerprompt}
>c=circleThrough(A,B,C): // lingkaran luar segitiga ABC
\end{eulerprompt}
\eulerimg{27}{images/Pekan 11-12_Fanny Erina Dewi_22305141005_EMT00-Geometry_Aplikom-010.png}
\begin{eulerprompt}
>R=getCircleRadius(c): // jari2 lingkaran luar 
\end{eulerprompt}
\eulerimg{27}{images/Pekan 11-12_Fanny Erina Dewi_22305141005_EMT00-Geometry_Aplikom-011.png}
\begin{eulerprompt}
>O=getCircleCenter(c): // titik pusat lingkaran c 
\end{eulerprompt}
\eulerimg{27}{images/Pekan 11-12_Fanny Erina Dewi_22305141005_EMT00-Geometry_Aplikom-012.png}
\begin{eulerprompt}
>plotPoint(O,"O"): // gambar titik "O"
\end{eulerprompt}
\eulerimg{27}{images/Pekan 11-12_Fanny Erina Dewi_22305141005_EMT00-Geometry_Aplikom-013.png}
\begin{eulerprompt}
>plotCircle(c,"Lingkaran luar segitiga ABC"):
\end{eulerprompt}
\eulerimg{27}{images/Pekan 11-12_Fanny Erina Dewi_22305141005_EMT00-Geometry_Aplikom-014.png}
\begin{eulercomment}
Tampilkan koordinat titik pusat dan jari-jari lingkaran luar.
\end{eulercomment}
\begin{eulerprompt}
>O, R
\end{eulerprompt}
\begin{euleroutput}
  [1.16667,  1.16667]
  1.17851130198
\end{euleroutput}
\begin{eulercomment}
Sekarang akan digambar lingkaran dalam segitiga ABC. Titik pusat lingkaran dalam adalah
titik potong garis-garis bagi sudut.
\end{eulercomment}
\begin{eulerprompt}
>l=angleBisector(A,C,B): // garis bagi <ACB
\end{eulerprompt}
\eulerimg{27}{images/Pekan 11-12_Fanny Erina Dewi_22305141005_EMT00-Geometry_Aplikom-015.png}
\begin{eulerprompt}
>g=angleBisector(C,A,B): // garis bagi <CAB
\end{eulerprompt}
\eulerimg{27}{images/Pekan 11-12_Fanny Erina Dewi_22305141005_EMT00-Geometry_Aplikom-016.png}
\begin{eulerprompt}
>P=lineIntersection(l,g) // titik potong kedua garis bagi sudut
\end{eulerprompt}
\begin{euleroutput}
  [0.86038,  0.86038]
\end{euleroutput}
\begin{eulercomment}
Tambahkan semuanya ke dalam plot.
\end{eulercomment}
\begin{eulerprompt}
>color(5); plotLine(l); plotLine(g); color(1): // gambar kedua garis bagi sudut
\end{eulerprompt}
\eulerimg{27}{images/Pekan 11-12_Fanny Erina Dewi_22305141005_EMT00-Geometry_Aplikom-017.png}
\begin{eulerprompt}
>plotPoint(P,"P"): // gambar titik potongnya
\end{eulerprompt}
\eulerimg{27}{images/Pekan 11-12_Fanny Erina Dewi_22305141005_EMT00-Geometry_Aplikom-018.png}
\begin{eulerprompt}
>r=norm(P-projectToLine(P,lineThrough(A,B))) // jari-jari lingkaran dalam
\end{eulerprompt}
\begin{euleroutput}
  0.509653732104
\end{euleroutput}
\begin{eulerprompt}
>plotCircle(circleWithCenter(P,r),"Lingkaran dalam segitiga ABC"): // gambar lingkaran dalam
\end{eulerprompt}
\eulerimg{27}{images/Pekan 11-12_Fanny Erina Dewi_22305141005_EMT00-Geometry_Aplikom-019.png}
\begin{eulerprompt}
>reset
\end{eulerprompt}
\begin{euleroutput}
  0
\end{euleroutput}
\eulersubheading{Latihan}
\begin{eulercomment}
1. Tentukan ketiga titik singgung lingkaran dalam dengan sisi-sisi
segitiga ABC.\\
2. Gambar segitiga dengan titik-titik sudut ketiga titik singgung
tersebut. Merupakan segitiga apakah itu?\\
3. Hitung luas segitiga tersebut.\\
4. Tunjukkan bahwa garis bagi sudut yang ke tiga juga melalui titik
pusat lingkaran dalam.\\
5. Gambar jari-jari lingkaran dalam.\\
6. Hitung luas lingkaran luar dan luas lingkaran dalam segitiga ABC.
Adakah hubungan antara luas kedua lingkaran tersebut dengan luas
segitiga ABC?
\end{eulercomment}
\begin{eulerprompt}
>setPlotRange(-0.5,2.5,-0.5,2.5);
>plotCircle(circleWithCenter(P,r),"Lingkaran dalam segitiga ABC");
>K = lineCircleIntersections(lineThrough(A,B),circleWithCenter(P,r))
\end{eulerprompt}
\begin{euleroutput}
  [0.5,  0.5]
\end{euleroutput}
\begin{eulerprompt}
>L = lineCircleIntersections(lineThrough(C,B),circleWithCenter(P,r))
\end{eulerprompt}
\begin{euleroutput}
  [0.632456,  1.31623]
\end{euleroutput}
\begin{eulerprompt}
>M = lineCircleIntersections(lineThrough(C,A),circleWithCenter(P,r))
\end{eulerprompt}
\begin{euleroutput}
  [1.31623,  0.632456]
\end{euleroutput}
\begin{eulerprompt}
>plotPoint(K,"K"); plotPoint(M,"M"); plotPoint(L,"L");
>plotSegment(K,L,"m");
>plotSegment(L,M,"n");
>plotSegment(M,K,"l"):
\end{eulerprompt}
\eulerimg{27}{images/Pekan 11-12_Fanny Erina Dewi_22305141005_EMT00-Geometry_Aplikom-020.png}
\begin{eulerprompt}
>distance(P,lineIntersection(lineThrough(A,B),l))
\end{eulerprompt}
\begin{euleroutput}
  0.509653732104
\end{euleroutput}
\begin{eulerprompt}
>getCircleRadius(circleWithCenter(P,r))
\end{eulerprompt}
\begin{euleroutput}
  0.509653732104
\end{euleroutput}
\begin{eulerprompt}
>reset
\end{eulerprompt}
\begin{euleroutput}
  0
\end{euleroutput}
\begin{eulercomment}
1. Tentukan ketiga titik singgung lingkaran dalam dengan sisi-sisi
segitiga ABC.\\
Titik singgung garis BC dengan lingkaran dalam.
\end{eulercomment}
\begin{eulerprompt}
>s=lineThrough(B,C)
\end{eulerprompt}
\begin{euleroutput}
  [-1,  2,  2]
\end{euleroutput}
\begin{eulerprompt}
>m=circleWithCenter(P,r)
\end{eulerprompt}
\begin{euleroutput}
  [0.86038,  0.86038,  0.509654]
\end{euleroutput}
\begin{eulerprompt}
>S=lineCircleIntersections(s,m)
\end{eulerprompt}
\begin{euleroutput}
  [0.632456,  1.31623]
\end{euleroutput}
\begin{eulercomment}
Titik singgung garis AC dengan lingkaran dalam.
\end{eulercomment}
\begin{eulerprompt}
>p=lineThrough(A,C)
\end{eulerprompt}
\begin{euleroutput}
  [-2,  1,  -2]
\end{euleroutput}
\begin{eulerprompt}
>Q=lineCircleIntersections(p,m)
\end{eulerprompt}
\begin{euleroutput}
  [1.31623,  0.632456]
\end{euleroutput}
\begin{eulercomment}
Titik singgung garis AB dengan lingkaran dalam.
\end{eulercomment}
\begin{eulerprompt}
>q=lineThrough(A,B)
\end{eulerprompt}
\begin{euleroutput}
  [-1,  -1,  -1]
\end{euleroutput}
\begin{eulerprompt}
>L=lineCircleIntersections(q,m)
\end{eulerprompt}
\begin{euleroutput}
  [0.5,  0.5]
\end{euleroutput}
\begin{eulercomment}
Jadi titik singgung lingkaran dalam dengan sisi-sisi segitiga adalah
(0.632456, 1.31623), (1.31632, 0.632456), (0.5, 0.5).

2. Gambar segitiga dengan titik-titik sudut ketiga titik singgung
tersebut.
\end{eulercomment}
\begin{eulerprompt}
>setPlotRange(-0.5,2.5,-0.5,2.5);
>plotSegment(S,Q,"a"):
\end{eulerprompt}
\eulerimg{27}{images/Pekan 11-12_Fanny Erina Dewi_22305141005_EMT00-Geometry_Aplikom-021.png}
\begin{eulerprompt}
>plotSegment(S,L,"b"):
\end{eulerprompt}
\eulerimg{27}{images/Pekan 11-12_Fanny Erina Dewi_22305141005_EMT00-Geometry_Aplikom-022.png}
\begin{eulerprompt}
>plotSegment(L,Q,"c"):
\end{eulerprompt}
\eulerimg{27}{images/Pekan 11-12_Fanny Erina Dewi_22305141005_EMT00-Geometry_Aplikom-023.png}
\begin{eulercomment}
3. Tunjukkan bahwa garis bagi sudut yang ke tiga juga melalui titik
pusat lingkaran dalam.
\end{eulercomment}
\begin{eulerprompt}
>P, r
\end{eulerprompt}
\begin{euleroutput}
  [0.86038,  0.86038]
  0.509653732104
\end{euleroutput}
\begin{eulerprompt}
>k=angleBisector(A,B,C)
\end{eulerprompt}
\begin{euleroutput}
  [-0.264911,  -1.63246,  -1.63246]
\end{euleroutput}
\begin{eulerprompt}
>color(2); plotLine(k):
\end{eulerprompt}
\eulerimg{27}{images/Pekan 11-12_Fanny Erina Dewi_22305141005_EMT00-Geometry_Aplikom-024.png}
\begin{eulercomment}
4. Gambar jari-jari lingkaran dalam.
\end{eulercomment}
\begin{eulerprompt}
>plotSegment(P,L,"r"):
\end{eulerprompt}
\eulerimg{27}{images/Pekan 11-12_Fanny Erina Dewi_22305141005_EMT00-Geometry_Aplikom-025.png}
\eulerheading{Contoh 2: Geometri Smbolik}
\begin{eulercomment}
Kita dapat menghitung geometri eksak dan simbolik menggunakan Maxima.

File geometry.e menyediakan fungsi yang sama (dan lebih banyak lagi)
di Maxima. Namun, kita dapat menggunakan komputasi simbolik sekarang.
\end{eulercomment}
\begin{eulerprompt}
>A &= [1,0]; B &= [0,1]; C &= [2,2]; // menentukan tiga titik A, B, C
\end{eulerprompt}
\begin{eulercomment}
Fungsi untuk garis dan lingkaran bekerja seperti fungsi Euler, tetapi
menyediakan komputasi simbolik.
\end{eulercomment}
\begin{eulerprompt}
>c &= lineThrough(B,C) // c=BC
\end{eulerprompt}
\begin{euleroutput}
  
                               [- 1, 2, 2]
  
\end{euleroutput}
\begin{eulercomment}
Kita bisa mendapatkan persamaan untuk sebuah garis dengan mudah.
\end{eulercomment}
\begin{eulerprompt}
>$getLineEquation(c,x,y), $solve(%,y) | expand // persamaan garis c
\end{eulerprompt}
\begin{eulerformula}
\[
2\,y-x=2
\]
\end{eulerformula}
\begin{eulerformula}
\[
\left[ y=\frac{x}{2}+1 \right] 
\]
\end{eulerformula}
\begin{eulerprompt}
>$getLineEquation(lineThrough([x1,y1],[x2,y2]),x,y), $solve(%,y) // persamaan garis melalui(x1, y1) dan (x2, y2)
\end{eulerprompt}
\begin{eulerformula}
\[
x\,\left({\it y_1}-{\it y_2}\right)+\left({\it x_2}-{\it x_1}
 \right)\,y={\it x_1}\,\left({\it y_1}-{\it y_2}\right)+\left(
 {\it x_2}-{\it x_1}\right)\,{\it y_1}
\]
\end{eulerformula}
\begin{eulerformula}
\[
\left[ y=\frac{-\left({\it x_1}-x\right)\,{\it y_2}-\left(x-
 {\it x_2}\right)\,{\it y_1}}{{\it x_2}-{\it x_1}} \right] 
\]
\end{eulerformula}
\begin{eulerprompt}
>$getLineEquation(lineThrough(A,[x1,y1]),x,y) // persamaan garis melalui A dan (x1, y1)
\end{eulerprompt}
\begin{eulerformula}
\[
\left({\it x_1}-1\right)\,y-x\,{\it y_1}=-{\it y_1}
\]
\end{eulerformula}
\begin{eulerprompt}
>h &= perpendicular(A,lineThrough(B,C)) // h melalui A tegak lurus BC
\end{eulerprompt}
\begin{euleroutput}
  
                                [2, 1, 2]
  
\end{euleroutput}
\begin{eulerprompt}
>Q &= lineIntersection(c,h) // Q titik potong garis c=BC dan h
\end{eulerprompt}
\begin{euleroutput}
  
                                   2  6
                                  [-, -]
                                   5  5
  
\end{euleroutput}
\begin{eulerprompt}
>$projectToLine(A,lineThrough(B,C)) // proyeksi A pada BC
\end{eulerprompt}
\begin{eulerformula}
\[
\left[ \frac{2}{5} , \frac{6}{5} \right] 
\]
\end{eulerformula}
\begin{eulerprompt}
>$distance(A,Q) // jarak AQ
\end{eulerprompt}
\begin{eulerformula}
\[
\frac{3}{\sqrt{5}}
\]
\end{eulerformula}
\begin{eulerprompt}
>cc &= circleThrough(A,B,C); $cc // (titik pusat dan jari-jari) lingkaran melalui A, B, C
\end{eulerprompt}
\begin{eulerformula}
\[
\left[ \frac{7}{6} , \frac{7}{6} , \frac{5}{3\,\sqrt{2}} \right] 
\]
\end{eulerformula}
\begin{eulerprompt}
>r&=getCircleRadius(cc); $r , $float(r) // tampilkan nilai jari-jari
\end{eulerprompt}
\begin{eulerformula}
\[
\frac{5}{3\,\sqrt{2}}
\]
\end{eulerformula}
\begin{eulerformula}
\[
1.178511301977579
\]
\end{eulerformula}
\begin{eulerprompt}
>$computeAngle(A,C,B) // nilai <ACB
\end{eulerprompt}
\begin{eulerformula}
\[
\arccos \left(\frac{4}{5}\right)
\]
\end{eulerformula}
\begin{eulerprompt}
>$solve(getLineEquation(angleBisector(A,C,B),x,y),y)[1] // persamaan garis bagi <ACB
\end{eulerprompt}
\begin{eulerformula}
\[
y=x
\]
\end{eulerformula}
\begin{eulerprompt}
>P &= lineIntersection(angleBisector(A,C,B),angleBisector(C,B,A)); $P // titik potong 2 garis bagi sudut
\end{eulerprompt}
\begin{eulerformula}
\[
\left[ \frac{\sqrt{2}\,\sqrt{5}+2}{6} , \frac{\sqrt{2}\,\sqrt{5}+2
 }{6} \right] 
\]
\end{eulerformula}
\begin{eulerprompt}
>P() // hasilnya sama dengan perhitungan sebelumnya
\end{eulerprompt}
\begin{euleroutput}
  [0.86038,  0.86038]
\end{euleroutput}
\eulersubheading{Garis dan Lingkaran yang Berpotongan}
\begin{eulercomment}
Tentu saja, kita juga bisa memotong garis dengan lingkaran, dan
lingkaran dengan lingkaran.
\end{eulercomment}
\begin{eulerprompt}
>A &:= [1,0]; c=circleWithCenter(A,4);
>B &:= [1,2]; C &:= [2,1]; l=lineThrough(B,C);
>setPlotRange(5); plotCircle(c); plotLine(l):
\end{eulerprompt}
\eulerimg{27}{images/Pekan 11-12_Fanny Erina Dewi_22305141005_EMT00-Geometry_Aplikom-039.png}
\begin{eulercomment}
Perpotongan garis dengan lingkaran menghasilkan dua titik dan jumlah
titik perpotongan.
\end{eulercomment}
\begin{eulerprompt}
>\{P1,P2,f\}=lineCircleIntersections(l,c);
>P1, P2, f
\end{eulerprompt}
\begin{euleroutput}
  [4.64575,  -1.64575]
  [-0.645751,  3.64575]
  2
\end{euleroutput}
\begin{eulerprompt}
>plotPoint(P1); plotPoint(P2):
\end{eulerprompt}
\eulerimg{27}{images/Pekan 11-12_Fanny Erina Dewi_22305141005_EMT00-Geometry_Aplikom-040.png}
\begin{eulercomment}
Hal yang sama juga terjadi pada Maxima.
\end{eulercomment}
\begin{eulerprompt}
>c &= circleWithCenter(A,4) // lingkaran dengan pusat A jari-jari 4
\end{eulerprompt}
\begin{euleroutput}
  
                                [1, 0, 4]
  
\end{euleroutput}
\begin{eulerprompt}
>l &= lineThrough(B,C) // garis l melalui B dan C
\end{eulerprompt}
\begin{euleroutput}
  
                                [1, 1, 3]
  
\end{euleroutput}
\begin{eulerprompt}
>$lineCircleIntersections(l,c) | radcan, // titik potong lingkaran c dan garis l
\end{eulerprompt}
\begin{eulerformula}
\[
\left[ \left[ \sqrt{7}+2 , 1-\sqrt{7} \right]  , \left[ 2-\sqrt{7}
  , \sqrt{7}+1 \right]  \right] 
\]
\end{eulerformula}
\begin{eulercomment}
Akan ditunjukkan bahwa sudut-sudut yang menghadap busur yang sama
adalah sama besar.
\end{eulercomment}
\begin{eulerprompt}
>C=A+normalize([-2,-3])*4; plotPoint(C); plotSegment(P1,C); plotSegment(P2,C):
\end{eulerprompt}
\eulerimg{27}{images/Pekan 11-12_Fanny Erina Dewi_22305141005_EMT00-Geometry_Aplikom-042.png}
\begin{eulerprompt}
>degprint(computeAngle(P1,C,P2))
\end{eulerprompt}
\begin{euleroutput}
  69°17'42.68''
\end{euleroutput}
\begin{eulerprompt}
>C=A+normalize([-4,-3])*4; plotPoint(C); plotSegment(P1,C); plotSegment(P2,C):
\end{eulerprompt}
\eulerimg{27}{images/Pekan 11-12_Fanny Erina Dewi_22305141005_EMT00-Geometry_Aplikom-043.png}
\begin{eulerprompt}
>degprint(computeAngle(P1,C,P2))
\end{eulerprompt}
\begin{euleroutput}
  69°17'42.68''
\end{euleroutput}
\begin{eulerprompt}
>insimg;
\end{eulerprompt}
\eulerimg{27}{images/Pekan 11-12_Fanny Erina Dewi_22305141005_EMT00-Geometry_Aplikom-044.png}
\eulersubheading{Garis Sumbu}
\begin{eulercomment}
Berikut adalah langkah-langkah menggambar garis sumbu ruas garis AB:

1. Gambar lingkaran dengan pusat A melalui B.\\
2. Gambar lingkaran dengan pusat B melalui A.\\
3. Tarik garis melallui kedua titik potong kedua lingkaran tersebut. Garis ini merupakan
garis sumbu (melalui titik tengah dan tegak lurus) AB.
\end{eulercomment}
\begin{eulerprompt}
>A=[2,2]; B=[-1,-2];
>c1=circleWithCenter(A,distance(A,B));
>c2=circleWithCenter(B,distance(A,B));
>\{P1,P2,f\}=circleCircleIntersections(c1,c2);
>l=lineThrough(P1,P2);
>setPlotRange(5); plotCircle(c1); plotCircle(c2);
>plotPoint(A); plotPoint(B); plotSegment(A,B); plotLine(l):
\end{eulerprompt}
\eulerimg{27}{images/Pekan 11-12_Fanny Erina Dewi_22305141005_EMT00-Geometry_Aplikom-045.png}
\begin{eulercomment}
Selanjutnya, kita melakukan hal yang sama di Maxima dengan koordinat
umum.
\end{eulercomment}
\begin{eulerprompt}
>A &= [a1,a2]; B &= [b1,b2];
>c1 &= circleWithCenter(A,distance(A,B));
>c2 &= circleWithCenter(B,distance(A,B));
>P &= circleCircleIntersections(c1,c2); P1 &= P[1]; P2 &= P[2];
\end{eulerprompt}
\begin{eulercomment}
Persamaan untuk persimpangan cukup rumit. Tetapi kita dapat
menyederhanakannya, jika kita menyelesaikan untuk y.
\end{eulercomment}
\begin{eulerprompt}
>g &= getLineEquation(lineThrough(P1,P2),x,y);
>$solve(g,y)
\end{eulerprompt}
\begin{eulerformula}
\[
\left[ y=\frac{-\left(2\,{\it b_1}-2\,{\it a_1}\right)\,x+{\it b_2}
 ^2+{\it b_1}^2-{\it a_2}^2-{\it a_1}^2}{2\,{\it b_2}-2\,{\it a_2}}
  \right] 
\]
\end{eulerformula}
\begin{eulercomment}
Ini memang sama dengan tegak lurus tengah, yang dihitung dengan cara
yang sama sekali berbeda.
\end{eulercomment}
\begin{eulerprompt}
>$solve(getLineEquation(middlePerpendicular(A,B),x,y),y)
\end{eulerprompt}
\begin{eulerformula}
\[
\left[ y=\frac{-\left(2\,{\it b_1}-2\,{\it a_1}\right)\,x+{\it b_2}
 ^2+{\it b_1}^2-{\it a_2}^2-{\it a_1}^2}{2\,{\it b_2}-2\,{\it a_2}}
  \right] 
\]
\end{eulerformula}
\begin{eulerprompt}
>h &=getLineEquation(lineThrough(A,B),x,y);
>$solve(h,y)
\end{eulerprompt}
\begin{eulerformula}
\[
\left[ y=\frac{\left({\it b_2}-{\it a_2}\right)\,x-{\it a_1}\,
 {\it b_2}+{\it a_2}\,{\it b_1}}{{\it b_1}-{\it a_1}} \right] 
\]
\end{eulerformula}
\begin{eulercomment}
Perhatikan hasil kali gradien garis g dan h adalah:

\end{eulercomment}
\begin{eulerformula}
\[
\frac{-(b_1-a_1)}{(b_2-a_2)}\times \frac{(b_2-a_2)}{(b_1-a_1)} = -1.
\]
\end{eulerformula}
\begin{eulercomment}
Artinya kedua garis tegak lurus.
\end{eulercomment}
\eulerheading{Contoh 3: Rumus Heron}
\begin{eulercomment}
Rumus Heron menyatakan bahwa luas segitiga dengan panjang sisi-sisi a, b dan c adalah:

\end{eulercomment}
\begin{eulerformula}
\[
L = \sqrt{s(s-a)(s-b)(s-c)}\quad \text{ dengan } s=(a+b+c)/2,
\]
\end{eulerformula}
\begin{eulercomment}
atau bisa ditulis dalam bentuk lain:

\end{eulercomment}
\begin{eulerformula}
\[
L = \frac{1}{4}\sqrt{(a+b+c)(b+c-a)(a+c-b)(a+b-c)}
\]
\end{eulerformula}
\begin{eulercomment}
Untuk membuktikan hal ini kita misalkan C(0,0), B(a,0) dan A(x,y), b=AC, c=AB. Luas segitiga
ABC adalah

\end{eulercomment}
\begin{eulerformula}
\[
L_{\triangle ABC}=\frac{1}{2}a\times y.
\]
\end{eulerformula}
\begin{eulercomment}
Nilai y didapat dengan menyelesaikan sistem persamaan:

\end{eulercomment}
\begin{eulerformula}
\[
x^2+y^2=b^2, \quad (x-a)^2+y^2=c^2.
\]
\end{eulerformula}
\begin{eulerprompt}
>load geometry
\end{eulerprompt}
\begin{euleroutput}
  Numerical and symbolic geometry.
\end{euleroutput}
\begin{eulerprompt}
>setPlotRange(-1,10,-1,8); plotPoint([0,0], "C(0,0)"); plotPoint([5.5,0], "B(a,0)");  ...
> plotPoint([7.5,6], "A(x,y)"):
\end{eulerprompt}
\eulerimg{27}{images/Pekan 11-12_Fanny Erina Dewi_22305141005_EMT00-Geometry_Aplikom-049.png}
\begin{eulerprompt}
>plotSegment([0,0],[5.5,0], "a",25); plotSegment([5.5,0],[7.5,6],"c",15);  ...
>plotSegment([0,0],[7.5,6],"b",25): 
\end{eulerprompt}
\eulerimg{27}{images/Pekan 11-12_Fanny Erina Dewi_22305141005_EMT00-Geometry_Aplikom-050.png}
\begin{eulerprompt}
>plotSegment([7.5,6],[7.5,0],"t=y",25):
\end{eulerprompt}
\eulerimg{27}{images/Pekan 11-12_Fanny Erina Dewi_22305141005_EMT00-Geometry_Aplikom-051.png}
\begin{eulerprompt}
>&assume(a>0); sol &= solve([x^2+y^2=b^2,(x-a)^2+y^2=c^2],[x,y])
\end{eulerprompt}
\begin{euleroutput}
  
                                    []
  
\end{euleroutput}
\begin{eulercomment}
Extrak solusi y.
\end{eulercomment}
\begin{eulerprompt}
>ysol &= y with sol[2][2]; $'y=sqrt(factor(ysol^2))
\end{eulerprompt}
\begin{euleroutput}
  Maxima said:
  part: invalid index of list or matrix.
   -- an error. To debug this try: debugmode(true);
  
  Error in:
  ysol &= y with sol[2][2]; $'y=sqrt(factor(ysol^2)) ...
                          ^
\end{euleroutput}
\begin{eulercomment}
Kita mendapatkan formula Heron.
\end{eulercomment}
\begin{eulerprompt}
>function H(a,b,c) &= sqrt(factor((ysol*a/2)^2)); $'H(a,b,c)=H(a,b,c)
\end{eulerprompt}
\begin{eulerformula}
\[
H\left(a , b , \left[ 1 , 0 , 4 \right] \right)=\frac{a\,\left| 
 {\it ysol}\right| }{2}
\]
\end{eulerformula}
\begin{eulerprompt}
>$'Luas=H(2,5,6) // luas segitiga dengan panjang sisi-sisi 2, 5, 6
\end{eulerprompt}
\begin{eulerformula}
\[
{\it Luas}=\left| {\it ysol}\right| 
\]
\end{eulerformula}
\begin{eulercomment}
Tentu saja, setiap segitiga persegi panjang adalah kasus yang
terkenal.
\end{eulercomment}
\begin{eulerprompt}
>H(3,4,5) //luas segitiga siku-siku dengan panjang sisi 3, 4, 5
\end{eulerprompt}
\begin{euleroutput}
  Variable or function ysol not found.
  Try "trace errors" to inspect local variables after errors.
  H:
      useglobal; return a*abs(ysol)/2 
  Error in:
  H(3,4,5) //luas segitiga siku-siku dengan panjang sisi 3, 4, 5 ...
          ^
\end{euleroutput}
\begin{eulercomment}
Dan jelas juga, bahwa ini adalah segitiga dengan luas maksimal dan
kedua sisinya 3 dan 4.
\end{eulercomment}
\begin{eulerprompt}
>aspect (1.5); plot2d(&H(3,4,x),1,7): // Kurva luas segitiga sengan panjang sisi 3, 4, x (1<= x <=7)
\end{eulerprompt}
\begin{euleroutput}
  Variable or function ysol not found.
  Error in expression: 3*abs(ysol)/2
  %ploteval:
      y0=f$(x[1],args());
  adaptiveevalone:
      s=%ploteval(g$,t;args());
  Try "trace errors" to inspect local variables after errors.
  plot2d:
      dw/n,dw/n^2,dw/n,auto;args());
\end{euleroutput}
\begin{eulercomment}
Kasus umum juga berfungsi.
\end{eulercomment}
\begin{eulerprompt}
>$solve(diff(H(a,b,c)^2,c)=0,c)
\end{eulerprompt}
\begin{euleroutput}
  Maxima said:
  diff: second argument must be a variable; found [1,0,4]
   -- an error. To debug this try: debugmode(true);
  
  Error in:
  $solve(diff(H(a,b,c)^2,c)=0,c) ...
                                ^
\end{euleroutput}
\begin{eulercomment}
Sekarang mari kita cari himpunan semua titik di mana b + c = d untuk
beberapa konstanta d. Diketahui bahwa ini adalah elips.
\end{eulercomment}
\begin{eulerprompt}
>s1 &= subst(d-c,b,sol[2]); $s1
\end{eulerprompt}
\begin{euleroutput}
  Maxima said:
  part: invalid index of list or matrix.
   -- an error. To debug this try: debugmode(true);
  
  Error in:
  s1 &= subst(d-c,b,sol[2]); $s1 ...
                           ^
\end{euleroutput}
\begin{eulercomment}
Dan mendapat persamaan seperti ini.
\end{eulercomment}
\begin{eulerprompt}
>function fx(a,c,d) &= rhs(s1[1]); $fx(a,c,d), function fy(a,c,d) &= rhs(s1[2]); $fy(a,c,d)
\end{eulerprompt}
\begin{eulerformula}
\[
0
\]
\end{eulerformula}
\begin{eulerformula}
\[
0
\]
\end{eulerformula}
\begin{eulercomment}
Sekarang kita bisa menggambar setnya. Sisi b bervariasi dari 1 hingga
4. Diketahui bahwa kita mendapatkan elips.
\end{eulercomment}
\begin{eulerprompt}
>aspect(1); plot2d(&fx(3,x,5),&fy(3,x,5),xmin=1,xmax=4,square=1):
\end{eulerprompt}
\eulerimg{27}{images/Pekan 11-12_Fanny Erina Dewi_22305141005_EMT00-Geometry_Aplikom-056.png}
\begin{eulercomment}
Kita dapat memeriksa persamaan umum elips ini, yaitu:

\end{eulercomment}
\begin{eulerformula}
\[
\frac{(x-x_m)^2}{u^2}+\frac{(y-y_m)}{v^2}=1,
\]
\end{eulerformula}
\begin{eulercomment}
di mana (xm,ym) adalah pusat, dan u dan v adalah setengah sumbu.
\end{eulercomment}
\begin{eulerprompt}
>$ratsimp((fx(a,c,d)-a/2)^2/u^2+fy(a,c,d)^2/v^2 with [u=d/2,v=sqrt(d^2-a^2)/2])
\end{eulerprompt}
\begin{eulerformula}
\[
\frac{a^2}{d^2}
\]
\end{eulerformula}
\begin{eulercomment}
Kita melihat bahwa tinggi dan luas segitiga adalah maksimal untuk x =
0. Jadi luas segitiga dengan a + b + c = d adalah maksimal, jika sama
sisi. Kami ingin mendapatkan ini secara analitis.
\end{eulercomment}
\begin{eulerprompt}
>eqns &= [diff(H(a,b,d-(a+b))^2,a)=0,diff(H(a,b,d-(a+b))^2,b)=0]; $eqns
\end{eulerprompt}
\begin{eulerformula}
\[
\left[ \frac{a\,{\it ysol}^2}{2}=0 , 0=0 \right] 
\]
\end{eulerformula}
\begin{eulercomment}
Kami mendapatkan beberapa minima, yang termasuk dalam segitiga dengan
satu sisi 0, dan solusi a=b=c=d/3.
\end{eulercomment}
\begin{eulerprompt}
>$solve(eqns,[a,b])
\end{eulerprompt}
\begin{eulerformula}
\[
\left[ \left[ a=0 , b={\it \%r_1} \right]  \right] 
\]
\end{eulerformula}
\begin{eulercomment}
Ada juga metode Lagrange, memaksimalkan H(a,b,c)ˆ2 terhadap a+b+d=d.
\end{eulercomment}
\begin{eulerprompt}
>&solve([diff(H(a,b,c)^2,a)=la,diff(H(a,b,c)^2,b)=la, ...
>   diff(H(a,b,c)^2,c)=la,a+b+c=d],[a,b,c,la])
\end{eulerprompt}
\begin{euleroutput}
  Maxima said:
  diff: second argument must be a variable; found [1,0,4]
   -- an error. To debug this try: debugmode(true);
  
  Error in:
  ... la,    diff(H(a,b,c)^2,c)=la,a+b+c=d],[a,b,c,la]) ...
                                                       ^
\end{euleroutput}
\begin{eulercomment}
Kita bisa membuat plot situasinya
\end{eulercomment}
\begin{eulercomment}
Pertama, atur poin di Maxima\\
.
\end{eulercomment}
\begin{eulerprompt}
>A &= at([x,y],sol[2]); $A
\end{eulerprompt}
\begin{euleroutput}
  Maxima said:
  part: invalid index of list or matrix.
   -- an error. To debug this try: debugmode(true);
  
  Error in:
  A &= at([x,y],sol[2]); $A ...
                       ^
\end{euleroutput}
\begin{eulerprompt}
>B &= [0,0]; $B, C &= [a,0]; $C
\end{eulerprompt}
\begin{eulerformula}
\[
\left[ 0 , 0 \right] 
\]
\end{eulerformula}
\begin{eulerformula}
\[
\left[ a , 0 \right] 
\]
\end{eulerformula}
\begin{eulercomment}
Kemudian atur rentang plot, dan plot poinnya.
\end{eulercomment}
\begin{eulerprompt}
>setPlotRange(0,5,-2,3); ...
>a=4; b=3; c=2; ...
>plotPoint(mxmeval("B"),"B"); plotPoint(mxmeval("C"),"C"); ...
>plotPoint(mxmeval("A"),"A"):
\end{eulerprompt}
\begin{euleroutput}
  Variable a1 not found!
  Use global variables or parameters for string evaluation.
  Error in Evaluate, superfluous characters found.
  Try "trace errors" to inspect local variables after errors.
  mxmeval:
      return evaluate(mxm(s));
  Error in:
  ... otPoint(mxmeval("C"),"C"); plotPoint(mxmeval("A"),"A"): ...
                                                       ^
\end{euleroutput}
\begin{eulercomment}
Plot garisnya
\end{eulercomment}
\begin{eulerprompt}
>plotSegment(mxmeval("A"),mxmeval("C")); ...
>plotSegment(mxmeval("B"),mxmeval("C")); ...
>plotSegment(mxmeval("B"),mxmeval("A")):
\end{eulerprompt}
\begin{euleroutput}
  Variable a1 not found!
  Use global variables or parameters for string evaluation.
  Error in Evaluate, superfluous characters found.
  Try "trace errors" to inspect local variables after errors.
  mxmeval:
      return evaluate(mxm(s));
  Error in:
  plotSegment(mxmeval("A"),mxmeval("C")); plotSegment(mxmeval("B ...
                          ^
\end{euleroutput}
\begin{eulercomment}
Hitung tengah tegak lurus di Maxima.
\end{eulercomment}
\begin{eulerprompt}
>h &= middlePerpendicular(A,B); g &= middlePerpendicular(B,C);
\end{eulerprompt}
\begin{eulercomment}
Dan bagian tengah dari keliling
\end{eulercomment}
\begin{eulerprompt}
>U &= lineIntersection(h,g);
\end{eulerprompt}
\begin{eulercomment}
Kita mendapatkan rumus untuk jari-jari lingkaran.
\end{eulercomment}
\begin{eulerprompt}
>&assume(a>0,b>0,c>0); $distance(U,B) | radcan
\end{eulerprompt}
\begin{eulerformula}
\[
\frac{\sqrt{{\it a_2}^2+{\it a_1}^2}\,\sqrt{{\it a_2}^2+{\it a_1}^2
 -2\,a\,{\it a_1}+a^2}}{2\,\left| {\it a_2}\right| }
\]
\end{eulerformula}
\begin{eulercomment}
Mari kita tambahkan ini ke plot.
\end{eulercomment}
\begin{eulerprompt}
>plotPoint(U()); ...
>plotCircle(circleWithCenter(mxmeval("U"),mxmeval("distance(U,C)"))):
\end{eulerprompt}
\begin{euleroutput}
  Variable a2 not found!
  Use global variables or parameters for string evaluation.
  Error in ^
  Error in expression: [a/2,(a2^2+a1^2-a*a1)/(2*a2)]
  Error in:
  plotPoint(U()); plotCircle(circleWithCenter(mxmeval("U"),mxmev ...
               ^
\end{euleroutput}
\begin{eulercomment}
Menggunakan geometri, kita mendapatkan rumus sederhana

\end{eulercomment}
\begin{eulerformula}
\[
\frac{a}{\sin(\alpha)}=2r
\]
\end{eulerformula}
\begin{eulercomment}
untuk radius. Kita dapat memeriksa, apakah ini benar dengan Maxima.
Maxima akan menfaktorkannya hanya jika kita mengkuadratkannya
\end{eulercomment}
\begin{eulerprompt}
>$c^2/sin(computeAngle(A,B,C))^2  | factor
\end{eulerprompt}
\begin{eulerformula}
\[
\left[ \frac{{\it a_2}^2+{\it a_1}^2}{{\it a_2}^2} , 0 , \frac{16\,
 \left({\it a_2}^2+{\it a_1}^2\right)}{{\it a_2}^2} \right] 
\]
\end{eulerformula}
\eulerheading{Contoh 4: Garis Euler dan Parabola}
\begin{eulercomment}
Garis euler adalah garis yang ditentukan dari segitiga yang tidak sama
sisi. Ini adalah garis tengahsegitiga, dan melewati beberapa titik
penting yang ditentukan dari segitiga, termasuk pusat ortosentrum,\\
sirkumenter, pusat massa, titik Exeter, dan pusat lingkaran sembilan
titik segitiga.

Untuk demonstrasi, kami menghitung dan memplot garis Euler dalam
segitiga.

Pertama, kami menentukan sudut segitiga di Euler. Kami menggunakan
definisi, yang terlihat dalam ekspresi simbolik.
\end{eulercomment}
\begin{eulerprompt}
>A::=[-1,-1]; B::=[2,0]; C::=[1,2];
\end{eulerprompt}
\begin{eulercomment}
Untuk memplot objek geometris, kami menyiapkan area plot, dan
menambahkan poin ke dalamnya. Semua plot objek geometris ditambahkan
ke plot saat ini.
\end{eulercomment}
\begin{eulerprompt}
>setPlotRange(3); plotPoint(A,"A"); plotPoint(B,"B"); plotPoint(C,"C");
\end{eulerprompt}
\begin{eulercomment}
Kita juga bisa menambahkan sisi segitiga
\end{eulercomment}
\begin{eulerprompt}
>plotSegment(A,B,""); plotSegment(B,C,""); plotSegment(C,A,""):
\end{eulerprompt}
\eulerimg{27}{images/Pekan 11-12_Fanny Erina Dewi_22305141005_EMT00-Geometry_Aplikom-064.png}
\begin{eulercomment}
Berikut adalah luas segitiga menggunakan rumus determinan. Tentu saja
kita harus mengambil nilai absolut dari hasil ini.
\end{eulercomment}
\begin{eulerprompt}
>$areaTriangle(A,B,C)
\end{eulerprompt}
\begin{eulerformula}
\[
-\frac{7}{2}
\]
\end{eulerformula}
\begin{eulercomment}
Kita dapat menghitung koefisien dari sisi c.
\end{eulercomment}
\begin{eulerprompt}
>c &= lineThrough(A,B)
\end{eulerprompt}
\begin{euleroutput}
  
                              [- 1, 3, - 2]
  
\end{euleroutput}
\begin{eulercomment}
Dan juga dapatkan rumus untuk baris ini.
\end{eulercomment}
\begin{eulerprompt}
>$getLineEquation(c,x,y)
\end{eulerprompt}
\begin{eulerformula}
\[
3\,y-x=-2
\]
\end{eulerformula}
\begin{eulercomment}
Untuk bentuk Hesse, kita perlu menentukan titik, sehingga titik
tersebut berada di sisi positif dari bentuk Hesse. Memasukkan titik
menghasilkan jarak positif ke garis.
\end{eulercomment}
\begin{eulerprompt}
>$getHesseForm(c,x,y,C), $at(%,[x=C[1],y=C[2]])
\end{eulerprompt}
\begin{eulerformula}
\[
\frac{3\,y-x+2}{\sqrt{10}}
\]
\end{eulerformula}
\begin{eulerformula}
\[
\frac{7}{\sqrt{10}}
\]
\end{eulerformula}
\begin{eulercomment}
Sekarang kita menghitung sirkit ABC.
\end{eulercomment}
\begin{eulerprompt}
>LL &= circleThrough(A,B,C); $getCircleEquation(LL,x,y)
\end{eulerprompt}
\begin{eulerformula}
\[
\left(y-\frac{5}{14}\right)^2+\left(x-\frac{3}{14}\right)^2=\frac{
 325}{98}
\]
\end{eulerformula}
\begin{eulerprompt}
>O &= getCircleCenter(LL); $O
\end{eulerprompt}
\begin{eulerformula}
\[
\left[ \frac{3}{14} , \frac{5}{14} \right] 
\]
\end{eulerformula}
\begin{eulercomment}
Plot lingkaran dan pusatnya. Cu dan U adalah simbolik. Kami
mengevaluasi ekspresi ini untuk Euler.
\end{eulercomment}
\begin{eulerprompt}
>plotCircle(LL()); plotPoint(O(),"O"):
\end{eulerprompt}
\eulerimg{27}{images/Pekan 11-12_Fanny Erina Dewi_22305141005_EMT00-Geometry_Aplikom-071.png}
\begin{eulercomment}
Kita dapat menghitung perpotongan ketinggian di ABC (orthocenter)
secara numerik dengan perintah berikut.
\end{eulercomment}
\begin{eulerprompt}
>H &= lineIntersection(perpendicular(A,lineThrough(C,B)),...
>  perpendicular(B,lineThrough(A,C))); $H
\end{eulerprompt}
\begin{eulerformula}
\[
\left[ \frac{11}{7} , \frac{2}{7} \right] 
\]
\end{eulerformula}
\begin{eulercomment}
Sekarang kita dapat menghitung garis Euler dari segitiga tersebut.
\end{eulercomment}
\begin{eulerprompt}
>el &= lineThrough(H,O); $getLineEquation(el,x,y)
\end{eulerprompt}
\begin{eulerformula}
\[
-\frac{19\,y}{14}-\frac{x}{14}=-\frac{1}{2}
\]
\end{eulerformula}
\begin{eulercomment}
Tambahkan ke plot kita
\end{eulercomment}
\begin{eulerprompt}
>plotPoint(H(),"H"); plotLine(el(),"Garis Euler"):
\end{eulerprompt}
\eulerimg{27}{images/Pekan 11-12_Fanny Erina Dewi_22305141005_EMT00-Geometry_Aplikom-074.png}
\begin{eulercomment}
Pusat gravitasi harus berada di garis ini.
\end{eulercomment}
\begin{eulerprompt}
>M &= (A+B+C)/3; $getLineEquation(el,x,y) with [x=M[1],y=M[2]]
\end{eulerprompt}
\begin{eulerformula}
\[
-\frac{1}{2}=-\frac{1}{2}
\]
\end{eulerformula}
\begin{eulerprompt}
>plotPoint(M(),"M"): // titik berat
\end{eulerprompt}
\eulerimg{27}{images/Pekan 11-12_Fanny Erina Dewi_22305141005_EMT00-Geometry_Aplikom-076.png}
\begin{eulercomment}
Teorinya mengatakan bahwa MH=2*MO. Kita perlu menyederhanakan dengan
radcan untuk mencapai ini.
\end{eulercomment}
\begin{eulerprompt}
>$distance(M,H)/distance(M,O)|radcan
\end{eulerprompt}
\begin{eulerformula}
\[
2
\]
\end{eulerformula}
\begin{eulercomment}
Fungsinya termasuk fungsi untuk sudut juga.
\end{eulercomment}
\begin{eulerprompt}
>$computeAngle(A,C,B), degprint(%())
\end{eulerprompt}
\begin{eulerformula}
\[
\arccos \left(\frac{4}{\sqrt{5}\,\sqrt{13}}\right)
\]
\end{eulerformula}
\begin{euleroutput}
  60°15'18.43''
\end{euleroutput}
\begin{eulercomment}
Persamaan untuk pusat lingkaran tidak terlalu bagus.
\end{eulercomment}
\begin{eulerprompt}
>Q &= lineIntersection(angleBisector(A,C,B),angleBisector(C,B,A))|radcan; $Q
\end{eulerprompt}
\begin{eulerformula}
\[
\left[ \frac{\left(2^{\frac{3}{2}}+1\right)\,\sqrt{5}\,\sqrt{13}-15
 \,\sqrt{2}+3}{14} , \frac{\left(\sqrt{2}-3\right)\,\sqrt{5}\,\sqrt{
 13}+5\,2^{\frac{3}{2}}+5}{14} \right] 
\]
\end{eulerformula}
\begin{eulercomment}
Mari kita hitung juga ekspresi jari-jari lingkaran yang tertulis
\end{eulercomment}
\begin{eulerprompt}
>r &= distance(Q,projectToLine(Q,lineThrough(A,B)))|ratsimp; $r
\end{eulerprompt}
\begin{eulerformula}
\[
\frac{\sqrt{\left(-41\,\sqrt{2}-31\right)\,\sqrt{5}\,\sqrt{13}+115
 \,\sqrt{2}+614}}{7\,\sqrt{2}}
\]
\end{eulerformula}
\begin{eulerprompt}
>LD &=  circleWithCenter(Q,r); // Lingkaran dalam
\end{eulerprompt}
\begin{eulercomment}
Mari kita tambahkan ini ke plot
\end{eulercomment}
\begin{eulerprompt}
>color(5); plotCircle(LD()):
\end{eulerprompt}
\eulerimg{27}{images/Pekan 11-12_Fanny Erina Dewi_22305141005_EMT00-Geometry_Aplikom-081.png}
\begin{eulerprompt}
>reset
\end{eulerprompt}
\begin{euleroutput}
  0
\end{euleroutput}
\eulersubheading{Parabola}
\begin{eulercomment}
Selanjutnya akan dicari persamaan tempat kedudukan titik-titik yang berjarak sama ke titik C
dan ke garis AB.
\end{eulercomment}
\begin{eulerprompt}
>p &= getHesseForm(lineThrough(A,B),x,y,C)-distance([x,y],C); $p='0
\end{eulerprompt}
\begin{eulerformula}
\[
\frac{3\,y-x+2}{\sqrt{10}}-\sqrt{\left(2-y\right)^2+\left(1-x
 \right)^2}=0
\]
\end{eulerformula}
\begin{eulercomment}
Persamaan tersebut dapat digambar menjadi satu dengan gambar sebelumnya.
\end{eulercomment}
\begin{eulerprompt}
>plot2d(p,level=0,add=1,contourcolor=6):
\end{eulerprompt}
\eulerimg{27}{images/Pekan 11-12_Fanny Erina Dewi_22305141005_EMT00-Geometry_Aplikom-083.png}
\begin{eulercomment}
Ini seharusnya menjadi beberapa fungsi, tetapi pemecah default Maxima
dapat menemukan solusi hanya,jika persamaan kita kuadratkan.
Akibatnya, kami mendapatkan solusi palsu.
\end{eulercomment}
\begin{eulerprompt}
>akar &= solve(getHesseForm(lineThrough(A,B),x,y,C)^2-distance([x,y],C)^2,y)
\end{eulerprompt}
\begin{euleroutput}
  
          [y = - 3 x - sqrt(70) sqrt(9 - 2 x) + 26, 
                                y = - 3 x + sqrt(70) sqrt(9 - 2 x) + 26]
  
\end{euleroutput}
\begin{eulercomment}
Solusi pertama adalah

maxima: akar[1]

Menambahkan solusi pertama ke pertunjukkan plot, bahwa itu memang
jalan yang kita cari. Teori mengatakan kepada kita bahwa itu adalah
parabola yang diputar.
\end{eulercomment}
\begin{eulerprompt}
>plot2d(&rhs(akar[1]),add=1):
\end{eulerprompt}
\eulerimg{27}{images/Pekan 11-12_Fanny Erina Dewi_22305141005_EMT00-Geometry_Aplikom-084.png}
\begin{eulerprompt}
>function g(x) &= rhs(akar[1]); $'g(x)= g(x)// fungsi yang mendefinisikan kurva di atas
\end{eulerprompt}
\begin{eulerformula}
\[
g\left(x\right)=-3\,x-\sqrt{70}\,\sqrt{9-2\,x}+26
\]
\end{eulerformula}
\begin{eulerprompt}
>T &=[-1, g(-1)]; // ambil sebarang titik pada kurva tersebut
>dTC &= distance(T,C); $fullratsimp(dTC), $float(%) // jarak T ke C
\end{eulerprompt}
\begin{eulerformula}
\[
\sqrt{1503-54\,\sqrt{11}\,\sqrt{70}}
\]
\end{eulerformula}
\begin{eulerformula}
\[
2.135605779339061
\]
\end{eulerformula}
\begin{eulerprompt}
>U &= projectToLine(T,lineThrough(A,B)); $U // proyeksi T pada garis AB 
\end{eulerprompt}
\begin{eulerformula}
\[
\left[ \frac{80-3\,\sqrt{11}\,\sqrt{70}}{10} , \frac{20-\sqrt{11}\,
 \sqrt{70}}{10} \right] 
\]
\end{eulerformula}
\begin{eulerprompt}
>dU2AB &= distance(T,U); $fullratsimp(dU2AB), $float(%) // jatak T ke AB
\end{eulerprompt}
\begin{eulerformula}
\[
\sqrt{1503-54\,\sqrt{11}\,\sqrt{70}}
\]
\end{eulerformula}
\begin{eulerformula}
\[
2.135605779339061
\]
\end{eulerformula}
\begin{eulercomment}
Ternyata jarak T ke C sama dengan jarak T ke AB. Coba Anda pilih titik T yang lain dan
ulangi perhitungan-perhitungan di atas untuk menunjukkan bahwa hasilnya juga sama.
\end{eulercomment}
\begin{eulercomment}

\begin{eulercomment}
\eulerheading{Contoh 5: Trigonometri Rasional}
\begin{eulercomment}
Ini terinspirasi oleh ceramah N.J.Wildberger. Dalam bukunya ”Proporsi
Agung”, Wildberger mengusulkan untuk menggantikan pengertian klasik
tentang jarak dan sudut dengan kuadransi dan penyebaran. Dengan
menggunakan ini, memang mungkin untuk menghindari fungsi trigonometri
dalam banyak contoh, dan tetap ”rasional”.

Berikut ini, saya memperkenalkan konsep, dan memecahkan beberapa
masalah. Saya menggunakan perhitungan simbolik Maxima di sini, yang
menyembunyikan keuntungan utama dari trigonometri rasional\\
bahwa perhitungan dapat dilakukan dengan kertas dan pensil saja. Anda
diundang untuk memeriksa hasil tanpa komputer.

Intinya adalah bahwa perhitungan rasional simbolis sering kali
menghasilkan hasil yang sederhana. Sebaliknya, trigonometri klasik
menghasilkan hasil trigonometri yang rumit, yang mengevaluasi ke
pendekatan numerik saja.
\end{eulercomment}
\begin{eulerprompt}
>load geometry;
\end{eulerprompt}
\begin{eulercomment}
Untuk pendahuluan pertama, kami menggunakan segitiga persegi panjang
dengan proporsi Mesir terkenal 3, 4 dan 5. Perintah berikut adalah
perintah Euler untuk memplot geometri bidang yang terdapat dalam\\
file Euler ”geometry.e”.
\end{eulercomment}
\begin{eulerprompt}
>C&:=[0,0]; A&:=[4,0]; B&:=[0,3]; ...
>setPlotRange(-1,5,-1,5); ...
>plotPoint(A,"A"); plotPoint(B,"B"); plotPoint(C,"C"); ...
>plotSegment(B,A,"c"); plotSegment(A,C,"b"); plotSegment(C,B,"a"); ...
>insimg(30);
\end{eulerprompt}
\eulerimg{27}{images/Pekan 11-12_Fanny Erina Dewi_22305141005_EMT00-Geometry_Aplikom-091.png}
\begin{eulercomment}
Tentu saja,

\end{eulercomment}
\begin{eulerformula}
\[
\sin(w_a)=\frac{a}{c},
\]
\end{eulerformula}
\begin{eulercomment}
di mana wa adalah sudut di A. Cara biasa untuk menghitung sudut ini,
adalah dengan melakukan invers dari fungsi sinus. Hasilnya adalah
sudut yang tidak dapat dicerna, yang hanya dapat dicetak secara
perkiraan.
\end{eulercomment}
\begin{eulerprompt}
>wa := arcsin(3/5); degprint(wa)
\end{eulerprompt}
\begin{euleroutput}
  36°52'11.63''
\end{euleroutput}
\begin{eulercomment}
Trigonometri rasional mencoba menghindari hal ini.

Pengertian pertama dari trigonometri rasional adalah kuadran, yang
menggantikan jarak. Faktanya, itu hanyalah kuadrat jarak. Berikut ini,
a, b, dan c menunjukkan kuadran sisi-sisinya.

Teorema Pythogoras menjadi a+b=c.
\end{eulercomment}
\begin{eulerprompt}
>a &= 3^2; b &= 4^2; c &= 5^2; &a+b=c
\end{eulerprompt}
\begin{euleroutput}
  
                                 25 = 25
  
\end{euleroutput}
\begin{eulercomment}
Gagasan kedua dari trigonometri rasional adalah penyebarannya. Spread
mengukur bukaan antar baris. Ini adalah 0, jika garis sejajar, dan 1,
jika garis persegi panjang. Ini adalah kuadrat dari sinus sudut\\
antara dua garis.


enyebaran garis AB dan AC pada gambar di atas didefinisikan sebagai


\end{eulercomment}
\begin{eulerformula}
\[
s_a = \sin(\alpha)^2 = \frac{a}{c},
\]
\end{eulerformula}
\begin{eulercomment}
di mana a dan c adalah kuadrat dari segitiga persegi panjang mana pun
dengan satu sudut di A.
\end{eulercomment}
\begin{eulerprompt}
>sa &= a/c; $sa
\end{eulerprompt}
\begin{eulerformula}
\[
\frac{9}{25}
\]
\end{eulerformula}
\begin{eulercomment}
Ini lebih mudah dihitung daripada sudut, tentu saja. Tetapi Anda
kehilangan properti yang sudut dapat ditambahkan dengan mudah.

Tentu saja, kita dapat mengubah nilai perkiraan sudut wa menjadi
sprad, dan mencetaknya sebagai\\
pecahan.
\end{eulercomment}
\begin{eulerprompt}
>fracprint(sin(wa)^2)
\end{eulerprompt}
\begin{euleroutput}
  9/25
\end{euleroutput}
\begin{eulercomment}
Hukum cosinus dari trgonometri klasik diterjemahkan menjadi ”hukum
silang” berikut.

\end{eulercomment}
\begin{eulerformula}
\[
(c+b-a)^2 = 4 b c \, (1-s_a)
\]
\end{eulerformula}
\begin{eulercomment}
Di sini a, b, dan c adalah kuadran dari sisi-sisi segitiga, dan sa
adalah sebaran di sudut A. Sisi a, seperti\\
biasa, berlawanan dengan sudut A.

Hukum ini diimplementasikan dalam file geometry.e yang kami muat ke
Euler.
\end{eulercomment}
\begin{eulerprompt}
>$crosslaw(aa,bb,cc,saa)
\end{eulerprompt}
\begin{eulerformula}
\[
\left[ \left({\it bb}-{\it aa}+\frac{7}{6}\right)^2 , \left(
 {\it bb}-{\it aa}+\frac{7}{6}\right)^2 , \left({\it bb}-{\it aa}+
 \frac{5}{3\,\sqrt{2}}\right)^2 \right] =\left[ \frac{14\,{\it bb}\,
 \left(1-{\it saa}\right)}{3} , \frac{14\,{\it bb}\,\left(1-{\it saa}
 \right)}{3} , \frac{5\,2^{\frac{3}{2}}\,{\it bb}\,\left(1-{\it saa}
 \right)}{3} \right] 
\]
\end{eulerformula}
\begin{eulercomment}
Dalam kasus kami, kita mendapatkan
\end{eulercomment}
\begin{eulerprompt}
>$crosslaw(a,b,c,sa)
\end{eulerprompt}
\begin{eulerformula}
\[
1024=1024
\]
\end{eulerformula}
\begin{eulercomment}
Mari kita gunakan crosslaw ini untuk mencari sebaran di A. Untuk
melakukan ini, kita menghasilkan crosslaw untuk kuadran a, b, dan c,
dan menyelesaikannya untuk sebaran yang tidak diketahui sa.

Anda dapat melakukan ini dengan tangan dengan mudah, tetapi saya
menggunakan Maxima. Tentu saja,kami mendapatkan hasilnya, kami sudah
mendapatkannya.
\end{eulercomment}
\begin{eulerprompt}
>$crosslaw(a,b,c,x), $solve(%,x)
\end{eulerprompt}
\begin{eulerformula}
\[
1024=1600\,\left(1-x\right)
\]
\end{eulerformula}
\begin{eulerformula}
\[
\left[ x=\frac{9}{25} \right] 
\]
\end{eulerformula}
\begin{eulercomment}
Kami sudah tahu ini. Definisi penyebaran adalah kasus khusus dari
hukum lintas hukum.

Kita juga bisa menyelesaikan ini untuk umum a, b, c. Hasilnya adalah
rumus yang menghitung sebaran sudut segitiga berdasarkan kuadran
ketiga sisinya.
\end{eulercomment}
\begin{eulerprompt}
>$solve(crosslaw(aa,bb,cc,x),x)
\end{eulerprompt}
\begin{eulerformula}
\[
\left[ \left[ \frac{168\,{\it bb}\,x+36\,{\it bb}^2+\left(-72\,
 {\it aa}-84\right)\,{\it bb}+36\,{\it aa}^2-84\,{\it aa}+49}{36} , 
 \frac{168\,{\it bb}\,x+36\,{\it bb}^2+\left(-72\,{\it aa}-84\right)
 \,{\it bb}+36\,{\it aa}^2-84\,{\it aa}+49}{36} , \frac{15\,2^{\frac{
 5}{2}}\,{\it bb}\,x+18\,{\it bb}^2+\left(-36\,{\it aa}-15\,2^{\frac{
 3}{2}}\right)\,{\it bb}+18\,{\it aa}^2-15\,2^{\frac{3}{2}}\,{\it aa}
 +25}{18} \right] =0 \right] 
\]
\end{eulerformula}
\begin{eulercomment}
Kita bisa membuat fungsi dari hasilnya. Fungsi seperti itu sudah
ditentukan dalam file geometry.e Euler.\\
.
\end{eulercomment}
\begin{eulerprompt}
>$spread(a,b,c)
\end{eulerprompt}
\begin{eulerformula}
\[
\frac{9}{25}
\]
\end{eulerformula}
\begin{eulercomment}
Sebagai contoh, kita bisa menggunakannya untuk menghitung sudut
segitiga bersisi

\end{eulercomment}
\begin{eulerformula}
\[
a, \quad a, \quad \frac{4a}{7}
\]
\end{eulerformula}
\begin{eulercomment}
Hasilnya rasional, yang tidak mudah didapat jika kita menggunakan
trigonometri klasik.
\end{eulercomment}
\begin{eulerprompt}
>$spread(a,a,4*a/7)
\end{eulerprompt}
\begin{eulerformula}
\[
\frac{6}{7}
\]
\end{eulerformula}
\begin{eulercomment}
Ini adalah sudut dalam derajat
\end{eulercomment}
\begin{eulerprompt}
>degprint(arcsin(sqrt(6/7)))
\end{eulerprompt}
\begin{euleroutput}
  67°47'32.44''
\end{euleroutput}
\eulersubheading{Contoh Lain}
\begin{eulercomment}
Sekarang, mari kita coba contoh yang lebih canggih.

Kami mengatur tiga sudut segitiga sebagai berikut.
\end{eulercomment}
\begin{eulerprompt}
>A&:=[1,2]; B&:=[4,3]; C&:=[0,4]; ...
>setPlotRange(-1,5,1,7); ...
>plotPoint(A,"A"); plotPoint(B,"B"); plotPoint(C,"C"); ...
>plotSegment(B,A,"c"); plotSegment(A,C,"b"); plotSegment(C,B,"a"); ...
>insimg;
\end{eulerprompt}
\eulerimg{27}{images/Pekan 11-12_Fanny Erina Dewi_22305141005_EMT00-Geometry_Aplikom-100.png}
\begin{eulercomment}
Menggunakan Pythogoras, mudah untuk menghitung jarak antara dua titik.
Saya pertama kali menggunakan jarak fungsi file Euler untuk geometri.
Jarak fungsi menggunakan geometri klasik.
\end{eulercomment}
\begin{eulerprompt}
>$distance(A,B)
\end{eulerprompt}
\begin{eulerformula}
\[
\sqrt{10}
\]
\end{eulerformula}
\begin{eulercomment}
Euler juga memiliki fungsi kuadrans antara dua titik.

Dalam contoh berikut, karena c + b bukan a, segitiga tidak persegi
panjang.
\end{eulercomment}
\begin{eulerprompt}
>c &= quad(A,B); $c, b &= quad(A,C); $b, a &= quad(B,C); $a,
\end{eulerprompt}
\begin{eulerformula}
\[
10
\]
\end{eulerformula}
\begin{eulerformula}
\[
5
\]
\end{eulerformula}
\begin{eulerformula}
\[
17
\]
\end{eulerformula}
\begin{eulercomment}
Pertama, mari kita hitung sudut tradisional. Fungsi computeAngle
menggunakan metode biasa berdasarkan perkalian titik dari dua vektor.
Hasilnya adalah beberapa pendekatan floating point.


\end{eulercomment}
\begin{eulerformula}
\[
A=<1,2>\quad B=<4,3>,\quad C=<0,4>
\]
\end{eulerformula}
\begin{eulerformula}
\[
\mathbf{a}=C-B=<-4,1>,\quad \mathbf{c}=A-B=<-3,-1>,\quad \beta=\angle ABC
\]
\end{eulerformula}
\begin{eulerformula}
\[
\mathbf{a}.\mathbf{c}=|\mathbf{a}|.|\mathbf{c}|\cos \beta
\]
\end{eulerformula}
\begin{eulerformula}
\[
\cos \angle ABC =\cos\beta=\frac{\mathbf{a}.\mathbf{c}}{|\mathbf{a}|.|\mathbf{c}|}=\frac{12-1}{\sqrt{17}\sqrt{10}}=\frac{11}{\sqrt{17}\sqrt{10}}
\]
\end{eulerformula}
\begin{eulerprompt}
>wb &= computeAngle(A,B,C); $wb, $(wb/pi*180)()
\end{eulerprompt}
\begin{eulerformula}
\[
\arccos \left(\frac{11}{\sqrt{10}\,\sqrt{17}}\right)
\]
\end{eulerformula}
\begin{euleroutput}
  32.4711922908
\end{euleroutput}
\begin{eulercomment}
Dengan menggunakan pensil dan kertas, kita dapat melakukan hal yang
sama dengan hukum silang. Kita masukkan kuadran a, b, dan c ke dalam
hukum silang dan selesaikan untuk x.
\end{eulercomment}
\begin{eulerprompt}
>$crosslaw(a,b,c,x), $solve(%,x), //(b+c-a)^=4b.c(1-x)
\end{eulerprompt}
\begin{eulerformula}
\[
4=200\,\left(1-x\right)
\]
\end{eulerformula}
\begin{eulerformula}
\[
\left[ x=\frac{49}{50} \right] 
\]
\end{eulerformula}
\begin{eulercomment}
Itulah yang dilakukan oleh fungsi spread yang didefinisikan dalam
"geometry.e".
\end{eulercomment}
\begin{eulerprompt}
>sb &= spread(b,a,c); $sb
\end{eulerprompt}
\begin{eulerformula}
\[
\frac{49}{170}
\]
\end{eulerformula}
\begin{eulercomment}
Maxima mendapatkan hasil yang sama dengan menggunakan trigonometri
biasa, jika kita memaksakannya. Ia menyelesaikan suku sin(arccos(...))
menjadi hasil pecahan. Sebagian besar siswa tidak dapat melakukan ini.
\end{eulercomment}
\begin{eulerprompt}
>$sin(computeAngle(A,B,C))^2
\end{eulerprompt}
\begin{eulerformula}
\[
\frac{49}{170}
\]
\end{eulerformula}
\begin{eulercomment}
Setelah kita memiliki penyebaran di B, kita dapat menghitung tinggi ha
di sisi a. Ingatlah bahwa

\end{eulercomment}
\begin{eulerformula}
\[
s_b=\frac{h_a}{c}
\]
\end{eulerformula}
\begin{eulercomment}
menurut definisi.
\end{eulercomment}
\begin{eulerprompt}
>ha &= c*sb; $ha
\end{eulerprompt}
\begin{eulerformula}
\[
\frac{49}{17}
\]
\end{eulerformula}
\begin{eulercomment}
Gambar berikut ini dibuat dengan program geometri C.a.R., yang dapat
menggambar kuadran dan penyebaran.

image: (20) Rational\_Geometry\_CaR.png

Menurut definisi, panjang ha adalah akar kuadrat dari kuadrannya.
\end{eulercomment}
\begin{eulerprompt}
>$sqrt(ha)
\end{eulerprompt}
\begin{eulerformula}
\[
\frac{7}{\sqrt{17}}
\]
\end{eulerformula}
\begin{eulercomment}
Sekarang kita dapat menghitung luas segitiga. Jangan lupa, bahwa kita
berurusan dengan kuadran!
\end{eulercomment}
\begin{eulerprompt}
>$sqrt(ha)*sqrt(a)/2
\end{eulerprompt}
\begin{eulerformula}
\[
\frac{7}{2}
\]
\end{eulerformula}
\begin{eulercomment}
Rumus penentu yang biasa menghasilkan hasil yang sama.
\end{eulercomment}
\begin{eulerprompt}
>$areaTriangle(B,A,C)
\end{eulerprompt}
\begin{eulerformula}
\[
\frac{7}{2}
\]
\end{eulerformula}
\eulersubheading{Rumus Heron}
\begin{eulercomment}
Sekarang, mari kita selesaikan masalah ini secara umum!
\end{eulercomment}
\begin{eulerprompt}
>&remvalue(a,b,c,sb,ha);
\end{eulerprompt}
\begin{eulercomment}
Pertama-tama kita menghitung penyebaran di B untuk segitiga dengan
sisi a, b, dan c. Kemudian kita menghitung luas kuadrat ("quadrea"?),
memfaktorkannya dengan Maxima, dan kita mendapatkan rumus Heron yang
terkenal.

Memang, hal ini sulit dilakukan dengan pensil dan kertas.
\end{eulercomment}
\begin{eulerprompt}
>$spread(b^2,c^2,a^2), $factor(%*c^2*a^2/4)
\end{eulerprompt}
\begin{eulerformula}
\[
\frac{-c^4-\left(-2\,b^2-2\,a^2\right)\,c^2-b^4+2\,a^2\,b^2-a^4}{4
 \,a^2\,c^2}
\]
\end{eulerformula}
\begin{eulerformula}
\[
\frac{\left(-c+b+a\right)\,\left(c-b+a\right)\,\left(c+b-a\right)\,
 \left(c+b+a\right)}{16}
\]
\end{eulerformula}
\eulersubheading{Aturan Penyebaran Tiga}
\begin{eulercomment}
Kerugian dari spread adalah bahwa mereka tidak lagi hanya menambahkan
sudut seperti.

Namun, tiga spread dari sebuah segitiga memenuhi aturan "triple
spread" berikut ini.
\end{eulercomment}
\begin{eulerprompt}
>&remvalue(sa,sb,sc); $triplespread(sa,sb,sc)
\end{eulerprompt}
\begin{eulerformula}
\[
\left({\it sc}+{\it sb}+{\it sa}\right)^2=2\,\left({\it sc}^2+
 {\it sb}^2+{\it sa}^2\right)+4\,{\it sa}\,{\it sb}\,{\it sc}
\]
\end{eulerformula}
\begin{eulercomment}
Aturan ini berlaku untuk tiga sudut yang berjumlah 180°.

\end{eulercomment}
\begin{eulerformula}
\[
\alpha+\beta+\gamma=\pi
\]
\end{eulerformula}
\begin{eulercomment}
Karena penyebaran dari

\end{eulercomment}
\begin{eulerformula}
\[
\alpha, \pi-\alpha
\]
\end{eulerformula}
\begin{eulercomment}
sama, aturan triple spread juga benar, jika

\end{eulercomment}
\begin{eulerformula}
\[
\alpha+\beta=\gamma
\]
\end{eulerformula}
\begin{eulercomment}
Karena penyebaran sudut negatifnya sama, aturan penyebaran tiga kali
lipat juga berlaku, jika

\end{eulercomment}
\begin{eulerformula}
\[
\alpha+\beta+\gamma=0
\]
\end{eulerformula}
\begin{eulercomment}
Contohnya, kita bisa menghitung penyebaran sudut 60°. Hasilnya adalah
3/4. Namun, persamaan ini memiliki solusi kedua, di mana semua
penyebarannya adalah 0.
\end{eulercomment}
\begin{eulerprompt}
>$solve(triplespread(x,x,x),x)
\end{eulerprompt}
\begin{eulerformula}
\[
\left[ x=\frac{3}{4} , x=0 \right] 
\]
\end{eulerformula}
\begin{eulercomment}
Penyebaran 90° jelas adalah 1. Jika dua sudut ditambahkan ke 90°,
penyebarannya akan menyelesaikan persamaan penyebaran tiga dengan a,
b, 1. Dengan perhitungan berikut, kita mendapatkan a + b = 1.
\end{eulercomment}
\begin{eulerprompt}
>$triplespread(x,y,1), $solve(%,x)
\end{eulerprompt}
\begin{eulerformula}
\[
\left(y+x+1\right)^2=2\,\left(y^2+x^2+1\right)+4\,x\,y
\]
\end{eulerformula}
\begin{eulerformula}
\[
\left[ x=1-y \right] 
\]
\end{eulerformula}
\begin{eulercomment}
Karena penyebaran 180°-t sama dengan penyebaran t, rumus penyebaran
tiga kali lipat juga berlaku, jika satu sudut adalah jumlah atau
selisih dari dua sudut lainnya.

Jadi kita dapat menemukan penyebaran sudut dua kali lipat. Perhatikan
bahwa ada dua solusi lagi. Kita jadikan ini sebuah fungsi.
\end{eulercomment}
\begin{eulerprompt}
>$solve(triplespread(a,a,x),x), function doublespread(a) &= factor(rhs(%[1]))
\end{eulerprompt}
\begin{eulerformula}
\[
\left[ x=4\,a-4\,a^2 , x=0 \right] 
\]
\end{eulerformula}
\begin{euleroutput}
  
                              - 4 (a - 1) a
  
\end{euleroutput}
\eulersubheading{Garis Pembagi Sudut}
\begin{eulercomment}
Ini adalah situasi yang sudah kita ketahui.
\end{eulercomment}
\begin{eulerprompt}
>C&:=[0,0]; A&:=[4,0]; B&:=[0,3]; ...
>setPlotRange(-1,5,-1,5); ...
>plotPoint(A,"A"); plotPoint(B,"B"); plotPoint(C,"C"); ...
>plotSegment(B,A,"c"); plotSegment(A,C,"b"); plotSegment(C,B,"a"); ...
>insimg;
\end{eulerprompt}
\eulerimg{27}{images/Pekan 11-12_Fanny Erina Dewi_22305141005_EMT00-Geometry_Aplikom-121.png}
\begin{eulercomment}
Mari kita hitung panjang garis bagi sudut di A. Tetapi kita ingin
menyelesaikannya untuk a, b, c secara umum.
\end{eulercomment}
\begin{eulerprompt}
>&remvalue(a,b,c);
\end{eulerprompt}
\begin{eulercomment}
Jadi, pertama-tama kita menghitung penyebaran sudut yang dibelah dua
di A, menggunakan rumus penyebaran tiga.

Masalah dengan rumus ini muncul lagi. Rumus ini memiliki dua solusi.
Kita harus memilih salah satu yang benar. Solusi lainnya mengacu pada
sudut terbagi dua 180°-wa.
\end{eulercomment}
\begin{eulerprompt}
>$triplespread(x,x,a/(a+b)), $solve(%,x), sa2 &= rhs(%[1]); $sa2
\end{eulerprompt}
\begin{eulerformula}
\[
\left(2\,x+\frac{a}{b+a}\right)^2=2\,\left(2\,x^2+\frac{a^2}{\left(
 b+a\right)^2}\right)+\frac{4\,a\,x^2}{b+a}
\]
\end{eulerformula}
\begin{eulerformula}
\[
\left[ x=\frac{-\sqrt{b}\,\sqrt{b+a}+b+a}{2\,b+2\,a} , x=\frac{
 \sqrt{b}\,\sqrt{b+a}+b+a}{2\,b+2\,a} \right] 
\]
\end{eulerformula}
\begin{eulerformula}
\[
\frac{-\sqrt{b}\,\sqrt{b+a}+b+a}{2\,b+2\,a}
\]
\end{eulerformula}
\begin{eulercomment}
Mari kita periksa persegi panjang Mesir.
\end{eulercomment}
\begin{eulerprompt}
>$sa2 with [a=3^2,b=4^2]
\end{eulerprompt}
\begin{eulerformula}
\[
\frac{1}{10}
\]
\end{eulerformula}
\begin{eulercomment}
Kita bisa mencetak sudut dalam Euler, setelah mentransfer penyebaran
ke radian.
\end{eulercomment}
\begin{eulerprompt}
>wa2 := arcsin(sqrt(1/10)); degprint(wa2)
\end{eulerprompt}
\begin{euleroutput}
  18°26'5.82''
\end{euleroutput}
\begin{eulercomment}
Titik P adalah perpotongan garis bagi sudut dengan sumbu y.
\end{eulercomment}
\begin{eulerprompt}
>P := [0,tan(wa2)*4]
\end{eulerprompt}
\begin{euleroutput}
  [0,  1.33333]
\end{euleroutput}
\begin{eulerprompt}
>plotPoint(P,"P"); plotSegment(A,P):
\end{eulerprompt}
\eulerimg{27}{images/Pekan 11-12_Fanny Erina Dewi_22305141005_EMT00-Geometry_Aplikom-126.png}
\begin{eulercomment}
Mari kita periksa sudut-sudutnya dalam contoh spesifik kita.
\end{eulercomment}
\begin{eulerprompt}
>computeAngle(C,A,P), computeAngle(P,A,B)
\end{eulerprompt}
\begin{euleroutput}
  0.321750554397
  0.321750554397
\end{euleroutput}
\begin{eulercomment}
Now we compute the length of the bisector AP.

Sekarang kita menghitung panjang garis bagi AP.

\end{eulercomment}
\begin{eulerformula}
\[
\frac{BC}{\sin(w_a)} = \frac{AC}{\sin(w_b)} = \frac{AB}{\sin(w_c)}
\]
\end{eulerformula}
\begin{eulercomment}
berlaku dalam segitiga apa pun. Kuadratkan, ini diterjemahkan ke dalam
apa yang disebut "hukum penyebaran"

\end{eulercomment}
\begin{eulerformula}
\[
\frac{a}{s_a} = \frac{b}{s_b} = \frac{c}{s_b}
\]
\end{eulerformula}
\begin{eulercomment}
di mana a, b, c menunjukkan kuadrannya.

Karena spread CPA adalah 1-sa2, kita mendapatkan bisa/1=b/(1-sa2) dan
bisa menghitung bisa (kuadran dari pembagi sudut).
\end{eulercomment}
\begin{eulerprompt}
>&factor(ratsimp(b/(1-sa2))); bisa &= %; $bisa
\end{eulerprompt}
\begin{eulerformula}
\[
\frac{2\,b\,\left(b+a\right)}{\sqrt{b}\,\sqrt{b+a}+b+a}
\]
\end{eulerformula}
\begin{eulercomment}
Mari kita periksa rumus ini untuk nilai-nilai Mesir kita.
\end{eulercomment}
\begin{eulerprompt}
>sqrt(mxmeval("at(bisa,[a=3^2,b=4^2])")), distance(A,P)
\end{eulerprompt}
\begin{euleroutput}
  4.21637021356
  4.21637021356
\end{euleroutput}
\begin{eulercomment}
Kita juga dapat menghitung P dengan menggunakan rumus penyebaran.
\end{eulercomment}
\begin{eulerprompt}
>py&=factor(ratsimp(sa2*bisa)); $py
\end{eulerprompt}
\begin{eulerformula}
\[
-\frac{b\,\left(\sqrt{b}\,\sqrt{b+a}-b-a\right)}{\sqrt{b}\,\sqrt{b+
 a}+b+a}
\]
\end{eulerformula}
\begin{eulercomment}
Nilainya sama dengan yang kita dapatkan dengan rumus trigonometri.
\end{eulercomment}
\begin{eulerprompt}
>sqrt(mxmeval("at(py,[a=3^2,b=4^2])"))
\end{eulerprompt}
\begin{euleroutput}
  1.33333333333
\end{euleroutput}
\eulersubheading{Sudut Akor}
\begin{eulercomment}
Lihatlah situasi berikut ini.
\end{eulercomment}
\begin{eulerprompt}
>setPlotRange(1.2); ...
>color(1); plotCircle(circleWithCenter([0,0],1)); ...
>A:=[cos(1),sin(1)]; B:=[cos(2),sin(2)]; C:=[cos(6),sin(6)]; ...
>plotPoint(A,"A"); plotPoint(B,"B"); plotPoint(C,"C"); ...
>color(3); plotSegment(A,B,"c"); plotSegment(A,C,"b"); plotSegment(C,B,"a"); ...
>color(1); O:=[0,0];  plotPoint(O,"0"); ...
>plotSegment(A,O); plotSegment(B,O); plotSegment(C,O,"r"); ...
>insimg;
\end{eulerprompt}
\eulerimg{27}{images/Pekan 11-12_Fanny Erina Dewi_22305141005_EMT00-Geometry_Aplikom-129.png}
\begin{eulercomment}
Kita dapat menggunakan Maxima untuk menyelesaikan rumus penyebaran
tiga untuk sudut-sudut di pusat O untuk r. Dengan demikian kita
mendapatkan rumus untuk jari-jari kuadrat dari pericircle dalam hal
kuadran sisi-sisinya.

Kali ini, Maxima menghasilkan beberapa angka nol yang rumit, yang kita
abaikan.
\end{eulercomment}
\begin{eulerprompt}
>&remvalue(a,b,c,r); // hapus nilai-nilai sebelumnya untuk perhitungan baru
>rabc &= rhs(solve(triplespread(spread(b,r,r),spread(a,r,r),spread(c,r,r)),r)[4]); $rabc
\end{eulerprompt}
\begin{eulerformula}
\[
-\frac{a\,b\,c}{c^2-2\,b\,c+a\,\left(-2\,c-2\,b\right)+b^2+a^2}
\]
\end{eulerformula}
\begin{eulercomment}
Kita dapat menjadikannya sebuah fungsi Euler.
\end{eulercomment}
\begin{eulerprompt}
>function periradius(a,b,c) &= rabc;
\end{eulerprompt}
\begin{eulercomment}
Mari kita periksa hasilnya untuk poin A, B, C.
\end{eulercomment}
\begin{eulerprompt}
>a:=quadrance(B,C); b:=quadrance(A,C); c:=quadrance(A,B);
\end{eulerprompt}
\begin{eulercomment}
Radiusnya 1.
\end{eulercomment}
\begin{eulerprompt}
>periradius(a,b,c)
\end{eulerprompt}
\begin{euleroutput}
  1
\end{euleroutput}
\begin{eulercomment}
Faktanya adalah, bahwa penyebaran CBA hanya bergantung pada b dan c.
Ini adalah teorema sudut akor.
\end{eulercomment}
\begin{eulerprompt}
>$spread(b,a,c)*rabc | ratsimp
\end{eulerprompt}
\begin{eulerformula}
\[
\frac{b}{4}
\]
\end{eulerformula}
\begin{eulercomment}
Faktanya, penyebarannya adalah b/(4r), dan kita melihat bahwa sudut
chord b adalah setengah dari sudut tengah.
\end{eulercomment}
\begin{eulerprompt}
>$doublespread(b/(4*r))-spread(b,r,r) | ratsimp
\end{eulerprompt}
\begin{eulerformula}
\[
0
\]
\end{eulerformula}
\begin{eulercomment}
\begin{eulercomment}
\eulerheading{Contoh 6: Jarak Minimal pada Bidang}
\begin{eulercomment}
\end{eulercomment}
\eulersubheading{Pernyataan Awal}
\begin{eulercomment}
Fungsi yang, pada sebuah titik M pada bidang, menetapkan jarak AM
antara titik tetap A dan M, memiliki garis-garis tingkat yang cukup
sederhana: lingkaran yang berpusat di A.
\end{eulercomment}
\begin{eulerprompt}
>&remvalue();
>A=[-1,-1];
>function d1(x,y):=sqrt((x-A[1])^2+(y-A[2])^2)
>fcontour("d1",xmin=-2,xmax=0,ymin=-2,ymax=0,hue=1, ...
>title="If you see ellipses, please set your window square"):
\end{eulerprompt}
\eulerimg{27}{images/Pekan 11-12_Fanny Erina Dewi_22305141005_EMT00-Geometry_Aplikom-133.png}
\begin{eulercomment}
dan grafiknya juga cukup sederhana: bagian atas kerucut:
\end{eulercomment}
\begin{eulerprompt}
>plot3d("d1",xmin=-2,xmax=0,ymin=-2,ymax=0):
\end{eulerprompt}
\eulerimg{27}{images/Pekan 11-12_Fanny Erina Dewi_22305141005_EMT00-Geometry_Aplikom-134.png}
\begin{eulercomment}
Tentu saja, nilai minimum 0 diperoleh di A.

\end{eulercomment}
\eulersubheading{Dua Poin}
\begin{eulercomment}
Sekarang kita lihat fungsi MA+MB di mana A dan B adalah dua titik
(tetap). Ini adalah "fakta yang terkenal" bahwa kurva level adalah
elips, titik fokusnya adalah A dan B; kecuali AB minimum yang konstan
pada segmen [AB]:
\end{eulercomment}
\begin{eulerprompt}
>B=[1,-1];
>function d2(x,y):=d1(x,y)+sqrt((x-B[1])^2+(y-B[2])^2)
>fcontour("d2",xmin=-2,xmax=2,ymin=-3,ymax=1,hue=1):
\end{eulerprompt}
\eulerimg{27}{images/Pekan 11-12_Fanny Erina Dewi_22305141005_EMT00-Geometry_Aplikom-135.png}
\begin{eulercomment}
Grafiknya lebih menarik:
\end{eulercomment}
\begin{eulerprompt}
>plot3d("d2",xmin=-2,xmax=2,ymin=-3,ymax=1):
\end{eulerprompt}
\eulerimg{27}{images/Pekan 11-12_Fanny Erina Dewi_22305141005_EMT00-Geometry_Aplikom-136.png}
\begin{eulercomment}
Pembatasan pada garis (AB) lebih terkenal:
\end{eulercomment}
\begin{eulerprompt}
>plot2d("abs(x+1)+abs(x-1)",xmin=-3,xmax=3):
\end{eulerprompt}
\eulerimg{27}{images/Pekan 11-12_Fanny Erina Dewi_22305141005_EMT00-Geometry_Aplikom-137.png}
\begin{eulercomment}
\end{eulercomment}
\eulersubheading{Poin ke 3}
\begin{eulercomment}
Sekarang, hal-hal menjadi kurang sederhana: Hal ini sedikit kurang
dikenal bahwa MA+MB+MC mencapai minimumnya pada satu titik di bidang,
tetapi untuk menentukannya tidak sesederhana itu:

1) Jika salah satu sudut segitiga ABC lebih dari 120° (katakanlah di
A), maka nilai minimumnya dicapai pada titik ini (katakanlah AB+AC).

Contoh:
\end{eulercomment}
\begin{eulerprompt}
>C=[-4,1];
>function d3(x,y):=d2(x,y)+sqrt((x-C[1])^2+(y-C[2])^2)
>plot3d("d3",xmin=-5,xmax=3,ymin=-4,ymax=4);
>insimg;
\end{eulerprompt}
\eulerimg{27}{images/Pekan 11-12_Fanny Erina Dewi_22305141005_EMT00-Geometry_Aplikom-138.png}
\begin{eulerprompt}
>fcontour("d3",xmin=-4,xmax=1,ymin=-2,ymax=2,hue=1,title="The minimum is on A");
>P=(A_B_C_A)'; plot2d(P[1],P[2],add=1,color=12);
>insimg;
\end{eulerprompt}
\eulerimg{27}{images/Pekan 11-12_Fanny Erina Dewi_22305141005_EMT00-Geometry_Aplikom-139.png}
\begin{eulercomment}
2) Tetapi jika semua sudut segitiga ABC kurang dari 120°, minimumnya
adalah pada titik F di bagian dalam segitiga, yang merupakan
satu-satunya titik yang melihat sisi-sisi ABC dengan sudut yang sama
(masing-masing 120°):
\end{eulercomment}
\begin{eulerprompt}
>C=[-0.5,1];
>plot3d("d3",xmin=-2,xmax=2,ymin=-2,ymax=2):
\end{eulerprompt}
\eulerimg{27}{images/Pekan 11-12_Fanny Erina Dewi_22305141005_EMT00-Geometry_Aplikom-140.png}
\begin{eulerprompt}
>fcontour("d3",xmin=-2,xmax=2,ymin=-2,ymax=2,hue=1,title="The Fermat point");
>P=(A_B_C_A)'; plot2d(P[1],P[2],add=1,color=12);
>insimg;
\end{eulerprompt}
\eulerimg{27}{images/Pekan 11-12_Fanny Erina Dewi_22305141005_EMT00-Geometry_Aplikom-141.png}
\begin{eulercomment}
Merupakan kegiatan yang menarik untuk merealisasikan gambar di atas
dengan perangkat lunak geometri; sebagai contoh, saya tahu sebuah
perangkat lunak yang ditulis dalam bahasa Java yang memiliki instruksi
"garis kontur"...

Semua hal di atas telah ditemukan oleh seorang hakim Perancis bernama
Pierre de Fermat; dia menulis surat kepada para ahli lain seperti
pendeta Marin Mersenne dan Blaise Pascal yang bekerja di bagian pajak
penghasilan. Jadi titik unik F sedemikian rupa sehingga FA+FB+FC
minimal, disebut titik Fermat dari segitiga. Namun tampaknya beberapa
tahun sebelumnya, Torriccelli dari Italia telah menemukan titik ini
sebelum Fermat menemukannya! Bagaimanapun juga, tradisinya adalah
mencatat titik F ini...

\end{eulercomment}
\eulersubheading{Empat poin}
\begin{eulercomment}
Langkah selanjutnya adalah menambahkan titik ke-4 D dan mencoba
meminimumkan MA+MB+MC+MD; misalkan Anda adalah operator TV kabel dan
ingin menemukan di bidang mana Anda harus meletakkan antena sehingga
Anda dapat memberi makan empat desa dan menggunakan panjang kabel
sesedikit mungkin!
\end{eulercomment}
\begin{eulerprompt}
>D=[1,1];
>function d4(x,y):=d3(x,y)+sqrt((x-D[1])^2+(y-D[2])^2)
>plot3d("d4",xmin=-1.5,xmax=1.5,ymin=-1.5,ymax=1.5):
\end{eulerprompt}
\eulerimg{27}{images/Pekan 11-12_Fanny Erina Dewi_22305141005_EMT00-Geometry_Aplikom-142.png}
\begin{eulerprompt}
>fcontour("d4",xmin=-1.5,xmax=1.5,ymin=-1.5,ymax=1.5,hue=1);
>P=(A_B_C_D)'; plot2d(P[1],P[2],points=1,add=1,color=12);
>insimg;
\end{eulerprompt}
\eulerimg{27}{images/Pekan 11-12_Fanny Erina Dewi_22305141005_EMT00-Geometry_Aplikom-143.png}
\begin{eulercomment}
Masih ada nilai minimum dan tidak ada simpul A, B, C, maupun D:
\end{eulercomment}
\begin{eulerprompt}
>function f(x):=d4(x[1],x[2])
>neldermin("f",[0.2,0.2])
\end{eulerprompt}
\begin{euleroutput}
  [0.142858,  0.142857]
\end{euleroutput}
\begin{eulercomment}
Tampaknya dalam kasus ini, koordinat titik optimal adalah rasional
atau mendekati rasional...

Karena ABCD adalah sebuah bujur sangkar, maka kita berharap bahwa
titik optimalnya adalah pusat dari ABCD:
\end{eulercomment}
\begin{eulerprompt}
>C=[-1,1];
>plot3d("d4",xmin=-1,xmax=1,ymin=-1,ymax=1):
\end{eulerprompt}
\eulerimg{27}{images/Pekan 11-12_Fanny Erina Dewi_22305141005_EMT00-Geometry_Aplikom-144.png}
\begin{eulerprompt}
>fcontour("d4",xmin=-1.5,xmax=1.5,ymin=-1.5,ymax=1.5,hue=1);
>P=(A_B_C_D)'; plot2d(P[1],P[2],add=1,color=12,points=1);
>insimg;
\end{eulerprompt}
\eulerimg{27}{images/Pekan 11-12_Fanny Erina Dewi_22305141005_EMT00-Geometry_Aplikom-145.png}
\eulerheading{Contoh 7: Bola Dandelin dengan Povray}
\begin{eulercomment}
Anda dapat menjalankan demonstrasi ini, jika Anda telah menginstal
Povray, dan pvengine.exe pada jalur program.

Pertama, kita menghitung jari-jari bola.

Jika Anda melihat gambar di bawah ini, Anda dapat melihat bahwa kita
membutuhkan dua lingkaran yang menyentuh dua garis yang membentuk
kerucut, dan satu garis yang membentuk bidang yang memotong kerucut.

Kita menggunakan file geometri.e dari Euler untuk hal ini.
\end{eulercomment}
\begin{eulerprompt}
>load geometry;
\end{eulerprompt}
\begin{eulercomment}
Pertama, dua garis yang membentuk kerucut.
\end{eulercomment}
\begin{eulerprompt}
>g1 &= lineThrough([0,0],[1,a])
\end{eulerprompt}
\begin{euleroutput}
  
                               [- a, 1, 0]
  
\end{euleroutput}
\begin{eulerprompt}
>g2 &= lineThrough([0,0],[-1,a])
\end{eulerprompt}
\begin{euleroutput}
  
                              [- a, - 1, 0]
  
\end{euleroutput}
\begin{eulercomment}
Kemudian baris ketiga.
\end{eulercomment}
\begin{eulerprompt}
>g &= lineThrough([-1,0],[1,1])
\end{eulerprompt}
\begin{euleroutput}
  
                               [- 1, 2, 1]
  
\end{euleroutput}
\begin{eulercomment}
Kami merencanakan semuanya sejauh ini.
\end{eulercomment}
\begin{eulerprompt}
>setPlotRange(-1,1,0,2);
>color(black); plotLine(g(),"")
>a:=2; color(blue); plotLine(g1(),""), plotLine(g2(),""):
\end{eulerprompt}
\eulerimg{27}{images/Pekan 11-12_Fanny Erina Dewi_22305141005_EMT00-Geometry_Aplikom-146.png}
\begin{eulercomment}
Sekarang, kita ambil titik umum pada sumbu y.
\end{eulercomment}
\begin{eulerprompt}
>P &= [0,u]
\end{eulerprompt}
\begin{euleroutput}
  
                                  [0, u]
  
\end{euleroutput}
\begin{eulercomment}
Hitung jarak ke g1.
\end{eulercomment}
\begin{eulerprompt}
>d1 &= distance(P,projectToLine(P,g1)); $d1
\end{eulerprompt}
\begin{eulerformula}
\[
\sqrt{\left(\frac{a^2\,u}{a^2+1}-u\right)^2+\frac{a^2\,u^2}{\left(a
 ^2+1\right)^2}}
\]
\end{eulerformula}
\begin{eulercomment}
Compute the distance to g.
\end{eulercomment}
\begin{eulerprompt}
>d &= distance(P,projectToLine(P,g)); $d
\end{eulerprompt}
\begin{eulerformula}
\[
\sqrt{\left(\frac{u+2}{5}-u\right)^2+\frac{\left(2\,u-1\right)^2}{
 25}}
\]
\end{eulerformula}
\begin{eulercomment}
Dan temukan pusat kedua lingkaran, yang jaraknya sama.
\end{eulercomment}
\begin{eulerprompt}
>sol &= solve(d1^2=d^2,u); $sol
\end{eulerprompt}
\begin{eulerformula}
\[
\left[ u=\frac{-\sqrt{5}\,\sqrt{a^2+1}+2\,a^2+2}{4\,a^2-1} , u=
 \frac{\sqrt{5}\,\sqrt{a^2+1}+2\,a^2+2}{4\,a^2-1} \right] 
\]
\end{eulerformula}
\begin{eulercomment}
Ada dua solusi.

Kami mengevaluasi solusi simbolis, dan menemukan kedua pusat, dan
kedua jarak.
\end{eulercomment}
\begin{eulerprompt}
>u := sol()
\end{eulerprompt}
\begin{euleroutput}
  [0.333333,  1]
\end{euleroutput}
\begin{eulerprompt}
>dd := d()
\end{eulerprompt}
\begin{euleroutput}
  [0.149071,  0.447214]
\end{euleroutput}
\begin{eulercomment}
Plot lingkaran-lingkaran ke dalam gambar.
\end{eulercomment}
\begin{eulerprompt}
>color(red);
>plotCircle(circleWithCenter([0,u[1]],dd[1]),"");
>plotCircle(circleWithCenter([0,u[2]],dd[2]),"");
>insimg;
\end{eulerprompt}
\eulerimg{27}{images/Pekan 11-12_Fanny Erina Dewi_22305141005_EMT00-Geometry_Aplikom-150.png}
\eulersubheading{Plot dengan Povray}
\begin{eulercomment}
Selanjutnya kita plot semuanya dengan Povray. Perhatikan bahwa Anda
mengubah perintah apa pun dalam urutan perintah Povray berikut ini,
dan jalankan kembali semua perintah dengan Shift-Return.

Pertama kita memuat fungsi povray.
\end{eulercomment}
\begin{eulerprompt}
>load povray;
>defaultpovray="C:\(\backslash\)Program Files\(\backslash\)POV-Ray\(\backslash\)v3.7\(\backslash\)bin\(\backslash\)pvengine.exe"
\end{eulerprompt}
\begin{euleroutput}
  C:\(\backslash\)Program Files\(\backslash\)POV-Ray\(\backslash\)v3.7\(\backslash\)bin\(\backslash\)pvengine.exe
\end{euleroutput}
\begin{eulercomment}
Kami menyiapkan pemandangan dengan tepat.
\end{eulercomment}
\begin{eulerprompt}
>povstart(zoom=11,center=[0,0,0.5],height=10°,angle=140°);
\end{eulerprompt}
\begin{eulercomment}
Selanjutnya kita tulis kedua bola tersebut ke file Povray.
\end{eulercomment}
\begin{eulerprompt}
>writeln(povsphere([0,0,u[1]],dd[1],povlook(red)));
>writeln(povsphere([0,0,u[2]],dd[2],povlook(red)));
\end{eulerprompt}
\begin{eulercomment}
Dan kerucutnya, transparan.
\end{eulercomment}
\begin{eulerprompt}
>writeln(povcone([0,0,0],0,[0,0,a],1,povlook(lightgray,1)));
\end{eulerprompt}
\begin{eulercomment}
Kami menghasilkan bidang yang terbatas pada kerucut.
\end{eulercomment}
\begin{eulerprompt}
>gp=g();
>pc=povcone([0,0,0],0,[0,0,a],1,"");
>vp=[gp[1],0,gp[2]]; dp=gp[3];
>writeln(povplane(vp,dp,povlook(blue,0.5),pc));
\end{eulerprompt}
\begin{eulercomment}
Sekarang kita menghasilkan dua titik pada lingkaran, di mana bola
menyentuh kerucut.
\end{eulercomment}
\begin{eulerprompt}
>function turnz(v) := return [-v[2],v[1],v[3]]
>P1=projectToLine([0,u[1]],g1()); P1=turnz([P1[1],0,P1[2]]);
>writeln(povpoint(P1,povlook(yellow)));
>P2=projectToLine([0,u[2]],g1()); P2=turnz([P2[1],0,P2[2]]);
>writeln(povpoint(P2,povlook(yellow)));
\end{eulerprompt}
\begin{eulercomment}
Kemudian, kita menghasilkan dua titik di mana bola-bola tersebut
menyentuh bidang. Ini adalah fokus elips.
\end{eulercomment}
\begin{eulerprompt}
>P3=projectToLine([0,u[1]],g()); P3=[P3[1],0,P3[2]];
>writeln(povpoint(P3,povlook(yellow)));
>P4=projectToLine([0,u[2]],g()); P4=[P4[1],0,P4[2]];
>writeln(povpoint(P4,povlook(yellow)));
\end{eulerprompt}
\begin{eulercomment}
Selanjutnya kita menghitung perpotongan P1P2 dengan bidang.
\end{eulercomment}
\begin{eulerprompt}
>t1=scalp(vp,P1)-dp; t2=scalp(vp,P2)-dp; P5=P1+t1/(t1-t2)*(P2-P1);
>writeln(povpoint(P5,povlook(yellow)));
\end{eulerprompt}
\begin{eulercomment}
Kami menghubungkan titik-titik dengan segmen garis.
\end{eulercomment}
\begin{eulerprompt}
>writeln(povsegment(P1,P2,povlook(yellow)));
>writeln(povsegment(P5,P3,povlook(yellow)));
>writeln(povsegment(P5,P4,povlook(yellow)));
\end{eulerprompt}
\begin{eulercomment}
Sekarang, kita menghasilkan pita abu-abu, di mana bola-bola menyentuh
kerucut.
\end{eulercomment}
\begin{eulerprompt}
>pcw=povcone([0,0,0],0,[0,0,a],1.01);
>pc1=povcylinder([0,0,P1[3]-defaultpointsize/2],[0,0,P1[3]+defaultpointsize/2],1);
>writeln(povintersection([pcw,pc1],povlook(gray)));
>pc2=povcylinder([0,0,P2[3]-defaultpointsize/2],[0,0,P2[3]+defaultpointsize/2],1);
>writeln(povintersection([pcw,pc2],povlook(gray)));
\end{eulerprompt}
\begin{eulercomment}
Mulai program Povray.
\end{eulercomment}
\begin{eulerprompt}
>povend();
\end{eulerprompt}
\eulerimg{28}{images/Pekan 11-12_Fanny Erina Dewi_22305141005_EMT00-Geometry_Aplikom-151.png}
\begin{eulercomment}
Untuk mendapatkan Anaglyph ini, kita perlu memasukkan semuanya ke
dalam fungsi scene. Fungsi ini akan digunakan dua kali nanti.
\end{eulercomment}
\begin{eulerprompt}
>function scene () ...
\end{eulerprompt}
\begin{eulerudf}
  global a,u,dd,g,g1,defaultpointsize;
  writeln(povsphere([0,0,u[1]],dd[1],povlook(red)));
  writeln(povsphere([0,0,u[2]],dd[2],povlook(red)));
  writeln(povcone([0,0,0],0,[0,0,a],1,povlook(lightgray,1)));
  gp=g();
  pc=povcone([0,0,0],0,[0,0,a],1,"");
  vp=[gp[1],0,gp[2]]; dp=gp[3];
  writeln(povplane(vp,dp,povlook(blue,0.5),pc));
  P1=projectToLine([0,u[1]],g1()); P1=turnz([P1[1],0,P1[2]]);
  writeln(povpoint(P1,povlook(yellow)));
  P2=projectToLine([0,u[2]],g1()); P2=turnz([P2[1],0,P2[2]]);
  writeln(povpoint(P2,povlook(yellow)));
  P3=projectToLine([0,u[1]],g()); P3=[P3[1],0,P3[2]];
  writeln(povpoint(P3,povlook(yellow)));
  P4=projectToLine([0,u[2]],g()); P4=[P4[1],0,P4[2]];
  writeln(povpoint(P4,povlook(yellow)));
  t1=scalp(vp,P1)-dp; t2=scalp(vp,P2)-dp; P5=P1+t1/(t1-t2)*(P2-P1);
  writeln(povpoint(P5,povlook(yellow)));
  writeln(povsegment(P1,P2,povlook(yellow)));
  writeln(povsegment(P5,P3,povlook(yellow)));
  writeln(povsegment(P5,P4,povlook(yellow)));
  pcw=povcone([0,0,0],0,[0,0,a],1.01);
  pc1=povcylinder([0,0,P1[3]-defaultpointsize/2],[0,0,P1[3]+defaultpointsize/2],1);
  writeln(povintersection([pcw,pc1],povlook(gray)));
  pc2=povcylinder([0,0,P2[3]-defaultpointsize/2],[0,0,P2[3]+defaultpointsize/2],1);
  writeln(povintersection([pcw,pc2],povlook(gray)));
  endfunction
\end{eulerudf}
\begin{eulercomment}
Anda memerlukan kacamata merah/sian untuk mengapresiasi efek berikut
ini.
\end{eulercomment}
\begin{eulerprompt}
>povanaglyph("scene",zoom=11,center=[0,0,0.5],height=10°,angle=140°);
\end{eulerprompt}
\eulerimg{27}{images/Pekan 11-12_Fanny Erina Dewi_22305141005_EMT00-Geometry_Aplikom-152.png}
\eulerheading{Contoh 8: Geometri Bumi}
\begin{eulercomment}
Di notebook ini, kami ingin melakukan beberapa komputasi bola.
Fungsi-fungsi tersebut terdapat dalam file "spherical.e" di folder
contoh. Kita perlu memuat file itu dulu.
\end{eulercomment}
\begin{eulerprompt}
>load "spherical.e";
\end{eulerprompt}
\begin{eulercomment}
Untuk memasukkan posisi geografis, kami menggunakan vektor dengan dua
koordinat dalam radian (utara dan timur, nilai negatif untuk selatan
dan barat). Berikut koordinat Kampus FMIPA UNY.
\end{eulercomment}
\begin{eulerprompt}
>FMIPA=[rad(-7,-46.467),rad(110,23.05)]
\end{eulerprompt}
\begin{euleroutput}
  [-0.13569,  1.92657]
\end{euleroutput}
\begin{eulercomment}
Anda dapat mencetak posisi ini dengan sposprint (cetak posisi bola).
\end{eulercomment}
\begin{eulerprompt}
>sposprint(FMIPA) // posisi garis lintang dan garis bujur FMIPA UNY
\end{eulerprompt}
\begin{euleroutput}
  S 7°46.467' E 110°23.050'
\end{euleroutput}
\begin{eulercomment}
Mari kita tambahkan dua kota lagi, Solo dan Semarang.
\end{eulercomment}
\begin{eulerprompt}
>Solo=[rad(-7,-34.333),rad(110,49.683)]; Semarang=[rad(-6,-59.05),rad(110,24.533)];
>sposprint(Solo), sposprint(Semarang),
\end{eulerprompt}
\begin{euleroutput}
  S 7°34.333' E 110°49.683'
  S 6°59.050' E 110°24.533'
\end{euleroutput}
\begin{eulercomment}
Pertama kita menghitung vektor dari satu bola ke bola lainnya pada
bola ideal. Vektor ini adalah [heading, distance] dalam radian. Untuk
menghitung jarak di bumi, kita mengalikan dengan jari-jari bumi pada
garis lintang 7°.
\end{eulercomment}
\begin{eulerprompt}
>br=svector(FMIPA,Solo); degprint(br[1]), br[2]*rearth(7°)->km // perkiraan jarak FMIPA-Solo
\end{eulerprompt}
\begin{euleroutput}
  65°20'26.60''
  53.8945384608
\end{euleroutput}
\begin{eulercomment}
Ini adalah perkiraan yang bagus. Rutinitas berikut menggunakan
perkiraan yang lebih baik. Pada jarak yang begitu dekat hasilnya
hampir sama.
\end{eulercomment}
\begin{eulerprompt}
>esdist(FMIPA,Semarang)->" km" // perkiraan jarak FMIPA-Semarang
\end{eulerprompt}
\begin{euleroutput}
  Commands must be separated by semicolon or comma!
  Found:  // perkiraan jarak FMIPA-Semarang (character 32)
  You can disable this in the Options menu.
  Error in:
  esdist(FMIPA,Semarang)->" km" // perkiraan jarak FMIPA-Semaran ...
                               ^
\end{euleroutput}
\begin{eulercomment}
Ada fungsi untuk heading, dengan mempertimbangkan bentuk bumi yang
elips. Sekali lagi, kami mencetak dengan cara yang canggih.
\end{eulercomment}
\begin{eulerprompt}
>sdegprint(esdir(FMIPA,Solo))
\end{eulerprompt}
\begin{euleroutput}
       65.34°
\end{euleroutput}
\begin{eulercomment}
Sudut segitiga melebihi 180° pada bola.
\end{eulercomment}
\begin{eulerprompt}
>asum=sangle(Solo,FMIPA,Semarang)+sangle(FMIPA,Solo,Semarang)+sangle(FMIPA,Semarang,Solo); degprint(asum)
\end{eulerprompt}
\begin{euleroutput}
  180°0'10.77''
\end{euleroutput}
\begin{eulercomment}
Ini dapat digunakan untuk menghitung luas segitiga. Catatan: Untuk
segitiga kecil, ini tidak akurat karena kesalahan pengurangan dalam
asum-pi.
\end{eulercomment}
\begin{eulerprompt}
>(asum-pi)*rearth(48°)^2->" km^2" // perkiraan luas segitiga FMIPA-Solo-Semarang
\end{eulerprompt}
\begin{euleroutput}
  Commands must be separated by semicolon or comma!
  Found:  // perkiraan luas segitiga FMIPA-Solo-Semarang (character 32)
  You can disable this in the Options menu.
  Error in:
  (asum-pi)*rearth(48°)^2->" km^2" // perkiraan luas segitiga FM ...
                                  ^
\end{euleroutput}
\begin{eulercomment}
Ada fungsi untuk ini, yang menggunakan garis lintang rata-rata
segitiga untuk menghitung jari-jari bumi, dan menangani kesalahan
pembulatan untuk segitiga yang sangat kecil.
\end{eulercomment}
\begin{eulerprompt}
>esarea(Solo,FMIPA,Semarang)->" km^2", //perkiraan yang sama dengan fungsi esarea()
\end{eulerprompt}
\begin{euleroutput}
  2123.64310526 km^2
\end{euleroutput}
\begin{eulercomment}
Kami juga dapat menambahkan vektor ke posisi. Vektor berisi heading
dan jarak, keduanya dalam radian. Untuk mendapatkan vektor, kami
menggunakan svector. Untuk menambahkan vektor ke posisi, kami
menggunakan saddvector.
\end{eulercomment}
\begin{eulerprompt}
>v=svector(FMIPA,Solo); sposprint(saddvector(FMIPA,v)), sposprint(Solo),
\end{eulerprompt}
\begin{euleroutput}
  S 7°34.333' E 110°49.683'
  S 7°34.333' E 110°49.683'
\end{euleroutput}
\begin{eulercomment}
Fungsi-fungsi ini mengasumsikan bola yang ideal. Hal yang sama di
bumi.
\end{eulercomment}
\begin{eulerprompt}
>sposprint(esadd(FMIPA,esdir(FMIPA,Solo),esdist(FMIPA,Solo))), sposprint(Solo),
\end{eulerprompt}
\begin{euleroutput}
  S 7°34.333' E 110°49.683'
  S 7°34.333' E 110°49.683'
\end{euleroutput}
\begin{eulercomment}
Mari kita beralih ke contoh yang lebih besar, Tugu Jogja dan Monas
Jakarta (menggunakan Google Earth untuk mencari koordinatnya).
\end{eulercomment}
\begin{eulerprompt}
>Tugu=[-7.7833°,110.3661°]; Monas=[-6.175°,106.811944°];
>sposprint(Tugu), sposprint(Monas)
\end{eulerprompt}
\begin{euleroutput}
  S 7°46.998' E 110°21.966'
  S 6°10.500' E 106°48.717'
\end{euleroutput}
\begin{eulercomment}
Menurut Google Earth, jaraknya 429,66 km. Kami mendapatkan perkiraan
yang bagus.
\end{eulercomment}
\begin{eulerprompt}
>esdist(Tugu,Monas)->" km" // perkiraan jarak Tugu Jogja - Monas Jakarta
\end{eulerprompt}
\begin{euleroutput}
  Commands must be separated by semicolon or comma!
  Found:  // perkiraan jarak Tugu Jogja - Monas Jakarta (character 32)
  You can disable this in the Options menu.
  Error in:
  esdist(Tugu,Monas)->" km" // perkiraan jarak Tugu Jogja - Mona ...
                           ^
\end{euleroutput}
\begin{eulercomment}
Judulnya sama dengan yang dihitung di Google Earth.
\end{eulercomment}
\begin{eulerprompt}
>degprint(esdir(Tugu,Monas))
\end{eulerprompt}
\begin{euleroutput}
  294°17'2.85''
\end{euleroutput}
\begin{eulercomment}
Namun, kita tidak lagi mendapatkan posisi target yang tepat, jika kita
menambahkan heading dan jarak ke posisi semula. Hal ini terjadi,
karena kita tidak menghitung fungsi invers secara tepat, tetapi
mengambil perkiraan jari-jari bumi di sepanjang jalan.
\end{eulercomment}
\begin{eulerprompt}
>sposprint(esadd(Tugu,esdir(Tugu,Monas),esdist(Tugu,Monas)))
\end{eulerprompt}
\begin{euleroutput}
  S 6°10.500' E 106°48.717'
\end{euleroutput}
\begin{eulercomment}
Namun, kesalahannya tidak besar.
\end{eulercomment}
\begin{eulerprompt}
>sposprint(Monas),
\end{eulerprompt}
\begin{euleroutput}
  S 6°10.500' E 106°48.717'
\end{euleroutput}
\begin{eulercomment}
Tentunya kita tidak bisa berlayar dengan tujuan yang sama dari satu
tujuan ke tujuan lainnya, jika kita ingin mengambil jalur terpendek.
Bayangkan, Anda terbang NE mulai dari titik mana pun di bumi. Kemudian
Anda akan berputar ke kutub utara. Lingkaran besar tidak mengikuti
arah yang konstan!

Perhitungan berikut menunjukkan bahwa kami jauh dari tujuan yang
benar, jika kami menggunakan tajuk yang sama selama perjalanan kami.
\end{eulercomment}
\begin{eulerprompt}
>dist=esdist(Tugu,Monas); hd=esdir(Tugu,Monas);
\end{eulerprompt}
\begin{eulercomment}
Sekarang kita tambahkan 10 kali sepersepuluh jaraknya, menggunakan
heading ke Monas, kita sampai di Tugu.
\end{eulercomment}
\begin{eulerprompt}
>p=Tugu; loop 1 to 10; p=esadd(p,hd,dist/10); end;
\end{eulerprompt}
\begin{eulercomment}
Hasilnya masih jauh.
\end{eulercomment}
\begin{eulerprompt}
>sposprint(p), skmprint(esdist(p,Monas))
\end{eulerprompt}
\begin{euleroutput}
  S 6°11.250' E 106°48.372'
       1.529km
\end{euleroutput}
\begin{eulercomment}
Sebagai contoh lain, mari kita ambil dua titik di bumi pada ketinggian
yang sama.
\end{eulercomment}
\begin{eulerprompt}
>P1=[30°,10°]; P2=[30°,50°];
\end{eulerprompt}
\begin{eulercomment}
Jalur terpendek dari P1 ke P2 bukanlah lingkaran dengan garis lintang
30°, tetapi jalur yang lebih pendek mulai 10° lebih jauh ke utara di
P1.
\end{eulercomment}
\begin{eulerprompt}
>sdegprint(esdir(P1,P2))
\end{eulerprompt}
\begin{euleroutput}
       79.69°
\end{euleroutput}
\begin{eulercomment}
Tapi, jika kita mengikuti pembacaan kompas ini, kita akan berputar ke
kutub utara! Jadi kita harus menyesuaikan arah tujuan kita di
sepanjang jalan. Untuk tujuan kasar, kami menyesuaikannya pada 1/10
dari jarak total.
\end{eulercomment}
\begin{eulerprompt}
>p=P1;  dist=esdist(P1,P2); ...
>  loop 1 to 10; dir=esdir(p,P2); sdegprint(dir), p=esadd(p,dir,dist/10); end;
\end{eulerprompt}
\begin{euleroutput}
       79.69°
       81.67°
       83.71°
       85.78°
       87.89°
       90.00°
       92.12°
       94.22°
       96.29°
       98.33°
\end{euleroutput}
\begin{eulercomment}
Jaraknya tidak tepat, karena kita akan menambahkan sedikit kesalahan,
jika kita mengikuti tajuk yang sama terlalu lama.
\end{eulercomment}
\begin{eulerprompt}
>skmprint(esdist(p,P2))
\end{eulerprompt}
\begin{euleroutput}
       0.203km
\end{euleroutput}
\begin{eulercomment}
Kita akan mendapatkan perkiraan yang baik, jika kita menyesuaikan arah
setiap 1/100 dari total jarak dari Tugu ke Monas.
\end{eulercomment}
\begin{eulerprompt}
>p=Tugu; dist=esdist(Tugu,Monas); ...
>  loop 1 to 100; p=esadd(p,esdir(p,Monas),dist/100); end;
>skmprint(esdist(p,Monas))
\end{eulerprompt}
\begin{euleroutput}
       0.000km
\end{euleroutput}
\begin{eulercomment}
Untuk keperluan navigasi, kita bisa mendapatkan urutan posisi GPS di
sepanjang Bundaran Hotel Indonesia menuju Monas dengan fungsi
navigate.
\end{eulercomment}
\begin{eulerprompt}
>load spherical; v=navigate(Tugu,Monas,10); ...
>  loop 1 to rows(v); sposprint(v[#]), end;
\end{eulerprompt}
\begin{euleroutput}
  S 7°46.998' E 110°21.966'
  S 7°37.422' E 110°0.573'
  S 7°27.829' E 109°39.196'
  S 7°18.219' E 109°17.834'
  S 7°8.592' E 108°56.488'
  S 6°58.948' E 108°35.157'
  S 6°49.289' E 108°13.841'
  S 6°39.614' E 107°52.539'
  S 6°29.924' E 107°31.251'
  S 6°20.219' E 107°9.977'
  S 6°10.500' E 106°48.717'
\end{euleroutput}
\begin{eulercomment}
Kami menulis sebuah fungsi, yang memplot bumi, dua posisi, dan posisi
di antaranya.
\end{eulercomment}
\begin{eulerprompt}
>function testplot ...
\end{eulerprompt}
\begin{eulerudf}
  useglobal;
  plotearth;
  plotpos(Tugu,"Tugu Jogja"); plotpos(Monas,"Tugu Monas");
  plotposline(v);
  endfunction
\end{eulerudf}
\begin{eulercomment}
Sekarang plot semuanya.
\end{eulercomment}
\begin{eulerprompt}
>plot3d("testplot",angle=25, height=6,>own,>user,zoom=4):
\end{eulerprompt}
\eulerimg{27}{images/Pekan 11-12_Fanny Erina Dewi_22305141005_EMT00-Geometry_Aplikom-153.png}
\begin{eulercomment}
Atau gunakan plot3d untuk mendapatkan tampilan anaglyph. Ini terlihat
sangat bagus dengan kacamata merah/cyan.
\end{eulercomment}
\begin{eulerprompt}
>plot3d("testplot",angle=25,height=6,distance=5,own=1,anaglyph=1,zoom=4):
\end{eulerprompt}
\eulerimg{27}{images/Pekan 11-12_Fanny Erina Dewi_22305141005_EMT00-Geometry_Aplikom-154.png}
\eulerheading{Latihan}
\begin{eulercomment}
1. Gambarlah segi-n beraturan jika diketahui titik pusat O, n, dan
jarak titik pusat ke titik-titik sudut segi-n tersebut (jari-jari
lingkaran luar segi-n), r.

Petunjuk:

- Besar sudut pusat yang menghadap masing-masing sisi segi-n adalah
(360/n).\\
- Titik-titik sudut segi-n merupakan perpotongan lingkaran luar segi-n
dan garis-garis yang melalui pusat dan saling membentuk sudut sebesar
kelipatan (360/n).\\
- Untuk n ganjil, pilih salah satu titik sudut adalah di atas.\\
- Untuk n genap, pilih 2 titik di kanan dan kiri lurus dengan titik
pusat.\\
- Anda dapat menggambar segi-3, 4, 5, 6, 7, dst beraturan.

\end{eulercomment}
\eulersubheading{Jawab}
\eulersubheading{}
\begin{eulercomment}
\end{eulercomment}
\begin{eulerprompt}
>o &:= [0,0]; c=circleWithCenter(o,5);
>color(1); setPlotRange(5); plotPoint(o); plotCircle(c):
\end{eulerprompt}
\eulerimg{27}{images/Pekan 11-12_Fanny Erina Dewi_22305141005_EMT00-Geometry_Aplikom-155.png}
\begin{eulerprompt}
>A=[-5,0]; plotPoint(A,"A");
>B=[5,0]; plotPoint(B,"B");
>plotSegment(A,B,""):
\end{eulerprompt}
\eulerimg{27}{images/Pekan 11-12_Fanny Erina Dewi_22305141005_EMT00-Geometry_Aplikom-156.png}
\begin{eulerprompt}
>c1=circleWithCenter(A,distance(A,o));
>c2=circleWithCenter(B,distance(B,o));
>k=circleCircleIntersections(c1,c);
>l=circleCircleIntersections(c,c2);
>m=circleCircleIntersections(c2,c);
>n=circleCircleIntersections(c,c1);
>r=lineThrough(k,m); s=lineThrough(l,n);
>setPlotRange(8); plotPoint(o); plotCircle(c); plotPoint(A,"A"); plotPoint(B,"B"); plotSegment(A,B,"");
>color(4); plotCircle(c1); plotCircle(c2); plotPoint(k); plotPoint(l); plotPoint(m); plotPoint(n)
>color(5); plotLine(r); plotLine(s):
\end{eulerprompt}
\eulerimg{27}{images/Pekan 11-12_Fanny Erina Dewi_22305141005_EMT00-Geometry_Aplikom-157.png}
\begin{eulerprompt}
>color(6); plotSegment(A,k,""); plotSegment(A,n,""); plotSegment(k,l,""); ...
>plotSegment(l,B,""); plotSegment(B,m,""); plotSegment(m,n,""):
\end{eulerprompt}
\eulerimg{27}{images/Pekan 11-12_Fanny Erina Dewi_22305141005_EMT00-Geometry_Aplikom-158.png}
\begin{eulercomment}
2. Gambarlah suatu parabola yang melalui 3 titik yang diketahui.

Petunjuk:\\
- Misalkan persamaan parabolanya y= ax\textasciicircum{}2+bx+c.\\
- Substitusikan koordinat titik-titik yang diketahui ke persamaan
tersebut.\\
- Selesaikan SPL yang terbentuk untuk mendapatkan nilai-nilai a, b, c.

\end{eulercomment}
\begin{eulerprompt}
>setPlotRange(5); A=[1,0]; B=[4,0]; C=[0,-4];
>plotPoint(A,"A"); plotPoint(B,"B"); plotPoint(C,"C"):
\end{eulerprompt}
\eulerimg{27}{images/Pekan 11-12_Fanny Erina Dewi_22305141005_EMT00-Geometry_Aplikom-159.png}
\begin{eulerprompt}
>sol &= solve([a+b=-c,16*a+4*b=-c,c=-4],[a,b,c])
\end{eulerprompt}
\begin{euleroutput}
  
                       [[a = - 1, b = 5, c = - 4]]
  
\end{euleroutput}
\begin{eulerprompt}
>function y&=-x^2+5*x-4
\end{eulerprompt}
\begin{euleroutput}
  
                                 2
                              - x  + 5 x - 4
  
\end{euleroutput}
\begin{eulerprompt}
>plot2d("-x^2+5*x-4",-5,5,-5,5):
\end{eulerprompt}
\eulerimg{27}{images/Pekan 11-12_Fanny Erina Dewi_22305141005_EMT00-Geometry_Aplikom-160.png}
\begin{eulercomment}
3. Gambarlah suatu segi-4 yang diketahui keempat titik sudutnya,
misalnya A, B, C, D.\\
\end{eulercomment}
\begin{eulerttcomment}
   - Tentukan apakah segi-4 tersebut merupakan segi-4 garis singgung
\end{eulerttcomment}
\begin{eulercomment}
(sisinya-sisintya merupakan garis singgung lingkaran yang sama yakni
lingkaran dalam segi-4 tersebut).\\
\end{eulercomment}
\begin{eulerttcomment}
   - Suatu segi-4 merupakan segi-4 garis singgung apabila keempat
\end{eulerttcomment}
\begin{eulercomment}
garis bagi sudutnya bertemu di satu titik.\\
\end{eulercomment}
\begin{eulerttcomment}
   - Jika segi-4 tersebut merupakan segi-4 garis singgung, gambar
\end{eulerttcomment}
\begin{eulercomment}
lingkaran dalamnya.\\
\end{eulercomment}
\begin{eulerttcomment}
   - Tunjukkan bahwa syarat suatu segi-4 merupakan segi-4 garis
\end{eulerttcomment}
\begin{eulercomment}
singgung apabila hasil kali panjang sisi-sisi yang berhadapan sama.

\end{eulercomment}
\begin{eulerprompt}
>setPlotRange(-5,5,-5,5);
>A=[-2,2]; plotPoint(A,"A");
>B=[2,2]; plotPoint(B,"B");
>C=[2,-2]; plotPoint(C,"C");
>D=[-2,-2]; plotPoint(D,"D"):
\end{eulerprompt}
\eulerimg{27}{images/Pekan 11-12_Fanny Erina Dewi_22305141005_EMT00-Geometry_Aplikom-161.png}
\begin{eulerprompt}
>plotSegment(A,B);
>plotSegment(B,C);
>plotSegment(C,D);
>plotSegment(D,A):
\end{eulerprompt}
\eulerimg{27}{images/Pekan 11-12_Fanny Erina Dewi_22305141005_EMT00-Geometry_Aplikom-162.png}
\begin{eulerprompt}
>plotSegment(A,C,"q1"):
\end{eulerprompt}
\eulerimg{27}{images/Pekan 11-12_Fanny Erina Dewi_22305141005_EMT00-Geometry_Aplikom-163.png}
\begin{eulerprompt}
>plotSegment(B,D,"q2"):
\end{eulerprompt}
\eulerimg{27}{images/Pekan 11-12_Fanny Erina Dewi_22305141005_EMT00-Geometry_Aplikom-164.png}
\begin{eulerprompt}
>q1=lineThrough(A,C);
>q2=lineThrough(B,D);
>p=lineIntersection(q1,q2);
>plotLine(q1); plotLine(q2):
\end{eulerprompt}
\eulerimg{27}{images/Pekan 11-12_Fanny Erina Dewi_22305141005_EMT00-Geometry_Aplikom-165.png}
\begin{eulerprompt}
>plotPoint(p, "P");
>r=norm(p-projectToLine(p,lineThrough(A,B)))
\end{eulerprompt}
\begin{euleroutput}
  2
\end{euleroutput}
\begin{eulerprompt}
>plotCircle(circleWithCenter(p,r),"lingkaran dalam segi-4 ABCD"):
\end{eulerprompt}
\eulerimg{27}{images/Pekan 11-12_Fanny Erina Dewi_22305141005_EMT00-Geometry_Aplikom-166.png}
\begin{eulerprompt}
>AB=norm(A-B) // panjang sisi AB
\end{eulerprompt}
\begin{euleroutput}
  4
\end{euleroutput}
\begin{eulerprompt}
>CD=norm(C-D) // panjang sisi CD
\end{eulerprompt}
\begin{euleroutput}
  4
\end{euleroutput}
\begin{eulerprompt}
>AD=norm(A-D) // panjang sisi AD
\end{eulerprompt}
\begin{euleroutput}
  4
\end{euleroutput}
\begin{eulerprompt}
>BC=norm(B-C) // panjang sisi BC
\end{eulerprompt}
\begin{euleroutput}
  4
\end{euleroutput}
\begin{eulerprompt}
>AB.CD
\end{eulerprompt}
\begin{euleroutput}
  16
\end{euleroutput}
\begin{eulerprompt}
>AD.BC
\end{eulerprompt}
\begin{euleroutput}
  16
\end{euleroutput}
\begin{eulercomment}
4. Gambarlah suatu ellips jika diketahui kedua titik fokusnya,
misalnya P dan Q. Ingat ellips dengan fokus P dan Q adalah tempat
kedudukan titik-titik yang jumlah jarak ke P dan ke Q selalu sama
(konstan).

\end{eulercomment}
\begin{eulerprompt}
>P=[-1,-1]; Q=[1,-1];
>function d1(x,y):=sqrt((x-P[1])^2+(y-P[2])^2)
>Q=[1,-1]; function d2(x,y):=sqrt((x-P[1])^2+(y-P[2])^2)+sqrt((x-Q[1])^2+(y-Q[2])^2)
>fcontour("d2",xmin=-2,xmax=2,ymin=-3,ymax=1,hue=1):
\end{eulerprompt}
\eulerimg{27}{images/Pekan 11-12_Fanny Erina Dewi_22305141005_EMT00-Geometry_Aplikom-167.png}
\begin{eulerprompt}
>reset
\end{eulerprompt}
\begin{euleroutput}
  0
\end{euleroutput}
\begin{eulerprompt}
>plot3d("d2",xmin=-2,xmax=2,ymin=-3,ymax=1):
\end{eulerprompt}
\eulerimg{27}{images/Pekan 11-12_Fanny Erina Dewi_22305141005_EMT00-Geometry_Aplikom-168.png}
\begin{eulerprompt}
>reset
\end{eulerprompt}
\begin{euleroutput}
  0
\end{euleroutput}
\begin{eulercomment}
5. Gambarlah suatu hiperbola jika diketahui kedua titik fokusnya,
misalnya P dan Q. Ingat ellips dengan fokus P dan Q adalah tempat
kedudukan titik-titik yang selisih jarak ke P dan ke Q selalu sama
(konstan).

\end{eulercomment}
\begin{eulerprompt}
>P=[-1,-1]; Q=[1,-1];
>function d1(x,y):=sqrt((x-p[1])^2+(y-p[2])^2)
>Q=[1,-1]; function d2(x,y):=sqrt((x-P[1])^2+(y-P[2])^2)+sqrt((x+Q[1])^2+(y+Q[2])^2)
>fcontour("d2",xmin=-2,xmax=2,ymin=-3,ymax=1,hue=1):
\end{eulerprompt}
\eulerimg{27}{images/Pekan 11-12_Fanny Erina Dewi_22305141005_EMT00-Geometry_Aplikom-169.png}
\begin{eulerprompt}
>reset
\end{eulerprompt}
\begin{euleroutput}
  0
\end{euleroutput}
\begin{eulerprompt}
>plot3d("d2",xmin=-2,xmax=2,ymin=-3,ymax=1):
\end{eulerprompt}
\eulerimg{27}{images/Pekan 11-12_Fanny Erina Dewi_22305141005_EMT00-Geometry_Aplikom-170.png}
\begin{eulerprompt}
>reset
\end{eulerprompt}
\begin{euleroutput}
  0
\end{euleroutput}
\begin{eulerprompt}
>plot2d("abs(x+1)+abs(x-1)",xmin=-3,xmax=3):
\end{eulerprompt}
\eulerimg{27}{images/Pekan 11-12_Fanny Erina Dewi_22305141005_EMT00-Geometry_Aplikom-171.png}
\begin{eulerprompt}
>reset
\end{eulerprompt}
\begin{euleroutput}
  0
\end{euleroutput}
\end{eulernotebook}

\chapter{EMT Statistika}
\begin{eulercomment}
\eulerheading{EMT Statistika}
\begin{eulercomment}
Dalam buku catatan ini, kami mendemonstrasikan plot statistik utama,
tes, dan distribusi dalam Euler.

Mari kita mulai dengan beberapa statistik deskriptif. Ini bukanlah
sebuah pengantar statistik. Jadi, Anda mungkin memerlukan latar
belakang untuk memahami detailnya.

Asumsikan pengukuran berikut. Kita ingin menghitung nilai rata-rata
dan deviasi standar yang diukur.
\end{eulercomment}
\begin{eulerprompt}
>M=[1000,1004,998,997,1002,1001,998,1004,998,997]; ...
>median(M), mean(M), dev(M),
\end{eulerprompt}
\begin{euleroutput}
  999
  999.9
  2.72641400622
\end{euleroutput}
\begin{eulercomment}
Kita dapat memplot plot kotak dan kumis untuk data tersebut. Dalam
kasus kami, tidak ada pencilan.
\end{eulercomment}
\begin{eulerprompt}
>aspect(1.75); boxplot(M):
\end{eulerprompt}
\eulerimg{15}{images/Pekan 13-14_Fanny Erina Dewi_22305141005_EMT00-Statistika_Aplikom-001.png}
\begin{eulercomment}
Kami menghitung probabilitas bahwa suatu nilai lebih besar dari 1005,
dengan mengasumsikan nilai yang diukur dari distribusi normal.

Semua fungsi untuk distribusi dalam Euler diakhiri dengan ...dis dan
menghitung distribusi probabilitas kumulatif (CPF).

\end{eulercomment}
\begin{eulerformula}
\[
\text{normaldis(x,m,d)}=\int_{-\infty}^x \frac{1}{d\sqrt{2\pi}}e^{-\frac{1}{2}(\frac{t-m}{d})^2}\ dt.
\]
\end{eulerformula}
\begin{eulercomment}
Kami mencetak hasilnya dalam \% dengan akurasi 2 digit menggunakan
fungsi cetak.
\end{eulercomment}
\begin{eulerprompt}
>print((1-normaldis(1005,mean(M),dev(M)))*100,2,unit=" %")
\end{eulerprompt}
\begin{euleroutput}
        3.07 %
\end{euleroutput}
\begin{eulercomment}
Untuk contoh berikutnya, kami mengasumsikan jumlah pria berikut ini
dalam rentang ukuran tertentu.
\end{eulercomment}
\begin{eulerprompt}
>r=155.5:4:187.5; v=[22,71,136,169,139,71,32,8];
\end{eulerprompt}
\begin{eulercomment}
Berikut ini adalah plot distribusinya.
\end{eulercomment}
\begin{eulerprompt}
>plot2d(r,v,a=150,b=200,c=0,d=190,bar=1,style="\(\backslash\)/"):
\end{eulerprompt}
\eulerimg{15}{images/Pekan 13-14_Fanny Erina Dewi_22305141005_EMT00-Statistika_Aplikom-002.png}
\begin{eulercomment}
Kita dapat memasukkan data mentah tersebut ke dalam tabel.

Tabel adalah sebuah metode untuk menyimpan data statistik. Tabel kita
harus berisi tiga kolom: Awal rentang, akhir rentang, jumlah orang
dalam rentang.

Tabel dapat dicetak dengan header. Kami menggunakan vektor string
untuk mengatur header.
\end{eulercomment}
\begin{eulerprompt}
>T:=r[1:8]' | r[2:9]' | v'; writetable(T,labc=["BB","BA","Frek"])
\end{eulerprompt}
\begin{euleroutput}
          BB        BA      Frek
       155.5     159.5        22
       159.5     163.5        71
       163.5     167.5       136
       167.5     171.5       169
       171.5     175.5       139
       175.5     179.5        71
       179.5     183.5        32
       183.5     187.5         8
\end{euleroutput}
\begin{eulercomment}
Jika kita membutuhkan nilai rata-rata dan statistik lain dari ukuran,
kita perlu menghitung titik tengah rentang. Kita dapat menggunakan dua
kolom pertama dari tabel kita untuk hal ini.

Simbol "\textbar{}" digunakan untuk memisahkan kolom, fungsi "writetable"
digunakan untuk menulis tabel, dengan opsi "labc" untuk menentukan
judul kolom.
\end{eulercomment}
\begin{eulerprompt}
>(T[,1]+T[,2])/2 // the midpoint of each interval
\end{eulerprompt}
\begin{euleroutput}
          157.5 
          161.5 
          165.5 
          169.5 
          173.5 
          177.5 
          181.5 
          185.5 
\end{euleroutput}
\begin{eulercomment}
Tetapi akan lebih mudah, untuk melipat rentang dengan vektor
[1/2,1/2].
\end{eulercomment}
\begin{eulerprompt}
>M=fold(r,[0.5,0.5])
\end{eulerprompt}
\begin{euleroutput}
  [157.5,  161.5,  165.5,  169.5,  173.5,  177.5,  181.5,  185.5]
\end{euleroutput}
\begin{eulercomment}
Sekarang kita dapat menghitung rata-rata dan deviasi sampel dengan
frekuensi yang diberikan.
\end{eulercomment}
\begin{eulerprompt}
>\{m,d\}=meandev(M,v); m, d,
\end{eulerprompt}
\begin{euleroutput}
  169.901234568
  5.98912964449
\end{euleroutput}
\begin{eulercomment}
Mari kita tambahkan distribusi normal dari nilai-nilai tersebut ke
dalam diagram batang di atas. Rumus untuk distribusi normal dengan
rata-rata m dan deviasi standar d adalah:

\end{eulercomment}
\begin{eulerformula}
\[
y=\frac{1}{d\sqrt{2\pi}}e^{\frac{-(x-m)^2}{2d^2}}.
\]
\end{eulerformula}
\begin{eulercomment}
Karena nilainya antara 0 dan 1, untuk memplotnya pada diagram batang,
nilai tersebut harus dikalikan dengan 4 kali jumlah data.
\end{eulercomment}
\begin{eulerprompt}
>plot2d("qnormal(x,m,d)*sum(v)*4", ...
>  xmin=min(r),xmax=max(r),thickness=3,add=1):
\end{eulerprompt}
\eulerimg{15}{images/Pekan 13-14_Fanny Erina Dewi_22305141005_EMT00-Statistika_Aplikom-003.png}
\eulerheading{Tables}
\begin{eulercomment}
Dalam direktori buku catatan ini, Anda akan menemukan file dengan
tabel. Data tersebut merupakan hasil survei. Berikut adalah empat
baris pertama dari file tersebut. Data berasal dari sebuah buku online
berbahasa Jerman "Einführung in die Statistik mit R" oleh A. Handl.
\end{eulercomment}
\begin{eulerprompt}
>printfile("table.dat",4);
\end{eulerprompt}
\begin{euleroutput}
  Person Sex Age Titanic Evaluation Tip Problem
  1 m 30 n . 1.80 n
  2 f 23 y g 1.80 n
  3 f 26 y g 1.80 y
\end{euleroutput}
\begin{eulercomment}
Tabel berisi 7 kolom angka atau token (string). Kita ingin membaca
tabel tersebut dari file. Pertama, kita menggunakan terjemahan kita
sendiri untuk token-token tersebut.

Untuk itu, kita mendefinisikan set token. Fungsi strtokens()
mendapatkan vektor string token dari string yang diberikan.
\end{eulercomment}
\begin{eulerprompt}
>mf:=["m","f"]; yn:=["y","n"]; ev:=strtokens("g vg m b vb");
\end{eulerprompt}
\begin{eulercomment}
Sekarang kita membaca tabel dengan terjemahan ini.

Argumen tok2, tok4, dan lain-lain adalah terjemahan dari kolom-kolom
tabel. Argumen-argumen ini tidak ada dalam daftar parameter
readtable(), jadi Anda harus menyediakannya dengan ":=".
\end{eulercomment}
\begin{eulerprompt}
>\{MT,hd\}=readtable("table.dat",tok2:=mf,tok4:=yn,tok5:=ev,tok7:=yn);
>load over statistics;
\end{eulerprompt}
\begin{eulercomment}
Untuk mencetak, kita perlu menentukan set token yang sama. Kami
mencetak empat baris pertama saja.
\end{eulercomment}
\begin{eulerprompt}
>writetable(MT[1:10],labc=hd,wc=5,tok2:=mf,tok4:=yn,tok5:=ev,tok7:=yn);
\end{eulerprompt}
\begin{euleroutput}
   Person  Sex  Age Titanic Evaluation  Tip Problem
        1    m   30       n          .  1.8       n
        2    f   23       y          g  1.8       n
        3    f   26       y          g  1.8       y
        4    m   33       n          .  2.8       n
        5    m   37       n          .  1.8       n
        6    m   28       y          g  2.8       y
        7    f   31       y         vg  2.8       n
        8    m   23       n          .  0.8       n
        9    f   24       y         vg  1.8       y
       10    m   26       n          .  1.8       n
\end{euleroutput}
\begin{eulercomment}
Tanda titik "." mewakili nilai yang tidak tersedia.

Jika kita tidak ingin menentukan token untuk terjemahan sebelumnya,
kita hanya perlu menentukan kolom mana yang berisi token dan bukan
angka.
\end{eulercomment}
\begin{eulerprompt}
>ctok=[2,4,5,7]; \{MT,hd,tok\}=readtable("table.dat",ctok=ctok);
\end{eulerprompt}
\begin{eulercomment}
Fungsi readtable() sekarang mengembalikan satu set token.
\end{eulercomment}
\begin{eulerprompt}
>tok
\end{eulerprompt}
\begin{euleroutput}
  m
  n
  f
  y
  g
  vg
\end{euleroutput}
\begin{eulercomment}
Tabel berisi entri dari file dengan token yang diterjemahkan menjadi
angka.

String khusus NA="." ditafsirkan sebagai "Tidak Tersedia", dan
mendapatkan NAN (bukan angka) dalam tabel. Terjemahan ini dapat diubah
dengan parameter NA, dan NAval.
\end{eulercomment}
\begin{eulerprompt}
>MT[1]
\end{eulerprompt}
\begin{euleroutput}
  [1,  1,  30,  2,  NAN,  1.8,  2]
\end{euleroutput}
\begin{eulercomment}
Berikut ini adalah isi tabel dengan angka yang tidak diterjemahkan.
\end{eulercomment}
\begin{eulerprompt}
>writetable(MT,wc=5)
\end{eulerprompt}
\begin{euleroutput}
      1    1   30    2    .  1.8    2
      2    3   23    4    5  1.8    2
      3    3   26    4    5  1.8    4
      4    1   33    2    .  2.8    2
      5    1   37    2    .  1.8    2
      6    1   28    4    5  2.8    4
      7    3   31    4    6  2.8    2
      8    1   23    2    .  0.8    2
      9    3   24    4    6  1.8    4
     10    1   26    2    .  1.8    2
     11    3   23    4    6  1.8    4
     12    1   32    4    5  1.8    2
     13    1   29    4    6  1.8    4
     14    3   25    4    5  1.8    4
     15    3   31    4    5  0.8    2
     16    1   26    4    5  2.8    2
     17    1   37    2    .  3.8    2
     18    1   38    4    5    .    2
     19    3   29    2    .  3.8    2
     20    3   28    4    6  1.8    2
     21    3   28    4    1  2.8    4
     22    3   28    4    6  1.8    4
     23    3   38    4    5  2.8    2
     24    3   27    4    1  1.8    4
     25    1   27    2    .  2.8    4
\end{euleroutput}
\begin{eulercomment}
Untuk kenyamanan, Anda dapat menaruh output dari readtable() ke dalam
sebuah daftar.
\end{eulercomment}
\begin{eulerprompt}
>Table=\{\{readtable("table.dat",ctok=ctok)\}\};
\end{eulerprompt}
\begin{eulercomment}
Dengan menggunakan kolom token yang sama dan token yang dibaca dari
file, kita dapat mencetak tabel. Kita dapat menentukan ctok, tok, dll.
atau menggunakan daftar Tabel.
\end{eulercomment}
\begin{eulerprompt}
>writetable(Table,ctok=ctok,wc=5);
\end{eulerprompt}
\begin{euleroutput}
   Person  Sex  Age Titanic Evaluation  Tip Problem
        1    m   30       n          .  1.8       n
        2    f   23       y          g  1.8       n
        3    f   26       y          g  1.8       y
        4    m   33       n          .  2.8       n
        5    m   37       n          .  1.8       n
        6    m   28       y          g  2.8       y
        7    f   31       y         vg  2.8       n
        8    m   23       n          .  0.8       n
        9    f   24       y         vg  1.8       y
       10    m   26       n          .  1.8       n
       11    f   23       y         vg  1.8       y
       12    m   32       y          g  1.8       n
       13    m   29       y         vg  1.8       y
       14    f   25       y          g  1.8       y
       15    f   31       y          g  0.8       n
       16    m   26       y          g  2.8       n
       17    m   37       n          .  3.8       n
       18    m   38       y          g    .       n
       19    f   29       n          .  3.8       n
       20    f   28       y         vg  1.8       n
       21    f   28       y          m  2.8       y
       22    f   28       y         vg  1.8       y
       23    f   38       y          g  2.8       n
       24    f   27       y          m  1.8       y
       25    m   27       n          .  2.8       y
\end{euleroutput}
\begin{eulercomment}
Fungsi tablecol() mengembalikan nilai kolom dari tabel, melewatkan
setiap baris dengan nilai NAN ("." dalam file), dan indeks kolom, yang
berisi nilai-nilai ini.
\end{eulercomment}
\begin{eulerprompt}
>\{c,i\}=tablecol(MT,[5,6]);
\end{eulerprompt}
\begin{eulercomment}
Kita dapat menggunakan ini untuk mengekstrak kolom dari tabel untuk
tabel baru.
\end{eulercomment}
\begin{eulerprompt}
>j=[1,5,6]; writetable(MT[i,j],labc=hd[j],ctok=[2],tok=tok)
\end{eulerprompt}
\begin{euleroutput}
      Person Evaluation       Tip
           2          g       1.8
           3          g       1.8
           6          g       2.8
           7         vg       2.8
           9         vg       1.8
          11         vg       1.8
          12          g       1.8
          13         vg       1.8
          14          g       1.8
          15          g       0.8
          16          g       2.8
          20         vg       1.8
          21          m       2.8
          22         vg       1.8
          23          g       2.8
          24          m       1.8
\end{euleroutput}
\begin{eulercomment}
Tentu saja, kita perlu mengekstrak tabel itu sendiri dari daftar Tabel
dalam kasus ini.
\end{eulercomment}
\begin{eulerprompt}
>MT=Table[1];
\end{eulerprompt}
\begin{eulercomment}
Tentu saja, kita juga dapat menggunakannya untuk menentukan nilai
rata-rata kolom atau nilai statistik lainnya.
\end{eulercomment}
\begin{eulerprompt}
>mean(tablecol(MT,6))
\end{eulerprompt}
\begin{euleroutput}
  2.175
\end{euleroutput}
\begin{eulercomment}
Fungsi getstatistics() mengembalikan elemen-elemen dalam sebuah
vektor, dan jumlahnya. Kita menerapkannya pada nilai "m" dan "f" pada
kolom kedua tabel kita.
\end{eulercomment}
\begin{eulerprompt}
>\{xu,count\}=getstatistics(tablecol(MT,2)); xu, count,
\end{eulerprompt}
\begin{euleroutput}
  [1,  3]
  [12,  13]
\end{euleroutput}
\begin{eulercomment}
Kita bisa mencetak hasilnya dalam tabel baru.
\end{eulercomment}
\begin{eulerprompt}
>writetable(count',labr=tok[xu])
\end{eulerprompt}
\begin{euleroutput}
           m        12
           f        13
\end{euleroutput}
\begin{eulercomment}
Fungsi selecttable() mengembalikan sebuah tabel baru dengan nilai
dalam satu kolom yang dipilih dari vektor indeks. Pertama, kita
mencari indeks dari dua nilai kita dalam tabel token.
\end{eulercomment}
\begin{eulerprompt}
>v:=indexof(tok,["g","vg"])
\end{eulerprompt}
\begin{euleroutput}
  [5,  6]
\end{euleroutput}
\begin{eulercomment}
Sekarang kita dapat memilih baris-baris dari tabel, yang memiliki
salah satu nilai dalam v di baris ke-5.
\end{eulercomment}
\begin{eulerprompt}
>MT1:=MT[selectrows(MT,5,v)]; i:=sortedrows(MT1,5);
\end{eulerprompt}
\begin{eulercomment}
Sekarang kita dapat mencetak tabel, dengan nilai yang diekstrak dan
diurutkan di kolom ke-5.
\end{eulercomment}
\begin{eulerprompt}
>writetable(MT1[i],labc=hd,ctok=ctok,tok=tok,wc=7);
\end{eulerprompt}
\begin{euleroutput}
   Person    Sex    Age Titanic Evaluation    Tip Problem
        2      f     23       y          g    1.8       n
        3      f     26       y          g    1.8       y
        6      m     28       y          g    2.8       y
       18      m     38       y          g      .       n
       16      m     26       y          g    2.8       n
       15      f     31       y          g    0.8       n
       12      m     32       y          g    1.8       n
       23      f     38       y          g    2.8       n
       14      f     25       y          g    1.8       y
        9      f     24       y         vg    1.8       y
        7      f     31       y         vg    2.8       n
       20      f     28       y         vg    1.8       n
       22      f     28       y         vg    1.8       y
       13      m     29       y         vg    1.8       y
       11      f     23       y         vg    1.8       y
\end{euleroutput}
\begin{eulercomment}
Untuk statistik berikutnya, kita ingin menghubungkan dua kolom tabel.
Jadi kita mengekstrak kolom 2 dan 4 dan mengurutkan tabel.
\end{eulercomment}
\begin{eulerprompt}
>i=sortedrows(MT,[2,4]);  ...
>  writetable(tablecol(MT[i],[2,4])',ctok=[1,2],tok=tok)
\end{eulerprompt}
\begin{euleroutput}
           m         n
           m         n
           m         n
           m         n
           m         n
           m         n
           m         n
           m         y
           m         y
           m         y
           m         y
           m         y
           f         n
           f         y
           f         y
           f         y
           f         y
           f         y
           f         y
           f         y
           f         y
           f         y
           f         y
           f         y
           f         y
\end{euleroutput}
\begin{eulercomment}
Dengan getstatistics(), kita juga dapat menghubungkan hitungan dalam
dua kolom tabel satu sama lain.
\end{eulercomment}
\begin{eulerprompt}
>MT24=tablecol(MT,[2,4]); ...
>\{xu1,xu2,count\}=getstatistics(MT24[1],MT24[2]); ...
>writetable(count,labr=tok[xu1],labc=tok[xu2])
\end{eulerprompt}
\begin{euleroutput}
                     n         y
           m         7         5
           f         1        12
\end{euleroutput}
\begin{eulercomment}
Tabel dapat ditulis ke sebuah file.
\end{eulercomment}
\begin{eulerprompt}
>filename="test.dat"; ...
>writetable(count,labr=tok[xu1],labc=tok[xu2],file=filename);
\end{eulerprompt}
\begin{eulercomment}
Kemudian kita dapat membaca tabel dari file tersebut.
\end{eulercomment}
\begin{eulerprompt}
>\{MT2,hd,tok2,hdr\}=readtable(filename,>clabs,>rlabs); ...
>writetable(MT2,labr=hdr,labc=hd)
\end{eulerprompt}
\begin{euleroutput}
                     n         y
           m         7         5
           f         1        12
\end{euleroutput}
\begin{eulercomment}
dan menghapus file tersebut
\end{eulercomment}
\begin{eulerprompt}
>fileremove(filename);
\end{eulerprompt}
\eulerheading{Distribusi}
\begin{eulercomment}
Dengan plot2d, ada metode yang sangat mudah untuk memplot distribusi
data eksperimen.
\end{eulercomment}
\begin{eulerprompt}
>p=normal(1,1000); //1000 random normal-distributed sample p
>plot2d(p,distribution=20,style="\(\backslash\)/"); // plot the random sample p
>plot2d("qnormal(x,0,1)",add=1): // add the standard normal distribution plot
\end{eulerprompt}
\eulerimg{15}{images/Pekan 13-14_Fanny Erina Dewi_22305141005_EMT00-Statistika_Aplikom-004.png}
\begin{eulercomment}
Perhatikan perbedaan antara plot batang (sampel) dan kurva normal
(distribusi sesungguhnya). Masukkan kembali ketiga perintah tersebut
untuk melihat hasil pengambilan sampel yang lain.
\end{eulercomment}
\begin{eulercomment}
Berikut ini adalah perbandingan 10 simulasi dari 1000 nilai
terdistribusi normal dengan menggunakan apa yang disebut plot kotak.
Plot ini menunjukkan median, kuartil 25\% dan 75\%, nilai minimal dan
maksimal, serta pencilan.
\end{eulercomment}
\begin{eulerprompt}
>p=normal(10,1000); boxplot(p):
\end{eulerprompt}
\eulerimg{15}{images/Pekan 13-14_Fanny Erina Dewi_22305141005_EMT00-Statistika_Aplikom-005.png}
\begin{eulercomment}
Untuk menghasilkan bilangan bulat acak, Euler memiliki intrandom. Mari
kita simulasikan pelemparan dadu dan memplot distribusinya.

Kita menggunakan fungsi getmultiplicities(v,x), yang menghitung
seberapa sering elemen-elemen dari v muncul di dalam x. Kemudian kita
memplot hasilnya menggunakan columnsplot().
\end{eulercomment}
\begin{eulerprompt}
>k=intrandom(1,6000,6);  ...
>columnsplot(getmultiplicities(1:6,k));  ...
>ygrid(1000,color=red):
\end{eulerprompt}
\eulerimg{15}{images/Pekan 13-14_Fanny Erina Dewi_22305141005_EMT00-Statistika_Aplikom-006.png}
\begin{eulercomment}
Meskipun intrandom(n,m,k) menghasilkan bilangan bulat yang
terdistribusi secara seragam dari 1 sampai k, adalah mungkin untuk
menggunakan distribusi bilangan bulat yang lain dengan randpint().

Pada contoh berikut, probabilitas untuk 1,2,3 adalah 0.4, 0.1, 0.5
secara berurutan.
\end{eulercomment}
\begin{eulerprompt}
>randpint(1,1000,[0.4,0.1,0.5]); getmultiplicities(1:3,%)
\end{eulerprompt}
\begin{euleroutput}
  [378,  102,  520]
\end{euleroutput}
\begin{eulercomment}
Euler dapat menghasilkan nilai acak dari lebih banyak distribusi.
Lihatlah ke dalam referensi.

Misalnya, kita mencoba distribusi eksponensial. Sebuah variabel acak
kontinu X dikatakan memiliki distribusi eksponensial, jika PDF-nya
diberikan oleh\\
\end{eulercomment}
\begin{eulerformula}
\[
f_X(x)=\lambda e^{-\lambda x},\quad x>0,\quad \lambda>0,
\]
\end{eulerformula}
\begin{eulercomment}
dengan parameter\\
\end{eulercomment}
\begin{eulerformula}
\[
\lambda=\frac{1}{\mu},\quad \mu \text{ is the mean, and denoted by } X \sim \text{Exponential}(\lambda).
\]
\end{eulerformula}
\begin{eulerprompt}
>plot2d(randexponential(1,1000,2),>distribution):
\end{eulerprompt}
\eulerimg{15}{images/Pekan 13-14_Fanny Erina Dewi_22305141005_EMT00-Statistika_Aplikom-007.png}
\begin{eulercomment}
Untuk banyak distribusi, Euler dapat menghitung fungsi distribusi dan
kebalikannya.
\end{eulercomment}
\begin{eulerprompt}
>plot2d("normaldis",-4,4): 
\end{eulerprompt}
\eulerimg{15}{images/Pekan 13-14_Fanny Erina Dewi_22305141005_EMT00-Statistika_Aplikom-008.png}
\begin{eulercomment}
Berikut ini adalah salah satu cara untuk memplot kuantil.
\end{eulercomment}
\begin{eulerprompt}
>plot2d("qnormal(x,1,1.5)",-4,6);  ...
>plot2d("qnormal(x,1,1.5)",a=2,b=5,>add,>filled):
\end{eulerprompt}
\eulerimg{15}{images/Pekan 13-14_Fanny Erina Dewi_22305141005_EMT00-Statistika_Aplikom-009.png}
\begin{eulerformula}
\[
\text{normaldis(x,m,d)}=\int_{-\infty}^x \frac{1}{d\sqrt{2\pi}}e^{-\frac{1}{2}(\frac{t-m}{d})^2}\ dt.
\]
\end{eulerformula}
\begin{eulercomment}
Probabilitas untuk berada di area hijau adalah sebagai berikut.
\end{eulercomment}
\begin{eulerprompt}
>normaldis(5,1,1.5)-normaldis(2,1,1.5)
\end{eulerprompt}
\begin{euleroutput}
  0.248662156979
\end{euleroutput}
\begin{eulercomment}
Hal ini dapat dihitung secara numerik dengan integral berikut ini.\\
\end{eulercomment}
\begin{eulerformula}
\[
\int_2^5 \frac{1}{1.5\sqrt{2\pi}}e^{-\frac{1}{2}(\frac{x-1}{1.5})^2}\ dx.
\]
\end{eulerformula}
\begin{eulerprompt}
>gauss("qnormal(x,1,1.5)",2,5)
\end{eulerprompt}
\begin{euleroutput}
  0.248662156979
\end{euleroutput}
\begin{eulercomment}
Mari kita bandingkan distribusi binomial dengan distribusi normal
dengan rata-rata dan deviasi yang sama. Fungsi invbindis()
menyelesaikan interpolasi linier antara nilai bilangan bulat.
\end{eulercomment}
\begin{eulerprompt}
>invbindis(0.95,1000,0.5), invnormaldis(0.95,500,0.5*sqrt(1000))
\end{eulerprompt}
\begin{euleroutput}
  525.516721219
  526.007419394
\end{euleroutput}
\begin{eulercomment}
Fungsi qdis() adalah densitas dari distribusi chi-square. Seperti
biasa, Euler memetakan vektor ke fungsi ini. Dengan demikian kita
mendapatkan plot semua distribusi chi-kuadrat dengan derajat 5 hingga
30 dengan mudah dengan cara berikut.
\end{eulercomment}
\begin{eulerprompt}
>plot2d("qchidis(x,(5:5:50)')",0,50):
\end{eulerprompt}
\eulerimg{15}{images/Pekan 13-14_Fanny Erina Dewi_22305141005_EMT00-Statistika_Aplikom-010.png}
\begin{eulercomment}
Euler memiliki fungsi-fungsi yang akurat untuk mengevaluasi
distribusi-distribusi. Mari kita periksa chidis() dengan sebuah
integral.

Penamaannya diusahakan untuk konsisten. Sebagai contoh,

- distribusi chi-kuadrat adalah chidis(),\\
- fungsi kebalikannya adalah invchidis(),\\
- densitasnya adalah qchidis().

Pelengkap dari distribusi (ekor atas) adalah chicdis().
\end{eulercomment}
\begin{eulerprompt}
>chidis(1.5,2), integrate("qchidis(x,2)",0,1.5)
\end{eulerprompt}
\begin{euleroutput}
  0.527633447259
  0.527633447259
\end{euleroutput}
\eulerheading{Distribusi Diskrit}
\begin{eulercomment}
Untuk menentukan distribusi diskrit Anda sendiri, Anda dapat
menggunakan metode berikut.

Pertama, kita tetapkan fungsi distribusinya.
\end{eulercomment}
\begin{eulerprompt}
>wd = 0|((1:6)+[-0.01,0.01,0,0,0,0])/6
\end{eulerprompt}
\begin{euleroutput}
  [0,  0.165,  0.335,  0.5,  0.666667,  0.833333,  1]
\end{euleroutput}
\begin{eulercomment}
Artinya, dengan probabilitas wd[i+1]-wd[i] kita menghasilkan nilai
acak i.

Ini hampir merupakan distribusi yang seragam. Mari kita definisikan
sebuah generator bilangan acak untuk ini. Fungsi find(v,x) menemukan
nilai x dalam vektor v. Fungsi ini juga dapat digunakan untuk vektor
x.
\end{eulercomment}
\begin{eulerprompt}
>function wrongdice (n,m) := find(wd,random(n,m))
\end{eulerprompt}
\begin{eulercomment}
Kesalahan ini sangat halus sehingga kita hanya bisa melihatnya setelah
melakukan iterasi yang sangat banyak.
\end{eulercomment}
\begin{eulerprompt}
>columnsplot(getmultiplicities(1:6,wrongdice(1,1000000))):
\end{eulerprompt}
\eulerimg{15}{images/Pekan 13-14_Fanny Erina Dewi_22305141005_EMT00-Statistika_Aplikom-011.png}
\begin{eulercomment}
Berikut ini adalah fungsi sederhana untuk memeriksa distribusi seragam
dari nilai 1... K dalam v. Kami menerima hasilnya, jika untuk semua
frekuensi

\end{eulercomment}
\begin{eulerformula}
\[
\left|f_i-\frac{1}{K}\right| < \frac{\delta}{\sqrt{n}}.
\]
\end{eulerformula}
\begin{eulerprompt}
>function checkrandom (v, delta=1) ...
\end{eulerprompt}
\begin{eulerudf}
    K=max(v); n=cols(v);
    fr=getfrequencies(v,1:K);
    return max(fr/n-1/K)<delta/sqrt(n);
    endfunction
\end{eulerudf}
\begin{eulercomment}
Memang fungsi ini menolak distribusi seragam.
\end{eulercomment}
\begin{eulerprompt}
>checkrandom(wrongdice(1,1000000))
\end{eulerprompt}
\begin{euleroutput}
  0
\end{euleroutput}
\begin{eulercomment}
Dan ini menerima generator acak bawaan.
\end{eulercomment}
\begin{eulerprompt}
>checkrandom(intrandom(1,1000000,6))
\end{eulerprompt}
\begin{euleroutput}
  1
\end{euleroutput}
\begin{eulercomment}
Kita dapat menghitung distribusi binomial. Pertama, ada binomialsum(),
yang mengembalikan probabilitas i atau kurang dari n percobaan.
\end{eulercomment}
\begin{eulerprompt}
>bindis(410,1000,0.4)
\end{eulerprompt}
\begin{euleroutput}
  0.751401349654
\end{euleroutput}
\begin{eulercomment}
Fungsi Beta invers digunakan untuk menghitung interval kepercayaan
Clopper-Pearson untuk parameter p. Tingkat defaultnya adalah alpha.

Arti dari interval ini adalah jika p berada di luar interval, hasil
yang diamati sebesar 410 dalam 1000 jarang terjadi.
\end{eulercomment}
\begin{eulerprompt}
>clopperpearson(410,1000)
\end{eulerprompt}
\begin{euleroutput}
  [0.37932,  0.441212]
\end{euleroutput}
\begin{eulercomment}
Perintah berikut ini adalah cara langsung untuk mendapatkan hasil di
atas. Tetapi untuk n yang besar, penjumlahan langsung tidak akurat dan
lambat.
\end{eulercomment}
\begin{eulerprompt}
>p=0.4; i=0:410; n=1000; sum(bin(n,i)*p^i*(1-p)^(n-i))
\end{eulerprompt}
\begin{euleroutput}
  0.751401349655
\end{euleroutput}
\begin{eulercomment}
Omong-omong, invbinsum() menghitung kebalikan dari binomialsum().
\end{eulercomment}
\begin{eulerprompt}
>invbindis(0.75,1000,0.4)
\end{eulerprompt}
\begin{euleroutput}
  409.932733047
\end{euleroutput}
\begin{eulercomment}
Dalam Bridge, kita mengasumsikan 5 kartu yang terbuka (dari 52 kartu)
di dua tangan (26 kartu). Mari kita hitung probabilitas distribusi
yang lebih buruk dari 3:2 (misalnya 0:5, 1:4, 4:1, atau 5:0).
\end{eulercomment}
\begin{eulerprompt}
>2*hypergeomsum(1,5,13,26)
\end{eulerprompt}
\begin{euleroutput}
  0.321739130435
\end{euleroutput}
\begin{eulercomment}
Ada juga simulasi distribusi multinomial.
\end{eulercomment}
\begin{eulerprompt}
>randmultinomial(10,1000,[0.4,0.1,0.5])
\end{eulerprompt}
\begin{euleroutput}
            381           100           519 
            376            91           533 
            417            80           503 
            440            94           466 
            406           112           482 
            408            94           498 
            395           107           498 
            399            96           505 
            428            87           485 
            400            99           501 
\end{euleroutput}
\eulerheading{Memplot Data}
\begin{eulercomment}
Untuk memplot data, kami mencoba hasil pemilihan umum Jerman sejak
tahun 1990, yang diukur dalam kursi.
\end{eulercomment}
\begin{eulerprompt}
>BW := [ ...
>1990,662,319,239,79,8,17; ...
>1994,672,294,252,47,49,30; ...
>1998,669,245,298,43,47,36; ...
>2002,603,248,251,47,55,2; ...
>2005,614,226,222,61,51,54; ...
>2009,622,239,146,93,68,76; ...
>2013,631,311,193,0,63,64];
\end{eulerprompt}
\begin{eulercomment}
Untuk pesta, kami menggunakan serangkaian nama.
\end{eulercomment}
\begin{eulerprompt}
>P:=["CDU/CSU","SPD","FDP","Gr","Li"];
\end{eulerprompt}
\begin{eulercomment}
Mari kita cetak persentasenya dengan baik.

Pertama kita ekstrak kolom-kolom yang diperlukan. Kolom 3 sampai 7
adalah kursi masing-masing partai, dan kolom 2 adalah jumlah total
kursi. kolom adalah tahun pemilihan.
\end{eulercomment}
\begin{eulerprompt}
>BT:=BW[,3:7]; BT:=BT/sum(BT); YT:=BW[,1]';
\end{eulerprompt}
\begin{eulercomment}
Kemudian kita mencetak statistik dalam bentuk tabel. Kita menggunakan
nama sebagai judul kolom, dan tahun sebagai judul baris. Lebar default
untuk kolom adalah wc = 10, tetapi kami lebih suka output yang lebih
padat. Kolom-kolom akan diperluas untuk label-label kolom, jika perlu.
\end{eulercomment}
\begin{eulerprompt}
>writetable(BT*100,wc=6,dc=0,>fixed,labc=P,labr=YT)
\end{eulerprompt}
\begin{euleroutput}
         CDU/CSU   SPD   FDP    Gr    Li
    1990      48    36    12     1     3
    1994      44    38     7     7     4
    1998      37    45     6     7     5
    2002      41    42     8     9     0
    2005      37    36    10     8     9
    2009      38    23    15    11    12
    2013      49    31     0    10    10
\end{euleroutput}
\begin{eulercomment}
Perkalian matriks berikut ini mengekstrak jumlah persentase dua partai
besar yang menunjukkan bahwa partai-partai kecil telah memperoleh
suara di parlemen hingga tahun 2009.
\end{eulercomment}
\begin{eulerprompt}
>BT1:=(BT.[1;1;0;0;0])'*100
\end{eulerprompt}
\begin{euleroutput}
  [84.29,  81.25,  81.1659,  82.7529,  72.9642,  61.8971,  79.8732]
\end{euleroutput}
\begin{eulercomment}
Ada juga plot statistik sederhana. Kita menggunakannya untuk
menampilkan garis dan titik secara bersamaan. Alternatif lainnya
adalah memanggil plot2d dua kali dengan \textgreater{}add.
\end{eulercomment}
\begin{eulerprompt}
>statplot(YT,BT1,"b"):
\end{eulerprompt}
\eulerimg{15}{images/Pekan 13-14_Fanny Erina Dewi_22305141005_EMT00-Statistika_Aplikom-012.png}
\begin{eulercomment}
Tentukan beberapa warna untuk masing-masing pihak.
\end{eulercomment}
\begin{eulerprompt}
>CP:=[rgb(0.5,0.5,0.5),red,yellow,green,rgb(0.8,0,0)];
\end{eulerprompt}
\begin{eulercomment}
Sekarang kita dapat memplot hasil pemilu 2009 dan perubahannya ke
dalam satu plot menggunakan figure. Kita dapat menambahkan vektor
kolom pada setiap plot.
\end{eulercomment}
\begin{eulerprompt}
>figure(2,1);  ...
>figure(1); columnsplot(BW[6,3:7],P,color=CP); ...
>figure(2); columnsplot(BW[6,3:7]-BW[5,3:7],P,color=CP);  ...
>figure(0):
\end{eulerprompt}
\eulerimg{15}{images/Pekan 13-14_Fanny Erina Dewi_22305141005_EMT00-Statistika_Aplikom-013.png}
\begin{eulercomment}
Plot data menggabungkan baris data statistik dalam satu plot.
\end{eulercomment}
\begin{eulerprompt}
>J:=BW[,1]'; DP:=BW[,3:7]'; ...
>dataplot(YT,BT',color=CP);  ...
>labelbox(P,colors=CP,styles="[]",>points,w=0.2,x=0.3,y=0.4):
\end{eulerprompt}
\eulerimg{15}{images/Pekan 13-14_Fanny Erina Dewi_22305141005_EMT00-Statistika_Aplikom-014.png}
\begin{eulercomment}
Plot kolom 3D menunjukkan deretan data statistik dalam bentuk kolom.
Kami menyediakan label untuk baris dan kolom. angle adalah sudut
pandang.
\end{eulercomment}
\begin{eulerprompt}
>columnsplot3d(BT,scols=P,srows=YT, ...
>  angle=30°,ccols=CP):
\end{eulerprompt}
\eulerimg{15}{images/Pekan 13-14_Fanny Erina Dewi_22305141005_EMT00-Statistika_Aplikom-015.png}
\begin{eulercomment}
Representasi lainnya adalah plot mosaik. Perhatikan bahwa kolom-kolom
pada plot mewakili kolom-kolom pada matriks di sini. Karena panjangnya
label CDU/CSU, kita mengambil jendela yang lebih kecil dari biasanya.
\end{eulercomment}
\begin{eulerprompt}
>shrinkwindow(>smaller);  ...
>mosaicplot(BT',srows=YT,scols=P,color=CP,style="#"); ...
>shrinkwindow():
\end{eulerprompt}
\eulerimg{15}{images/Pekan 13-14_Fanny Erina Dewi_22305141005_EMT00-Statistika_Aplikom-016.png}
\begin{eulercomment}
Kita juga bisa membuat diagram lingkaran. Karena warna hitam dan
kuning membentuk sebuah koalisi, kita menyusun ulang elemen-elemennya.
\end{eulercomment}
\begin{eulerprompt}
>i=[1,3,5,4,2]; piechart(BW[6,3:7][i],color=CP[i],lab=P[i]):
\end{eulerprompt}
\eulerimg{15}{images/Pekan 13-14_Fanny Erina Dewi_22305141005_EMT00-Statistika_Aplikom-017.png}
\begin{eulercomment}
Berikut ini jenis plot yang lain.
\end{eulercomment}
\begin{eulerprompt}
>starplot(normal(1,10)+4,lab=1:10,>rays):
\end{eulerprompt}
\eulerimg{15}{images/Pekan 13-14_Fanny Erina Dewi_22305141005_EMT00-Statistika_Aplikom-018.png}
\begin{eulercomment}
Some plots in plot2d are good for statics. Here is an impulse plot of random data, uniformly
distributed in [0,1].
\end{eulercomment}
\begin{eulerprompt}
>plot2d(makeimpulse(1:10,random(1,10)),>bar):
\end{eulerprompt}
\eulerimg{15}{images/Pekan 13-14_Fanny Erina Dewi_22305141005_EMT00-Statistika_Aplikom-019.png}
\begin{eulercomment}
Tetapi untuk data yang terdistribusi secara eksponensial, kita mungkin
memerlukan plot logaritmik.
\end{eulercomment}
\begin{eulerprompt}
>logimpulseplot(1:10,-log(random(1,10))*10):
\end{eulerprompt}
\eulerimg{15}{images/Pekan 13-14_Fanny Erina Dewi_22305141005_EMT00-Statistika_Aplikom-020.png}
\begin{eulercomment}
Fungsi columnsplot() lebih mudah digunakan, karena hanya membutuhkan
sebuah vektor nilai. Selain itu, fungsi ini dapat mengatur labelnya
menjadi apa pun yang kita inginkan, kita telah mendemonstrasikan hal
ini dalam tutorial ini.

Berikut ini adalah aplikasi lain, di mana kita menghitung karakter
dalam sebuah kalimat dan memplot statistik.
\end{eulercomment}
\begin{eulerprompt}
>v=strtochar("the quick brown fox jumps over the lazy dog"); ...
>w=ascii("a"):ascii("z"); x=getmultiplicities(w,v); ...
>cw=[]; for k=w; cw=cw|char(k); end; ...
>columnsplot(x,lab=cw,width=0.05):
\end{eulerprompt}
\eulerimg{15}{images/Pekan 13-14_Fanny Erina Dewi_22305141005_EMT00-Statistika_Aplikom-021.png}
\begin{eulercomment}
Anda juga dapat menetapkan sumbu secara manual.
\end{eulercomment}
\begin{eulerprompt}
>n=10; p=0.4; i=0:n; x=bin(n,i)*p^i*(1-p)^(n-i); ...
>columnsplot(x,lab=i,width=0.05,<frame,<grid); ...
>yaxis(0,0:0.1:1,style="->",>left); xaxis(0,style="."); ...
>label("p",0,0.25), label("i",11,0); ...
>textbox(["Binomial distribution","with p=0.4"]):
\end{eulerprompt}
\eulerimg{15}{images/Pekan 13-14_Fanny Erina Dewi_22305141005_EMT00-Statistika_Aplikom-022.png}
\begin{eulercomment}
The following is a way to plot the frequencies of numbers in a vector.

We create a vector of integer random numbers 1 to 6.
\end{eulercomment}
\begin{eulerprompt}
>v:=intrandom(1,10,10)
\end{eulerprompt}
\begin{euleroutput}
  [8,  5,  8,  8,  6,  8,  8,  3,  5,  5]
\end{euleroutput}
\begin{eulercomment}
Kemudian ekstrak nomor unik dalam v.
\end{eulercomment}
\begin{eulerprompt}
>vu:=unique(v)
\end{eulerprompt}
\begin{euleroutput}
  [3,  5,  6,  8]
\end{euleroutput}
\begin{eulercomment}
Dan memplot frekuensi dalam plot kolom.
\end{eulercomment}
\begin{eulerprompt}
>columnsplot(getmultiplicities(vu,v),lab=vu,style="/"):
\end{eulerprompt}
\eulerimg{15}{images/Pekan 13-14_Fanny Erina Dewi_22305141005_EMT00-Statistika_Aplikom-023.png}
\begin{eulercomment}
Kami ingin mendemonstrasikan fungsi untuk distribusi nilai empiris.
\end{eulercomment}
\begin{eulerprompt}
>x=normal(1,20);
\end{eulerprompt}
\begin{eulercomment}
Fungsi empdist(x,vs) membutuhkan larik nilai yang telah diurutkan.
Jadi kita harus mengurutkan x sebelum dapat menggunakannya.
\end{eulercomment}
\begin{eulerprompt}
>xs=sort(x);
\end{eulerprompt}
\begin{eulercomment}
Kemudian kita memplot distribusi empiris dan beberapa batang kepadatan
ke dalam satu plot. Alih-alih plot batang untuk distribusi, kali ini
kami menggunakan plot gigi gergaji.
\end{eulercomment}
\begin{eulerprompt}
>figure(2,1); ...
>figure(1); plot2d("empdist",-4,4;xs); ...
>figure(2); plot2d(histo(x,v=-4:0.2:4,<bar));  ...
>figure(0):
\end{eulerprompt}
\eulerimg{15}{images/Pekan 13-14_Fanny Erina Dewi_22305141005_EMT00-Statistika_Aplikom-024.png}
\begin{eulercomment}
Plot sebaran mudah dilakukan di Euler dengan plot titik biasa. Grafik
berikut ini menunjukkan bahwa X dan X+Y berkorelasi positif secara
jelas.
\end{eulercomment}
\begin{eulerprompt}
>x=normal(1,100); plot2d(x,x+rotright(x),>points,style=".."):
\end{eulerprompt}
\eulerimg{15}{images/Pekan 13-14_Fanny Erina Dewi_22305141005_EMT00-Statistika_Aplikom-025.png}
\begin{eulercomment}
Sering kali, kita ingin membandingkan dua sampel dari distribusi yang
berbeda. Hal ini dapat dilakukan dengan plot kuantil-kuantil.

Untuk pengujian, kami mencoba distribusi student-t dan distribusi
eksponensial.
\end{eulercomment}
\begin{eulerprompt}
>x=randt(1,1000,5); y=randnormal(1,1000,mean(x),dev(x)); ...
>plot2d("x",r=6,style="--",yl="normal",xl="student-t",>vertical); ...
>plot2d(sort(x),sort(y),>points,color=red,style="x",>add):
\end{eulerprompt}
\eulerimg{15}{images/Pekan 13-14_Fanny Erina Dewi_22305141005_EMT00-Statistika_Aplikom-026.png}
\begin{eulercomment}
Plot ini dengan jelas menunjukkan bahwa nilai yang terdistribusi
normal cenderung lebih kecil pada ujung yang ekstrim.

Jika kita memiliki dua distribusi dengan ukuran yang berbeda, kita
dapat memperluas distribusi yang lebih kecil atau memperkecil
distribusi yang lebih besar. Fungsi berikut ini bagus untuk keduanya.
Fungsi ini mengambil nilai median dengan persentase antara 0 dan 1.
\end{eulercomment}
\begin{eulerprompt}
>function medianexpand (x,n) := median(x,p=linspace(0,1,n-1));
\end{eulerprompt}
\begin{eulercomment}
Mari kita bandingkan dua distribusi yang sama.
\end{eulercomment}
\begin{eulerprompt}
>x=random(1000); y=random(400); ...
>plot2d("x",0,1,style="--"); ...
>plot2d(sort(medianexpand(x,400)),sort(y),>points,color=red,style="x",>add):
\end{eulerprompt}
\eulerimg{15}{images/Pekan 13-14_Fanny Erina Dewi_22305141005_EMT00-Statistika_Aplikom-027.png}
\eulerheading{Regresi dan Korelasi}
\begin{eulercomment}
Regresi linier dapat dilakukan dengan fungsi polyfit() atau berbagai
fungsi kecocokan.

Sebagai permulaan, kita mencari garis regresi untuk data univariat
dengan polyfit(x,y,1).
\end{eulercomment}
\begin{eulerprompt}
>x=1:10; y=[2,3,1,5,6,3,7,8,9,8]; writetable(x'|y',labc=["x","y"])
\end{eulerprompt}
\begin{euleroutput}
           x         y
           1         2
           2         3
           3         1
           4         5
           5         6
           6         3
           7         7
           8         8
           9         9
          10         8
\end{euleroutput}
\begin{eulercomment}
Kami ingin membandingkan kecocokan tanpa bobot dan dengan bobot.
Pertama, koefisien dari kecocokan linier.
\end{eulercomment}
\begin{eulerprompt}
>p=polyfit(x,y,1)
\end{eulerprompt}
\begin{euleroutput}
  [0.733333,  0.812121]
\end{euleroutput}
\begin{eulercomment}
Sekarang, koefisien dengan bobot yang menekankan nilai terakhir.
\end{eulercomment}
\begin{eulerprompt}
>w &= "exp(-(x-10)^2/10)"; pw=polyfit(x,y,1,w=w(x))
\end{eulerprompt}
\begin{euleroutput}
  [4.71566,  0.38319]
\end{euleroutput}
\begin{eulercomment}
Kami menempatkan semuanya ke dalam satu plot untuk titik-titik dan
garis regresi, dan untuk bobot yang digunakan.
\end{eulercomment}
\begin{eulerprompt}
>figure(2,1);  ...
>figure(1); statplot(x,y,"b",xl="Regression"); ...
>  plot2d("evalpoly(x,p)",>add,color=blue,style="--"); ...
>  plot2d("evalpoly(x,pw)",5,10,>add,color=red,style="--"); ...
>figure(2); plot2d(w,1,10,>filled,style="/",fillcolor=red,xl=w); ...
>figure(0):
\end{eulerprompt}
\eulerimg{15}{images/Pekan 13-14_Fanny Erina Dewi_22305141005_EMT00-Statistika_Aplikom-028.png}
\begin{eulercomment}
Untuk contoh lain, kita membaca survei tentang siswa, usia mereka,
usia orang tua mereka, dan jumlah saudara kandung dari sebuah file.

Tabel ini berisi "m" dan "f" pada kolom kedua. Kita menggunakan
variabel tok2 untuk mengatur terjemahan yang tepat dan bukannya
membiarkan readtable() mengumpulkan terjemahan.
\end{eulercomment}
\begin{eulerprompt}
>\{MS,hd\}:=readtable("table1.dat",tok2:=["m","f"]);  ...
>writetable(MS,labc=hd,tok2:=["m","f"]);
\end{eulerprompt}
\begin{euleroutput}
      Person       Sex       Age    Mother    Father  Siblings
           1         m        29        58        61         1
           2         f        26        53        54         2
           3         m        24        49        55         1
           4         f        25        56        63         3
           5         f        25        49        53         0
           6         f        23        55        55         2
           7         m        23        48        54         2
           8         m        27        56        58         1
           9         m        25        57        59         1
          10         m        24        50        54         1
          11         f        26        61        65         1
          12         m        24        50        52         1
          13         m        29        54        56         1
          14         m        28        48        51         2
          15         f        23        52        52         1
          16         m        24        45        57         1
          17         f        24        59        63         0
          18         f        23        52        55         1
          19         m        24        54        61         2
          20         f        23        54        55         1
\end{euleroutput}
\begin{eulercomment}
Bagaimana usia saling bergantung satu sama lain? Kesan pertama datang
dari scatterplot berpasangan.
\end{eulercomment}
\begin{eulerprompt}
>scatterplots(tablecol(MS,3:5),hd[3:5]):
\end{eulerprompt}
\eulerimg{15}{images/Pekan 13-14_Fanny Erina Dewi_22305141005_EMT00-Statistika_Aplikom-029.png}
\begin{eulercomment}
Jelas bahwa usia ayah dan ibu saling bergantung satu sama lain. Mari
kita tentukan dan plot garis regresinya.
\end{eulercomment}
\begin{eulerprompt}
>cs:=MS[,4:5]'; ps:=polyfit(cs[1],cs[2],1)
\end{eulerprompt}
\begin{euleroutput}
  [17.3789,  0.740964]
\end{euleroutput}
\begin{eulercomment}
Ini jelas merupakan model yang salah. Garis regresinya adalah s = 17 +
0,74t, di mana t adalah usia ibu dan s adalah usia ayah. Perbedaan
usia mungkin sedikit bergantung pada usia, tetapi tidak terlalu
banyak.

Sebaliknya, kami menduga fungsi seperti s = a + t. Kemudian a adalah
rata-rata dari s-t. Ini adalah perbedaan usia rata-rata antara ayah
dan ibu.
\end{eulercomment}
\begin{eulerprompt}
>da:=mean(cs[2]-cs[1])
\end{eulerprompt}
\begin{euleroutput}
  3.65
\end{euleroutput}
\begin{eulercomment}
Mari kita plotkan ini ke dalam satu scatter plot.
\end{eulercomment}
\begin{eulerprompt}
>plot2d(cs[1],cs[2],>points);  ...
>plot2d("evalpoly(x,ps)",color=red,style=".",>add);  ...
>plot2d("x+da",color=blue,>add):
\end{eulerprompt}
\eulerimg{15}{images/Pekan 13-14_Fanny Erina Dewi_22305141005_EMT00-Statistika_Aplikom-030.png}
\begin{eulercomment}
Berikut ini adalah plot kotak dari kedua usia tersebut. Ini hanya
menunjukkan, bahwa usia keduanya berbeda.
\end{eulercomment}
\begin{eulerprompt}
>boxplot(cs,["mothers","fathers"]):
\end{eulerprompt}
\eulerimg{15}{images/Pekan 13-14_Fanny Erina Dewi_22305141005_EMT00-Statistika_Aplikom-031.png}
\begin{eulercomment}
Sangat menarik bahwa perbedaan dalam median tidak sebesar perbedaan
dalam mean.
\end{eulercomment}
\begin{eulerprompt}
>median(cs[2])-median(cs[1])
\end{eulerprompt}
\begin{euleroutput}
  1.5
\end{euleroutput}
\begin{eulercomment}
Koefisien korelasi menunjukkan korelasi positif.
\end{eulercomment}
\begin{eulerprompt}
>correl(cs[1],cs[2])
\end{eulerprompt}
\begin{euleroutput}
  0.7588307236
\end{euleroutput}
\begin{eulercomment}
Korelasi peringkat adalah ukuran untuk urutan yang sama dalam kedua
vektor. Korelasi ini juga cukup positif.
\end{eulercomment}
\begin{eulerprompt}
>rankcorrel(cs[1],cs[2])
\end{eulerprompt}
\begin{euleroutput}
  0.758925292358
\end{euleroutput}
\eulerheading{Membuat Fungsi baru}
\begin{eulercomment}
Tentu saja, bahasa EMT dapat digunakan untuk memprogram fungsi baru.
Sebagai contoh, kita mendefinisikan fungsi kemiringan.

lateks: \textbackslash{}text\{sk\}(x) = \textbackslash{}dfrac\{\textbackslash{}sqrt\{n\} \textbackslash{}sum\_i (x\_i-m)\textasciicircum{}3\}\{\textbackslash{}left(\textbackslash{}sum\_i
(x\_i-m)\textasciicircum{}2\textbackslash{}right)\textasciicircum{}\{3/2\}\}

di mana m adalah rata-rata dari x.
\end{eulercomment}
\begin{eulerprompt}
>function skew (x:vector) ...
\end{eulerprompt}
\begin{eulerudf}
  m=mean(x);
  return sqrt(cols(x))*sum((x-m)^3)/(sum((x-m)^2))^(3/2);
  endfunction
\end{eulerudf}
\begin{eulercomment}
Seperti yang Anda lihat, kita dapat dengan mudah menggunakan bahasa
matriks untuk mendapatkan implementasi yang sangat singkat dan
efisien. Mari kita coba fungsi ini.
\end{eulercomment}
\begin{eulerprompt}
>data=normal(20); skew(normal(10))
\end{eulerprompt}
\begin{euleroutput}
  -0.198710316203
\end{euleroutput}
\begin{eulercomment}
Berikut ini adalah fungsi lain, yang disebut koefisien kemencengan
Pearson.
\end{eulercomment}
\begin{eulerprompt}
>function skew1 (x) := 3*(mean(x)-median(x))/dev(x)
>skew1(data)
\end{eulerprompt}
\begin{euleroutput}
  -0.0801873249135
\end{euleroutput}
\eulerheading{Simulasi Monte Carlo}
\begin{eulercomment}
Euler dapat digunakan untuk mensimulasikan kejadian acak. Kita telah
melihat contoh sederhana di atas. Berikut ini adalah contoh lainnya,
yang mensimulasikan 1000 kali pelemparan 3 dadu, dan menanyakan
distribusi dari jumlah tersebut.
\end{eulercomment}
\begin{eulerprompt}
>ds:=sum(intrandom(1000,3,6))';  fs=getmultiplicities(3:18,ds)
\end{eulerprompt}
\begin{euleroutput}
  [5,  17,  35,  44,  75,  97,  114,  116,  143,  116,  104,  53,  40,
  22,  13,  6]
\end{euleroutput}
\begin{eulercomment}
Kita bisa merencanakan ini sekarang.
\end{eulercomment}
\begin{eulerprompt}
>columnsplot(fs,lab=3:18):
\end{eulerprompt}
\eulerimg{15}{images/Pekan 13-14_Fanny Erina Dewi_22305141005_EMT00-Statistika_Aplikom-032.png}
\begin{eulercomment}
Untuk menentukan distribusi yang diharapkan tidaklah mudah. Kami
menggunakan rekursi tingkat lanjut untuk hal ini.

Fungsi berikut ini menghitung jumlah cara angka k dapat
direpresentasikan sebagai jumlah n angka dalam rentang 1 hingga m.
Fungsi ini bekerja secara rekursif dengan cara yang jelas.
\end{eulercomment}
\begin{eulerprompt}
>function map countways (k; n, m) ...
\end{eulerprompt}
\begin{eulerudf}
    if n==1 then return k>=1 && k<=m
    else
      sum=0; 
      loop 1 to m; sum=sum+countways(k-#,n-1,m); end;
      return sum;
    end;
  endfunction
\end{eulerudf}
\begin{eulercomment}
Berikut ini adalah hasil dari tiga lemparan dadu.
\end{eulercomment}
\begin{eulerprompt}
>countways(5:25,5,5)
\end{eulerprompt}
\begin{euleroutput}
  [1,  5,  15,  35,  70,  121,  185,  255,  320,  365,  381,  365,  320,
  255,  185,  121,  70,  35,  15,  5,  1]
\end{euleroutput}
\begin{eulerprompt}
>cw=countways(3:18,3,6)
\end{eulerprompt}
\begin{euleroutput}
  [1,  3,  6,  10,  15,  21,  25,  27,  27,  25,  21,  15,  10,  6,  3,
  1]
\end{euleroutput}
\begin{eulercomment}
Kami menambahkan nilai yang diharapkan ke plot.
\end{eulercomment}
\begin{eulerprompt}
>plot2d(cw/6^3*1000,>add); plot2d(cw/6^3*1000,>points,>add):
\end{eulerprompt}
\eulerimg{15}{images/Pekan 13-14_Fanny Erina Dewi_22305141005_EMT00-Statistika_Aplikom-033.png}
\begin{eulercomment}
Untuk simulasi lainnya, deviasi nilai rata-rata dari n variabel acak
berdistribusi normal 0-1 adalah 1/sqrt(n).
\end{eulercomment}
\begin{eulerprompt}
>longformat; 1/sqrt(10)
\end{eulerprompt}
\begin{euleroutput}
  0.316227766017
\end{euleroutput}
\begin{eulercomment}
Mari kita periksa dengan sebuah simulasi. Kami menghasilkan 10.000
kali 10 vektor acak.
\end{eulercomment}
\begin{eulerprompt}
>M=normal(10000,10); dev(mean(M)')
\end{eulerprompt}
\begin{euleroutput}
  0.319493614817
\end{euleroutput}
\begin{eulerprompt}
>plot2d(mean(M)',>distribution):
\end{eulerprompt}
\eulerimg{15}{images/Pekan 13-14_Fanny Erina Dewi_22305141005_EMT00-Statistika_Aplikom-034.png}
\begin{eulercomment}
Median dari 10 bilangan acak berdistribusi normal 0-1 memiliki deviasi
yang lebih besar.
\end{eulercomment}
\begin{eulerprompt}
>dev(median(M)')
\end{eulerprompt}
\begin{euleroutput}
  0.374460271535
\end{euleroutput}
\begin{eulercomment}
Karena kita dapat dengan mudah menghasilkan jalan acak, kita dapat
mensimulasikan proses Wiener. Kami mengambil 1000 langkah dari 1000
proses. Kami kemudian memplot deviasi standar dan rata-rata dari
langkah ke-n dari proses-proses ini bersama dengan nilai yang
diharapkan dalam warna merah.
\end{eulercomment}
\begin{eulerprompt}
>n=1000; m=1000; M=cumsum(normal(n,m)/sqrt(m)); ...
>t=(1:n)/n; figure(2,1); ...
>figure(1); plot2d(t,mean(M')'); plot2d(t,0,color=red,>add); ...
>figure(2); plot2d(t,dev(M')'); plot2d(t,sqrt(t),color=red,>add); ...
>figure(0):
\end{eulerprompt}
\eulerimg{15}{images/Pekan 13-14_Fanny Erina Dewi_22305141005_EMT00-Statistika_Aplikom-035.png}
\eulerheading{Tes}
\begin{eulercomment}
Tes adalah alat yang penting dalam statistik. Dalam Euler, banyak tes
yang diterapkan. Semua tes ini mengembalikan kesalahan yang kita
terima jika kita menolak hipotesis nol.

Sebagai contoh, kita menguji lemparan dadu untuk distribusi yang
seragam. Pada 600 lemparan, kita mendapatkan nilai berikut, yang kita
masukkan ke dalam uji chi-kuadrat.
\end{eulercomment}
\begin{eulerprompt}
>chitest([90,103,114,101,103,89],dup(100,6)')
\end{eulerprompt}
\begin{euleroutput}
  0.498830517952
\end{euleroutput}
\begin{eulercomment}
Uji chi-square juga memiliki mode, yang menggunakan simulasi Monte
Carlo untuk menguji statistik. Hasilnya seharusnya hampir sama.
Parameter \textgreater{}p menginterpretasikan vektor y sebagai vektor probabilitas.
\end{eulercomment}
\begin{eulerprompt}
>chitest([90,103,114,101,103,89],dup(1/6,6)',>p,>montecarlo)
\end{eulerprompt}
\begin{euleroutput}
  0.526
\end{euleroutput}
\begin{eulercomment}
Kesalahan ini terlalu besar. Jadi kita tidak bisa menolak distribusi
seragam. Ini tidak membuktikan bahwa dadu kita adil. Tetapi kita tidak
bisa menolak hipotesis kita.

Selanjutnya kita buat 1000 lemparan dadu dengan menggunakan generator
bilangan acak, dan lakukan pengujian yang sama.
\end{eulercomment}
\begin{eulerprompt}
>n=1000; t=random([1,n*6]); chitest(count(t*6,6),dup(n,6)')
\end{eulerprompt}
\begin{euleroutput}
  0.528028118442
\end{euleroutput}
\begin{eulercomment}
Mari kita uji nilai rata-rata 100 dengan uji-t.
\end{eulercomment}
\begin{eulerprompt}
>s=200+normal([1,100])*10; ...
>ttest(mean(s),dev(s),100,200)
\end{eulerprompt}
\begin{euleroutput}
  0.0218365848476
\end{euleroutput}
\begin{eulercomment}
Fungsi ttest() membutuhkan nilai rata-rata, deviasi, jumlah data, dan
nilai rata-rata untuk diuji.

Sekarang mari kita periksa dua pengukuran untuk mean yang sama. Kita
tolak hipotesis bahwa kedua pengukuran tersebut memiliki nilai
rata-rata yang sama, jika hasilnya \textless{} 0,05.
\end{eulercomment}
\begin{eulerprompt}
>tcomparedata(normal(1,10),normal(1,10))
\end{eulerprompt}
\begin{euleroutput}
  0.38722000942
\end{euleroutput}
\begin{eulercomment}
Jika kita menambahkan bias pada satu distribusi, kita akan mendapatkan
lebih banyak penolakan. Ulangi simulasi ini beberapa kali untuk
melihat efeknya.
\end{eulercomment}
\begin{eulerprompt}
>tcomparedata(normal(1,10),normal(1,10)+2)
\end{eulerprompt}
\begin{euleroutput}
  5.60009101758e-07
\end{euleroutput}
\begin{eulercomment}
Pada contoh berikut, kita membuat 20 lemparan dadu secara acak
sebanyak 100 kali dan menghitung jumlah dadu yang muncul. Rata-rata
harus ada 20/6 = 3,3 mata dadu.
\end{eulercomment}
\begin{eulerprompt}
>R=random(100,20); R=sum(R*6<=1)'; mean(R)
\end{eulerprompt}
\begin{euleroutput}
  3.28
\end{euleroutput}
\begin{eulercomment}
Sekarang kita bandingkan jumlah satu dengan distribusi binomial.
Pertama, kita memplot distribusi angka satu.
\end{eulercomment}
\begin{eulerprompt}
>plot2d(R,distribution=max(R)+1,even=1,style="\(\backslash\)/"):
\end{eulerprompt}
\eulerimg{15}{images/Pekan 13-14_Fanny Erina Dewi_22305141005_EMT00-Statistika_Aplikom-036.png}
\begin{eulerprompt}
>t=count(R,21);
\end{eulerprompt}
\begin{eulercomment}
Kemudian kami menghitung nilai yang diharapkan.
\end{eulercomment}
\begin{eulerprompt}
>n=0:20; b=bin(20,n)*(1/6)^n*(5/6)^(20-n)*100;
\end{eulerprompt}
\begin{eulercomment}
Kami harus mengumpulkan beberapa angka untuk mendapatkan kategori yang
cukup besar.
\end{eulercomment}
\begin{eulerprompt}
>t1=sum(t[1:2])|t[3:7]|sum(t[8:21]); ...
>b1=sum(b[1:2])|b[3:7]|sum(b[8:21]);
\end{eulerprompt}
\begin{eulercomment}
Uji chi-square menolak hipotesis bahwa distribusi kita adalah
distribusi binomial, jika hasilnya \textless{}0,05.
\end{eulercomment}
\begin{eulerprompt}
>chitest(t1,b1)
\end{eulerprompt}
\begin{euleroutput}
  0.53921579764
\end{euleroutput}
\begin{eulercomment}
Contoh berikut ini berisi hasil dari dua kelompok orang (laki-laki dan
perempuan, katakanlah) yang memberikan suara untuk satu dari enam
partai.
\end{eulercomment}
\begin{eulerprompt}
>A=[23,37,43,52,64,74;27,39,41,49,63,76];  ...
>  writetable(A,wc=6,labr=["m","f"],labc=1:6)
\end{eulerprompt}
\begin{euleroutput}
             1     2     3     4     5     6
       m    23    37    43    52    64    74
       f    27    39    41    49    63    76
\end{euleroutput}
\begin{eulercomment}
Kami ingin menguji independensi suara dari jenis kelamin. Uji tabel
chi\textasciicircum{}2 melakukan hal ini. Hasilnya terlalu besar untuk menolak
independensi. Jadi kita tidak dapat mengatakan, jika pemungutan suara
tergantung pada jenis kelamin dari data ini.
\end{eulercomment}
\begin{eulerprompt}
>tabletest(A)
\end{eulerprompt}
\begin{euleroutput}
  0.990701632326
\end{euleroutput}
\begin{eulercomment}
Berikut ini adalah tabel yang diharapkan, jika kita mengasumsikan
frekuensi pemungutan suara yang diamati.
\end{eulercomment}
\begin{eulerprompt}
>writetable(expectedtable(A),wc=6,dc=1,labr=["m","f"],labc=1:6)
\end{eulerprompt}
\begin{euleroutput}
             1     2     3     4     5     6
       m  24.9  37.9  41.9  50.3  63.3  74.7
       f  25.1  38.1  42.1  50.7  63.7  75.3
\end{euleroutput}
\begin{eulercomment}
Kita dapat menghitung koefisien kontingensi yang telah dikoreksi.
Karena koefisien ini sangat dekat dengan 0, kami menyimpulkan bahwa
pemungutan suara tidak bergantung pada jenis kelamin.
\end{eulercomment}
\begin{eulerprompt}
>contingency(A)
\end{eulerprompt}
\begin{euleroutput}
  0.0427225484717
\end{euleroutput}
\begin{eulercomment}
\begin{eulercomment}
\eulerheading{Beberapa Tes Lainnya}
\begin{eulercomment}
Selanjutnya kita menggunakan analisis varians (uji F) untuk menguji
tiga sampel data yang terdistribusi secara normal dengan nilai
rata-rata yang sama. Metode ini disebut ANOVA (analisis varians).
Dalam Euler, fungsi varanalysis() digunakan.
\end{eulercomment}
\begin{eulerprompt}
>x1=[109,111,98,119,91,118,109,99,115,109,94]; mean(x1),
\end{eulerprompt}
\begin{euleroutput}
  106.545454545
\end{euleroutput}
\begin{eulerprompt}
>x2=[120,124,115,139,114,110,113,120,117]; mean(x2),
\end{eulerprompt}
\begin{euleroutput}
  119.111111111
\end{euleroutput}
\begin{eulerprompt}
>x3=[120,112,115,110,105,134,105,130,121,111]; mean(x3)
\end{eulerprompt}
\begin{euleroutput}
  116.3
\end{euleroutput}
\begin{eulerprompt}
>varanalysis(x1,x2,x3)
\end{eulerprompt}
\begin{euleroutput}
  0.0138048221371
\end{euleroutput}
\begin{eulercomment}
Ini berarti, kami menolak hipotesis nilai rata-rata yang sama. Kami
melakukan ini dengan probabilitas kesalahan sebesar 1,3\%.

Ada juga uji median, yang menolak sampel data dengan distribusi
rata-rata yang berbeda dengan menguji median dari sampel gabungan.
\end{eulercomment}
\begin{eulerprompt}
>a=[56,66,68,49,61,53,45,58,54];
>b=[72,81,51,73,69,78,59,67,65,71,68,71];
>mediantest(a,b)
\end{eulerprompt}
\begin{euleroutput}
  0.0241724220052
\end{euleroutput}
\begin{eulercomment}
Uji lain tentang kesetaraan adalah uji peringkat. Uji ini jauh lebih
tajam daripada uji median.
\end{eulercomment}
\begin{eulerprompt}
>ranktest(a,b)
\end{eulerprompt}
\begin{euleroutput}
  0.00199969612469
\end{euleroutput}
\begin{eulercomment}
Dalam contoh berikut ini, kedua distribusi memiliki rata-rata yang
sama.
\end{eulercomment}
\begin{eulerprompt}
>ranktest(random(1,100),random(1,50)*3-1)
\end{eulerprompt}
\begin{euleroutput}
  0.129608141484
\end{euleroutput}
\begin{eulercomment}
Sekarang mari kita coba mensimulasikan dua perawatan a dan b yang
diterapkan pada orang yang berbeda.
\end{eulercomment}
\begin{eulerprompt}
>a=[8.0,7.4,5.9,9.4,8.6,8.2,7.6,8.1,6.2,8.9];
>b=[6.8,7.1,6.8,8.3,7.9,7.2,7.4,6.8,6.8,8.1];
\end{eulerprompt}
\begin{eulercomment}
Uji signum memutuskan, apakah a lebih baik daripada b.
\end{eulercomment}
\begin{eulerprompt}
>signtest(a,b)
\end{eulerprompt}
\begin{euleroutput}
  0.0546875
\end{euleroutput}
\begin{eulercomment}
Ini adalah kesalahan yang terlalu besar. Kita tidak dapat menolak
bahwa a sama baiknya dengan b.

Uji Wilcoxon lebih tajam daripada uji ini, tetapi bergantung pada
nilai kuantitatif dari perbedaan.
\end{eulercomment}
\begin{eulerprompt}
>wilcoxon(a,b)
\end{eulerprompt}
\begin{euleroutput}
  0.0296680599405
\end{euleroutput}
\begin{eulercomment}
Mari kita coba dua pengujian lagi dengan menggunakan rangkaian yang
dihasilkan.
\end{eulercomment}
\begin{eulerprompt}
>wilcoxon(normal(1,20),normal(1,20)-1)
\end{eulerprompt}
\begin{euleroutput}
  0.0068706451766
\end{euleroutput}
\begin{eulerprompt}
>wilcoxon(normal(1,20),normal(1,20))
\end{eulerprompt}
\begin{euleroutput}
  0.275145971064
\end{euleroutput}
\eulerheading{Bilangan Acak}
\begin{eulercomment}
Berikut ini adalah tes untuk generator bilangan acak. Euler
menggunakan generator yang sangat bagus, jadi kita tidak perlu
mengharapkan adanya masalah.

Pertama, kita akan membangkitkan sepuluh juta bilangan acak dalam
[0,1].
\end{eulercomment}
\begin{eulerprompt}
>n:=10000000; r:=random(1,n);
\end{eulerprompt}
\begin{eulercomment}
Selanjutnya, kami menghitung jarak antara dua angka yang kurang dari
0,05.
\end{eulercomment}
\begin{eulerprompt}
>a:=0.05; d:=differences(nonzeros(r<a));
\end{eulerprompt}
\begin{eulercomment}
Terakhir, kami memplot berapa kali, setiap jarak yang terjadi, dan
membandingkannya dengan nilai yang diharapkan.
\end{eulercomment}
\begin{eulerprompt}
>m=getmultiplicities(1:100,d); plot2d(m); ...
>  plot2d("n*(1-a)^(x-1)*a^2",color=red,>add):
\end{eulerprompt}
\eulerimg{15}{images/Pekan 13-14_Fanny Erina Dewi_22305141005_EMT00-Statistika_Aplikom-037.png}
\begin{eulercomment}
Menghapus data.
\end{eulercomment}
\begin{eulerprompt}
>remvalue n;
\end{eulerprompt}
\begin{eulercomment}
\begin{eulercomment}
\eulerheading{Pengantar untuk Pengguna Proyek R}
\begin{eulercomment}
Jelas, EMT tidak bersaing dengan R sebagai sebuah paket statistik.
Namun, ada banyak prosedur dan fungsi statistik yang tersedia di EMT
juga. Jadi EMT dapat memenuhi kebutuhan dasar. Bagaimanapun, EMT hadir
dengan paket numerik dan sistem aljabar komputer.

Buku ini diperuntukkan bagi Anda yang sudah terbiasa dengan R, tetapi
perlu mengetahui perbedaan sintaks EMT dan R. Kami mencoba memberikan
gambaran umum mengenai hal-hal yang jelas dan kurang jelas yang perlu
Anda ketahui.

Selain itu, kami juga membahas cara-cara untuk bertukar data di antara
kedua sistem tersebut.
\end{eulercomment}
\begin{eulercomment}
Perhatikan bahwa ini adalah pekerjaan yang sedang berlangsung.
\end{eulercomment}
\eulerheading{Sintaks Dasar}
\begin{eulercomment}
Hal pertama yang Anda pelajari dalam R adalah membuat sebuah vektor.
Dalam EMT, perbedaan utamanya adalah operator : dapat mengambil ukuran
langkah. Selain itu, operator ini memiliki daya ikat yang rendah.
\end{eulercomment}
\begin{eulerprompt}
>n=10; 0:n/20:n-1
\end{eulerprompt}
\begin{euleroutput}
  [0,  0.5,  1,  1.5,  2,  2.5,  3,  3.5,  4,  4.5,  5,  5.5,  6,  6.5,
  7,  7.5,  8,  8.5,  9]
\end{euleroutput}
\begin{eulercomment}
Fungsi c() tidak ada. Anda dapat menggunakan vektor untuk
menggabungkan beberapa hal.

Contoh berikut ini, seperti banyak contoh lainnya, berasal dari
"Interoduksi ke R" yang disertakan dengan proyek R. Jika Anda membaca
PDF ini, Anda akan menemukan bahwa saya mengikuti alurnya dalam
tutorial ini.
\end{eulercomment}
\begin{eulerprompt}
>x=[10.4, 5.6, 3.1, 6.4, 21.7]; [x,0,x]
\end{eulerprompt}
\begin{euleroutput}
  [10.4,  5.6,  3.1,  6.4,  21.7,  0,  10.4,  5.6,  3.1,  6.4,  21.7]
\end{euleroutput}
\begin{eulercomment}
Operator titik dua dengan ukuran langkah EMT digantikan oleh fungsi
seq() dalam R. Kita dapat menulis fungsi ini dalam EMT.
\end{eulercomment}
\begin{eulerprompt}
>function seq(a,b,c) := a:b:c; ...
>seq(0,-0.1,-1)
\end{eulerprompt}
\begin{euleroutput}
  [0,  -0.1,  -0.2,  -0.3,  -0.4,  -0.5,  -0.6,  -0.7,  -0.8,  -0.9,  -1]
\end{euleroutput}
\begin{eulercomment}
Fungsi rep() dari R tidak ada dalam EMT. Untuk input vektor, dapat
dituliskan sebagai berikut.
\end{eulercomment}
\begin{eulerprompt}
>function rep(x:vector,n:index) := flatten(dup(x,n)); ...
>rep(x,2)
\end{eulerprompt}
\begin{euleroutput}
  [10.4,  5.6,  3.1,  6.4,  21.7,  10.4,  5.6,  3.1,  6.4,  21.7]
\end{euleroutput}
\begin{eulercomment}
Perhatikan bahwa "=" atau ":=" digunakan untuk penugasan. Operator
"-\textgreater{}" digunakan untuk unit dalam EMT.
\end{eulercomment}
\begin{eulerprompt}
>125km -> " miles"
\end{eulerprompt}
\begin{euleroutput}
  77.6713990297 miles
\end{euleroutput}
\begin{eulercomment}
Operator "\textless{}-" untuk penugasan menyesatkan, dan bukan ide yang baik
untuk R. Berikut ini akan membandingkan a dan -4 dalam EMT.
\end{eulercomment}
\begin{eulerprompt}
>a=2; a<-4
\end{eulerprompt}
\begin{euleroutput}
  0
\end{euleroutput}
\begin{eulercomment}
Dalam R, "a\textless{}-4\textless{}3" bisa digunakan, tetapi "a\textless{}-4\textless{}-3" tidak. Saya juga
mengalami ambiguitas yang sama di EMT, tetapi saya mencoba untuk
menghilangkannya.

EMT dan R memiliki vektor dengan tipe boolean. Tetapi dalam EMT, angka
0 dan 1 digunakan untuk merepresentasikan salah dan benar. Dalam R,
nilai benar dan salah tetap dapat digunakan dalam aritmatika biasa
seperti dalam EMT.
\end{eulercomment}
\begin{eulerprompt}
>x<5, %*x
\end{eulerprompt}
\begin{euleroutput}
  [0,  0,  1,  0,  0]
  [0,  0,  3.1,  0,  0]
\end{euleroutput}
\begin{eulercomment}
EMT melempar kesalahan atau menghasilkan NAN tergantung pada flag
"kesalahan".
\end{eulercomment}
\begin{eulerprompt}
>errors off; 0/0, isNAN(sqrt(-1)), errors on;
\end{eulerprompt}
\begin{euleroutput}
  NAN
  1
\end{euleroutput}
\begin{eulercomment}
String sama saja dalam R dan EMT. Keduanya berada di lokal saat ini,
bukan di Unicode.

Dalam R ada paket-paket untuk Unicode. Dalam EMT, sebuah string dapat
berupa string Unicode. Sebuah string Unicode dapat diterjemahkan ke
pengkodean lokal dan sebaliknya. Selain itu, u"..." dapat berisi
entitas HTML.
\end{eulercomment}
\begin{eulerprompt}
>u"&#169; Ren&eacut; Grothmann"
\end{eulerprompt}
\begin{euleroutput}
  © René Grothmann
\end{euleroutput}
\begin{eulercomment}
Berikut ini mungkin atau mungkin tidak ditampilkan dengan benar pada
sistem Anda sebagai A dengan titik dan tanda hubung di atasnya. Hal
ini tergantung pada jenis huruf yang Anda gunakan.
\end{eulercomment}
\begin{eulerprompt}
>chartoutf([480])
\end{eulerprompt}
\begin{euleroutput}
  Ǡ
\end{euleroutput}
\begin{eulercomment}
Penggabungan string dilakukan dengan "+" atau "\textbar{}". Ini dapat
menyertakan angka, yang akan dicetak dalam format saat ini.
\end{eulercomment}
\begin{eulerprompt}
>"pi = "+pi
\end{eulerprompt}
\begin{euleroutput}
  pi = 3.14159265359
\end{euleroutput}
\eulerheading{Pengindeksan}
\begin{eulercomment}
Sebagian besar waktu, ini akan bekerja seperti pada R.

Tetapi EMT akan menginterpretasikan indeks negatif dari bagian
belakang vektor, sementara R menginterpretasikan x[n] sebagai x tanpa
elemen ke-n.
\end{eulercomment}
\begin{eulerprompt}
>x, x[1:3], x[-2]
\end{eulerprompt}
\begin{euleroutput}
  [10.4,  5.6,  3.1,  6.4,  21.7]
  [10.4,  5.6,  3.1]
  6.4
\end{euleroutput}
\begin{eulercomment}
Perilaku R dapat dicapai dalam EMT dengan drop().
\end{eulercomment}
\begin{eulerprompt}
>drop(x,2)
\end{eulerprompt}
\begin{euleroutput}
  [10.4,  3.1,  6.4,  21.7]
\end{euleroutput}
\begin{eulercomment}
Vektor logika tidak diperlakukan secara berbeda dengan indeks di EMT,
berbeda dengan R. Anda harus mengekstrak elemen-elemen yang bukan nol
terlebih dahulu di EMT.
\end{eulercomment}
\begin{eulerprompt}
>x, x>5, x[nonzeros(x>5)]
\end{eulerprompt}
\begin{euleroutput}
  [10.4,  5.6,  3.1,  6.4,  21.7]
  [1,  1,  0,  1,  1]
  [10.4,  5.6,  6.4,  21.7]
\end{euleroutput}
\begin{eulercomment}
Sama seperti di R, vektor indeks dapat berisi pengulangan.
\end{eulercomment}
\begin{eulerprompt}
>x[[1,2,2,1]]
\end{eulerprompt}
\begin{euleroutput}
  [10.4,  5.6,  5.6,  10.4]
\end{euleroutput}
\begin{eulercomment}
Namun pemberian nama untuk indeks tidak dimungkinkan dalam EMT. Untuk
paket statistik, hal ini mungkin sering diperlukan untuk memudahkan
akses ke elemen-elemen vektor.

Untuk meniru perilaku ini, kita dapat mendefinisikan sebuah fungsi
sebagai berikut.
\end{eulercomment}
\begin{eulerprompt}
>function sel (v,i,s) := v[indexof(s,i)]; ...
>s=["first","second","third","fourth"]; sel(x,["first","third"],s)
\end{eulerprompt}
\begin{euleroutput}
  
  Trying to overwrite protected function sel!
  Error in:
  function sel (v,i,s) := v[indexof(s,i)]; ... ...
               ^
  
  Trying to overwrite protected function sel!
  Error in:
  function sel (v,i,s) := v[indexof(s,i)]; ... ...
               ^
  [10.4,  3.1]
\end{euleroutput}
\eulerheading{Tipe Data}
\begin{eulercomment}
EMT memiliki lebih banyak tipe data yang tetap dibandingkan R. Jelas,
dalam R terdapat vektor yang berkembang. Anda bisa mengatur sebuah
vektor numerik kosong v dan memberikan sebuah nilai pada elemen v[17].
Hal ini tidak mungkin dilakukan dalam EMT.

Hal berikut ini sedikit tidak efisien.
\end{eulercomment}
\begin{eulerprompt}
>v=[]; for i=1 to 10000; v=v|i; end;
\end{eulerprompt}
\begin{eulercomment}
EMT sekarang akan membuat vektor dengan v dan i yang ditambahkan pada
tumpukan dan menyalin vektor tersebut kembali ke variabel global v.

Semakin efisien mendefinisikan vektor.
\end{eulercomment}
\begin{eulerprompt}
>v=zeros(10000); for i=1 to 10000; v[i]=i; end;
\end{eulerprompt}
\begin{eulercomment}
Untuk mengubah jenis tanggal di EMT, Anda dapat menggunakan fungsi
seperti complex().
\end{eulercomment}
\begin{eulerprompt}
>complex(1:4)
\end{eulerprompt}
\begin{euleroutput}
  [ 1+0i ,  2+0i ,  3+0i ,  4+0i  ]
\end{euleroutput}
\begin{eulercomment}
Konversi ke string hanya dapat dilakukan untuk tipe data dasar. Format
saat ini digunakan untuk penggabungan string sederhana. Tetapi ada
fungsi-fungsi seperti print() atau frac().

Untuk vektor, Anda dapat dengan mudah menulis fungsi Anda sendiri.
\end{eulercomment}
\begin{eulerprompt}
>function tostr (v) ...
\end{eulerprompt}
\begin{eulerudf}
  s="[";
  loop 1 to length(v);
     s=s+print(v[#],2,0);
     if #<length(v) then s=s+","; endif;
  end;
  return s+"]";
  endfunction
\end{eulerudf}
\begin{eulerprompt}
>tostr(linspace(0,1,10))
\end{eulerprompt}
\begin{euleroutput}
  [0.00,0.10,0.20,0.30,0.40,0.50,0.60,0.70,0.80,0.90,1.00]
\end{euleroutput}
\begin{eulercomment}
Untuk komunikasi dengan Maxima, ada sebuah fungsi convertmxm(), yang
juga dapat digunakan untuk memformat vektor untuk output.
\end{eulercomment}
\begin{eulerprompt}
>convertmxm(1:10)
\end{eulerprompt}
\begin{euleroutput}
  [1,2,3,4,5,6,7,8,9,10]
\end{euleroutput}
\begin{eulercomment}
Untuk Latex, perintah tex dapat digunakan untuk mendapatkan perintah
Latex.
\end{eulercomment}
\begin{eulerprompt}
>tex(&[1,2,3])
\end{eulerprompt}
\begin{euleroutput}
  \(\backslash\)left[ 1 , 2 , 3 \(\backslash\)right] 
\end{euleroutput}
\eulerheading{Faktor dan Tabel}
\begin{eulercomment}
Pada pengantar R terdapat sebuah contoh dengan apa yang disebut
faktor.

Berikut ini adalah daftar wilayah dari 30 negara bagian.
\end{eulercomment}
\begin{eulerprompt}
>austates = ["tas", "sa", "qld", "nsw", "nsw", "nt", "wa", "wa", ...
>"qld", "vic", "nsw", "vic", "qld", "qld", "sa", "tas", ...
>"sa", "nt", "wa", "vic", "qld", "nsw", "nsw", "wa", ...
>"sa", "act", "nsw", "vic", "vic", "act"];
\end{eulerprompt}
\begin{eulercomment}
Asumsikan, kita memiliki pendapatan yang sesuai di setiap negara
bagian.
\end{eulercomment}
\begin{eulerprompt}
>incomes = [60, 49, 40, 61, 64, 60, 59, 54, 62, 69, 70, 42, 56, ...
>61, 61, 61, 58, 51, 48, 65, 49, 49, 41, 48, 52, 46, ...
>59, 46, 58, 43];
\end{eulerprompt}
\begin{eulercomment}
Sekarang, kita ingin menghitung rata-rata pendapatan di wilayah
tersebut. Sebagai sebuah program statistik, R memiliki fungsi factor()
dan tappy() untuk hal ini.

EMT dapat melakukan hal ini dengan mencari indeks dari wilayah-wilayah
di dalam daftar unik dari wilayah-wilayah tersebut.
\end{eulercomment}
\begin{eulerprompt}
>auterr=sort(unique(austates)); f=indexofsorted(auterr,austates)
\end{eulerprompt}
\begin{euleroutput}
  [6,  5,  4,  2,  2,  3,  8,  8,  4,  7,  2,  7,  4,  4,  5,  6,  5,  3,
  8,  7,  4,  2,  2,  8,  5,  1,  2,  7,  7,  1]
\end{euleroutput}
\begin{eulercomment}
Pada titik ini, kita dapat menulis fungsi perulangan kita sendiri
untuk melakukan berbagai hal untuk satu faktor saja.

Atau kita dapat meniru fungsi tapply() dengan cara berikut.
\end{eulercomment}
\begin{eulerprompt}
>function map tappl (i; f$:call, cat, x) ...
\end{eulerprompt}
\begin{eulerudf}
  u=sort(unique(cat));
  f=indexof(u,cat);
  return f$(x[nonzeros(f==indexof(u,i))]);
  endfunction
\end{eulerudf}
\begin{eulercomment}
Ini sedikit tidak efisien, karena menghitung wilayah unik untuk setiap
i, tetapi berfungsi.
\end{eulercomment}
\begin{eulerprompt}
>tappl(auterr,"mean",austates,incomes)
\end{eulerprompt}
\begin{euleroutput}
  [44.5,  57.3333333333,  55.5,  53.6,  55,  60.5,  56,  52.25]
\end{euleroutput}
\begin{eulercomment}
Perhatikan bahwa ini bekerja untuk setiap vektor wilayah.
\end{eulercomment}
\begin{eulerprompt}
>tappl(["act","nsw"],"mean",austates,incomes)
\end{eulerprompt}
\begin{euleroutput}
  [44.5,  57.3333333333]
\end{euleroutput}
\begin{eulercomment}
Sekarang, paket statistik EMT mendefinisikan tabel seperti halnya di
R. Fungsi readtable() dan writetable() dapat digunakan untuk input dan
output.

Jadi kita dapat mencetak rata-rata pendapatan negara di wilayah dengan
cara yang ramah.
\end{eulercomment}
\begin{eulerprompt}
>writetable(tappl(auterr,"mean",austates,incomes),labc=auterr,wc=7)
\end{eulerprompt}
\begin{euleroutput}
      act    nsw     nt    qld     sa    tas    vic     wa
     44.5  57.33   55.5   53.6     55   60.5     56  52.25
\end{euleroutput}
\begin{eulercomment}
Kita juga dapat mencoba meniru perilaku R sepenuhnya.

Faktor-faktor tersebut harus disimpan dengan jelas dalam sebuah
koleksi dengan jenis dan kategorinya (negara bagian dan wilayah dalam
contoh kita). Untuk EMT, kita menambahkan indeks yang telah dihitung
sebelumnya.
\end{eulercomment}
\begin{eulerprompt}
>function makef (t) ...
\end{eulerprompt}
\begin{eulerudf}
  ## Factor data
  ## Returns a collection with data t, unique data, indices.
  ## See: tapply
  u=sort(unique(t));
  return \{\{t,u,indexofsorted(u,t)\}\};
  endfunction
\end{eulerudf}
\begin{eulerprompt}
>statef=makef(austates);
\end{eulerprompt}
\begin{eulercomment}
Sekarang elemen ketiga dari koleksi ini akan berisi indeks.
\end{eulercomment}
\begin{eulerprompt}
>statef[3]
\end{eulerprompt}
\begin{euleroutput}
  [6,  5,  4,  2,  2,  3,  8,  8,  4,  7,  2,  7,  4,  4,  5,  6,  5,  3,
  8,  7,  4,  2,  2,  8,  5,  1,  2,  7,  7,  1]
\end{euleroutput}
\begin{eulercomment}
Sekarang kita dapat meniru tapply() dengan cara berikut. Ini akan
mengembalikan sebuah tabel sebagai kumpulan data tabel dan judul
kolom.
\end{eulercomment}
\begin{eulerprompt}
>function tapply (t:vector,tf,f$:call) ...
\end{eulerprompt}
\begin{eulerudf}
  ## Makes a table of data and factors
  ## tf : output of makef()
  ## See: makef
  uf=tf[2]; f=tf[3]; x=zeros(length(uf));
  for i=1 to length(uf);
     ind=nonzeros(f==i);
     if length(ind)==0 then x[i]=NAN;
     else x[i]=f$(t[ind]);
     endif;
  end;
  return \{\{x,uf\}\};
  endfunction
\end{eulerudf}
\begin{eulercomment}
Kami tidak menambahkan banyak pemeriksaan tipe di sini. Satu-satunya
tindakan pencegahan adalah kategori (faktor) yang tidak memiliki data.
Tetapi kita harus memeriksa panjang t yang benar dan kebenaran koleksi
tf.

Tabel ini bisa dicetak sebagai sebuah tabel dengan writetable().
\end{eulercomment}
\begin{eulerprompt}
>writetable(tapply(incomes,statef,"mean"),wc=7)
\end{eulerprompt}
\begin{euleroutput}
      act    nsw     nt    qld     sa    tas    vic     wa
     44.5  57.33   55.5   53.6     55   60.5     56  52.25
\end{euleroutput}
\eulerheading{Larik}
\begin{eulercomment}
EMT hanya memiliki dua dimensi untuk array. Tipe data ini disebut
matriks. Akan lebih mudah untuk menulis fungsi untuk dimensi yang
lebih tinggi atau pustaka C untuk ini.

R memiliki lebih dari dua dimensi. Dalam R, larik adalah sebuah vektor
dengan sebuah bidang dimensi.

Dalam EMT, sebuah vektor adalah sebuah matriks dengan satu baris. Ini
bisa dibuat menjadi sebuah matriks dengan redim().
\end{eulercomment}
\begin{eulerprompt}
>shortformat; X=redim(1:20,4,5)
\end{eulerprompt}
\begin{euleroutput}
          1         2         3         4         5 
          6         7         8         9        10 
         11        12        13        14        15 
         16        17        18        19        20 
\end{euleroutput}
\begin{eulercomment}
Ekstraksi baris dan kolom, atau sub-matriks, sama seperti di R.
\end{eulercomment}
\begin{eulerprompt}
>X[,2:3]
\end{eulerprompt}
\begin{euleroutput}
          2         3 
          7         8 
         12        13 
         17        18 
\end{euleroutput}
\begin{eulercomment}
Namun, dalam R dimungkinkan untuk mengatur daftar indeks tertentu dari
vektor ke suatu nilai. Hal yang sama juga dapat dilakukan dalam EMT
hanya dengan sebuah perulangan.
\end{eulercomment}
\begin{eulerprompt}
>function setmatrixvalue (M, i, j, v) ...
\end{eulerprompt}
\begin{eulerudf}
  loop 1 to max(length(i),length(j),length(v))
     M[i\{#\},j\{#\}] = v\{#\};
  end;
  endfunction
\end{eulerudf}
\begin{eulercomment}
Kami mendemonstrasikan hal ini untuk menunjukkan bahwa matriks
diteruskan dengan referensi dalam EMT. Jika Anda tidak ingin mengubah
matriks asli M, Anda perlu menyalinnya dalam fungsi.
\end{eulercomment}
\begin{eulerprompt}
>setmatrixvalue(X,1:3,3:-1:1,0); X,
\end{eulerprompt}
\begin{euleroutput}
          1         2         0         4         5 
          6         0         8         9        10 
          0        12        13        14        15 
         16        17        18        19        20 
\end{euleroutput}
\begin{eulercomment}
Hasil kali luar dalam EMT hanya dapat dilakukan di antara vektor. Hal
ini otomatis karena bahasa matriks. Satu vektor harus berupa vektor
kolom dan vektor baris.
\end{eulercomment}
\begin{eulerprompt}
>(1:5)*(1:5)'
\end{eulerprompt}
\begin{euleroutput}
          1         2         3         4         5 
          2         4         6         8        10 
          3         6         9        12        15 
          4         8        12        16        20 
          5        10        15        20        25 
\end{euleroutput}
\begin{eulercomment}
Dalam pengantar PDF untuk R ada sebuah contoh, yang menghitung
distribusi ab-cd untuk a, b, c, d yang dipilih dari 0 sampai n secara
acak. Solusinya dalam R adalah membentuk sebuah matriks 4 dimensi dan
menjalankan table() di atasnya.

Tentu saja, ini bisa dicapai dengan sebuah perulangan. Tetapi
perulangan tidak efektif dalam EMT atau R. Dalam EMT, kita bisa
menulis perulangan dalam C dan itu adalah solusi tercepat.

Tetapi kita ingin meniru perilaku R. Untuk ini, kita perlu meratakan
perkalian ab dan membuat sebuah matriks ab-cd.
\end{eulercomment}
\begin{eulerprompt}
>a=0:6; b=a'; p=flatten(a*b); q=flatten(p-p'); ...
>u=sort(unique(q)); f=getmultiplicities(u,q); ...
>statplot(u,f,"h"):
\end{eulerprompt}
\eulerimg{15}{images/Pekan 13-14_Fanny Erina Dewi_22305141005_EMT00-Statistika_Aplikom-038.png}
\begin{eulercomment}
Selain kelipatan yang tepat, EMT dapat menghitung frekuensi dalam
vektor.
\end{eulercomment}
\begin{eulerprompt}
>getfrequencies(q,-50:10:50)
\end{eulerprompt}
\begin{euleroutput}
  [0,  23,  132,  316,  602,  801,  333,  141,  53,  0]
\end{euleroutput}
\begin{eulercomment}
Cara yang paling mudah untuk memplot ini sebagai distribusi adalah
sebagai berikut.
\end{eulercomment}
\begin{eulerprompt}
>plot2d(q,distribution=11):
\end{eulerprompt}
\eulerimg{15}{images/Pekan 13-14_Fanny Erina Dewi_22305141005_EMT00-Statistika_Aplikom-039.png}
\begin{eulercomment}
Tetapi juga memungkinkan untuk menghitung jumlah dalam interval yang
dipilih sebelumnya. Tentu saja, berikut ini menggunakan
getfrequencies() secara internal.

Karena fungsi histo() mengembalikan frekuensi, kita perlu
menskalakannya sehingga integral di bawah grafik batang adalah 1.
\end{eulercomment}
\begin{eulerprompt}
>\{x,y\}=histo(q,v=-55:10:55); y=y/sum(y)/differences(x); ...
>plot2d(x,y,>bar,style="/"):
\end{eulerprompt}
\eulerimg{15}{images/Pekan 13-14_Fanny Erina Dewi_22305141005_EMT00-Statistika_Aplikom-040.png}
\eulerheading{Daftar}
\begin{eulercomment}
EMT memiliki dua jenis daftar. Yang pertama adalah daftar global yang
dapat diubah, dan yang kedua adalah jenis daftar yang tidak dapat
diubah. Kita tidak peduli dengan daftar global di sini.

Tipe daftar yang tidak dapat diubah disebut koleksi dalam EMT. Ia
berperilaku seperti struktur dalam C, tetapi elemen-elemennya hanya
diberi nomor dan tidak diberi nama.
\end{eulercomment}
\begin{eulerprompt}
>L=\{\{"Fred","Flintstone",40,[1990,1992]\}\}
\end{eulerprompt}
\begin{euleroutput}
  Fred
  Flintstone
  40
  [1990,  1992]
\end{euleroutput}
\begin{eulercomment}
Saat ini elemen-elemen tersebut tidak memiliki nama, meskipun nama
dapat ditetapkan untuk tujuan khusus. Elemen-elemen tersebut diakses
dengan angka.
\end{eulercomment}
\begin{eulerprompt}
>(L[4])[2]
\end{eulerprompt}
\begin{euleroutput}
  1992
\end{euleroutput}
\begin{eulercomment}
\begin{eulercomment}
\eulerheading{Input dan Output File (Membaca dan Menulis Data)}
\begin{eulercomment}
Anda mungkin sering ingin mengimpor matriks data dari sumber lain ke
EMT. Tutorial ini akan menjelaskan kepada Anda tentang berbagai cara
untuk melakukan hal tersebut. Fungsi yang sederhana adalah
writematrix() dan readmatrix().

Mari kita tunjukkan bagaimana cara membaca dan menulis sebuah vektor
real ke sebuah file.
\end{eulercomment}
\begin{eulerprompt}
>a=random(1,100); mean(a), dev(a),
\end{eulerprompt}
\begin{euleroutput}
  0.49815
  0.28037
\end{euleroutput}
\begin{eulercomment}
Untuk menulis data ke sebuah berkas, kita menggunakan fungsi
writematrix().

Karena pengenalan ini kemungkinan besar berada di sebuah direktori, di
mana pengguna tidak memiliki akses tulis, kami menulis data ke
direktori home pengguna. Untuk notebook sendiri, hal ini tidak
diperlukan, karena file data akan ditulis ke dalam direktori yang
sama.
\end{eulercomment}
\begin{eulerprompt}
>filename="test.dat";
\end{eulerprompt}
\begin{eulercomment}
Sekarang kita tuliskan vektor kolom a' ke dalam file. Hal ini akan
menghasilkan satu angka pada setiap baris file.
\end{eulercomment}
\begin{eulerprompt}
>writematrix(a',filename);
\end{eulerprompt}
\begin{eulercomment}
Untuk membaca data, kita menggunakan readmatrix().
\end{eulercomment}
\begin{eulerprompt}
>a=readmatrix(filename)';
\end{eulerprompt}
\begin{eulercomment}
Dan hapus file tersebut.
\end{eulercomment}
\begin{eulerprompt}
>fileremove(filename);
>mean(a), dev(a),
\end{eulerprompt}
\begin{euleroutput}
  0.49815
  0.28037
\end{euleroutput}
\begin{eulercomment}
Fungsi writematrix() atau writetable() dapat dikonfigurasi untuk
bahasa lain.

Sebagai contoh, jika Anda memiliki sistem bahasa Indonesia (titik
desimal dengan koma), Excel Anda membutuhkan nilai dengan koma desimal
yang dipisahkan oleh titik koma dalam file csv (defaultnya adalah
nilai yang dipisahkan dengan koma). File "test.csv" berikut ini akan
muncul di folder cuurent Anda.
\end{eulercomment}
\begin{eulerprompt}
>filename="test.csv"; ...
>writematrix(random(5,3),file=filename,separator=",");
\end{eulerprompt}
\begin{eulercomment}
Anda sekarang dapat membuka file ini dengan Excel Indonesia secara
langsung.
\end{eulercomment}
\begin{eulerprompt}
>fileremove(filename);
\end{eulerprompt}
\begin{eulercomment}
Terkadang kita memiliki string dengan token seperti berikut ini.
\end{eulercomment}
\begin{eulerprompt}
>s1:="f m m f m m m f f f m m f";  ...
>s2:="f f f m m f f";
\end{eulerprompt}
\begin{eulercomment}
Untuk menandai ini, kita mendefinisikan vektor token.
\end{eulercomment}
\begin{eulerprompt}
>tok:=["f","m"]
\end{eulerprompt}
\begin{euleroutput}
  f
  m
\end{euleroutput}
\begin{eulercomment}
Kemudian kita dapat menghitung berapa kali setiap token muncul dalam
string, dan memasukkan hasilnya ke dalam tabel.
\end{eulercomment}
\begin{eulerprompt}
>M:=getmultiplicities(tok,strtokens(s1))_ ...
>  getmultiplicities(tok,strtokens(s2));
\end{eulerprompt}
\begin{eulercomment}
Tulis tabel dengan tajuk token.
\end{eulercomment}
\begin{eulerprompt}
>writetable(M,labc=tok,labr=1:2,wc=8)
\end{eulerprompt}
\begin{euleroutput}
                 f       m
         1       6       7
         2       5       2
\end{euleroutput}
\begin{eulercomment}
Untuk statika, EMT dapat membaca dan menulis tabel.
\end{eulercomment}
\begin{eulerprompt}
>file="test.dat"; open(file,"w"); ...
>writeln("A,B,C"); writematrix(random(3,3)); ...
>close();
\end{eulerprompt}
\begin{eulercomment}
File terlihat seperti ini.
\end{eulercomment}
\begin{eulerprompt}
>printfile(file)
\end{eulerprompt}
\begin{euleroutput}
  A,B,C
  0.7003664386138074,0.1875530821001213,0.3262339279660414
  0.5926249243193858,0.1522927283984059,0.368140583062521
  0.8065535209872989,0.7265910840408142,0.7332619844597152
  
\end{euleroutput}
\begin{eulercomment}
Fungsi readtable() dalam bentuknya yang paling sederhana dapat membaca
ini dan mengembalikan sebuah koleksi nilai dan baris judul.
\end{eulercomment}
\begin{eulerprompt}
>L=readtable(file,>list);
\end{eulerprompt}
\begin{eulercomment}
Koleksi ini dapat dicetak dengan writetable() ke buku catatan, atau ke
sebuah file.
\end{eulercomment}
\begin{eulerprompt}
>writetable(L,wc=10,dc=5)
\end{eulerprompt}
\begin{euleroutput}
           A         B         C
     0.70037   0.18755   0.32623
     0.59262   0.15229   0.36814
     0.80655   0.72659   0.73326
\end{euleroutput}
\begin{eulercomment}
Matriks nilai adalah elemen pertama dari L. Perhatikan bahwa mean()
dalam EMT menghitung nilai rata-rata dari baris-baris matriks.
\end{eulercomment}
\begin{eulerprompt}
>mean(L[1])
\end{eulerprompt}
\begin{euleroutput}
    0.40472 
    0.37102 
    0.75547 
\end{euleroutput}
\eulerheading{File CSV}
\begin{eulercomment}
Pertama, mari kita tulis sebuah matriks ke dalam sebuah file. Untuk
keluarannya, kami membuat file di direktori kerja saat ini.
\end{eulercomment}
\begin{eulerprompt}
>file="test.csv";  ...
>M=random(3,3); writematrix(M,file);
\end{eulerprompt}
\begin{eulercomment}
Berikut ini adalah isi file ini.
\end{eulercomment}
\begin{eulerprompt}
>printfile(file)
\end{eulerprompt}
\begin{euleroutput}
  0.8221197733097619,0.821531098722547,0.7771240608094004
  0.8482947121863489,0.3237767724883862,0.6501422353377985
  0.1482301827518109,0.3297459716109594,0.6261901074210923
  
\end{euleroutput}
\begin{eulercomment}
CVS ini dapat dibuka di sistem bahasa Inggris ke Excel dengan klik dua
kali. Jika Anda mendapatkan file seperti itu pada sistem Jerman, Anda
perlu mengimpor data ke Excel dengan memperhatikan titik desimal.

Namun, titik desimal juga merupakan format default untuk EMT. Anda
dapat membaca sebuah matriks dari sebuah file dengan readmatrix().
\end{eulercomment}
\begin{eulerprompt}
>readmatrix(file)
\end{eulerprompt}
\begin{euleroutput}
    0.82212   0.82153   0.77712 
    0.84829   0.32378   0.65014 
    0.14823   0.32975   0.62619 
\end{euleroutput}
\begin{eulercomment}
Dimungkinkan untuk menulis beberapa matriks ke dalam satu file.
Perintah open() dapat membuka file untuk menulis dengan parameter "w".
Standarnya adalah "r" untuk membaca.
\end{eulercomment}
\begin{eulerprompt}
>open(file,"w"); writematrix(M); writematrix(M'); close();
\end{eulerprompt}
\begin{eulercomment}
Matriks-matriks tersebut dipisahkan oleh sebuah baris kosong. Untuk
membaca matriks, buka file dan panggil readmatrix() beberapa kali.
\end{eulercomment}
\begin{eulerprompt}
>open(file); A=readmatrix(); B=readmatrix(); A==B, close();
\end{eulerprompt}
\begin{euleroutput}
          1         0         0 
          0         1         0 
          0         0         1 
\end{euleroutput}
\begin{eulercomment}
Di Excel atau spreadsheet serupa, Anda dapat mengekspor matriks
sebagai CSV (nilai yang dipisahkan dengan koma). Pada Excel 2007,
gunakan "save as" dan "format lain", lalu pilih "CSV". Pastikan, tabel
saat ini hanya berisi data yang ingin Anda ekspor.

Berikut ini adalah contohnya.
\end{eulercomment}
\begin{eulerprompt}
>printfile("excel-data.csv")
\end{eulerprompt}
\begin{euleroutput}
  0;1000;1000
  1;1051,271096;1072,508181
  2;1105,170918;1150,273799
  3;1161,834243;1233,67806
  4;1221,402758;1323,129812
  5;1284,025417;1419,067549
  6;1349,858808;1521,961556
  7;1419,067549;1632,31622
  8;1491,824698;1750,6725
  9;1568,312185;1877,610579
  10;1648,721271;2013,752707
\end{euleroutput}
\begin{eulercomment}
Seperti yang Anda lihat, sistem Jerman saya menggunakan titik koma
sebagai pemisah dan koma desimal. Anda dapat mengubahnya di pengaturan
sistem atau di Excel, tetapi tidak perlu untuk membaca matriks ke
dalam EMT.

Cara termudah untuk membaca ini ke dalam Euler adalah readmatrix().
Semua koma digantikan oleh titik dengan parameter \textgreater{}comma. Untuk CSV
bahasa Inggris, hilangkan saja parameter ini.
\end{eulercomment}
\begin{eulerprompt}
>M=readmatrix("excel-data.csv",>comma)
\end{eulerprompt}
\begin{euleroutput}
          0      1000      1000 
          1    1051.3    1072.5 
          2    1105.2    1150.3 
          3    1161.8    1233.7 
          4    1221.4    1323.1 
          5      1284    1419.1 
          6    1349.9      1522 
          7    1419.1    1632.3 
          8    1491.8    1750.7 
          9    1568.3    1877.6 
         10    1648.7    2013.8 
\end{euleroutput}
\begin{eulercomment}
Mari kita rencanakan ini.
\end{eulercomment}
\begin{eulerprompt}
>plot2d(M'[1],M'[2:3],>points,color=[red,green]'):
\end{eulerprompt}
\eulerimg{15}{images/Pekan 13-14_Fanny Erina Dewi_22305141005_EMT00-Statistika_Aplikom-041.png}
\begin{eulercomment}
Ada beberapa cara yang lebih mendasar untuk membaca data dari file.
Anda dapat membuka file dan membaca angka baris demi baris. Fungsi
getvectorline() akan membaca angka dari sebuah baris data. Secara
default, fungsi ini mengharapkan sebuah titik desimal. Tetapi fungsi
ini juga dapat menggunakan koma desimal, jika Anda memanggil
setdecimaldot(",") sebelum menggunakan fungsi ini.

Fungsi berikut ini adalah contohnya. Fungsi ini akan berhenti pada
akhir file atau baris kosong.
\end{eulercomment}
\begin{eulerprompt}
>function myload (file) ...
\end{eulerprompt}
\begin{eulerudf}
  open(file);
  M=[];
  repeat
     until eof();
     v=getvectorline(3);
     if length(v)>0 then M=M_v; else break; endif;
  end;
  return M;
  close(file);
  endfunction
\end{eulerudf}
\begin{eulerprompt}
>myload(file)
\end{eulerprompt}
\begin{euleroutput}
    0.82212   0.82153   0.77712 
    0.84829   0.32378   0.65014 
    0.14823   0.32975   0.62619 
\end{euleroutput}
\begin{eulercomment}
Anda juga dapat membaca semua angka dalam file tersebut dengan
getvector().
\end{eulercomment}
\begin{eulerprompt}
>open(file); v=getvector(10000); close(); redim(v[1:9],3,3)
\end{eulerprompt}
\begin{euleroutput}
    0.82212   0.82153   0.77712 
    0.84829   0.32378   0.65014 
    0.14823   0.32975   0.62619 
\end{euleroutput}
\begin{eulercomment}
Dengan demikian, sangat mudah untuk menyimpan vektor nilai, satu nilai
di setiap baris dan membaca kembali vektor ini.
\end{eulercomment}
\begin{eulerprompt}
>v=random(1000); mean(v)
\end{eulerprompt}
\begin{euleroutput}
  0.50303
\end{euleroutput}
\begin{eulerprompt}
>writematrix(v',file); mean(readmatrix(file)')
\end{eulerprompt}
\begin{euleroutput}
  0.50303
\end{euleroutput}
\eulerheading{Menggunakan Tabel}
\begin{eulercomment}
Tabel dapat digunakan untuk membaca atau menulis data numerik. Sebagai
contoh, kita menulis tabel dengan judul baris dan kolom ke file.
\end{eulercomment}
\begin{eulerprompt}
>file="test.tab"; M=random(3,3);  ...
>open(file,"w");  ...
>writetable(M,separator=",",labc=["one","two","three"]);  ...
>close(); ...
>printfile(file)
\end{eulerprompt}
\begin{euleroutput}
  one,two,three
        0.09,      0.39,      0.86
        0.39,      0.86,      0.71
         0.2,      0.02,      0.83
\end{euleroutput}
\begin{eulercomment}
File ini dapat diimpor ke Excel.

Untuk membaca file di EMT, kita menggunakan readtable().
\end{eulercomment}
\begin{eulerprompt}
>\{M,headings\}=readtable(file,>clabs); ...
>writetable(M,labc=headings)
\end{eulerprompt}
\begin{euleroutput}
         one       two     three
        0.09      0.39      0.86
        0.39      0.86      0.71
         0.2      0.02      0.83
\end{euleroutput}
\eulerheading{Menganalisis Garis}
\begin{eulercomment}
Anda bahkan dapat mengevaluasi setiap baris dengan tangan. Misalkan,
kita memiliki baris dengan format berikut.
\end{eulercomment}
\begin{eulerprompt}
>line="2020-11-03,Tue,1'114.05"
\end{eulerprompt}
\begin{euleroutput}
  2020-11-03,Tue,1'114.05
\end{euleroutput}
\begin{eulercomment}
Pertama, kita dapat memberi tanda pada garis tersebut.
\end{eulercomment}
\begin{eulerprompt}
>vt=strtokens(line)
\end{eulerprompt}
\begin{euleroutput}
  2020-11-03
  Tue
  1'114.05
\end{euleroutput}
\begin{eulercomment}
Kemudian, kita dapat mengevaluasi setiap elemen garis dengan
menggunakan evaluasi yang sesuai.
\end{eulercomment}
\begin{eulerprompt}
>day(vt[1]),  ...
>indexof(["mon","tue","wed","thu","fri","sat","sun"],tolower(vt[2])),  ...
>strrepl(vt[3],"'","")()
\end{eulerprompt}
\begin{euleroutput}
  7.3816e+05
  2
  1114
\end{euleroutput}
\begin{eulercomment}
Dengan menggunakan ekspresi reguler, Anda dapat mengekstrak hampir
semua informasi dari sebuah baris data.

Anggaplah kita memiliki baris dokumen HTML berikut ini.
\end{eulercomment}
\begin{eulerprompt}
>line="<tr><td>1145.45</td><td>5.6</td><td>-4.5</td><tr>"
\end{eulerprompt}
\begin{euleroutput}
  <tr><td>1145.45</td><td>5.6</td><td>-4.5</td><tr>
\end{euleroutput}
\begin{eulercomment}
Untuk mengekstrak ini, kita menggunakan ekspresi reguler, yang mencari

\end{eulercomment}
\begin{eulerttcomment}
 - tanda kurung tutup >,
 - setiap string yang tidak mengandung tanda kurung dengan
\end{eulerttcomment}
\begin{eulercomment}
sub-pencocokan "(...)",\\
\end{eulercomment}
\begin{eulerttcomment}
 - kurung pembuka dan kurung penutup menggunakan solusi terpendek,
 - sekali lagi, semua string yang tidak mengandung tanda kurung,
 - dan sebuah kurung pembuka <.
\end{eulerttcomment}
\begin{eulercomment}

Ekspresi reguler agak sulit untuk dipelajari tetapi sangat kuat.
\end{eulercomment}
\begin{eulerprompt}
>\{pos,s,vt\}=strxfind(line,">([^<>]+)<.+?>([^<>]+)<");
\end{eulerprompt}
\begin{eulercomment}
Hasilnya adalah posisi kecocokan, string yang cocok, dan vektor string
untuk sub-cocokan.
\end{eulercomment}
\begin{eulerprompt}
>for k=1:length(vt); vt[k](), end;
\end{eulerprompt}
\begin{euleroutput}
  1145.5
  5.6
\end{euleroutput}
\begin{eulercomment}
Berikut ini adalah fungsi yang membaca semua item numerik antara \textless{}td\textgreater{}
dan \textless{}/td\textgreater{}.
\end{eulercomment}
\begin{eulerprompt}
>function readtd (line) ...
\end{eulerprompt}
\begin{eulerudf}
  v=[]; cp=0;
  repeat
     \{pos,s,vt\}=strxfind(line,"<td.*?>(.+?)</td>",cp);
     until pos==0;
     if length(vt)>0 then v=v|vt[1]; endif;
     cp=pos+strlen(s);
  end;
  return v;
  endfunction
\end{eulerudf}
\begin{eulerprompt}
>readtd(line+"<td>non-numerical</td>")
\end{eulerprompt}
\begin{euleroutput}
  1145.45
  5.6
  -4.5
  non-numerical
\end{euleroutput}
\eulerheading{Membaca dari Web}
\begin{eulercomment}
Situs web atau file dengan URL dapat dibuka di EMT dan dapat dibaca
baris demi baris.

Dalam contoh, kita membaca versi saat ini dari situs EMT. Kami
menggunakan ekspresi reguler untuk memindai "Versi ..." dalam judul.
\end{eulercomment}
\begin{eulerprompt}
>function readversion () ...
\end{eulerprompt}
\begin{eulerudf}
  urlopen("http://www.euler-math-toolbox.de/Programs/Changes.html");
  repeat
    until urleof();
    s=urlgetline();
    k=strfind(s,"Version ",1);
    if k>0 then substring(s,k,strfind(s,"<",k)-1), break; endif;
  end;
  urlclose();
  endfunction
\end{eulerudf}
\begin{eulerprompt}
>readversion
\end{eulerprompt}
\begin{euleroutput}
  Version 2022-05-18
\end{euleroutput}
\eulerheading{Input dan Output Variabel}
\begin{eulercomment}
Anda dapat menulis variabel dalam bentuk definisi Euler ke file atau
ke baris perintah.
\end{eulercomment}
\begin{eulerprompt}
>writevar(pi,"mypi");
\end{eulerprompt}
\begin{euleroutput}
  mypi = 3.141592653589793;
\end{euleroutput}
\begin{eulercomment}
Untuk pengujian, kami membuat file Euler di direktori kerja EMT.
\end{eulercomment}
\begin{eulerprompt}
>file="test.e"; ...
>writevar(random(2,2),"M",file); ...
>printfile(file,3)
\end{eulerprompt}
\begin{euleroutput}
  M = [ ..
  0.5991820585590205, 0.7960280262224293;
  0.5167243983231363, 0.2996684599070898];
\end{euleroutput}
\begin{eulercomment}
Sekarang kita dapat memuat file tersebut. Ini akan mendefinisikan
matriks M.
\end{eulercomment}
\begin{eulerprompt}
>load(file); show M,
\end{eulerprompt}
\begin{euleroutput}
  M = 
    0.59918   0.79603 
    0.51672   0.29967 
\end{euleroutput}
\begin{eulercomment}
Sebagai catatan, jika writevar() digunakan pada sebuah variabel, maka
ia akan mencetak definisi variabel dengan nama variabel tersebut.
\end{eulercomment}
\begin{eulerprompt}
>writevar(M); writevar(inch$)
\end{eulerprompt}
\begin{euleroutput}
  M = [ ..
  0.5991820585590205, 0.7960280262224293;
  0.5167243983231363, 0.2996684599070898];
  inch$ = 0.0254;
\end{euleroutput}
\begin{eulercomment}
Kita juga dapat membuka file baru atau menambahkan ke file yang sudah
ada. Dalam contoh ini, kami menambahkan ke file yang telah dibuat
sebelumnya.
\end{eulercomment}
\begin{eulerprompt}
>open(file,"a"); ...
>writevar(random(2,2),"M1"); ...
>writevar(random(3,1),"M2"); ...
>close();
>load(file); show M1; show M2;
\end{eulerprompt}
\begin{euleroutput}
  M1 = 
    0.30287   0.15372 
     0.7504   0.75401 
  M2 = 
    0.27213 
   0.053211 
    0.70249 
\end{euleroutput}
\begin{eulercomment}
Untuk menghapus file, gunakan fileremove().
\end{eulercomment}
\begin{eulerprompt}
>fileremove(file);
\end{eulerprompt}
\begin{eulercomment}
Sebuah vektor baris dalam sebuah file tidak membutuhkan koma, jika
setiap angka berada dalam baris baru. Mari kita buat file seperti itu,
menulis setiap baris satu per satu dengan writeln().
\end{eulercomment}
\begin{eulerprompt}
>open(file,"w"); writeln("M = ["); ...
>for i=1 to 5; writeln(""+random()); end; ...
>writeln("];"); close(); ...
>printfile(file)
\end{eulerprompt}
\begin{euleroutput}
  M = [
  0.344851384551
  0.0807510017715
  0.876519562911
  0.754157709472
  0.688392638934
  ];
\end{euleroutput}
\begin{eulerprompt}
>load(file); M
\end{eulerprompt}
\begin{euleroutput}
  [0.34485,  0.080751,  0.87652,  0.75416,  0.68839]
\end{euleroutput}
\eulerheading{Latihan}
\begin{eulercomment}
1. Nilai ulangan harian bahasa Inggris kelas VII SMP Negeri 2 Sleman
berturut turut adalah:
80,90,78,68,56,88,100,78,86,56,78,80,98,54,78,90,100,96,86\\
Tentukan mean, median dan dev
\end{eulercomment}
\begin{eulerprompt}
>M=[80,90,78,68,56,88,100,78,86,56,78,80,98,54,78,90,100,96,86];...
>median(M), mean(M), dev(M)
\end{eulerprompt}
\begin{euleroutput}
  80
  81.053
  14.304
\end{euleroutput}
\begin{eulercomment}
Kemudian menghitung distribusi probabilitas kumulatif dan didapatkan
hasilnya 0\%
\end{eulercomment}
\begin{eulerprompt}
>print((1-normaldis(1005,mean(M),dev(M)))*100,2,unit=" %")
\end{eulerprompt}
\begin{euleroutput}
        0.00 %
\end{euleroutput}
\begin{eulercomment}
2. Perhitungan jumlah uang yang diperlukan dalam masalam pembelian
Bahan bahan dalam menjual samroll seperti rice papper, mie, box dapat
menggunakan perhitungan matrix. Misalnya harga nya secara berturut
turut 5,4,2 ribu rupiah. Banyak pesanan yang diperlukan berturut-turut
adalah 30,10,10 buah. Berapa jumlah uang yang diperlukan.

\end{eulercomment}
\begin{eulerttcomment}
 
\end{eulerttcomment}
\begin{eulerprompt}
>a=[5;4;2]
\end{eulerprompt}
\begin{euleroutput}
          5 
          4 
          2 
\end{euleroutput}
\begin{eulerprompt}
>b=[30;10;10]
\end{eulerprompt}
\begin{euleroutput}
         30 
         10 
         10 
\end{euleroutput}
\begin{eulerprompt}
>a*b
\end{eulerprompt}
\begin{euleroutput}
        150 
         40 
         20 
\end{euleroutput}
\begin{eulercomment}
Maka uang yang diperlukan adalah 150+40+20= Rp 210.000,-

3. Pada 50 lemparan koin, jumlah angka yang diharapkan dengan nilai
rata rata 25 dan deviasi standar\\
Hitunglah probabilitas untuk mendapatkan lebih dari 30 muncul angka,
dan ketika probabilitasnya kurang dari 0,1\% approximating distribusi
binomial dengan distribusi normal.
\end{eulercomment}
\begin{eulerprompt}
>n=50; p=0.5;...
>m=n*p; s=sqrt(n*p*(1-p));...
>1-normaldis(30,m,s)
\end{eulerprompt}
\begin{euleroutput}
  0.07865
\end{euleroutput}
\begin{eulerprompt}
>ceil(invnormaldis(99.9%,m,s))
\end{eulerprompt}
\begin{euleroutput}
  36
\end{euleroutput}
\begin{eulerprompt}
>1-bindis(30,50,0.5)
\end{eulerprompt}
\begin{euleroutput}
  0.05946
\end{euleroutput}
\begin{eulerprompt}
>invbindis(99.9%,50,0.5)
\end{eulerprompt}
\begin{euleroutput}
  35.361
\end{euleroutput}
\begin{eulercomment}
4. Disajikan data urut yaitu
45,48,49,50,52,52,52,53,53,54,54,54,54,54,56,56,
56,56,57,57,58,58,58,58,58,58,58,59,59,60,60,60,
62,62,62,63,63,64,64,65,67,68,69,70,70,71,73,74.\\
Buatlah distribusi frekuensi berdasarkan data diatas!\\
Penyeleaian:\\
- Menentukan range\\
\end{eulercomment}
\begin{eulerttcomment}
  range= nilai maks-nilai min
       = 74-45
       = 29
\end{eulerttcomment}
\begin{eulercomment}
- Menentukan banyak kelas dengan aturan struges.\\
\end{eulercomment}
\begin{eulerttcomment}
  = 1+3,3 log n, n banyaknya data
  = 1+3,3 log 48
  = 6,64
  = 7
\end{eulerttcomment}
\begin{eulercomment}
- Menentukan panjang kelas\\
\end{eulercomment}
\begin{eulerformula}
\[
p=\frac {range}{banyak kelas}
\]
\end{eulerformula}
\begin{eulerformula}
\[
p=\frac {29}{7}
\]
\end{eulerformula}
\begin{eulerformula}
\[
p= 4.14=5
\]
\end{eulerformula}
\begin{eulercomment}
Berdasarkan pertimbangan beberapa unsur dalam data urut diatas yaitu
nilai minimum 45, nilai maksimum 74, banyak kelas yaitu 7, dan panjang
kelas yaitu 5 maka dapat dibuat tabel distribusi frekuensi dengan
batas bawah kelas pertama yaitu 43 dan batas atas kelas ketujuh yaitu
77. Sehingga dapat ditentukan tepi bawah kelas pertama yaitu
43-0.5=42.5 dan tepi atas kelas ketujuh yaitu 77+0.5=77.5.
\end{eulercomment}
\begin{eulerprompt}
>r=42.5:5:77.5; v=[1,6,13,15,6,5,2];
>T:=r[1:7]' | r[2:8]' | v'; writetable(T,labc=["TB","TA","Frek"])
\end{eulerprompt}
\begin{euleroutput}
          TB        TA      Frek
        42.5      47.5         1
        47.5      52.5         6
        52.5      57.5        13
        57.5      62.5        15
        62.5      67.5         6
        67.5      72.5         5
        72.5      77.5         2
\end{euleroutput}
\begin{eulercomment}
Mencari titik tengah
\end{eulercomment}
\begin{eulerprompt}
>(T[,1]+T[,2])/2 // the midpoint of each interval
\end{eulerprompt}
\begin{euleroutput}
         45 
         50 
         55 
         60 
         65 
         70 
         75 
\end{euleroutput}
\begin{eulercomment}
Sajian dalam bentuk histogram
\end{eulercomment}
\begin{eulerprompt}
>plot2d(r,v,a=40,b=80,c=0,d=20,bar=1,style="\(\backslash\)/"):
\end{eulerprompt}
\eulerimg{15}{images/Pekan 13-14_Fanny Erina Dewi_22305141005_EMT00-Statistika_Aplikom-042.png}
\begin{eulercomment}
5. Seorang pelatih tembak ingin mengevaluasi nilai ketangkasan delapan
anak buahnya jenis senapan yang dipakai M-16 dengan jarak 300 meter
dan masing-masing mendapat nilai 76,85,70,65,40,70,50,dan 80.
Berapakah rata-rata nilai ketangkasan delapan anak tersebut?\\
Penyelesain:
\end{eulercomment}
\begin{eulerprompt}
>x=[76,85,70,65,40,70,50,80]; mean(x),
\end{eulerprompt}
\begin{euleroutput}
  67
\end{euleroutput}
\begin{eulercomment}
Diketahui:\\
\end{eulercomment}
\begin{eulerformula}
\[
\sum x_i={76+85+70+65+40+70+50+80}=536
\]
\end{eulerformula}
\begin{eulerttcomment}
                 n = 8
\end{eulerttcomment}
\begin{eulerformula}
\[
\bar{X} = \frac{\sum x_i}{n}
\]
\end{eulerformula}
\begin{eulerformula}
\[
\bar{X} = \frac{536}{8}
\]
\end{eulerformula}
\begin{eulerformula}
\[
\bar{X} = 67
\]
\end{eulerformula}
\begin{eulercomment}
Sehingga rata-rata nilai ketangkasan delapan anak tersebut adalah 67

6. Diketahui sebuah data hasil nilai Ujian Akhir Semester mata kuliah
Filsafat Ilmu 11 mahasiswa sebagai berikut:
85,90,80,95,50,75,30,60,65,40,70.\\
Tentukan nilai median dari data tersebut!\\
Penyelesaian:\\
Diketahui bahwa kasus ini merupakan data yang berjumlah ganjil,
sehingga nilai median untuk kasus ini adalah sama dengan data yang
memiliki nilai di urutan paling tengah yang memiliki nomor urut k.\\
\end{eulercomment}
\begin{eulerformula}
\[
k = \frac{n+1}{2}
\]
\end{eulerformula}
\begin{eulerformula}
\[
k = \frac{11+1}{2}
\]
\end{eulerformula}
\begin{eulerformula}
\[
k = 6
\]
\end{eulerformula}
\begin{eulerformula}
\[
Me = X_6 = 70
\]
\end{eulerformula}
\begin{eulercomment}
Jadi nilai median dari data hasil Ujian Akhir Semester(UAS) mata
kuliah Filsafat Ilmu 11 mahasiswa adalah 70
\end{eulercomment}
\begin{eulerprompt}
>data=[30,40,50,60,65,70,75,80,85,90,95];
>urut=sort(data)
\end{eulerprompt}
\begin{euleroutput}
  [30,  40,  50,  60,  65,  70,  75,  80,  85,  90,  95]
\end{euleroutput}
\begin{eulerprompt}
>median(data)
\end{eulerprompt}
\begin{euleroutput}
  70
\end{euleroutput}
\begin{eulercomment}
7. Berikut adalah data hasil dari pengukuran berat badan 30 siswa SMA
Negeri 1 Sleman. Dari ke 30 siswa, mayoritas siswa memiliki berat
badan yang ideal.\\
Siswa yang mempunyai:\\
berat badan dalam rentang 36-40 kg sebanyak 2 orang,  \\
berat badan dalam rentang 41-45 kg sebanyak 8 orang,\\
berat badan dalam rentang 46-50 kg sebanyak 9 orang,\\
berat badan dalam rentang 51-55 kg sebanyak 6 orang,\\
berat badan dalam rentang 56-60 kg sebanyak 3 orang, dan  \\
berat badan 61-65 kg sebanyak 2 orang.\\
Tentukan modus dari data hasil pengukuran berat badan 30 siswa di SMA
tersebut!
\end{eulercomment}
\begin{eulerprompt}
>35-0.5
\end{eulerprompt}
\begin{euleroutput}
  34.5
\end{euleroutput}
\begin{eulerprompt}
>(40-36)+1
\end{eulerprompt}
\begin{euleroutput}
  5
\end{euleroutput}
\begin{eulerprompt}
>65+0.5
\end{eulerprompt}
\begin{euleroutput}
  65.5
\end{euleroutput}
\begin{eulerprompt}
>r=34.5:5:65.5; v=[2,8,9,6,3,2];
>T:=r[1:6]' | r[2:7]' | v'; writetable(T,labc=["TB","TA","frek"])
\end{eulerprompt}
\begin{euleroutput}
          TB        TA      frek
        34.5      39.5         2
        39.5      44.5         8
        44.5      49.5         9
        49.5      54.5         6
        54.5      59.5         3
        59.5      64.5         2
\end{euleroutput}
\begin{eulerprompt}
>Tb=44.5, p=5, d1=1, d2=3
\end{eulerprompt}
\begin{euleroutput}
  44.5
  5
  1
  3
\end{euleroutput}
\begin{eulerprompt}
>Tb+p*d1/(d1+d2)
\end{eulerprompt}
\begin{euleroutput}
  45.75
\end{euleroutput}
\begin{eulercomment}
Jadi, modusnya adalah 45.75
\end{eulercomment}
\end{eulernotebook}
\end{document}


